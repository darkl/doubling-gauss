\documentclass[12pt, reqno]{amsart}
\usepackage[normalem]{ulem}

\usepackage{amsmath, amsthm, amssymb}
\usepackage{enumerate}
\tolerance=500
\setlength{\emergencystretch}{3em}
\usepackage[margin=1.0in]{geometry}
\usepackage{xcolor}
\definecolor{cite}{rgb}{0.30,0.60,1.00}
\definecolor{url}{rgb}{0.00,0.00,0.80}
\definecolor{link}{rgb}{0.40,0.10,0.20}
\usepackage[pdfusetitle,colorlinks,linkcolor=link,urlcolor=url,citecolor=cite,pagebackref,breaklinks]{hyperref}
\usepackage{graphicx}
\usepackage{cleveref}
\usepackage{mathdots}
\usepackage{tikz-cd}
\usepackage{comment}
\usepackage{xypic}
\usepackage{mathtools}

% new environment
\newtheorem{theorem}{Theorem}[section]
\newtheorem{claim}[theorem]{Claim}
\newtheorem{proposition}[theorem]{Proposition}
\newtheorem{lemma}[theorem]{Lemma}
\newtheorem{conjecture}[theorem]{Conjecture}
\newtheorem{corollary}[theorem]{Corollary}

\theoremstyle{definition}
\newtheorem{definition}[theorem]{Definition}
\theoremstyle{definition}
\newtheorem{remark}[theorem]{Remark}
\theoremstyle{definition}
\newtheorem{example}[theorem]{Example}

%% frequently used symbols
% Common math notions
\newcommand{\zIntegers}{\mathbb{Z}}
\newcommand{\rReal}{\mathbb{R}}
\newcommand{\cComplex}{\mathbb{C}}
\newcommand{\multiplicativegroup}[1]{#1^{\times}}
\newcommand{\RealPart}{\mathrm{Re}}
\newcommand{\detQuadratic}{{\det}_{\quadraticExtension}}
\newcommand{\Hom}{\mathrm{Hom}}
\newcommand{\EndomorphismRing}{\operatorname{End}}
\newcommand{\Span}{\mathrm{span}}
\newcommand{\Supp}{\mathrm{supp}}
\newcommand{\Stab}{\mathrm{stab}}
\newcommand{\idmap}{\mathrm{id}}
\newcommand{\conjugate}[1]{\overline{#1}}
\newcommand{\indicatorFunction}[1]{\delta_{#1}}
\newcommand{\isomorphic}{\cong}
\newcommand{\lengthof}{\ell}
\newcommand{\abs}[1]{\left|#1\right|}
\newcommand{\sizeof}[1]{\left|#1\right|}
\newcommand{\lcm}{\operatorname{lcm}}
\newcommand{\hermitianSpace}{\mathrm{V}}
\newcommand{\xIsotropic}{\mathrm{X}}
\newcommand{\yIsotropic}{\mathrm{Y}}
\newcommand{\similitudeCharacter}{\operatorname{sim}}
\newcommand{\etale}{\'etale }
\newcommand{\Erdelyi}{Erd{\'e}lyi}
\newcommand{\Toth}{T{\'o}th}

% Inner product
\newcommand{\innerproduct}[2]{\left\langle #1,#2\right\rangle}
\newcommand{\Norm}[1]{\left\Vert #1\right\Vert }
\newcommand{\standardForm}[2]{\left\langle #1,#2\right\rangle}

% Representation theory
\newcommand{\fieldCharacter}{\psi}
\newcommand{\centralCharacter}[1]{\omega_{#1}}
\newcommand{\Ind}[3]{\mathrm{Ind}_{#1}^{#2}\left(#3\right)}
\newcommand{\ind}[3]{\mathrm{ind}_{#1}^{#2}\left(#3\right)}
\newcommand{\Whittaker}{\mathcal{W}}
\newcommand{\Contragradient}[1]{#1^{\vee}}
\newcommand{\underlyingVectorSpace}[1]{V_{#1}}
\newcommand{\representationDeclaration}[1]{#1}
\newcommand{\besselFunction}{\mathcal{J}}
\newcommand{\besselFunctionOfFiniteFieldRepresentation}{\besselFunction_{\finiteFieldRepresentation, \fieldCharacter}}
\newcommand{\SpehRepresentation}[2]{\Delta\left(#1, #2\right)}

\newcommand{\gbesselSpehFunction}[2]{\mathcal{BS}_{#1, #2}}
\newcommand{\besselSpehFunction}[2]{\mathcal{BS}_{\SpehRepresentation{#1}{#2}, \fieldCharacter}}
\newcommand{\fourierTransform}[2]{\mathcal{F}_{#1}#2}
\newcommand{\GKGammaFactor}[3]{\gamma^{\mathrm{GK}}\left(#1 \times #2, #3\right)}
\newcommand{\LocalGKGammaFactor}[4]{\gamma^{\mathrm{GK}}\left(#1, #2 \times #3, #4\right)}

\newcommand{\GKPreGammaFactor}[3]{\Gamma^{\mathrm{GK}}\left(#1 \times #2, #3\right)}
\newcommand{\gGJPreGammaFactor}[3]{\Gamma\left(#1 \times #2, #3\right)}
\newcommand{\GJPreGammaFactor}[2]{\Gamma\left(#1, #2\right)}
\newcommand{\Irr}{\mathrm{Irr}}

% Group theory macros
\newcommand{\rquot}[2]{{#1}\slash{#2}}
\newcommand{\lquot}[2]{{#1}\backslash{#2}}
\newcommand{\grpIndex}[2]{\left[#1:#2\right]}

% Matrices macros
\newcommand{\transpose}[1]{\, {}^{t}#1}
\newcommand{\inverseTranspose}[1]{#1^{\iota}}
\newcommand{\involution}[1]{#1^{c}}
\newcommand{\minusInvolution}[1]{#1^{-c}}
\newcommand{\involutionPlusOne}[1]{#1^{1+c}}
\newcommand{\IdentityMatrix}[1]{I_{#1}}
\newcommand{\diag}{\mathrm{diag}}
\newcommand{\antidiag}{\operatorname{\mathrm{antidiag}}}
\newcommand{\trace}{\operatorname{tr}}
\newcommand{\GL}{\mathrm{GL}}
\newcommand{\SO}{\mathrm{SO}}
\newcommand{\GSO}{\mathrm{GSO}}
\newcommand{\Sp}{\mathrm{Sp}}
\newcommand{\GSp}{\mathrm{GSp}}
\newcommand{\UnitaryGroup}{\mathrm{U}}
\newcommand{\UnipotentSubgroup}{U}
\newcommand{\UnipotentRadical}{N}
\newcommand{\ParabolicSubgroup}{P}
\newcommand{\GroupExtension}[1]{\widetilde{#1}}

% Finite field macros
\newcommand{\FieldNorm}[2]{\mathrm{N}_{#1:#2}}
\newcommand{\aFieldNorm}{\mathrm{N}}
\newcommand{\finiteField}{\mathbb{F}}
\newcommand{\quadraticExtension}{\mathbb{E}}
\newcommand{\finiteFieldExtension}[1]{\finiteField_{#1}}
\newcommand{\quadraticFieldExtension}[1]{\quadraticExtension_{#1}}
\newcommand{\NormOneGroup}[1]{\finiteFieldExtension{#1}^{\aFieldNorm = 1}}
\newcommand{\algebraicClosure}[1]{\overline{#1}}
\newcommand{\Galois}{\operatorname{Gal}}
\newcommand{\Frobenius}{\operatorname{Fr}}
\newcommand{\restrictionOfScalars}[3]{\operatorname{Res}_{#1 \slash #2}{#3}}
\newcommand{\multiplcativeScheme}{\algebraicGroup{G}_m}
\newcommand{\affineLine}{\mathbb{A}^1}
\newcommand{\squareMatrix}{\operatorname{Mat}}
\newcommand{\Mat}[2]{\operatorname{Mat}_{#1 \times #2}}
\newcommand{\GaussSum}[2]{\mathcal{G}\left(#1, #2\right)}
\newcommand{\dblJacobiSum}[2]{\mathcal{J}_{\pm}^{\mathrm{dbl}}\left(#1, #2\right)}
\newcommand{\posDblJacobiSum}[2]{\mathcal{J}_{+}^{\mathrm{dbl}}\left(#1, #2\right)}
\newcommand{\negDblJacobiSum}[2]{\mathcal{J}_{-}^{\mathrm{dbl}}\left(#1, #2\right)}
\newcommand{\JacobiKernel}[1]{\Phi_{#1}}
\newcommand{\posJacobiKernel}[1]{\Phi^{+}_{#1}}
\newcommand{\negJacobiKernel}[1]{\Phi^{-}_{#1}}
\newcommand{\genJacobiKernel}[1]{\Phi^{\pm}_{#1}}
\newcommand{\genHermitianJacobiKernel}[2]{\Phi^{\pm}_{#1,#2}}

\newcommand{\GaussSumScalar}[2]{\mathrm{G}\left(#1, #2\right)}
\newcommand{\dblJacobiSumScalar}[2]{\mathrm{J}_{\pm}^{\mathrm{dbl}}\left(#1, #2\right)}
\newcommand{\posDblJacobiSumScalar}[2]{\mathrm{J}_{+}^{\mathrm{dbl}}\left(#1, #2\right)}
\newcommand{\negDblJacobiSumScalar}[2]{\mathrm{J}_{-}^{\mathrm{dbl}}\left(#1, #2\right)}
\newcommand{\dblVirtualJacobiSumScalar}[2]{\mathrm{j}_{\pm}^{\mathrm{dbl}}\left(#1, #2\right)}
\newcommand{\dblPosVirtualJacobiSumScalar}[2]{\mathrm{j}_{+}^{\mathrm{dbl}}\left(#1, #2\right)}
\newcommand{\dblNegVirtualJacobiSumScalar}[2]{\mathrm{j}_{-}^{\mathrm{dbl}}\left(#1, #2\right)}
\newcommand{\posDblVirtualJacobiSumScalar}[2]{\mathrm{j}_{+}^{\mathrm{dbl}}\left(#1, #2\right)}
\newcommand{\dblGammaFactor}[3]{\Gamma^{\mathrm{dbl}}\left(#1 \times #2, #3\right)}
\newcommand{\dblGammaFactorSpace}[4]{\Gamma^{\mathrm{dbl}}_{#1}\left(#2 \times #3, #4\right)}
\newcommand{\dblLanglandsGammaFactorSpace}[4]{\gamma^{\mathrm{dbl}}_{#1}\left(#2 \times #3, #4\right)}
\newcommand{\GaussSumCharacter}[3]{\tau\left(#1 \times #2, #3\right)}
\newcommand{\ladicnumbers}{\algebraicClosure{\mathbb{Q}_{\ell}}}
\newcommand{\IsometryGroup}{\mathrm{Isom}}
\newcommand{\lieAlgebra}{\mathfrak{g}}
\newcommand{\DeligneLusztigInduction}[2]{\mathrm{R}_{#1}^{#2}}
\newcommand{\algebraicGroup}[1]{\boldsymbol{\mathrm{#1}}}
\newcommand{\LusztigSeries}[2]{\mathcal{E}\left(#1, (#2)\right)}
\newcommand{\characteristicPolynomial}{\operatorname{CharPoly}}
\newcommand{\DualFrobeniusFixedPoints}[2][\Frobenius^{\ast}]{\algebraicGroup{#2}^{\ast #1}}
\newcommand{\FrobeniusFixedPoints}[2][\Frobenius]{\algebraicGroup{#2}^{#1}}
\newcommand{\CharacterLattice}[1]{X^{\ast}\left(#1\right)}
\newcommand{\CocharacterLattice}[1]{X_{\ast}\left(#1\right)}
\newcommand{\SymmetricGroup}{\mathfrak{S}}


\newcommand{\calvin}[1]{\textcolor{orange}{\sffamily ((CALVIN: #1))}}
\newcommand{\elad}[1]{\textcolor{blue}{\sffamily ((ELAD: #1))}}

\hypersetup{pdfauthor={Calvin Yost-Wolff, Elad Zelingher},
	pdfsubject={Number theory, Representation theory},
	pdfkeywords={Kloosterman sums, Exponential sums}}

\title[Doubling method exponential sums]{On exponential sums arising from the classical doubling method}

\author{Calvin Yost-Wolff}
\address{Department of Mathematics, University of Michigan, 3084 East Hall, 530 Church Street, Ann Arbor, MI 48109-1043 USA}
\email{calvinyw@umich.edu}

\author{Elad Zelingher}
\address{Department of Mathematics, University of Michigan, 1844 East Hall, 530 Church Street, Ann Arbor, MI 48109-1043 USA}
\email{eladz@umich.edu}

\subjclass[2010]{20C33, 11L05, 11T24}

% 11L05    Gauss and Kloosterman sums; generalizations
% 11T24    Other character sums and Gauss sums
% 20C33    Representations of finite groups of Lie type

\begin{document}

\begin{abstract}
\end{abstract}
\maketitle

\tableofcontents

\section{Introduction}

Gauss sums are prominent objects in number theory...

Let $\finiteField$ be a finite field with $q$ elements and let $\fieldCharacter \colon \finiteField \to \multiplicativegroup{\cComplex}$ be a non-trivial character. Given a character $\chi \colon \multiplicativegroup{\finiteField} \to \multiplicativegroup{\cComplex}$ the normalized Gauss sum corresponding to the data $\left(\chi, \fieldCharacter\right)$ is defined by the formula
$$\tau\left(\chi, \fieldCharacter\right) = -\frac{1}{\sqrt{q}}\sum_{x \in \multiplicativegroup{\finiteField}} \chi\left(x\right) \fieldCharacter\left(x\right).$$

In the 1960s, Kondo \cite{Kondo1963} introduced an interesting exponential sum, which can be thought of as a higher dimensional version of the Gauss sum defined above. Let $\pi$ be an irreducible representation of $\GL_n\left(\finiteField\right)$. Then the sum $$\GaussSum{\pi}{\fieldCharacter} = q^{-\frac{n^2}{2}} \sum_{g \in \GL_n\left(\finiteField\right)} \fieldCharacter\left(\trace g\right) \pi\left(g\right)$$
defines an operator in $\Hom_{\GL_n\left(\finiteField\right)}\left(\pi, \pi\right)$. Since $\pi$ is irreducible, by Schur's lemma there exists a scalar $\GaussSumScalar{\pi}{\fieldCharacter} \in \cComplex$ such that $$\GaussSum{\pi}{\fieldCharacter} = \GaussSumScalar{\pi}{\fieldCharacter} \cdot \idmap_{\pi}.$$

In \cite{Kondo1963}, Kondo explicitly computed $\GaussSumScalar{\pi}{\fieldCharacter}$. His result shows that $\GaussSumScalar{\pi}{\fieldCharacter}$ can be expressed as a product of Gauss sums as above associated to the multiplicative characters that correspond to $\pi$ in the Deligne--Lusztig parameterization.

In \cite{Zelingher2024}, the second author utilized Kondo's result to explicitly express twisted matrix Kloosterman sums in terms of Hall--Littlewood polynomials evaluated at the eigenvalues of the Frobenius action on the corresponding twisted Kloosterman sheaf. As a result, he resolved a conjecture of \Erdelyi{}--\Toth{} \cite{ErdelyiToth2024}.

It is natural to ask whether the results of Kondo in \cite{Kondo1963} and the results of the second author in \cite{Zelingher2024} mentioned above have generalizations to other groups besides $\GL_n\left(\finiteField\right)$. An attempt to generalize Kondo's result was given by Saito--Shinoda in \cite{SaitoShinoda2000}. Given a classical group $G$, embedded naturally as a subgroup of a general linear group, and an irreducible representation $\pi$ of $G$, they considered the analogous Gauss sum
$$\sum_{g \in G} \pi\left(g\right) \fieldCharacter\left(\trace g\right).$$

As before, by Schur's lemma, this sum is scalar multiple of $\idmap_{\pi}$. However, the computations of Saito--Shinoda show that this scalar is not well behaved, even for the case $\Sp_{4}\left(\finiteField\right)$. The main issue with this naive Gauss sum is that it is \emph{not constant on Lusztig series}.

An alternative approach for defining an analog of Kondo's Gauss sum comes from the theory of automorphic representations. Not long after Kondo's result was published, Godement and Jacquet \cite{GodementJacquet1972} defined a theory for standard $L$-functions for $\GL_n$. Nowadays it is known that the finite field analog of the Godemenet--Jacquet gamma factor corresponding to an irreducible representation $\pi$ of $\GL_n\left(\finiteField\right)$ coincides with Kondo's Gauss sum of $\pi$, see \cite[Section 2]{Macdonald80}.

Since Godement--Jacquet developed their theory for the standard $L$-functions, many other constructions for $L$-functions attached to different representations have appeared in the literature. Unlike the construction of Godement--Jacquet, most of these constructions require the representations in question to satisfy certain genericity conditions. In the 1980s, Piatetski-Shapiro and Rallis \cite{PiatetskiShapiroRallis1986, GelbartPiatetskiShapiroRallis1987} discovered a new construction of (twisted) standard $L$-functions for classical groups. Their construction, known as ``the doubling method'', does not require the representations in question to satisfy any genericity condition, and it is similar to the construction of Godement--Jacquet in some sense. In fact, it is closely related to the construction of Godement--Jacquet, which shows up as certain inner integrals when considering parabolically induced representations of the classical group.

Thus, a promising way to obtain a well behaved exponential sum for classical groups is to consider the finite field analog of the ``doubling method'' construction of Piatetski-Shapiro--Rallis and analyzing the formula for the gamma factor that this theory yields. This was done for $\Sp_{2n}\left(\finiteField\right)$ by Jun Chang \cite{Chang1997} in her doctoral thesis. In the upcoming work of Girsch and the second author \cite{GirschZelingher2025}, the finite field theory analog of the doubling method is developed for all classical groups.

The computations of \cite{Chang1997} and \cite{GirschZelingher2025} show that, generically speaking, the finite field doubling method gamma factor for classical groups is given by exponential sums that resemble Jacobi sums. Let $G \subset \GL\left(\hermitianSpace\right)$ be a classical group, corresponding to the space $\left(\hermitianSpace, \innerproduct{\cdot}{\cdot}\right)$, and let $\lieAlgebra \subset \EndomorphismRing\left(\hermitianSpace\right)$ be the Lie algebra of $G$. Given an irreducible representation $\pi$ of $G$ and a character $\chi \colon Z\left(\GL\left(\hermitianSpace\right)\right) \to \multiplicativegroup{\cComplex}$ not trivial on the subgroup of scalars that preserve the form $\innerproduct{\cdot}{\cdot}$, the exponential sums arising from the finite field doubling method gamma factor of $\pi$ are given by the formula
$$\dblJacobiSum{\pi}{\chi} = \frac{1}{\sizeof{\lieAlgebra}^{1 \slash 2}} \sum_{\substack{g \in G\\
\det\left(g \pm \idmap_{\hermitianSpace}\right) \ne 0}} \pi\left(g\right) \chi\left(\det\left(\idmap_{\hermitianSpace} \pm g\right)\right).$$

Notice that $$\negDblJacobiSum{\pi}{\chi} = \frac{1}{\sizeof{\lieAlgebra}^{1/2}}\sum_{\substack{g \in G, h \in \GL\left(\hermitianSpace\right)\\
g + h = \idmap_{\hermitianSpace}}} \pi\left(g\right) \chi\left(\det h\right),$$
which resembles a classical Jacobi sum attached to characters $\chi_1, \chi_2 \colon \multiplicativegroup{\finiteField} \to \multiplicativegroup{\cComplex}$:
$$\sum_{\substack{x_1,x_2 \in \multiplicativegroup{\finiteField}\\
		x_1 + x_2 = 1}} \chi_1\left(x\right) \chi_2\left(x_2\right).$$

As before, we have that $\posDblJacobiSum{\pi}{\chi}$ and $\negDblJacobiSum{\pi}{\chi}$ lie in $\Hom_G\left(\pi, \pi\right)$. Since $\pi$ is irreducible, by Schur's lemma there exists scalars $\dblJacobiSumScalar{\pi}{\chi} \in \multiplicativegroup{\cComplex}$ such that $$\dblJacobiSum{\pi}{\chi} = \dblJacobiSumScalar{\pi}{\chi} \cdot \idmap_{\pi}.$$

\calvin{I think I would like to include a main theorem here even if it's not super precise just to let the reader know what we highlight what we about the sum}
\begin{theorem}[\Cref{thm:doubling-method-gamma-factor-for-deligne-lusztig}, \Cref{thm:computation-of-doubling-gauss-sum-scalar-for-deligne-lusztig-characters}]
	For a classical group $G$ and a multiplicative character $\chi$ as above, $\genHermitianJacobiKernel{\hermitianSpace}{\chi}$ are stable functions. We have that
	$$\posDblJacobiSumScalar{\pi}{\chi} =
		\varepsilon_{\algebraicGroup{H}} c_{\hermitianSpace}\left(\chi, \fieldCharacter\right) g_T\left(\chi, \theta, \fieldCharacter\right),$$
		and that
	$$\negDblJacobiSumScalar{\pi}{\chi} = \begin{dcases} 0 & G \text{ is an odd special orthogonal group},\\
		\centralCharacter{\pi}\left(-1\right) \posDblJacobiSumScalar{\pi}{\chi} &  \text{otherwise.}
	\end{dcases}$$
Here $c_{\hermitianSpace}\left(\chi, \fieldCharacter\right)$ is an explicit factor and
$g_T\left(\chi, \theta, \fieldCharacter\right)$ is a product of Gauss sums, see \Cref{subsec:normalization-factor} and \Cref{thm:computation-of-doubling-gauss-sum-scalar-for-deligne-lusztig-characters}.
\end{theorem}
In this paper, we study the doubling method Jacobi sums $\dblJacobiSum{\pi}{\chi}$. We compute the scalars $\dblJacobiSumScalar{\pi}{\chi}$ in terms of the Deligne--Lusztig data corresponding to $\pi$ and show that, up to some well understood factors, they match the Kondo Gauss sum of the conjectural functorial transfer of $\pi$ to a general linear group. This is done by combining a couple of ingredients. The first ingredient is a lemma of Saito--Shinoda \cite{SaitoShinoda2000} that allows us to explicitly compute the analogous character sums for Deligne--Lusztig virtual representations $R_{T, \theta}$.  The second ingredient, which is a key ingredient for this computation, is that these Jacobi sums are well behaved, in the sense that they are constant on Lusztig series. To show this, we utilize a multiplicativity property of these Jacobi sums from \cite{GirschZelingher2025} and combine it with a careful analysis of the Deligne--Lusztig data corresponding to $\pi$ after functorial transfer. 


\elad{Write about the Braverman--Kazhdan proposal that generalizes Godement--Jacquet to all groups and all representations, but is not explicit.}

\section{Preliminaries}
Let $\finiteField$ be a finite field with $q$ elements, where $2 \nmid q$, and let $\fieldCharacter \colon \finiteField \to \multiplicativegroup{\cComplex}$ be a non-trivial character. Fix an algebraic closure $\algebraicClosure{\finiteField}$ of $\finiteField$. For any $k \ge 1$, let $\finiteFieldExtension{k} \slash \finiteField$ be the unique field extension of degree $k$ contained in $\algebraicClosure{\finiteField}$. If $k' \mid k$, let $\FieldNorm{k}{k'} \colon \multiplicativegroup{\finiteFieldExtension{k}} \to \multiplicativegroup{\finiteFieldExtension{k'}}$ be the norm map.

Given a partition $\lambda = \left(\lambda_1, \dots, \lambda_\ell\right)$, we denote the \etale $\finiteField$-algebra $$\finiteFieldExtension{\lambda} = \prod_{i=1}^{\ell} \finiteFieldExtension{\lambda_i}$$ and define
$$\multiplicativegroup{\finiteFieldExtension{\lambda}} = \prod_{i=1}^{\ell} \multiplicativegroup{\finiteFieldExtension{\lambda_i}}.$$

\elad{Need to define $2 \lambda$ and $\FieldNorm{2\lambda}{\lambda}$ and $\NormOneGroup{2 \lambda}$}

\subsection{Representation theory of $\GL_n\left(\finiteField\right)$}

\subsection{Gauss sums}

Let $\tau$ be an irreducible representation of $\GL_n\left(\finiteField\right)$. 
For a character $\chi \colon \multiplicativegroup{\finiteField} \to \multiplicativegroup{\cComplex}$, denote the \emph{twisted Gauss sum}
$$\GaussSum{\tau \times \chi}{\fieldCharacter} = q^{-\frac{n^2}{2}} \sum_{g \in \GL_n\left(\finiteField\right)} \tau\left(g\right) \chi\left(\det g\right) \fieldCharacter\left(\trace g\right).$$
This element lies in $\Hom_{\GL_n\left(\finiteField\right)}\left(\tau, \tau\right)$ and therefore by Schur's lemma there exists a complex number $\GaussSumScalar{\tau \times \chi}{\fieldCharacter} \in \cComplex$ such that
$$\GaussSum{\tau \times \chi}{\fieldCharacter} = \GaussSumScalar{\tau \times \chi}{\fieldCharacter} \cdot \idmap_\tau.$$ The computation of this scalar is known due to Kondo \cite{Kondo1963}.

In order to state Kondo's result, we need to introduce some Gauss sum notation which will be used throughout the paper.

If $\alpha \colon \multiplicativegroup{\finiteFieldExtension{n}} \to \multiplicativegroup{\cComplex}$ is a character, denote the \emph{Gauss sum} $$\tau\left(\alpha, \fieldCharacter_n\right) = -q^{-\frac{n}{2}}\sum_{x \in \multiplicativegroup{\finiteFieldExtension{n}}} \alpha\left(x\right) \fieldCharacter_n\left(x\right),$$where $\fieldCharacter_n \colon \finiteFieldExtension{n} \to \multiplicativegroup{\cComplex}$ is given by $\fieldCharacter_n = \fieldCharacter \circ \trace_{\finiteFieldExtension{n} \slash \finiteField}$. If $\chi \colon \multiplicativegroup{\finiteField} \to \multiplicativegroup{\cComplex}$ is a character and if $\alpha$ is as above, denote the \emph{twisted Gauss sum}
$$\tau\left(\alpha \times \chi, \fieldCharacter_n\right) = -q^{-\frac{n}{2}}\sum_{x \in \multiplicativegroup{\finiteFieldExtension{n}}} \alpha\left(x\right) \chi\left( \FieldNorm{n}{1}\left(x\right)\right) \fieldCharacter_n\left(x\right).$$

We are now ready to state Kondo's result.
%test
\begin{theorem}[Kondo {\cite{Kondo1963}}]
	Suppose that $\tau$ is an irreducible representation of $\GL_n\left(\finiteField\right)$ with cuspidal support $\left\{\tau_1,\dots, \tau_r\right\}$, where $n_1 + \dots + n_r = n$, and where for every $j$, $\tau_j$ is an irreducible cuspidal representation of $\GL_{n_j}\left(\finiteField\right)$ corresponding to the Frobenius orbit of a regular character $\theta_j \colon \multiplicativegroup{\finiteFieldExtension{n_j}} \to \multiplicativegroup{\cComplex}$. Then
	$$\GaussSumScalar{\tau \times \chi}{\fieldCharacter} = \left(-1\right)^n \cdot \prod_{j=1}^r \GaussSumCharacter{\theta_j}{\chi}{\fieldCharacter_{n_j}}.$$
\end{theorem}

We will need the following lemma that allows us to determine whether a degenerate Kondo Gauss sum vanishes.
\begin{lemma}\label{lem:sum-vanishes-for-singular-matrices}
	For any singular matrix $X \in \squareMatrix_n\left(\finiteField\right)$ and any $1 \ne \chi \colon \multiplicativegroup{\finiteField} \to \multiplicativegroup{\cComplex}$, the sum
	$$\sum_{h \in \GL_n\left(\finiteField\right)} \chi\left(\det h\right) \fieldCharacter\left(\trace\left(Xh\right)\right)$$
	is zero.
\end{lemma}
\begin{proof}
	Write $$X = h_1 \begin{pmatrix}
		\IdentityMatrix{n-r}\\
		& 0_r
	\end{pmatrix} h_2,$$
	where $1 \le r \le n$ and $h_1, h_2 \in \GL_n\left(\finiteField\right)$. Then
	$$\fieldCharacter\left(\trace Xh\right) = \fieldCharacter\left(\trace\left( \begin{pmatrix}
		\IdentityMatrix{n-r}\\
		& 0_r
	\end{pmatrix} h_2 h h_1\right)\right).$$
	Changing variables $h \mapsto h_2^{-1} h h_1^{-1},$
	we obtain the inner sum
	$$\sum_{h \in \GL_n\left(\finiteField\right)} \fieldCharacter\left(\trace \begin{pmatrix}
		\IdentityMatrix{n-r}\\
		& 0_r
	\end{pmatrix} h \right) \chi\left(\det h\right),$$
	which equals
	$$\frac{1}{\sizeof{\multiplicativegroup{\finiteField}}} \sum_{a \in \multiplicativegroup{\finiteField}} \sum_{h \in \GL_n\left(\finiteField\right)} \fieldCharacter\left(\trace \begin{pmatrix}
		\IdentityMatrix{n-r}\\
		& 0_r
	\end{pmatrix} \begin{pmatrix}
		\IdentityMatrix{n-1}\\
		& a
	\end{pmatrix} h \right) \chi\left(\det h\right).$$
	Changing variables $h \mapsto \left(\begin{smallmatrix}
		\IdentityMatrix{n-1}\\
		& a^{-1}
	\end{smallmatrix}\right) h$, we get the inner sum $$\sum_{a \in \multiplicativegroup{\finiteField}} \chi^{-1}\left(a\right),$$
	which vanishes because $\chi \ne 1$.
\end{proof}

\subsection{Classical groups}

In this section, we recall the notion of classical groups and their corresponding similitude groups. These are the main groups which we will study in this paper.

\subsubsection{$\epsilon$-sesquilinear spaces}
Let $\quadraticExtension \slash \finiteField$ be a field extension of degree $1$ or $2$ contained in $\algebraicClosure{\finiteField}$. For any $k \ge 1$, let $\quadraticFieldExtension{k} \slash \quadraticExtension$ be the unique field extension of degree $k$ contained in $\algebraicClosure{\finiteField}$. Let $x \mapsto \involution{x}$ be the generator of $\Galois\left(\quadraticExtension \slash \finiteField\right)$.

Let $\hermitianSpace$ be a vector space of dimension $n$ over $\quadraticExtension$, equipped with a non-degenerate sesquilinear form $\innerproduct{\cdot}{\cdot} \colon \hermitianSpace \times \hermitianSpace \to \quadraticExtension$ which is $\epsilon_{\hermitianSpace}$-symmetric for $\epsilon_{\hermitianSpace} \in \left\{\pm 1\right\}$. By this we mean that:
\begin{enumerate}
	\item For every $x_1,x_2,y \in \hermitianSpace$, $$\innerproduct{x_1 + x_2}{y} = \innerproduct{x_1}{y} + \innerproduct{x_2}{y}.$$
	\item For every $x,y \in \hermitianSpace$ and $t \in \quadraticExtension$, $$\innerproduct{tx}{y} = t\innerproduct{x}{y}.$$
	\item ($\epsilon_{\hermitianSpace}$-symmetric) For every $x,y \in \hermitianSpace$, $$\innerproduct{x}{y} = \epsilon_{\hermitianSpace} \involution{\innerproduct{y}{x}}.$$
	\item (non-degenerate) For every $0 \ne x \in \hermitianSpace$ there exists $y \in \hermitianSpace$ such that $$\innerproduct{x}{y} \ne 0.$$
\end{enumerate}

\subsubsection{Isometry groups}\label{subsec:isometry-groups}
Let $\IsometryGroup \left(\hermitianSpace\right)$ be the isometry group of $\hermitianSpace$, consisting of all the elements of $\GL_{\quadraticExtension}\left(\hermitianSpace\right)$ that satisfy $\innerproduct{gx}{gy} = \innerproduct{x}{y}$ for every $x,y \in \hermitianSpace$. We denote by $G$ the identity component of $\IsometryGroup\left(\hermitianSpace\right)$. For the purposes of later sections, it would be useful to realize $G$ as the $\finiteField$ points of an algebraic group $\algebraicGroup{G}$. We explain how the possible options for $G$ and the corresponding group $\algebraicGroup{G}$ in the following list.

Let $$w_n = \begin{pmatrix}
	& & & 1\\
	& & 1\\
	& \iddots\\
	1
\end{pmatrix} \in \GL_n\left(\finiteField\right)$$ be the longest Weyl element.
\calvin{we should standardize an order between Unitary, Special Orthogonal Symplectic and try to kinda stick with it throughout.}
\begin{enumerate}
	\item (Special orthogonal groups): $\quadraticExtension = \finiteField$ and $\epsilon_{\hermitianSpace} = 1$. It follows by definition that every element $g \in \IsometryGroup\left(\hermitianSpace\right)$ satisfies $\left(\det g\right)^2 = 1$.  In this case, we obtain the connected components of the identity by adding a restriction on the determinant: We have that $G$ consists of all elements $g \in \IsometryGroup\left(\hermitianSpace\right)$ with $\det g = 1$. There are two cases here.
	\begin{enumerate}
		\item Split case: in this case, we can write $$\hermitianSpace = \begin{dcases}
		\xIsotropic \oplus \yIsotropic & \dim_{\finiteField} \hermitianSpace = 2n\\
		\xIsotropic \oplus \finiteField v \oplus \yIsotropic & \dim_{\finiteField}\hermitianSpace = 2n + 1
		\end{dcases}$$ where $\xIsotropic$ and $\yIsotropic$ are totally isotropic (that is, for every $x \in \xIsotropic$, $\innerproduct{x}{x} = 0$ and similarly for $\yIsotropic$) and these spaces are in duality with respect to $\innerproduct{\cdot}{\cdot}$, i.e., the map $\yIsotropic \to \Hom_{\finiteField}\left(\xIsotropic, \finiteField\right)$ given by $y \mapsto \left(x \mapsto \innerproduct{x}{y}\right)$ is an isomorphism. In the odd dimensional case, $v \in \hermitianSpace$ is an element orthogonal to $\xIsotropic$ and $\yIsotropic$ that satisfies $\innerproduct{v}{v} \ne 0$.
		
		In either case, denote $N = \dim_{\finiteField} \hermitianSpace$. We have that $G$ is isomorphic to $\SO^{+}_{N}\left(\finiteField\right) \subset  \GL_{N}\left(\finiteField\right)$, defined as the $\finiteField$-points of $$\algebraicGroup{\SO}^{+}_{N} = \left\{g \in \algebraicGroup{\GL}_{N} \mid \transpose{g} w_{N} g  = w_{N},\,\, \det g = 1\right\}.$$
		
		In the odd-dimensional case $N = 2n+1$, we are always in this situation, and we simply denote $\SO_{2n+1}\left(\finiteField\right) = \SO_{2n+1}^{+}\left(\finiteField\right)$ and $\algebraicGroup{\SO}_{2n+1} = \algebraicGroup{\SO}_{2n+1}^{+}$.
		\item Non-split case: In this case, $\dim_{\finiteField} \hermitianSpace = 2n$ and there exists a decomposition $$\hermitianSpace = \xIsotropic \oplus \hermitianSpace' \oplus \yIsotropic,$$
		where $\xIsotropic$ and $\yIsotropic$ are totally isotropic spaces in duality and $\hermitianSpace'$ is an antisotropic space (that is, for any $0 \ne v' \in \hermitianSpace'$, we have $\innerproduct{v'}{v'} \ne 0$) of dimension 2. Let $d$ be an element of $\multiplicativegroup{\finiteField}$ that is not a square. Denote $$B_{n,d} = \begin{pmatrix}
			\IdentityMatrix{n-1}\\
			& & 1 &\\
			& -d & &\\
			& & & \IdentityMatrix{n-1}
		\end{pmatrix} \cdot w_{2n}.$$
		In this case, we have that $G$ is isomorphic to $\SO^{-}_{2n}\left(\finiteField\right) \subset  \GL_{2n}\left(\finiteField\right)$, defined as the $\finiteField$-points of $$\algebraicGroup{\SO}^{-}_{2n} = \left\{g \in \algebraicGroup{\GL}_{2n} \mid \transpose{g} B_{n,d} g = B_{n,d},\,\, \det g = 1\right\}.$$
	\end{enumerate}
	\item (Symplectic groups): $\quadraticExtension = \finiteField$ and $\epsilon_{\hermitianSpace} = -1$. In this case, $\dim_{\finiteField} \hermitianSpace = 2n$ for some positive integer $n$ and $G$ is isomorphic to the group $\Sp_{2n}\left(\finiteField\right) \subset  \GL_{2n}\left(\finiteField\right)$, defined as the $\finiteField$-points of $$\algebraicGroup{\Sp}_{2n} = \left\{g \in \algebraicGroup{\GL}_{2n} \mid \transpose{g} \begin{pmatrix}
		& w_n\\
		-w_n
	\end{pmatrix} g = \begin{pmatrix}
		& w_n\\
		-w_n
	\end{pmatrix}\right\}.$$
	\item (Unitary groups): $\quadraticExtension \ne \finiteField$. Denote $\dim_{\quadraticExtension} \hermitianSpace = n$. In this case, the group $G = \GroupExtension{G}$ is isomorphic to the group $\UnitaryGroup_n\left(\finiteField\right) \subset\GL_{n}\left(\quadraticExtension\right)$ defined as the $\finiteField$-points of $$\algebraicGroup{\UnitaryGroup}_n = \left\{ g \in \restrictionOfScalars{\quadraticExtension}{\finiteField}{\algebraicGroup{\GL}_n} \mid \involution{\left(\transpose{g}\right)} w_n g = w_n \right\}.$$
\end{enumerate}

Let $\lieAlgebra$ be the Lie algebra of $G$, consisting of all elements $A \in \EndomorphismRing_{\quadraticExtension}\left(\hermitianSpace\right)$ satisfying $\innerproduct{Ax}{y} + \involution{\innerproduct{x}{AY}} = 0$ for every $x, y \in \hermitianSpace$.

\subsubsection{The group $\tilde{G}$ and similitude groups}\label{subsec:similitute-groups}
In this section, we will introduce a group $\tilde{\algebraicGroup{G}}$ containing $\algebraicGroup{G}$ that has connected center. This will be important for later, as the Deligne--Lusztig theory  is easier  for $\tilde{\algebraicGroup{G}}$.

In the special odd orthogonal group case (that is, $\epsilon_{\hermitianSpace} = 1$, $\quadraticExtension = \finiteField$ and $\dim_{\finiteField} \hermitianSpace$ is odd) and in the unitary case (that is, $\quadraticExtension \ne \finiteField$), we have that $\algebraicGroup{G}$ has connected center and we denote $\tilde{\algebraicGroup{G}} = \algebraicGroup{G}$.

Let us assume that $\quadraticExtension = \finiteField$ and that $\dim_{\finiteField} \hermitianSpace$ is even. In this case, $\algebraicGroup{G}$ does not have a connected center. We extend $\algebraicGroup{G}$ to a group $\tilde{\algebraicGroup{G}}$ with connected center by considering the similitude group corresponding to $\hermitianSpace$.

First, let $\GroupExtension{\IsometryGroup\left(\hermitianSpace\right)}$ be the similitude group version of $\IsometryGroup\left(\hermitianSpace\right)$, consisting of the elements $g \in \GL_{\quadraticExtension}\left(\hermitianSpace\right)$ with the property that there exists a constant $\similitudeCharacter\left(g\right) \in \multiplicativegroup{\quadraticExtension}$ such that for every $\innerproduct{gx}{gy} = \similitudeCharacter\left(g\right) \innerproduct{x}{y}$ for every $x,y \in \hermitianSpace$. Let $\GroupExtension{G}$ be the identity component of $\GroupExtension{\IsometryGroup\left(\hermitianSpace\right)}$. The group $\GroupExtension{G}$ has connected center and contains $G$.

We write a more explicit description of $\tilde{G}$ and $\tilde{\algebraicGroup{G}}$.
\begin{enumerate}
	\item (Even special orthogonal groups): Denote $\dim_{\finiteField} \hermitianSpace = 2n$. Every element $g \in \widetilde{\IsometryGroup\left(\hermitianSpace\right)}$ satisfies $\left(\det g\right)^2 = \similitudeCharacter\left(g\right)^{2n}$. Similarly to before, we obtain the connected components of the identity by adding a restriction on the determinant. The group $\GroupExtension{G}$ consists of all elements $g \in \GroupExtension{\IsometryGroup\left(\hermitianSpace\right)}$ such that $\det g = \similitudeCharacter\left(g\right)^n$. We have the two cases from before here:
	\begin{enumerate}
		\item (Split case): The group $\GroupExtension{G}$ is isomorphic to the group $\GSO^{+}_{2n}\left(\finiteField\right) \subset \GL_{2n}\left(\finiteField\right)$, defined as the projection to the first coordinate of the $\finiteField$-points of $$\algebraicGroup{\GSO}^{+}_{2n} = \left\{\left(g, \lambda\right) \in \algebraicGroup{\GL}_{2n} \times \multiplcativeScheme \mid \transpose{g} w_{2n} g = \lambda w_{2n},\,\, \det g = \lambda^n\right\}.$$
		\item (Non-split case): The group $\GroupExtension{G}$ is isomorphic to the group $\GSO^{-}_{2n}\left(\finiteField\right) \subset \GL_{2n}\left(\finiteField\right)$, defined as the projection to the first coordinate of the $\finiteField$-points of $$\algebraicGroup{\GSO}^{-}_{2n} = \left\{\left(g, \lambda\right) \in \algebraicGroup{\GL}_{2n} \times \multiplcativeScheme \mid \transpose{g} B_{n,d} g = \lambda B_{n,d},\,\, \det g = \lambda^n\right\}.$$
	\end{enumerate}
	\item The group $\GroupExtension{G}$ is isomorphic to the group $\GSp_{2n}\left(\finiteField\right) \subset \GL_{2n}\left(\finiteField\right)$, defined as the projection to the first coordinate of the $\finiteField$-points of $$\algebraicGroup{\GSp}_{2n} = \left\{\left(g, \lambda\right) \in \algebraicGroup{\GL}_{2n} \times \multiplcativeScheme \mid \transpose{g} \begin{pmatrix}
		& w_n\\
		-w_n
	\end{pmatrix} g = \lambda \begin{pmatrix}
		& w_n\\
		-w_n
	\end{pmatrix}\right\}.$$
\end{enumerate}


\subsubsection{Rational Tori}
A \emph{rational torus} $\algebraicGroup{T} \subset \algebraicGroup{G}$ over $\finiteField$ is a torus defined over $\finiteField$. 
Over a finite field, $\algebraicGroup{G}(\finiteField)$ conjugacy classes of maximal rank rational tori \calvin{check if we need connected or something} are parameterized by  Frobenius order twisted conjugacy classes in the Weyl group. In particular if we fix a rational torus $\algebraicGroup{T}_0$ with rank $\mathrm{rk} \algebraicGroup{G}$ and an element $g \in \algebraicGroup{G}(\algebraicClosure{\finiteField})$ such that $g^{-1}\Frobenius(g) \in N(\algebraicGroup{T}_0)$ then the torus $g\algebraicGroup{T}_0g^{-1}$ is rational, and its $\algebraicGroup{G}(\finiteField)$-conjugacy class is determined  by the Frobeinus-twisted conjugacy class of the image of the Lang map $L \colon \algebraicGroup{G}\left(\algebraicClosure{\finiteField}\right) \to \algebraicGroup{G}\left(\algebraicClosure{\finiteField}\right)$ defined as $L\left(g\right) = g^{-1}\Frobenius(g)$ in $W(\algebraicGroup{T}_0)(\algebraicClosure{\mathbb{F}})$, see \cite[Section 3.3]{Carter1985}.

%since $\Frobenius(g\algebraicGroup{T}_0g^{-1}) = g\algebraicGroup{T}_0g^{-1}$
%could really demonstrate this by stating Land map is surjective + that all maximal tori are conjugate over an algebraiclly closed field

%since $\Frobenius(g\algebraicGroup{T}_0g^{-1}) = g\algebraicGroup{T}_0g^{-1}$
%could really demonstrate this by stating Land map is surjective + that all maximal tori are conjugate over an algebraiclly closed field

Let us denote $$\left(\quadraticExtension, x \mapsto \involution{x}\right) = \begin{dcases}
	\left(\finiteFieldExtension{2}, x \mapsto x^q\right)  & \algebraicGroup{G} \text{ is a unitary group},\\
	\left(\finiteField, x \mapsto x\right) & \text{otherwise.}
\end{dcases}$$

Let $\algebraicGroup{A}_n \subset \algebraicGroup{\GL}_n$ be the diagonal torus. We denote for an element $g \in \algebraicGroup{\GL}_n$, $g^{\ast} = w_n \transpose{g}^{-1} w_n$, where $w_n$ is the long Weyl element (see \Cref{subsec:isometry-groups}). Using $\algebraicGroup{A}_n$ we define standard (mostly split) tori in our groups as follows.
\begin{center}
\begin{tabular}{|c|c|}
	\hline 
		$\algebraicGroup{G}$ & $\algebraicGroup{T}_0$ \tabularnewline
		\hline 
		\hline 
		$\algebraicGroup{\UnitaryGroup}_{2n}$ & $\left\{ \begin{pmatrix}
			a\\
			& \involution{\left(a^{\ast}\right)}
		\end{pmatrix}  \mid a \in \restrictionOfScalars{\quadraticExtension}{\finiteField}{\algebraicGroup{A}_n} \right\}$ \tabularnewline
		\hline 
		$\algebraicGroup{\UnitaryGroup}_{2n+1}$ & $\left\{ \begin{pmatrix}
	a \\
	& t\\
	& & \involution{\left(a^{\ast}\right)}
\end{pmatrix}  \mid a \in \restrictionOfScalars{\quadraticExtension}{\finiteField}{\algebraicGroup{A}_n},\,\, t \in \algebraicGroup{U}_1 \right\}$ \tabularnewline
\hline 
	$\algebraicGroup{\SO}_{2n+1}$ & $\left\{ \begin{pmatrix}
		a\\
		& 1\\
		& & a^{\ast}
	\end{pmatrix} \mid a \in \algebraicGroup{A}_n \right\}$ \tabularnewline
	\hline 
	$\algebraicGroup{\Sp}_{2n}$ or $\algebraicGroup{\SO}^{+}_{2n}$ & $\left\{ \begin{pmatrix}
		a\\
		& a^{\ast}
	\end{pmatrix} \mid a \in \algebraicGroup{A}_n \right\}$ \tabularnewline
	\hline 
	$\algebraicGroup{\SO}_{2n}^{-}$ & $\left\{ \begin{pmatrix}
		a\\
		& h\\
		& & a^{\ast}
	\end{pmatrix} \mid a \in \algebraicGroup{A}_{n-1}\,\, h \in \algebraicGroup{\SO}_2^{-} \right\}$ \tabularnewline
	\hline		
	\end{tabular}
\end{center}
Here, $\algebraicGroup{T}_{0}$ is split for all rows except for the last row. We also write down the standard tori for the relevant corresponding similitude groups.
\begin{center}
	\begin{tabular}{|c|c|}
		\hline 
		$\algebraicGroup{G}$ & $\algebraicGroup{T}_0$ \tabularnewline
		\hline 
		\hline 
		$\algebraicGroup{\GSp}_{2n}$ or $\algebraicGroup{\GSO}^{+}_{2n}$ & $\left\{\left(\begin{pmatrix}
			\lambda a\\
			& a^{\ast}
		\end{pmatrix}, \lambda\right) \mid a \in \algebraicGroup{A}_n,\,\, \lambda \in \multiplcativeScheme \right\}$ \tabularnewline
		\hline 
		$\algebraicGroup{\GSO}_{2n}^{-}$ & $\left\{ \left(\begin{pmatrix}
			\algebraicGroup{\aFieldNorm}_{\finiteFieldExtension{2} \slash \finiteField}\left(h\right) a\\
			& h\\
			& & a^{\ast}
		\end{pmatrix}, \algebraicGroup{\aFieldNorm}_{\finiteFieldExtension{2} \slash \finiteField}\left(h\right)\right) \mid a \in \algebraicGroup{A}_{n-1}\,\, h \in \restrictionOfScalars{\finiteFieldExtension{2}}{\finiteField}{\multiplcativeScheme} \right\}$ \tabularnewline
		\hline		
	\end{tabular}
\end{center}
Here we realize $\restrictionOfScalars{\finiteFieldExtension{2}}{\finiteField}{\multiplcativeScheme}$ as $$\restrictionOfScalars{\finiteFieldExtension{2}}{\finiteField}{\multiplcativeScheme} = \left\{ \left(\begin{pmatrix}
	a & b\\
	db & a
\end{pmatrix}, \lambda\right) \mid a^2 - db^2 = \lambda \in \multiplcativeScheme \right\},$$
where $d$ is as in \Cref{subsec:isometry-groups}.

In the unitary group case, the action of $\Frobenius$ on the Weyl group $W\left(\algebraicGroup{T}_{0}\right)$ is conjugation by $w_n$. In the $\algebraicGroup{\SO}^-_{2n}$ case, the action of $\Frobenius$ on the Weyl group $W\left(\algebraicGroup{T}_{0}\right)$ is the outer automorphism on the Dynkin diagram of type $D_n$. In our matrix representation, $\Frobenius$ acts on the Weyl group by conjugation by the matrix $$\left(-n,n\right) \coloneq \begin{pmatrix}\IdentityMatrix{n-1} & 0 & 0 & 0\\
	0 & 1 & 0 & 0\\
	0 & 0 & -1 & 0\\
	0 & 0 & 0 & \IdentityMatrix{n-1}
\end{pmatrix}.$$ In all other cases, the action of $\Frobenius$ on $W(\algebraicGroup{T}_0)$ is trivial. We now parameterize the Weyl group $\Frobenius$-twisted conjugacy classes and their corresponding tori.

\begin{enumerate}
    \item Unitary groups: The Weyl group $W\left(\algebraicGroup{T}_{0}\right)$ is isomorphic to the symmetric group $\SymmetricGroup_n$. If $\dot{w} \in \algebraicGroup{\UnitaryGroup}_n$ is a lift of $w \in W\left(\algebraicGroup{T}_{0}\right)$, the map $w \mapsto \dot{w} \cdot w_n$ sends a $w_n$-twisted $\SymmetricGroup_n$ conjugacy class to a $\SymmetricGroup_n$ conjugacy class. Thus twisted conjugacy classes are determined by the cycle partition $\lambda \vdash n$ associated to $\dot{w} \cdot w_n$. Let us denote by $\lambda^+$ the partition formed by taking the even parts of $\lambda$ and by $\lambda^{-}$ the partition formed by taking the odd parts of $\lambda$. The corresponding algebraic torus is
	$$^{L^{-1}w}\algebraicGroup{T}_0 \cong \restrictionOfScalars{\finiteFieldExtension{\lambda^{+}}}{\finiteField}{\multiplcativeScheme} \times \restrictionOfScalars{\finiteFieldExtension{\lambda^{-}}}{\finiteField}{\algebraicGroup{\UnitaryGroup}_1}.$$
	% The embedding into the unitary group requires conjugation by \diag(1,1,1,\sqrt{d},\sqrt{d},\sqrt{d}). I would rather not write it.
	
	%The embedding into $\algebraicGroup{\UnitaryGroup}_n$ is given by the map %\begin{equation}\label{eq:torus-embedding-unitary-group}
	%	\left( \left(g_i\right)_{i=1}^{\lengthof\left(\lambda^+\right)}, \left(h_i\right)_{i=1}^{\lengthof\left(\lambda^-\right)}\right) \mapsto \diag(g_1,\dots, g_{\lengthof\left(\lambda^+\right)}, h_1^{\ast}, \dots, h_{\lengthof\left(\lambda^{-}\right)}^{\ast}, \involution{(g^{\ast}_{\lengthof\left(\lambda^+\right)})}, \dots, \involution{(g^{\ast}_1)}).
	%\end{equation}
	See also \cite[Section 2.3]{ThiemVinroot2009} and \cite[Section 2.2]{SaitoShinoda2000}.
    \item (Odd special orthogonal and symplectic groups)
    For $\algebraicGroup{\Sp}_{2n}$ or $\algebraicGroup{\SO}_{2n+1}$, the Weyl group is isomorphic to $$\left\{\pm 1\right\}^n \rtimes \SymmetricGroup_n,$$ which we can identify with the group signed permutations. The conjugacy classes of the Weyl group are classified by pairs of partitions $(\lambda^+,\lambda^-)$ with $\sizeof{\lambda^+} + \sizeof{\lambda^-} = n$. See \cite[Section 2.2]{KonvalinkaMatjavPfeiffer2011} for more details. Let $(\lambda^+,\lambda^-)$ correspond to the conjugacy class of $w$ in the Weyl group. Then
	$$^{L^{-1}w}\algebraicGroup{T}_0 \cong \restrictionOfScalars{\finiteFieldExtension{\lambda^{+}}}{\finiteField}{\multiplcativeScheme} \times \restrictionOfScalars{\finiteFieldExtension{\lambda^{-}}}{\finiteField}{\algebraicGroup{\UnitaryGroup}_1}.$$
		See also \cite[Section 3.2 Part (B)]{Zalesski2018}.
        \item (Even special orthogonal groups) The Weyl group of $\algebraicGroup{SO}_{2n}^{\pm}$ is isomorphic to $$\left\{ \left(x_1,\dots,x_n\right) \in \left\{\pm 1\right\}^n \mid \prod_{i=1}^n x_i = 1\right\} \rtimes \SymmetricGroup_n,$$ which can be realized the subgroup of the group of signed permutations, consisting of all even signed permutations. Using this realization, we may attach to $w$ in the Weyl group a pair of partitions $\left(\lambda^+, \lambda^-\right)$ as before, with $\sizeof{\lambda^+} + \sizeof{\lambda^-} = n$. If $\lambda^{-}$ is not the empty partition, the pair $\left(\lambda^+, \lambda^-\right)$ determines the conjugacy class of $w$. Otherwise, there are two different conjugacy classes of the Weyl group corresponding to the same pair $\left(\lambda^{+}, \lambda^{-}\right)$. These two conjugacy class correspond to isomorphic tori, and these tori are not conjugate in $\algebraicGroup{\SO}_{2n}^{\pm}$. See \cite[Section 2.3]{KonvalinkaMatjavPfeiffer2011} for more details.
        \begin{enumerate}
        	\item (Split case): For $\algebraicGroup{\SO}_{2n}^+$ we have that
        	$$^{L^{-1}w}\algebraicGroup{T}_0 \cong \restrictionOfScalars{\finiteFieldExtension{\lambda^{+}}}{\finiteField}{\multiplcativeScheme} \times \restrictionOfScalars{\finiteFieldExtension{\lambda^{-}}}{\finiteField}{\algebraicGroup{\UnitaryGroup}_1}.$$
        	See also \cite[Section 3.2 Parts (C) and (D)]{Zalesski2018}.        	
        	\item (Non-split case): As in the unitary group case, for $\algebraicGroup{SO}_{2n}^-$, the map $w \mapsto \dot{w}\circ (-n,n)$ sends a $(-n,n)$-twisted conjugacy class in the even signed permutation group to a conjugacy class in the signed permutation group. Let $(\lambda^+,\lambda^-)$ correspond to the signed permutation group conjugacy class of $\dot{w} \circ (-n,n)$. Then
        	$$^{L^{-1}w}\algebraicGroup{T}_0 \cong \restrictionOfScalars{\finiteFieldExtension{\lambda^{+}}}{\finiteField}{\multiplcativeScheme} \times \restrictionOfScalars{\finiteFieldExtension{\lambda^{-}}}{\finiteField}{\algebraicGroup{\UnitaryGroup}_1}.$$
			See also \cite[Section 3.2 Part (D)]{Zalesski2018}. Notice that we differ from that \cite{Zalesski2018} since we do not incorporate the element $\left(-n, n\right)$ in the Frobenius root.
        \end{enumerate}
    \end{enumerate}
	The following theorem summarizes these cases.
	
	\begin{theorem}
		A maximal rational torus $\algebraicGroup{T}$ of $\algebraicGroup{G} = \algebraicGroup{\UnitaryGroup}_{n}, \algebraicGroup{\Sp}_{2n}, \algebraicGroup{\SO}_{2n + 1}, \algebraicGroup{\SO}_{2n}^{\pm}$ is isomorphic to 	$$\algebraicGroup{T} \cong \restrictionOfScalars{\finiteFieldExtension{\lambda^{+}}}{\finiteField}{\multiplcativeScheme} \times \restrictionOfScalars{\finiteFieldExtension{\lambda^{-}}}{\finiteField}{\algebraicGroup{\UnitaryGroup}_1},$$
		where $\left(\lambda^+, \lambda^-\right)$ is a pair of partitions such that $\sizeof{\lambda^+} + \sizeof{\lambda^-} = n$, where in the unitary group case we require that $\lambda^+$ consists only of even parts and that $\lambda^-$ consists only of odd parts. We have that $$\algebraicGroup{T}^{\Frobenius} \cong \multiplicativegroup{\finiteFieldExtension{\lambda^+}} \times \NormOneGroup{2 \lambda^{-}}.$$ The embedding $\algebraicGroup{T}^{\Frobenius} \to \algebraicGroup{G}^{\Frobenius}$ is of the form $$\left(x, y\right) \mapsto
		\begin{pmatrix}
			\iota_+\left(x\right)\\
			& \iota_{-}\left(y\right) &\\
			 & & \involution{\iota_+^{\ast}\left(x\right)}
		\end{pmatrix},$$
		where $\iota_+ \colon \multiplicativegroup{\finiteFieldExtension{\lambda^+}} \to \GL_{\frac{\sizeof{\lambda^+}}{\grpIndex{\quadraticExtension}{\finiteField}}}\left(\quadraticExtension\right)$ and $\iota_- \colon \NormOneGroup{2 \lambda^{-}} \to G'$ are embeddings. Here, $G'$ is a group of the same type as $\algebraicGroup{G}^{\Frobenius}$, with $n$ replaced by $\sizeof{\lambda^{-}}$.
		
		In the odd special orthogonal case, the embedding $\iota_{-}$ can be chosen to fix the middle standard vector $e_{n+1}$.
	\end{theorem}
    
    
    In addition, we write the tori classification for the similitude groups we consider.
	\begin{enumerate}
		\item For $\algebraicGroup{\GSp}_{2n}$ or $\algebraicGroup{\GSO}_{2n}^+$, suppose that $(\lambda^+,\lambda^-)$ corresponds to the conjugacy class of $w$ as an element of the group of signed permutations. Then
			\begin{align*}
				^{L^{-1}w}\algebraicGroup{T}_0 \cong& \left\{\left(x,y,z\right) \in  \restrictionOfScalars{\finiteFieldExtension{\lambda^+}}{\finiteField}{\multiplcativeScheme} \times \restrictionOfScalars{\finiteFieldExtension{2\lambda^-}}{\finiteField}{\multiplcativeScheme}  \times \multiplcativeScheme \mid \FieldNorm{2\lambda^-}{\lambda^-}(y) = z \right\}.
			\end{align*}
			\item For $\algebraicGroup{\GSO}_{2n}^-$, let $(\lambda^+,\lambda^-)$ correspond to the conjugacy class of $w\circ (-n,n)$ viewed as an element of the group of signed permutations. Then
			\begin{align*}
				^{L^{-1}w}\algebraicGroup{T}_0 \cong& \left\{\left(x,y,z\right) \in  \restrictionOfScalars{\finiteFieldExtension{\lambda^+}}{\finiteField}{\multiplcativeScheme} \times \restrictionOfScalars{\finiteFieldExtension{2\lambda^-}}{\finiteField}{\multiplcativeScheme}  \times \multiplcativeScheme \mid \FieldNorm{2\lambda^-}{\lambda^-}(y) = z \right\}.
			\end{align*}
	\end{enumerate}

	The following theorem summarizes these cases.
	\begin{theorem}
	A maximal rational torus $\algebraicGroup{T}$ of $\algebraicGroup{G} =  \algebraicGroup{\GSp}_{2n}, \algebraicGroup{\GSO}_{2n}^{\pm}$ is isomorphic to \begin{align*}
				^{L^{-1}w}\algebraicGroup{T}_0 \cong& \left\{\left(x,y,z\right) \in  \restrictionOfScalars{\finiteFieldExtension{\lambda^+}}{\finiteField}{\multiplcativeScheme} \times \restrictionOfScalars{\finiteFieldExtension{2\lambda^-}}{\finiteField}{\multiplcativeScheme}  \times \multiplcativeScheme \mid \FieldNorm{2\lambda^-}{\lambda^-}(y) = z \right\}.
	\end{align*}
	where $\left(\lambda^+, \lambda^-\right)$ is a pair of partitions such that $\sizeof{\lambda^+} + \sizeof{\lambda^-} = n$. We have that $$\algebraicGroup{T}^{\Frobenius} \cong \left\{\left(x, y, z\right) \in \multiplicativegroup{\finiteFieldExtension{\lambda^+}} \times \multiplicativegroup{\finiteFieldExtension{2 \lambda^-}} \times \multiplicativegroup{\finiteField} \mid \FieldNorm{2 \lambda^{-}}{\lambda^{-}}\left(y\right) = z \right\}.$$ The embedding $\algebraicGroup{T}^{\Frobenius} \to \algebraicGroup{G}^{\Frobenius}$ is of the form $$\left(x, y, z\right) \mapsto
	\begin{pmatrix}
		z \cdot \iota_+ \left(x\right)\\
		& \iota_{-}\left(y,z\right) &\\
		& & \iota_+^{\ast}\left(x\right)
	\end{pmatrix},$$
	where $\iota_+ \colon \multiplicativegroup{\finiteFieldExtension{\lambda^+}} \to \GL_{\sizeof{\lambda^+}}\left(\finiteField\right)$ and $$\iota_- \colon \left\{ \left(y, z\right)\in \multiplicativegroup{\finiteFieldExtension{2 \lambda^-}} \times \multiplicativegroup{\finiteField} \mid \FieldNorm{2 \lambda^{-}}{\lambda^{-}}\left(y\right) = z \text{ for all } i \right\} \to G'$$ are embeddings. Here, $G'$ is a group of the same type as $\algebraicGroup{G}^{\Frobenius}$, with $n$ replaced by $\sizeof{\lambda^{-}}$.
\end{theorem}

\subsection{Deligne--Lusztig theory}
\elad{Write a 101-intro to Deligne--Lusztig but not too detailed}
For each maximal rational torus $\algebraicGroup{T}$ in $\algebraicGroup{G}$, Deligne and Lusztig contructed a geometric space with commuting actions of $\algebraicGroup{T}(\finiteField)$ and $\algebraicGroup{G}$.
The $\ell$-adic \'etale Euler characteristic of this space is a $\algebraicGroup{T}(\finiteField) \times \algebraicGroup{G}(\finiteField)$ virtual representation, and taking the $\algebraicGroup{T}(\finiteField)$ $\theta$-invariants gives a $\algebraicGroup{G}(\finiteField)$ virtual representation $R_T^G(\theta)$.
The $R_T^G(\theta)$ only depends on the $\algebraicGroup{G}(\finiteField)$-conjugacy class of $(\algebraicGroup{T},\theta)$ and
the collection of virtual representations $\{R_T^G(\theta)\}$ contain every irreducible $\algebraicGroup{G}(\finiteField)$-representation as a sub-virtual representation.

\begin{proposition}[{\cite[Corollary 12.7]{DigneMichel1991}}]
\label{prop:semisimple_pair_with_RTtheta}
Treating $R_T^G(\theta)$ as a class function of $\algebraicGroup{G}(\finiteField)$ via the character of $R_T^G(\theta)$, for any class function $f$ on $\algebraicGroup{G}(\finiteField)$ which satisfies $f(g) = f(s)$ when $g=su$ is the Jordan decompositoin of $g$, we have
\[
	\langle R_T^G(\theta),f \rangle_{\algebraicGroup{G}(\finiteField)} = \langle \theta, f|_{\algebraicGroup{T}(\finiteField)}\rangle_{\algebraicGroup{T}(\finiteField)}
\]
in other words:
\[
	\frac{1}{|\algebraicGroup{G}(\finiteField)|}\sum_{g \in \algebraicGroup{G}(\finiteField)}\left(R_T^G(\theta)\right)(g)f(g) = \frac{1}{|\algebraicGroup{T}(\finiteField)|}\sum_{t \in  \algebraicGroup{T}(\finiteField)} \theta(t)f(t)
\]
\end{proposition}

\subsubsection{Geometric Conjugacy and Lusztig Series}
Let $\ell$ be a prime number not equal to the characteristic of $\finiteField$. Fix an isomorphism $\ladicnumbers \cong \cComplex$.

Let $\algebraicGroup{G}$ be a reductive group over $\finiteField$ and fix a maximal $\Frobenius$-stable torus $\algebraicGroup{T} \subset \algebraicGroup{G}$ defined over $\finiteField$. We denote by $$\CharacterLattice{\algebraicGroup{T}} = \Hom_{\algebraicClosure{\finiteField}}\left(\algebraicGroup{T}, \multiplcativeScheme\right)$$ the \emph{character lattice} of $\algebraicGroup{T}$ and by $$\CocharacterLattice{\algebraicGroup{T}} = \Hom_{\algebraicClosure{\finiteField}}\left(\multiplcativeScheme, \algebraicGroup{T}\right)$$ the \emph{cocharacter lattice} of $\algebraicGroup{T}$, where we regard $\algebraicGroup{T}$ and $\multiplcativeScheme$ as algebraic groups over $\finiteField$. Recall that $\CharacterLattice{\algebraicGroup{T}}$ and $\CocharacterLattice{\algebraicGroup{T}}$ are free $\zIntegers$-modules of the same finite rank. The assignment $\CharacterLattice{\algebraicGroup{T}} \times \CocharacterLattice{\algebraicGroup{T}} \to \Hom_{\finiteField}\left(\multiplcativeScheme, \multiplcativeScheme\right)$ given by composition $\left(x,y\right) \mapsto \left(z \mapsto x\left(y\left(z\right)\right)\right)$ defines a pairing $\innerproduct{\cdot}{\cdot} \colon \CharacterLattice{\algebraicGroup{T}} \times \CocharacterLattice{\algebraicGroup{T}} \to \zIntegers$ by the rule $$x\left(y\left(z\right)\right) = z^{\innerproduct{x}{y}}.$$

The \emph{dual group over $\finiteField$} of $\algebraicGroup{G}$ with respect to the torus $\algebraicGroup{T}$ corresponds to the group $\algebraicGroup{G}^*$ with a dual torus $\algebraicGroup{T}^*$ and root datum
\[
(\CharacterLattice{\algebraicGroup{T}^*},\Phi_{\algebraicGroup{T}^*},\CocharacterLattice{\algebraicGroup{T}^*},\Phi_{\algebraicGroup{T}^*}^\vee) \cong (\CocharacterLattice{\algebraicGroup{T}},\Phi_{\algebraicGroup{T}}^\vee,\CharacterLattice{\algebraicGroup{T}},\Phi_{\algebraicGroup{T}})
\]
such that the isomorphism respects the Frobenius actions on $\CharacterLattice{\cdot}$ and $\CocharacterLattice{\cdot}$.

Fix an isomorphism $\multiplicativegroup{\algebraicClosure{\finiteField}} \cong (\mathbb{Q}/\mathbb{Z})_{p'}$ and an embedding $\exp:(\mathbb{Q}/\mathbb{Z})_{p'} \to \overline{\mathbb{Q}_\ell}^\times$. 
By \cite[Proposition 13.7 (ii)]{DigneMichel1991}, we have an exact sequence
\begin{equation}
	\label{eq:DL}
	\xymatrix{0 \ar[r] & \CocharacterLattice{\algebraicGroup{T}}_{(\mathbb{Q}/\mathbb{Z})_{p'}} \ar[r]^{\Frobenius-\idmap} & \CocharacterLattice{\algebraicGroup{T}}_{(\mathbb{Q}/\mathbb{Z})_{p'}} \ar[r] & \algebraicGroup{T} \ar[r] & 0}
\end{equation}
with the last map being $y \mapsto \aFieldNorm_{\algebraicGroup{T}^{\Frobenius^n}:\algebraicGroup{T}^{\Frobenius}}(y(\zeta_n))$
for $\zeta_n$ being a $q^n-1$ root of unity corresponding to $1/(q^n-1) \in (\mathbb{Q}/\mathbb{Z})_{p'}$ and sufficiently large $n$ (large enough so that $\algebraicGroup{T}\left(\finiteFieldExtension{n}\right)$ is split). Here, $$\aFieldNorm_{\algebraicGroup{T}^{\Frobenius^n}:\algebraicGroup{T}^{\Frobenius}}(t) = t\cdot \Frobenius(t)\cdot \Frobenius^2(t)\cdot  \hdots \cdot \Frobenius^{n-1}(t)$$ is the torus norm map.

We say that $(\algebraicGroup{T},\theta)$ is a \emph{torus character pair} for $\algebraicGroup{G}^{\Frobenius}$ if $\algebraicGroup{T} \subset \algebraicGroup{G}$ is a maximal torus and $\theta \colon \algebraicGroup{T}^{\Frobenius} \to \multiplicativegroup{\ladicnumbers}$ is a character. By the discussion above, given a torus character pair  $(\algebraicGroup{T},\theta)$ for $\algebraicGroup{G}^{\Frobenius}$, we may conjugate $\algebraicGroup{T}$ to a fixed torus $\algebraicGroup{T}'$ and associate to $\theta$ a twisted Frobenius (twisted by an element of $W\left(\algebraicGroup{T}\right)$ in the twisted Weyl group conjugacy class corresponding to $\algebraicGroup{T}$) stable element $x_{\algebraicGroup{T},\theta}$ of $\CharacterLattice{\algebraicGroup{T}'}_{(\mathbb{Q}/\mathbb{Z})_{p'}}$ via the equality
\[
\exp(\innerproduct{x_{\algebraicGroup{T},\theta}}{y}) = \theta(\aFieldNorm_{\algebraicGroup{T}^{\Frobenius^n}:\algebraicGroup{T}^{\Frobenius}}(y(\zeta_n)))
\]
for large enough $n$. This $x_{\algebraicGroup{T},\theta}$ is well defined up to the Weyl group, and in fact, by \cite[Corollary 13.9]{DigneMichel1991} there is a correspondence between torus character pairs $(\algebraicGroup{T},\theta)$ and $\Frobenius$-stable Weyl group orbits in $\CharacterLattice{\algebraicGroup{T}'}_{(\mathbb{Q}/\mathbb{Z})_{p'}}$.  Using the given the identification $\CharacterLattice{\algebraicGroup{T}} = \CocharacterLattice{\algebraicGroup{T}^{\ast}}$ and composing with \eqref{eq:DL} (with $\algebraicGroup{T}$ replaced with $\algebraicGroup{T}^{\ast}$), we associate to a torus character pair a Frobenius stable semisimple conjugacy class in $\algebraicGroup{G}^*$.

Denote the Frobenius semisimple conjugacy class associated to a torus character pair $(\algebraicGroup{T},\theta)$ by $(s_{\algebraicGroup{T},\theta})$. Similarly, denote the $\Frobenius$-stable Weyl group orbit in $\CharacterLattice{\algebraicGroup{T}}_{(\mathbb{Q}/\mathbb{Z})_{p'}}$ corresponding to $(\algebraicGroup{T},\theta)$ by $(x_{\algebraicGroup{T},\theta}).$

We say that the torus character pairs $(\algebraicGroup{T},\theta)$ and $(\algebraicGroup{T}',\theta')$ for $\algebraicGroup{G}$ are \emph{geometrically conjugate} if the geometric conjugacy classes $(s_{\algebraicGroup{T},\theta})$ and $(s_{\algebraicGroup{T}',\theta'})$ are equal. This is equivalent to the following: there exists $n \ge 1$ and $h \in {\algebraicGroup{G}}^{\Frobenius^n}$ such that $h \algebraicGroup{T}^{\Frobenius^n} h^{-1} = \left(\algebraicGroup{T}'\right)^{\Frobenius^n}$ and such that $$\theta\left(\aFieldNorm_{\algebraicGroup{T}^{\Frobenius^n}:\algebraicGroup{T}^{\Frobenius}}\left(t\right)\right) = \theta'\left(\aFieldNorm_{\left(\algebraicGroup{T}'\right)^{\Frobenius^n}:\left(\algebraicGroup{T}'\right)^{\Frobenius}}\left(h t h^{-1}\right)\right),$$ for any $t \in \algebraicGroup{T}^{\Frobenius^n}$.

\begin{example}
	\elad{Correct example or remove it}\calvin{They are geo conjugate in $\algebraicGroup{Sp}_4$}
	Take $\algebraicGroup{G}' = \algebraicGroup{\GL}_2$. Let $\algebraicGroup{T}'_2 = \restrictionOfScalars{\finiteFieldExtension{2}}{\finiteField}{\multiplcativeScheme}$ and $$\algebraicGroup{T}_2 = \left\{x \in \restrictionOfScalars{\finiteFieldExtension{2}}{\finiteField}{\multiplcativeScheme} \mid x \involution{x} = 1 \right\}.$$
	Then $\left(\algebraicGroup{T}'_2\right)^{\Frobenius} = \multiplicativegroup{\finiteFieldExtension{2}}$ and $\algebraicGroup{T}_2^{\Frobenius} = \NormOneGroup{2}$. Let $\theta \colon \NormOneGroup{2} \to \multiplicativegroup{\ladicnumbers}$ be a character, and let $\theta' \colon \multiplicativegroup{\finiteFieldExtension{2}} \to \multiplicativegroup{\ladicnumbers}$ be the character given bv $$\theta'\left(x\right) = \theta\left(x^{1-q}\right).$$
	We claim that the torus character pairs $\left(\algebraicGroup{T}_2 \times \algebraicGroup{T}_2, \theta \times \theta\right)$ and $\left(\algebraicGroup{T}'_2, \theta'\right)$ are geometrically conjugate. Indeed, if $m \ge 2$ is even, then $\left(\algebraicGroup{T}'_2\right)^{\Frobenius^{m}} = \multiplicativegroup{\finiteFieldExtension{m}} \times \multiplicativegroup{\finiteFieldExtension{m}}$ and $\algebraicGroup{T}_2^{\Frobenius^m} = \multiplicativegroup{\finiteFieldExtension{m}}$. The norm map $\left(\algebraicGroup{T}'_2\right)^{\Frobenius^{m}} \to \left(\algebraicGroup{T}'_2\right)^{\Frobenius}$ is given by $\multiplicativegroup{\finiteFieldExtension{m}} \times \multiplicativegroup{\finiteFieldExtension{m}} \ni \left(x,y\right) \mapsto \FieldNorm{m}{2}\left(x\right) \FieldNorm{m}{2}\left(y\right)$, while the norm map $\algebraicGroup{T}_2^{\Frobenius^{m}} \to \algebraicGroup{T}_2^{\Frobenius}$ is given by $\multiplicativegroup{\finiteFieldExtension{m}} \ni x \mapsto \FieldNorm{m}{2}\left(x\right)^{1-q}$. We thus have
	$$\left(\theta \times \theta\right) \circ \aFieldNorm_{ \left(\algebraicGroup{T}_2 \times \algebraicGroup{T}_2\right)^{\Frobenius^m} : \left(\algebraicGroup{T}_2 \times \algebraicGroup{T}_2\right)^{\Frobenius} } = \theta' \circ \aFieldNorm_{\left(\algebraicGroup{T}'_2\right)^{\Frobenius^m} \colon \left(\algebraicGroup{T}'_2\right)^{\Frobenius}}.$$
\end{example}

The virtual representations $R_T^G(\theta)$ and $R_{T'}^G(\theta')$ share an irreducible representation only if $(T,\theta),(T',\theta')$ are geometrically conjugate. This yields a natural partitioning of the irreducible representations of $\algebraicGroup{G}(\finiteField)$: the \emph{geometric Lusztig series} $\LusztigSeries{\algebraicGroup{G}^{\Frobenius}}{s}$ associated to a Frobenius stable geometric conjugacy class $(s) \subset \algebraicGroup{G}^*$ consists of all (equivalence classes of) irreducible representations $\pi$ which appear with non-zero multiplicity in $R_{\algebraicGroup{T}, \theta}$ for some $(T,\theta)$ such that $(s_{\algebraicGroup{T},\theta}) = (s)$.

\begin{center}
	\begin{tabular}{|c|c|c|c|c|c|} \hline
		$\algebraicGroup{G}$ & $\algebraicGroup{T} \subset \algebraicGroup{G}$ & $\Frobenius$ action on $\CharacterLattice{\algebraicGroup{T}}$ & $\algebraicGroup{G}^{\ast}$ & $\algebraicGroup{T}^{\ast} \subset \algebraicGroup{G}^{\ast}$ \tabularnewline \hline \hline
		$\algebraicGroup{\UnitaryGroup}_{2n}$ & $\restrictionOfScalars{\quadraticExtension}{\finiteField}{\multiplcativeScheme^{n}}$ & $\left(-1\right)^{\times n}$ & $\algebraicGroup{\UnitaryGroup}_{2n}$ &  $\restrictionOfScalars{\quadraticExtension}{\finiteField}{\multiplcativeScheme^{n}}$ \tabularnewline \hline
		$\algebraicGroup{\UnitaryGroup}_{2n+1}$ & $\restrictionOfScalars{\quadraticExtension}{\finiteField}{\multiplcativeScheme^{n}} \times \algebraicGroup{\UnitaryGroup}_1$ & $\left(-1\right)^{\times n}$ & $\algebraicGroup{\UnitaryGroup}_{2n+1}$ &  $\restrictionOfScalars{\quadraticExtension}{\finiteField}{\multiplcativeScheme^{n} \times \algebraicGroup{\UnitaryGroup}}_1$ \tabularnewline \hline		 				 
		$\algebraicGroup{\SO}_{2n+1}$ & $\multiplcativeScheme^n$ & 1 &$\algebraicGroup{\Sp}_{2n}$ &  $\multiplcativeScheme^n$ \tabularnewline \hline
		$\algebraicGroup{\Sp}_{2n}$ & $\multiplcativeScheme^n$ & 1 &$\algebraicGroup{\SO}_{2n+1}$ & $\multiplcativeScheme^n$ \tabularnewline \hline		 
		$\algebraicGroup{\SO}^{+}_{2n}$ & $\multiplcativeScheme^n$ & 1 &$\algebraicGroup{\SO}^{+}_{2n}$ &  $\multiplcativeScheme^n$ \tabularnewline \hline
		$\algebraicGroup{\SO}^{-}_{2n}$ & $\multiplcativeScheme^{n-1} \times \algebraicGroup{\UnitaryGroup}_1$ & $\left(1\right)^{\times {\left(n-1\right)}} \times \left(-1\right)$ &$\algebraicGroup{\SO}^{-}_{2n}$ &  $\multiplcativeScheme^{n-1} \times \algebraicGroup{\UnitaryGroup}_1$ \tabularnewline \hline
		$\algebraicGroup{\GSp}_{2n}$ & $\multiplcativeScheme \times \multiplcativeScheme^{n}$ & 1 &$\algebraicGroup{GSpin}_{2n+1}$ &  $\multiplcativeScheme \times \multiplcativeScheme^{n}$ \tabularnewline \hline
		$\algebraicGroup{\GSO}^{+}_{2n}$ & $\multiplcativeScheme \times \multiplcativeScheme^{n}$ & 1 & $\algebraicGroup{GSpin}_{2n}^{+}$ &  $\multiplcativeScheme \times \multiplcativeScheme^{n}$ \tabularnewline \hline
		$\algebraicGroup{\GSO}^{-}_{2n}$ & $\multiplcativeScheme^{n-1} \times \restrictionOfScalars{\finiteFieldExtension{2}}{\finiteField}{\multiplcativeScheme}$ & $\left(1\right)^{\times {\left(n-1\right)}} \times \sigma^{\ast}$ & $\algebraicGroup{GSpin}_{2n}^{-}$ &  $\multiplcativeScheme^{n-1} \times \restrictionOfScalars{\finiteFieldExtension{2}}{\finiteField}{\multiplcativeScheme}$ \tabularnewline \hline
	\end{tabular}
\end{center}
\calvin{correct the tori and how Fr acts on cocharacters and the Weyl group}
For all the pairs $\algebraicGroup{T} \subset \algebraicGroup{G}$ in this list, the Frobenius root $\Frobenius$ of $\algebraicGroup{G}$ acts trivially on the Weyl group $W\left(\algebraicGroup{T}\right)$. In the last row, $\sigma \colon \restrictionOfScalars{\finiteFieldExtension{2}}{\finiteField}{\multiplcativeScheme} \to \restrictionOfScalars{\finiteFieldExtension{2}}{\finiteField}{\multiplcativeScheme}$ is the automorphism $a + b\sqrt{d} \mapsto a - b\sqrt{d}$ where $d \in \multiplicativegroup{\finiteField}$ is a non-square.

\section{Doubling method Jacobi sums}

\subsection{Definition}

\subsubsection{The case of general linear groups}

Let $\chi \colon \multiplicativegroup{\finiteField} \to \multiplicativegroup{\cComplex}$ be a non-trivial character.

Consider the following assignment $\GL_k\left(\finiteField\right) \to \cComplex$
$$\Phi_{\chi}\left(g\right) = \begin{dcases}
	\chi\left(\det\left(\IdentityMatrix{k}+g\right)\right) & \text{if }\det\left(\IdentityMatrix{k}+g\right) \ne 0,\\
	0 & \text{otherwise.}
\end{dcases}$$
It is clear that $\Phi_{\chi}$ is a class function of $\GL_k\left(\finiteField\right)$.

Given an irreducible representation $\tau$ of $\GL_k\left(\finiteField\right)$, we consider the following \emph{doubling method Jacobi sum}:
$$\posDblJacobiSum{\tau}{\chi} = q^{-\frac{k^2}{2}} \sum_{g \in \GL_k\left(\finiteField\right)} \Phi_{\chi}\left(g\right) \tau\left(g\right).$$
Then $\posDblJacobiSum{\tau}{\chi}$ defines an element of $\Hom_{\GL_k\left(\finiteField\right)}\left(\tau, \tau\right)$. Therefore, by Schur's lemma there exists a complex number $\posDblJacobiSumScalar{\tau}{\chi} \in \cComplex$ such that $$\posDblJacobiSum{\tau}{\chi} = \posDblJacobiSumScalar{\tau}{\chi} \cdot \idmap_\tau.$$ 

The goal of this section is to express $\posDblJacobiSumScalar{\tau}{\chi}$ in terms of Kondo's Gauss sum.

\begin{proposition}\label{prop:doubling-for-gln-in-terms-of-kondo}
	We have the identity
	$$\Phi_{\chi}\left(g\right) = \chi\left(-1\right)^k q^{-\frac{k^2}{2}} \GaussSumScalar{\chi_{\GL_k}}{\fieldCharacter} \sum_{\substack{x, y \in \GL_k\left(\finiteField\right)\\
			y^{-1} x = g}} \fieldCharacter\left(\trace x\right) \chi^{-1}\left(\det y\right) \fieldCharacter\left(\trace y\right).$$
\end{proposition}
\begin{proof}
	We write $$\Phi_{\chi}\left(g\right) = \frac{1}{\sizeof{\squareMatrix_k\left(\finiteField\right)}}\sum_{X \in \squareMatrix_k\left(\finiteField\right)} \sum_{h \in \GL_k\left(\finiteField\right)} \fieldCharacter\left(\trace \left(X\left(g+\IdentityMatrix{k}-h\right)\right)\right) \chi\left(\det h\right).$$
	By \Cref{lem:sum-vanishes-for-singular-matrices}, we can reduce the sum to $X \in \GL_k\left(\finiteField\right)$. Thus	$$\Phi_{\chi}\left(g\right) = \frac{1}{\sizeof{\squareMatrix_k\left(\finiteField\right)}} \sum_{X, h \in \GL_k\left(\finiteField\right)} \fieldCharacter\left(\trace \left(X\left(g+\IdentityMatrix{k}-h\right)\right)\right) \chi\left(\det h\right).$$
	Changing variables $h \mapsto X^{-1} h$, we get
	$$\Phi_{\chi}\left(g\right) = \frac{1}{\sizeof{\squareMatrix_k\left(\finiteField\right)}} \sum_{X, h \in \GL_k\left(\finiteField\right)} \fieldCharacter\left(\trace \left(Xg\right)\right) \fieldCharacter\left(-\trace h\right) \chi\left(\det h\right) \chi^{-1}\left(\det X\right) \fieldCharacter\left(\trace X\right).$$
	Changing variables again $h \mapsto -h$ and setting $Xg = x$ and $X = y$, we get  
	$$\Phi_{\chi}\left(g\right) = \chi\left(-1\right)^k q^{-\frac{k^2}{2}} \GaussSumScalar{\chi_{\GL_k}}{\fieldCharacter} \sum_{\substack{x, y \in \GL_k\left(\finiteField\right)\\
			y^{-1} x = g}} \fieldCharacter\left(\trace x\right) \chi^{-1}\left(\det y\right) \fieldCharacter\left(\trace y\right).$$
\end{proof}

We now use this to compute $\posDblJacobiSum{\tau}{\chi}$. We have that $$\posDblJacobiSum{\tau}{\chi} = q^{-k^2} \chi\left(-1\right)^k \GaussSumScalar{\chi_{\GL_k}}{\fieldCharacter} \sum_{x, y \in \GL_k\left(\finiteField\right)} \fieldCharacter\left(\trace x\right) \chi^{-1}\left(\det y\right) \fieldCharacter\left(\trace y\right) \tau\left(y^{-1} x\right).$$
The last equality implies the following result.
\begin{theorem}\label{thm:gln-doubling-gauss-sum-in-terms-of-kondo}For any irreducible representation $\tau$ of $\GL_k\left(\finiteField\right)$ and any non-trivial character $\chi \colon \multiplicativegroup{\finiteField} \to \multiplicativegroup{\cComplex}$, the following identity holds:
	$$\posDblJacobiSumScalar{\tau}{\chi} = \chi\left(-1\right)^k \GaussSumScalar{\chi}{\fieldCharacter}^k \GaussSumScalar{\tau}{\fieldCharacter} \GaussSumScalar{\tau^{\vee} \times \chi^{-1}}{\fieldCharacter}.$$
\end{theorem}

We can define analogous sums $\negDblJacobiSum{\tau}{\chi}$ and $\negDblJacobiSumScalar{\tau}{\chi}$ by the formulas
$$\negDblJacobiSum{\tau}{\chi} = q^{-\frac{k^2}{2}} \sum_{g \in \GL_k\left(\finiteField\right)} \Phi_{\chi}\left(-g\right) \tau\left(g\right) = \negDblJacobiSumScalar{\tau}{\chi} \cdot \idmap_{\tau}.$$
Then we have the simple relation
$$\negDblJacobiSumScalar{\tau}{\chi} = \centralCharacter{\tau}\left(-1\right) \posDblJacobiSumScalar{\tau}{\chi}.$$

\subsubsection{The case of classical groups}

Let $\left(\hermitianSpace, \innerproduct{\cdot}{\cdot}\right)$ be a non-degenerate $\epsilon_{\hermitianSpace}$-sesquilinear hermitian space. Let $G$ and $\tilde{G}$ be as in Sections \ref{subsec:isometry-groups} and \ref{subsec:similitute-groups}.

Let $\chi \colon \multiplicativegroup{\quadraticExtension} \to \multiplicativegroup{\cComplex}$ be a character. As before, the assignment $\JacobiKernel{\chi} \colon \GL_{\quadraticExtension}\left(\hermitianSpace\right) \to \cComplex$ given by $$\JacobiKernel{\chi}\left(g\right) = \begin{dcases}
	\chi\left(\detQuadratic\left( \idmap_{\hermitianSpace} + g\right)\right) & \text{if }\detQuadratic\left( \idmap_{\hermitianSpace} + g\right) \ne 0\\
	0 & \text{otherwise,}
\end{dcases}$$
is a class function of $\GL_{\quadraticExtension}\left(\hermitianSpace\right)$. Denote for $g \in \GL_{\quadraticExtension}\left(\hermitianSpace\right)$, $\posJacobiKernel{\chi}\left(g\right) = \JacobiKernel{\chi}\left(g\right)$ and $\negJacobiKernel{\chi}\left(g\right) = \JacobiKernel{\chi}\left(-g\right)$.

We define $\genHermitianJacobiKernel{\hermitianSpace}{\chi} \colon \tilde{G} \to \cComplex$ by $$\genHermitianJacobiKernel{\hermitianSpace}{\chi} =
	\begin{dcases}
		\genJacobiKernel{\chi}\left(g\right) & g \in G,\\
		0 & \text{otherwise.}
	\end{dcases}$$
Let $\pi$ be an irreducible representation of $H$ where $H \in \{G, \GroupExtension{G}\}$. Denote $$\dblJacobiSum{\pi}{\chi} = \frac{1}{\sqrt{\sizeof{\lieAlgebra}}} \sum_{g \in H} \pi\left(g\right) \genHermitianJacobiKernel{\hermitianSpace}{\chi}\left(g\right) = \frac{1}{\sqrt{\sizeof{\lieAlgebra}}} \sum_{g \in G} \pi\left(g\right) \genJacobiKernel{\chi}\left(g\right).$$
Since $\genJacobiKernel{\chi}$ is a class function of $\GL_{\quadraticExtension}\left(\hermitianSpace\right)$, and since $G$ is a normal subgroup of $H$, we have that $\genHermitianJacobiKernel{\hermitianSpace}{\chi}$ is a class function of $H$. It follows that $\dblJacobiSum{\pi}{\chi} \in \Hom_{H}\left(\pi, \pi\right)$. By Schur's lemma, there exists a constant $\dblJacobiSumScalar{\pi}{\fieldCharacter} \in \cComplex$, such that $$\dblJacobiSum{\pi}{\chi} = \dblJacobiSumScalar{\pi}{\chi} \cdot \idmap_\pi.$$
We call $\posDblJacobiSumScalar{\pi}{\chi}$ and $\negDblJacobiSumScalar{\pi}{\chi}$ the \emph{doubling method Jacobi sums}.

\subsection{Multiplicativity}
The goal of this section is to understand how the doubling method Jacobi sums behave under parabolic induction.

\elad{TODO: work on section}

Let $\xIsotropic$ and $\yIsotropic$ be isotropic spaces of $\hermitianSpace$ of dimension $k$, such that $\xIsotropic$ and $\yIsotropic$ are in duality with respect to form $\innerproduct{\cdot}{\cdot}$. Let us write $$\hermitianSpace = \xIsotropic \oplus \hermitianSpace' \oplus \yIsotropic,$$
where $\hermitianSpace' \subset \hermitianSpace$ is a non-degenerate subspace, orthogonal to $\xIsotropic$ and $\yIsotropic$. Let $P$ be the parabolic subgroup of $H$, consisting of all elements stabilizing the flag $$0 \subset \xIsotropic \subset \xIsotropic \oplus \hermitianSpace' \subset \xIsotropic \oplus \hermitianSpace' \oplus \yIsotropic = \hermitianSpace.$$
Let $G' = \IsometryGroup\left(\hermitianSpace'\right)$. Write $P = L \ltimes N$, where $L$ is the Levi part of $P$ and $N$ is the unipotent radical of $P$. Then $L$ is isomorphic to $H' \times \GL_k\left(\quadraticExtension\right)$, where $H' \in \{G',\GroupExtension{G'}\}$ is of the same type as $H$. 

Let $\fieldCharacter_{\quadraticExtension} \colon \quadraticExtension \to \multiplicativegroup{\cComplex}$ be the character  $\fieldCharacter_{\quadraticExtension} = \fieldCharacter \circ \trace_{\quadraticExtension \slash \finiteField}$. Similarly, for any $k \ge 1$, let $\fieldCharacter_{\quadraticFieldExtension{k}} \colon \quadraticFieldExtension{k} \to \multiplicativegroup{\cComplex}$ be the character $\fieldCharacter_{\quadraticFieldExtension{k}} = \fieldCharacter \circ \trace_{\quadraticFieldExtension{k} \slash \finiteField}$.

Given a multiplicative character $\chi \colon \multiplicativegroup{\quadraticExtension} \to \multiplicativegroup{\cComplex}$, let $\involutionPlusOne{\chi} \colon \multiplicativegroup{\quadraticExtension} \to \multiplicativegroup{\cComplex}$ be the character given by
$$\involutionPlusOne{\chi}\left(x\right) = \chi\left(x \cdot \involution{x}\right).$$ Similarly, let $\minusInvolution{\chi} \colon \multiplicativegroup{\quadraticExtension} \to \multiplicativegroup{\cComplex}$ be the character given by
$$\minusInvolution{\chi}\left(x\right) = \chi^{-1}\left(\involution{x}\right).$$

\begin{theorem}\label{thm:multiplicativity-in-terms-of-gauss-sums}
	Let $\pi'$ be an irreducible representation of $H'$ and let $\tau$ be an irreducible representation of $\GL_k\left(\quadraticExtension\right)$. Then for any irreducible representation $\pi$ of $H$ which appears as a subrepresentation the parabolic induction $\rho = \Ind{P}{H}{\tau \overline{\otimes} \pi'}$ and any $\chi$ such that $\involutionPlusOne{\chi} \ne 1$, we have
	$$\dblJacobiSumScalar{\pi}{\chi} = \centralCharacter{\tau}\left(\pm 1\right) \chi\left(\pm 1\right)^k \GaussSumScalar{\involutionPlusOne{\chi}}{\fieldCharacter_{\quadraticExtension}}^k \GaussSumScalar{\tau \times \minusInvolution{\chi}}{\fieldCharacter_{\quadraticExtension}} \GaussSumScalar{\Contragradient{\tau} \times \chi^{-1}}{\fieldCharacter_{\quadraticExtension}} \dblJacobiSumScalar{\pi'}{\chi}.$$
\end{theorem}

This theorem was proved in \cite{GirschZelingher2025} under the assumption that $\pi' \subset \Ind{P}{H'}{\tau' \overline{\otimes} \pi''}$, where $\pi''$ is an irreducible cuspidal representation of $H''$ and $\tau'$ is an irreducible representation of $\GL_{k'}\left(\finiteField\right)$ such that $\chi$ nor $\minusInvolution{\chi}$ appear in the cuspidal support of $\tau''$. \elad{TODO: check the exact condition. The Gauss sums appearing here suggest that we should replace $\chi$ with $\involution{\chi}$ and $\minusInvolution{\chi}$ with $\chi^{-1}$}

In order to prove a more general form of this theorem, let $v_{\pi'} \in \pi'$ and $v_{\tau} \in \tau$. Consider $f \in \rho$ defined as follows. The function $f \in \rho$ is the unique element supported on $P$, such that $f\left(\idmap_{\hermitianSpace}\right) = v_{\tau} \otimes v_{\pi'}$. The projection $f$ to any non-zero invariant subspace of $\rho$ is non-zero. We claim the following.
\begin{proposition}\label{prop:multiplicativity-of-jacobi-sum-given-support}
	Suppose that $\dblJacobiSum{\rho}{\chi} f$ is supported on $P$. Then \Cref{thm:multiplicativity-in-terms-of-gauss-sums} is true.
\end{proposition}
\begin{proof}
	We compute $\left(\dblJacobiSum{\rho}{\chi} f\right)\left(\idmap_{\hermitianSpace}\right)$. It is given by
	\begin{equation}\label{eq:recursive-doubling-gauss-sum}
		\frac{1}{\sqrt{\sizeof{\lieAlgebra}}} \sum_{\substack{p \in P \cap G\\
				\detQuadratic\left(\idmap_{\hermitianSpace}\pm p\right) \ne 0}} \chi\left(\detQuadratic\left(\idmap_{\hermitianSpace} \pm p\right)\right) \left(\tau \overline{\otimes} \pi'\right)\left(p\right) v_{\tau} \otimes v_{\pi'}.
	\end{equation}
	Decomposing the sum \eqref{eq:recursive-doubling-gauss-sum} as a sum over $L$ and $N$, and using the fact that if $p \in P \cap G$ has Levi part with image $\left(a, g'\right) \in \GL_k\left(\quadraticExtension\right) \times G'$ then $$\detQuadratic\left(\idmap_{\hermitianSpace} \pm p\right) = \left(\pm 1\right)^k \detQuadratic\left(\involution{a}\right)^{-1}\detQuadratic\left(\IdentityMatrix{k} \pm a\right) \det\left(\involution{\left(\IdentityMatrix{k} \pm a\right)}\right) \detQuadratic\left(\idmap_{\hermitianSpace'} \pm g'\right),$$
	and using the fact that $\sizeof{\lieAlgebra} = \sizeof{\lieAlgebra'} \sizeof{\squareMatrix_k\left(\finiteField\right)} \sizeof{N}^2$, where $\lieAlgebra'$ is the Lie algebra of $G'$,
	we get that \eqref{eq:recursive-doubling-gauss-sum} equals
	\begin{equation}
		\chi\left(\pm 1\right)^k q^{-\frac{k^2}{2}} \sum_{a \in \GL_k\left(\quadraticExtension\right)} \genJacobiKernel{\involutionPlusOne{\chi}}\left(a\right) \minusInvolution{\chi}\left(\detQuadratic a\right) \tau\left(a\right) v_{\tau} \otimes \frac{1}{\sqrt{\sizeof{\lieAlgebra'}}} \sum_{g' \in G'} \genHermitianJacobiKernel{\hermitianSpace}{\chi}\left(g'\right) \pi'\left(g'\right) v_{\pi'},
	\end{equation}
	which in turn is
	\begin{align*}
		&\chi\left(\pm 1\right)^k \dblJacobiSum{\tau \otimes \minusInvolution{\chi_{\GL_k}}}{\involutionPlusOne{\chi}} v_{\tau} \otimes \dblJacobiSum{\pi'}{\chi} v_{\pi'}\\
		=& \chi\left(\pm 1\right)^k\dblJacobiSumScalar{\tau \otimes \minusInvolution{\chi_{\GL_k}}}{\involutionPlusOne{\chi}} \dblJacobiSumScalar{\pi'}{\chi} v_{\tau} \otimes v_{\pi'}.
	\end{align*}
	
	It follows that $$\dblJacobiSum{\rho}{\chi} f = \chi\left(\pm 1\right)^k\dblJacobiSumScalar{\tau \otimes \minusInvolution{\chi_{\GL_k}}}{\involutionPlusOne{\chi}} \dblJacobiSumScalar{\pi'}{\chi} f.$$
	
	Write $\rho = \bigoplus_i \sigma_i$ where $\sigma_i$ are irreducible representations of $\sigma_i$ and the $\sigma_i$'s need not be non-isomorphic. By writing $f = \sum_{i} f_i$ where $0 \ne f_i \in \sigma_i$, we have that $$\sum_i \dblJacobiSum{\sigma_i}{\chi} f_i = \dblJacobiSum{\rho}{\chi} f,$$
	and therefore $$\dblJacobiSum{\sigma_i}{\chi} f_i = \chi\left(\pm 1\right)^k\dblJacobiSumScalar{\tau \otimes \minusInvolution{\chi_{\GL_k}}}{\involutionPlusOne{\chi}} \dblJacobiSumScalar{\pi'}{\chi} f_i,$$
	for every $i$. The result now follows.
\end{proof}
The assumption in \Cref{prop:multiplicativity-of-jacobi-sum-given-support} holds in certain cases. For example, if $\rho$ is an irreducible representation, this assumption holds. It seems difficult to prove that this assumption holds in general. However, it is possible in a special case which suffices given the results of \cite{GirschZelingher2025}.

\begin{proposition}
	Suppose that $k=1$. Then the assumption of \Cref{prop:multiplicativity-of-jacobi-sum-given-support} holds.
\end{proposition}
\begin{proof}\elad{Work on this}
	Notice that for $x \in G$, $$\left(\dblJacobiSum{\rho}{\chi} f\right)\left(x\right) = \frac{1}{\sqrt{\sizeof{\lieAlgebra}}} \sum_{g \in G} f\left(xg\right) \genJacobiKernel{\chi}\left(g\right) = \frac{1}{\sqrt{\sizeof{\lieAlgebra}}} \sum_{p \in G \cap P} f\left(p\right) \genJacobiKernel{\chi}\left(x^{-1} p\right),$$
	which can be rewritten as
	$$\left(\dblJacobiSum{\rho}{\chi} f\right)\left(x\right) = \frac{1}{\sqrt{\sizeof{\lieAlgebra}}} \sum_{p \in G \cap P}  \genJacobiKernel{\chi}\left(x^{-1} p\right) \left(\tau \overline{\otimes} \pi'\right)\left(p\right) v_\tau \otimes v_{\pi'}.$$
	As before, we may write
	$$\genJacobiKernel{\chi}\left(g\right) = q^{-k^2} \sum_{h \in \GL_{\quadraticExtension}\left(\hermitianSpace\right)} \sum_{A \in \EndomorphismRing_{\quadraticExtension}\left(\hermitianSpace\right)} \chi\left(\detQuadratic h\right) \fieldCharacter_{\quadraticExtension}\left(\trace\left(A \left(\idmap_{\hermitianSpace} \pm g - h \right)\right)\right).$$
	Hence, \begin{align*}
		q^{k^2} \sum_{p \in G \cap P} \genJacobiKernel{\chi}\left(xp\right) \left(\tau \overline{\otimes} \pi'\right)\left(p\right) =& \sum_{A \in \EndomorphismRing_{\quadraticExtension}\left(\hermitianSpace\right)} \sum_{h \in \GL_{\quadraticExtension}\left(\hermitianSpace\right)} \chi\left(\detQuadratic h\right)\fieldCharacter_{\quadraticExtension}\left(\trace\left(-Ah\right)\right) \\
		& \times \sum_{p \in P} \fieldCharacter_{\quadraticExtension}\left(\trace\left(A \left(\idmap_{\hermitianSpace} \pm x^{-1} p\right)\right)\right) \left(\tau \overline{\otimes} \pi'\right)\left(p\right).
	\end{align*}
	Assume that $\chi \ne 1$. By \Cref{lem:sum-vanishes-for-singular-matrices}, we can reduce the sum to $A \in \GL_{\quadraticExtension}\left(\hermitianSpace\right)$. Let us replace $h$ with $-A^{-1} h$. We get the sum
	\begin{align*}
		& \chi\left(-1\right)^k \sum_{A \in \GL_{\quadraticExtension}\left(\hermitianSpace\right)} \chi^{-1}\left(\detQuadratic A\right) \sum_{h \in \GL_{\quadraticExtension}\left(\hermitianSpace\right)} \chi\left(\detQuadratic h\right)\fieldCharacter_{\quadraticExtension}\left(\trace h\right)\\
		& \times \sum_{p \in P} \fieldCharacter_{\quadraticExtension}\left(\trace\left(A \left(\idmap_{\hermitianSpace} \pm x^{-1}p\right)\right)\right) \left(\tau \overline{\otimes} \pi'\right)\left(p\right).
	\end{align*}
	Consider the inner sum over $P$, written as $$\sum_{l \in L} \sum_{u \in N} \fieldCharacter_{\quadraticExtension}\left(\trace\left(\pm A x^{-1} l u\right)\right) \left(\tau \overline{\otimes} \pi'\right)\left(l\right).$$
	The inner sum over $N$ will vanish unless $\pm Ax^{-1} \in P$ \elad{This is not true in general}. Thus we have
	\begin{align*}
		q^{k^2} \sum_{p \in G \cap P} \genJacobiKernel{\chi}\left(xp\right) \left(\tau \overline{\otimes} \pi'\right)\left(p\right) =& \sum_{h \in \GL_{\quadraticExtension}\left(\hermitianSpace\right)}  \chi\left(\detQuadratic\left(xh\right)\right) \fieldCharacter_{\quadraticExtension}\left(\trace h\right) \\
		& \times \sum_{p' \in G \cap P} \sum_{p \in G \cap P} \chi^{-1}\left(\detQuadratic p'\right) \fieldCharacter_{\quadraticExtension}\left(\trace\left(p' \left( p \mp x  \right)\right)\right) \left(\tau \overline{\otimes} \pi'\right)\left(p\right).
	\end{align*}
	By decomposing the sum over $p' \in P$ into a sum over $N$ and $L$, we see that the inner sum over $N$ will vanish unless $p \mp x \in P$, which implies that $x \in P \cap G$.
	
	Hence, we have that $\dblJacobiSum{\rho}{\chi} f$ is supported on $P$. 
\end{proof}

\subsubsection{Gamma factors}\label{subsec:normalization-factor}
In order to make the computations more tolerable, it is beneficial to work with fully multiplicative functions. We introduce the doubling method gamma factors.

Let $\chi \colon \multiplicativegroup{\quadraticExtension} \to \multiplicativegroup{\cComplex}$ be a character.

We define a normalization factor as follows $$c_{\hermitianSpace}\left(\chi, \fieldCharacter\right) = \begin{dcases}
	\tau\left(\chi^2, \fieldCharacter\right)^{\frac{\dim_{\finiteField} \hermitianSpace}{2}} & \quadraticExtension = \finiteField,\,\, \dim_{\finiteField} \hermitianSpace \text{ is even,}\\
	\chi\left(2\right) \tau\left(\chi^2, \fieldCharacter\right)^{\frac{\dim_{\finiteField} \hermitianSpace - 1}{2}} & \quadraticExtension = \finiteField,\,\, \dim_{\finiteField} \hermitianSpace \text{ is odd,}	\\
	\tau\left(\chi\restriction_{\multiplicativegroup{\finiteField}}, \fieldCharacter\right)^{\dim_{\quadraticExtension} \hermitianSpace} & \quadraticExtension \ne \finiteField.
\end{dcases}$$
Notice that if $\quadraticExtension \ne \finiteField$ then by the Hasse--Davenport relation $\tau\left(\chi\restriction_{\multiplicativegroup{\finiteField}}, \fieldCharacter\right)^{2} = \tau\left(\involutionPlusOne{\chi}, \fieldCharacter_{\quadraticExtension}\right)$.

For an irreducible representation  $\tau$ of $\GL_k\left(\quadraticExtension\right)$, define $$\dblGammaFactor{\tau}{\chi}{\fieldCharacter} = \GaussSumScalar{\tau \times \minusInvolution{\chi}}{\fieldCharacter} \GaussSumScalar{\Contragradient{\tau} \times \chi^{-1}}{\fieldCharacter}.$$
For an irreducible representation $\pi$ of $G$, define the \emph{doubling method gamma factor} $$\dblGammaFactorSpace{\hermitianSpace}{\pi}{\chi}{\fieldCharacter} = \frac{\posDblJacobiSumScalar{\pi}{\chi}}{c_{\hermitianSpace}\left(\chi, \fieldCharacter\right)}.$$
Using this notation and using the multiplicativity property of Kondo's Gauss sum and \Cref{thm:multiplicativity-in-terms-of-gauss-sums}, we have the following multiplicativity property.

\begin{theorem}\label{thm:multiplicativity-in-terms-of-gamma-factors}
	\begin{enumerate}
		\item If $\tau_1$ and $\tau_2$ are irreducible representations of $\GL_{k_1}\left(\finiteField\right)$ and $\GL_{k_2}\left(\finiteField\right)$, respectively, then for any irreducible subrepresentation $\tau$ of the parabolic induction $\tau_1 \circ \tau_2$, we have
		$$\dblGammaFactor{\tau}{\chi}{\fieldCharacter} = \dblGammaFactor{\tau_1}{\chi}{\fieldCharacter} \dblGammaFactor{\tau_2}{\chi}{\fieldCharacter}.$$
		\item If $\tau$, $\pi'$ and $\pi$ are as in \Cref{thm:multiplicativity-in-terms-of-gauss-sums} and $\chi \ne 1$, then
		$$\dblGammaFactorSpace{\hermitianSpace}{\pi}{\chi}{\fieldCharacter} = \dblGammaFactorSpace{\hermitianSpace'}{\pi'}{\chi}{\fieldCharacter} \dblGammaFactor{\tau}{\chi}{\fieldCharacter}.$$
	\end{enumerate}
\end{theorem}

\begin{remark}
	The ``correct'' gamma factor (expected by Langlands functoriality) is given by $$\dblLanglandsGammaFactorSpace{\hermitianSpace}{\pi}{\chi}{\fieldCharacter} = \dblGammaFactorSpace{\hermitianSpace}{\pi}{\chi}{\fieldCharacter} \cdot \begin{dcases}
				\left(-1\right)^n & \quadraticExtension = \finiteField \text{ and } \epsilon_{\hermitianSpace} = 1\\
				\left(-1\right)^{n+1} \tau\left(\chi^{-1}, \fieldCharacter\right) & \quadraticExtension = \finiteField \text{ and } \epsilon_{\hermitianSpace} = -1 \\
				1 & \text{otherwise.}
	\end{dcases},$$
	where $n = \left\lfloor \frac{\dim_{\finiteField} \hermitianSpace}{2} \right\rfloor$.
	However, we will not need this version for our purposes. \elad{Check odd orthogonal case. Probably needs a sign modification (the formula needs to be $\left(-1\right)^{\text{dual group dimension}} \times \text{product of Gauss sums}$). $\varepsilon_{\algebraicGroup{G}}$ contributes half of the dimension so that's why we add another half here.}
\end{remark}

\subsection{Pairing with Deligne--Lusztig characters}
\calvin{old title: Computation of character sums attached to Deligne--Lusztig data}

The goal of the next two sections is to compute $\dblJacobiSumScalar{\pi}{\chi}$ for $\pi$ that appears in a Deligne--Lusztig virtual character $R_{T,\theta}$ for a suitable maximal torus $T$ and $\theta \colon T \to \multiplicativegroup{\cComplex}$.

In this section, we focus on the computing the analogous character sum corresponding to $\dblJacobiSumScalar{\pi}{\fieldCharacter}$ for the virtual character $R_{T,\theta}$.

Let us realize $H = \algebraicGroup{H}^{\Frobenius}$, that is, $H$ is the group consisting fixed points of the Frobenius map $\Frobenius$ acting on the connected reductive algebraic group $\algebraicGroup{H}$, where $\algebraicGroup{H} = \tilde{\algebraicGroup{G}}$ if $H = \tilde{G}$ and $\algebraicGroup{H} = \algebraicGroup{G}$ if $H = G$.

Given a class function $F \colon H \to \cComplex$ (i.e., a function that is invariant under conjugation by elements of $G$), let us define $$\dblVirtualJacobiSumScalar{F}{\chi} = \frac{1}{\sqrt{\sizeof{\lieAlgebra}}} \sum_{g \in G} F\left(g\right) \genHermitianJacobiKernel{\hermitianSpace}{\chi}\left(g\right).$$

It is clear that if $\pi$ is an irreducible representation of $H$ then $$\dblVirtualJacobiSumScalar{\trace \pi}{\chi} = \dblJacobiSumScalar{\pi}{\chi} \cdot \dim \pi.$$
Notice that the assignments $g \mapsto \posJacobiKernel{\chi}\left(g\right)$ and $g \mapsto \negJacobiKernel{\chi}\left(g\right)$ only depend on the semisimple part of $g \in H$. By \cite[Theorem in Section 1.2]{SaitoShinoda2000}, we have the following result. \calvin{does this also follow from \Cref{prop:semisimple_pair_with_RTtheta}? Maybe we should say something about it} \elad{Can you compare the proofs?}\calvin{they use the same proof: basically they just use the character formula for RTtheta}
\begin{proposition}\label{prop:reduction-of-gauss-sum-to-torus}
	For any $\Frobenius$-stable maximal torus $\algebraicGroup{T}$ of $\algebraicGroup{H}$, and any character $\theta \colon T \to \multiplicativegroup{\cComplex}$, where $T = \algebraicGroup{T}^{\Frobenius}$, we have
	$$ \dblVirtualJacobiSumScalar{R_{T, \theta}}{\chi} = \frac{\grpIndex{H}{T}}{\sqrt{\sizeof{\lieAlgebra}}} \sum_{t \in T} \theta\left(t\right) \genHermitianJacobiKernel{\hermitianSpace}{\chi}\left(t\right).$$
\end{proposition}
In the special case that $\theta$ is in general position, the virtual character $\varepsilon_{\algebraicGroup{H}} \varepsilon_{\algebraicGroup{T}} R_{T, \theta}$ equals $\trace \pi$ for some irreducible representation $\pi$ of $H$ and we have that for this $\pi$,
$$\dblJacobiSumScalar{\pi}{\chi} = \varepsilon_{\algebraicGroup{H}} \varepsilon_{\algebraicGroup{T}} \frac{\sizeof{H}_p}{\sqrt{\sizeof{\lieAlgebra}}} \sum_{t \in T} \theta\left(t\right) \genHermitianJacobiKernel{\hermitianSpace}{\chi}\left(t\right).$$
Here $\varepsilon_{\algebraicGroup{H}} = \left(-1\right)^{\mathrm{rel.rank} \algebraicGroup{H}}$ and $\varepsilon_{\algebraicGroup{T}} = \left(-1\right)^{\mathrm{rel.rank} \algebraicGroup{T}}$, and $\sizeof{H}_p$ is the size of the $p$-Sylow group of $H$, where $p$ is the characteristic of $\finiteField$. 

\subsubsection{Split torus computation I}
Let $\alpha \colon \multiplicativegroup{\quadraticFieldExtension{k}} \to \multiplicativegroup{\cComplex}$ be a character and let $\chi \colon \multiplicativegroup{\quadraticExtension} \to \multiplicativegroup{\cComplex}$ be a non-trivial character.

The goal of this section is to compute $$\sum_{\substack{x \in \multiplicativegroup{\quadraticFieldExtension{k}}\\
x \ne -1}} \alpha\left(x\right) \chi\left(\aFieldNorm_{\quadraticFieldExtension{k} \slash \quadraticExtension}\left(1+x\right)\right).$$
This computation will be useful for the next section, which in turn will be useful for invoking \Cref{prop:reduction-of-gauss-sum-to-torus}.

Let us rewrite our sum as
$$\frac{1}{\sizeof{\quadraticFieldExtension{k}}} \sum_{z \in \quadraticFieldExtension{k}} \sum_{x \in  \multiplicativegroup{\quadraticFieldExtension{k}}} \sum_{y \in \multiplicativegroup{\quadraticFieldExtension{k}}}\alpha\left(x\right) \chi\left(\aFieldNorm_{\quadraticFieldExtension{k} \slash \quadraticExtension}\left(y\right)\right) \fieldCharacter_{\quadraticFieldExtension{k}}\left(\left(y - \left(x + 1\right)\right)z\right).$$
Since $\chi$ is non-trivial, when $z = 0$, the inner sum over $y$ will vanish. Hence, we have that our sum can be rewritten as
$$\frac{1}{\sizeof{\quadraticFieldExtension{k}}} \sum_{z \in \multiplicativegroup{\quadraticFieldExtension{k}}} \sum_{x \in  \multiplicativegroup{\quadraticFieldExtension{k}}} \sum_{y \in \multiplicativegroup{\quadraticFieldExtension{k}}}\alpha\left(x\right) \chi\left(\aFieldNorm_{\quadraticFieldExtension{k} \slash \quadraticExtension}\left(y\right)\right) \fieldCharacter_{\quadraticFieldExtension{k}}\left(\left(y - \left(x + 1\right)\right)z\right).$$
Replacing $x$ with $-z^{-1} x$ and $y$ with $z^{-1} y$ and $z$ with $-z$, this becomes
$$\frac{1}{\sizeof{\quadraticFieldExtension{k}}} \sum_{z \in \multiplicativegroup{\quadraticFieldExtension{k}}} \alpha^{-1}\left(z\right) \chi^{-1}\left(\aFieldNorm_{\quadraticFieldExtension{k} \slash \quadraticExtension}\left(z\right)\right) \fieldCharacter_{\quadraticFieldExtension{k}}\left(z\right) \sum_{x \in  \multiplicativegroup{\quadraticFieldExtension{k}}} \alpha\left(x\right) \fieldCharacter_{\quadraticFieldExtension{k}}\left(x\right) \sum_{y \in \multiplicativegroup{\quadraticFieldExtension{k}}} \chi\left(\aFieldNorm_{\quadraticFieldExtension{k} \slash \quadraticExtension}\left(y\right)\right) \fieldCharacter_{\quadraticFieldExtension{k}}\left(y\right).$$
Hence, we have the identity
$$\sum_{\substack{x \in \multiplicativegroup{\quadraticFieldExtension{k}}\\
x \ne -1}} \alpha\left(x\right) \chi\left(\aFieldNorm_{\quadraticFieldExtension{k} \slash \quadraticExtension}\left(x+1\right)\right) = -\sizeof{\quadraticFieldExtension{k}}^{1/2} \GaussSumCharacter{\alpha^{-1}}{\chi^{-1}}{\fieldCharacter_{\quadraticFieldExtension{k}}} \tau\left(\alpha, \fieldCharacter_{\quadraticFieldExtension{k}}\right) \tau\left(\chi \circ \aFieldNorm_{\quadraticFieldExtension{k} \slash \quadraticExtension}, \fieldCharacter_{\quadraticFieldExtension{k}}\right).$$

\subsubsection{Split torus computation II}
We use the results from the previous section to compute the doubling method Jacobi sum for a torus of the form $\multiplicativegroup{\quadraticFieldExtension{k}}$. Let $\alpha \colon \multiplicativegroup{\quadraticFieldExtension{k}} \to \multiplicativegroup{\cComplex}$ and $\chi \colon \multiplicativegroup{\quadraticExtension} \to \multiplicativegroup{\cComplex}$ be characters such that $\involutionPlusOne{\chi} \ne 1$. Our goal is to compute $$\sum_{\substack{x \in \multiplicativegroup{\quadraticFieldExtension{k}}\\
		x \ne -1}} \alpha \left(x\right) \chi\left(\aFieldNorm_{\quadraticFieldExtension{k} \slash \quadraticExtension}\left(1 + x\right)\right) \chi\left(\aFieldNorm_{\quadraticFieldExtension{k} \slash \quadraticExtension}\left(1 + \minusInvolution{x}\right)\right).$$
This can be rewritten as
$$\sum_{\substack{x \in \multiplicativegroup{\quadraticFieldExtension{k}}\\
		x \ne -1}} \alpha \left(x\right) \minusInvolution{\chi}\left(\aFieldNorm_{\quadraticFieldExtension{k} \slash \quadraticExtension}\left(x\right)\right) \involutionPlusOne{\chi}\left(\aFieldNorm_{\quadraticFieldExtension{k} \slash \quadraticExtension}\left(1 + x\right)\right).$$
By the previous section, this equals
\begin{equation*}
	-\sizeof{\quadraticExtension}^{\frac{k}{2}} \GaussSumCharacter{\alpha^{-1}}{\chi^{-1}}{\fieldCharacter_{\quadraticFieldExtension{k}}} \GaussSumCharacter{\alpha}{\minusInvolution{\chi}}{\fieldCharacter_{\quadraticFieldExtension{k}}} \tau\left(\involutionPlusOne{\chi}, \fieldCharacter_{\quadraticExtension}\right)^k,
\end{equation*}
which equals
\begin{equation*}
	-\sizeof{\quadraticExtension}^{\frac{k}{2}} \GaussSumCharacter{\alpha^{-1}}{\chi^{-1}}{\fieldCharacter_{\quadraticFieldExtension{k}}} \GaussSumCharacter{\involution{\alpha}}{\chi^{-1}}{\fieldCharacter_{\quadraticFieldExtension{k}}} \tau\left(\involutionPlusOne{\chi}, \fieldCharacter_{\quadraticExtension}\right)^k,
\end{equation*}
Thus we obtained the following.
\begin{theorem}We have the identity
\begin{align*}
	& \sum_{\substack{x \in \multiplicativegroup{\quadraticFieldExtension{k}}\\
			x \ne -1}} \alpha \left(x\right) \chi\left(\aFieldNorm_{\quadraticFieldExtension{k} \slash \quadraticExtension}\left(1 + x\right)\right) \chi\left(\aFieldNorm_{\quadraticFieldExtension{k} \slash \quadraticExtension}\left(1 + \minusInvolution{x}\right)\right) \\
	=& -\sizeof{\quadraticExtension}^{\frac{k}{2}} \GaussSumCharacter{\alpha^{-1}}{\chi^{-1}}{\fieldCharacter_{\quadraticFieldExtension{k}}} \GaussSumCharacter{\involution{\alpha}}{\chi^{-1}}{\fieldCharacter_{\quadraticFieldExtension{k}}} \tau\left(\involutionPlusOne{\chi}, \fieldCharacter_{\quadraticExtension}\right)^k.
\end{align*}
\end{theorem}

\subsubsection{Elliptic torus computation}
In the next two sections we compute the doubling method Jacobi sum for a torus of the form $\NormOneGroup{2m}$.

Let $\theta \colon \NormOneGroup{2m} \to \multiplicativegroup{\cComplex}$ and $\chi \colon \multiplicativegroup{\quadraticExtension} \to \multiplicativegroup{\cComplex}$ be non-trivial characters, such that $\involutionPlusOne{\chi} \ne 1$. Our goal is to compute $$\sum_{\substack{x \in \NormOneGroup{2m}\\
		x \ne -1}} \theta \left(x\right) \chi\left(\aFieldNorm_{\finiteFieldExtension{2m} \slash \quadraticExtension}\left(1 + x\right)\right).$$

We separate two cases: the case $\quadraticExtension = \finiteField$ and the case $\quadraticExtension \ne \finiteField$.

\subsubsection{Elliptic torus computation: the case $\quadraticExtension = \finiteField$}
In this section, we compute the above exponential sum for the case $\quadraticExtension = \finiteField$. Since $\chi^2 \ne 1$, it follows that $\theta\left(x^{1-q^m}\right) \ne \chi \circ \aFieldNorm_{\finiteFieldExtension{2m} \slash \quadraticExtension}\left(x\right)$ for some $x \in \multiplicativegroup{\finiteFieldExtension{2m}}$.

As usual, we rewrite the sum as follows
$$q^{-2m} \sum_{z \in \finiteFieldExtension{2m}} \sum_{y \in \multiplicativegroup{\finiteFieldExtension{2m}}} \sum_{x \in \NormOneGroup{2m}} \theta \left(x\right) \chi\left(\aFieldNorm_{\finiteFieldExtension{2m} \slash \finiteField}\left(y\right)\right) \fieldCharacter_{2m}\left(z\left(y-\left(1+x\right)\right)\right).$$

If $z=0$ then the sum over $y$ will vanish. Hence we can reduce the sum to $z \in \multiplicativegroup{\finiteFieldExtension{2m}}$. Replacing $y$ with $z^{-1} y$ and $z$ with $-z$, we get the sum
$$-q^{-m} \tau\left(\chi \circ \aFieldNorm_{\finiteFieldExtension{2m} \slash \finiteField}, \fieldCharacter_{2m}\right) \sum_{z \in \multiplicativegroup{\finiteFieldExtension{2m}}} \chi^{-1}\left(\aFieldNorm_{\finiteFieldExtension{2m} \slash \finiteField}\left(z\right)\right) \fieldCharacter_{2m}\left(z\right)  \sum_{x \in \NormOneGroup{2m}} \theta \left(x\right) \fieldCharacter_{2m}\left(xz\right).$$
To proceed, we use the Hilbert 90 map $\multiplicativegroup{\finiteFieldExtension{2m}} \to \NormOneGroup{2m}$ given by $t \mapsto t^{1 - q^m}$. This map is surjective, and its kernel is $\multiplicativegroup{\finiteFieldExtension{m}}$. Define $\theta' \colon \multiplicativegroup{\finiteFieldExtension{2m}} \to \multiplicativegroup{\cComplex}$ by $\theta'\left(t\right) = \theta\left(t^{1-q^m}\right)$. Replacing the sum over $x \in \NormOneGroup{2m}$ with a sum over $t \in \multiplicativegroup{\finiteFieldExtension{2m}}$, we get \begin{align*}
	& \sum_{z \in \multiplicativegroup{\finiteFieldExtension{2m}}} \chi^{-1}\left(\aFieldNorm_{\finiteFieldExtension{2m} \slash \finiteField}\left(z\right)\right) \fieldCharacter_{2m}\left(z\right) \sum_{x \in \NormOneGroup{2m}} \theta \left(x\right) \fieldCharacter_{2m}\left(xz\right) \\
	= & \frac{1}{q^m-1}\sum_{z \in \multiplicativegroup{\finiteFieldExtension{2m}}} \sum_{t \in \multiplicativegroup{\finiteFieldExtension{2m}}} \chi^{-1}\left(\aFieldNorm_{\finiteFieldExtension{2m} \slash \finiteField}\left(z\right)\right) \fieldCharacter_{2m}\left(z^{q^m}\right) \theta' \left(t\right) \fieldCharacter_{2m}\left(t^{1-q^m} z\right).
\end{align*}
Replacing $z$ with $t^{q^m} z$, this becomes
\begin{align*}
	\frac{1}{q^m-1}\sum_{t,z \in \multiplicativegroup{\finiteFieldExtension{2m}}} \chi^{-1}\left(\aFieldNorm_{\finiteFieldExtension{2m} \slash \finiteField}\left(z\right)\right) \chi^{-1}\left(\aFieldNorm_{\finiteFieldExtension{2m} \slash \finiteField}\left(t\right)\right) \theta' \left(t\right) \fieldCharacter_{2m}\left(\trace_{\finiteFieldExtension{2m} \slash \finiteFieldExtension{m}}\left(z\right) t\right).
\end{align*}
Since $\chi \circ \aFieldNorm_{\finiteFieldExtension{2m} \slash \finiteField} \ne \theta'$, if $\trace_{\finiteFieldExtension{2m} \slash \finiteFieldExtension{m}}\left(z\right) = 0$, we have that the inner sum over $t$ is zero. Hence, we may reduce the sum over $z$ to $z$ such that $\trace_{\finiteFieldExtension{2m} \slash \finiteFieldExtension{m}}\left(z\right) \ne 0$. Using the fact that $\theta'$ is trivial on $\multiplicativegroup{\finiteFieldExtension{m}}$, we have after replacing variables $t \mapsto \frac{t}{\trace_{\finiteFieldExtension{2m} \slash \finiteFieldExtension{m}}\left(z\right)}$,
\begin{align*}
	\frac{1}{q^m-1}\sum_{\substack{t,z \in \multiplicativegroup{\finiteFieldExtension{2m}}\\
			\trace_{\finiteFieldExtension{2m} \slash \finiteFieldExtension{m}}\left(z\right) \ne 0}} \chi^{-1}\left(\aFieldNorm_{\finiteFieldExtension{2m} \slash \finiteField}\left(\frac{z}{\trace_{\finiteFieldExtension{2m} \slash \finiteFieldExtension{m}}\left(z\right) }\right)\right) \chi^{-1}\left(\aFieldNorm_{\finiteFieldExtension{2m} \slash \finiteField}\left(t\right)\right) \theta' \left(t\right) \fieldCharacter_{2m}\left(t\right).
\end{align*}
By the appendix, this equals
\begin{align*}
	q^{\frac{m}{2}} \tau\left(\chi^{-1} \circ \aFieldNorm_{\finiteFieldExtension{m} \slash \finiteField}, \fieldCharacter_m\right)^2 \tau\left(\chi^{2} \circ \aFieldNorm_{\finiteFieldExtension{m} \slash \finiteField}, \fieldCharacter_m\right) \left(\sum_{t \in \multiplicativegroup{\finiteFieldExtension{2m}}} \chi^{-1}\left(\aFieldNorm_{\finiteFieldExtension{2m} \slash \finiteField}\left(t\right)\right) \theta' \left(t\right) \fieldCharacter_{2m}\left(t\right)\right).
\end{align*}

To summarize, we have
\begin{align*}
	& \sum_{\substack{x \in \NormOneGroup{2m}\\
			x \ne -1}} \theta \left(x\right) \chi\left(\aFieldNorm_{\finiteFieldExtension{2m} \slash \finiteField}\left(1 + x\right)\right) \\
		=& q^{\frac{m}{2}} \tau\left(\chi^{-1}, \fieldCharacter\right)^{2m} \tau\left(\chi, \fieldCharacter\right)^{2m} \tau\left(\chi^{2}, \fieldCharacter\right)^m \tau\left(\theta' \times \chi^{-1}, \fieldCharacter_{2m}\right),
\end{align*}
where $\theta'\left(t\right) = \theta\left(t^{1-q^m}\right)$.
Since $\tau\left(\chi^{-1}, \fieldCharacter\right) = \chi\left(-1\right) \conjugate{\tau\left(\chi, \fieldCharacter\right)}$ and since $\chi \ne 1$, this implies the following. \begin{theorem}
	We have the identity $$\sum_{\substack{x \in \NormOneGroup{2m}\\
			x \ne -1}} \theta \left(x\right) \chi\left(\aFieldNorm_{\finiteFieldExtension{2m} \slash \finiteField}\left(1 + x\right)\right) = q^{\frac{m}{2}} \tau\left(\chi^{2}, \fieldCharacter\right)^m \tau\left(\theta' \times \chi^{-1}, \fieldCharacter_{2m}\right).$$
\end{theorem}

\subsubsection{Elliptic torus computation: the case $\quadraticExtension \ne \finiteField$}
In this section, we compute the above exponential sum for the case $\quadraticExtension = \finiteFieldExtension{2}$. Let $\chi \colon \multiplicativegroup{\finiteFieldExtension{2}} \to \multiplicativegroup{\cComplex}$. We wish to compute the following sum.
$$\sum_{\substack{x \in \NormOneGroup{2m}\\
		x \ne -1}} \theta \left(x\right) \chi\left(\aFieldNorm_{\finiteFieldExtension{2m} \slash \finiteFieldExtension{2}}\left(1 + x\right)\right).$$
By the appendix, this equals
$$q^{\frac{m}{2}} \GaussSumCharacter{\left(\theta'\right)^{-1}}{\chi^{-1}}{\fieldCharacter_{2m}} \tau\left(\chi \restriction_{\multiplicativegroup{\finiteField}} \circ \FieldNorm{m}{1}, \fieldCharacter_m\right),$$
which equals
$$q^{\frac{m}{2}} \tau\left(\chi \restriction_{\multiplicativegroup{\finiteField}}, \fieldCharacter\right)^m \GaussSumCharacter{\left(\theta'\right)^{-1}}{\chi^{-1}}{\fieldCharacter_{2m}}.$$

Thus, we obtained the following.
\begin{theorem}
	We have the following identity $$\sum_{\substack{x \in \NormOneGroup{2m}\\
			x \ne -1}} \theta \left(x\right) \chi\left(\aFieldNorm_{\finiteFieldExtension{2m} \slash \finiteFieldExtension{2}}\left(1 + x\right)\right) = q^{\frac{m}{2}} \tau\left(\chi \restriction_{\multiplicativegroup{\finiteField}}, \fieldCharacter\right)^m \GaussSumCharacter{\left(\theta'\right)^{-1}}{\chi^{-1}}{\fieldCharacter_{2m}}$$
\end{theorem}

\subsubsection{Computation for Deligne--Lusztig characters}
Combining the results of the previous sections with \Cref{prop:reduction-of-gauss-sum-to-torus}, we arrive at the following result.

\begin{theorem}\label{thm:computation-of-doubling-gauss-sum-scalar-for-deligne-lusztig-characters}
	Suppose that $T \cong \prod_{j=1}^r \multiplicativegroup{\quadraticFieldExtension{k_j}} \times \prod_{i=1}^s \finiteFieldExtension{2m_i}^1$ is a maximal torus of $H$. Suppose that $\theta \colon T \to \multiplicativegroup{\cComplex}$ is a character and that under the above isomorphism $\theta = \alpha_1 \times \dots \times \alpha_r \times \theta_1 \times \dots \times \theta_s$, where $\alpha_j \colon \multiplicativegroup{\quadraticFieldExtension{k_j}} \to \multiplicativegroup{\cComplex}$ and $\theta_i \colon \finiteFieldExtension{2m_i}^1 \to \multiplicativegroup{\cComplex}$ are characters for every $i$ and $j$. Denote \begin{align*}
		g_T\left(\chi, \theta, \fieldCharacter\right)
		= & \prod_{i=1}^s \GaussSumCharacter{\left(\theta'_i\right)^{-1}}{\chi^{-1}}{\fieldCharacter_{2m_i}} \cdot \prod_{j=1}^r \GaussSumCharacter{\involution{\alpha_j}}{\chi^{-1}}{\fieldCharacter_{\quadraticFieldExtension{k_j}}} \GaussSumCharacter{\alpha_j^{-1}}{\chi^{-1}}{\fieldCharacter_{\quadraticFieldExtension{k_j}}}.
	\end{align*} Then
	\begin{align*}
		 \dblPosVirtualJacobiSumScalar{R_{T, \theta}}{\chi} = \varepsilon_{\algebraicGroup{H}} R_{T,\theta}\left(1\right) c_{\hermitianSpace}\left(\chi, \fieldCharacter\right) g_T\left(\chi, \theta, \fieldCharacter\right),
	\end{align*}
	and \begin{align*}
		\dblNegVirtualJacobiSumScalar{R_{T, \theta}}{\chi} = \begin{dcases}
			0 & \dim_{\finiteField} \hermitianSpace \text{ is odd.}\\
			\dblPosVirtualJacobiSumScalar{R_{T, \theta}}{\chi} \cdot \prod_{i=1}^s \theta_i\left(-1\right) \cdot \prod_{j=1}^r \alpha_j\left(-1\right) & \text{ otherwise.}
		\end{dcases}
	\end{align*}
\end{theorem}
\begin{proof}
	Notice that if $\dim_{\finiteField} \hermitianSpace$ is odd then $\negJacobiKernel{\chi}\left(t\right) = 0$ for any $t \in T$, as $\idmap_{\hermitianSpace} - t$ always has determinant zero.
	
	We have that if $t$ corresponds to $\left(a_1,\dots,a_r,t_1,\dots,t_s\right) \in \prod_{j=1}^r \multiplicativegroup{\quadraticFieldExtension{k_j}} \times \prod_{i=1}^s \NormOneGroup{2m_i}$ such that $\idmap_{\hermitianSpace} + t$ is invertible then \begin{align*}
		\posJacobiKernel{\chi}\left(t\right) =& \prod_{i=1}^s \chi\left(\FieldNorm{2m_i}{1}\left(1 + t_i\right)\right) \cdot \prod_{j=1}^r \chi\left(\aFieldNorm_{\quadraticFieldExtension{k_j} / \quadraticExtension}\left(1 + a_j\right)\right) \chi\left(\aFieldNorm_{\quadraticFieldExtension{k_j} / \quadraticExtension}\left(1 + \minusInvolution{a_j}\right)\right) \\
		& \times  \begin{dcases}
			\chi\left(2\right) & \dim_{\finiteField} \hermitianSpace \text{ is odd},\\
			1 & \text{otherwise}
		\end{dcases}.
	\end{align*}
	If $t$ corresponds to $\left(a_1,\dots,a_r,t_1,\dots,t_s\right) \in \prod_{j=1}^r \multiplicativegroup{\quadraticFieldExtension{k_j}} \times \prod_{i=1}^s \NormOneGroup{2m_i}$ such that $\idmap_{\hermitianSpace} - t$ is invertible then \begin{align*}
		\negJacobiKernel{\chi}\left(t\right) =& \begin{dcases}
			0 & \dim_{\finiteField} \hermitianSpace \text{ is odd},\\
			\posJacobiKernel{\chi}\left(-t\right) & \text{otherwise.}
		\end{dcases}
	\end{align*}
	

	
	Hence our theorem follows from the results mentioned above and from the formula $$R_T\left(1\right) = \varepsilon_{\algebraicGroup{H}} \varepsilon_{\algebraicGroup{T}} \frac{\grpIndex{H}{T}}{\sizeof{H}_p},$$ the fact that $$ \sizeof{\lieAlgebra} = \sizeof{H}_p^2 \cdot q^{\frac{\dim_{\finiteField} \hermitianSpace}{2}},$$
	and the fact \begin{equation*}
		\varepsilon_{\algebraicGroup{T}} = \left(-1\right)^r.
	\end{equation*}
\end{proof}
\begin{remark}
	We have $$\varepsilon_{\algebraicGroup{G}} = \left(-1\right)^{\left\lfloor \frac{\dim_{\finiteField} \hermitianSpace}{2}\right\rfloor}.$$ If $\finiteField = \quadraticExtension$ and $\dim_{\finiteField} \hermitianSpace$ is even then we also have $$\varepsilon_{\algebraicGroup{\GroupExtension{G}}} = \left(-1\right)^{\frac{\dim_{\finiteField} \hermitianSpace}{2} + 1}.$$
\end{remark}
\begin{remark}
	Suppose that $\pi$ is any irreducible representation of $H$ such that $\trace \pi$ appears with non-zero coefficient in $R_{T,\theta}$. Then the central character of $\pi$ satisfies
	\begin{align*}
		\centralCharacter{\pi}\left(-1\right) = \theta\left(-1\right) =& \prod_{j=1}^r \alpha_j\left(-1\right) \cdot \prod_{i=1}^s \theta_i\left(-1\right).
	\end{align*}
	Thus \begin{align*}
		\dblNegVirtualJacobiSumScalar{R_{T, \theta}}{\chi} =& R_{T,\theta}\left(1\right) c_{\hermitianSpace}\left(\chi, \fieldCharacter\right) \centralCharacter{\pi}\left(-1\right) g_T\left(\chi, \theta, \fieldCharacter\right).
	\end{align*}
\end{remark}

\subsection{Invariance of $\langle \JacobiKernel{}, R_T(\theta) \rangle$ under geometric conjugacy}

Let $f \colon \algebraicGroup{G}^{\Frobenius} \to \cComplex$ be a class function. Then by Schur's lemma for any irreducible representation $\pi$, there exists $\gamma_{f}\left(\pi\right) \in \cComplex$ such that $$\sum_{g \in G} f(g)\pi(g) = \gamma_{f}\left(\pi\right) \cdot \idmap_{\pi}.$$

\begin{definition}
	We say that  $f \colon \algebraicGroup{G}^{\Frobenius} \to \cComplex$ is a \emph{stable function} on $\algebraicGroup{G}^{\Frobenius}$ if it satisfies the following property: if $\pi$ and $\pi'$ are irreducible representations of $\algebraicGroup{G}^{\Frobenius}$ such that lie in the same geometric Lusztig series, then $$\gamma_{f}\left(\pi\right) = \gamma_{f}\left(\pi'\right).$$
(See also \cite[Section 4]{LaumonLetellier2023} and \cite[Section 4.1]{ChenBhattacharya2024})
	\end{definition}


Here is a basic example of a stable class function.
\begin{proposition}
\label{prop:central_char}
    If $f \colon \algebraicGroup{G}^{\Frobenius} \to \cComplex$ is supported on the center $\algebraicGroup{Z}^{\Frobenius}$ of $\algebraicGroup{G}^{\Frobenius}$, then $f$ is a stable function.
\end{proposition}
\begin{proof}
By \cite[Lemma 2.2]{Malle2007}, if $\pi$ and $\pi'$ lie in the same geometric Lusztig series, then they have the same central character. The result now immediately follows.
\end{proof}
%adding in the commented out sketched proof just in case:
\begin{comment}
\begin{proof}
The Deligne Lusztig variety is a subset of the group, for any $z \in \algebraicGroup{Z}(\finiteField)$ we can move the action of $z \in algebraicGroup{G}(\finiteField)$ to the action of $z \in \algebraicGroup{T}(\finiteField)$
Thus it suffices to show that $\theta|_{\algebraicGroup{Z}(\finiteField))} = \theta'|_{\algebraicGroup{Z}(\finiteField)}$
But this follows from geometric conjugacy: for large enough $n$, $\theta \circ \fieldNorm{n}{1}|_{\algebraicGroup{Z}(\finiteField))} = \theta'\circ \fieldNorm{n}{1}|_{\algebraicGroup{Z}(\finiteField)}$
and the Norm map is surjective.
\end{proof}
\end{comment}


The main result of this paper is 
\begin{theorem}
\label{thm:Phi_stable}
	The assignments $H \to \cComplex$ given by $$h \mapsto \genHermitianJacobiKernel{\hermitianSpace}{h}$$ are stable functions. Equivalently, the assignments $$\pi \mapsto \dblJacobiSumScalar{\pi}{\chi}$$ are constant on geometric Lusztig series.
\end{theorem}
In the $\GL_n$ case this follows from \Cref{thm:gln-doubling-gauss-sum-in-terms-of-kondo} along with \calvin{Hasse-Davenport? Something else?}.
In this section we prove a slightly weaker result:

\begin{lemma}
\label{lem:indep_geo_conj}
    If $(\algebraicGroup{T},\theta)$ and $(\algebraicGroup{T}', \theta)$ are geometrically conjugate, then $$\frac{\dblVirtualJacobiSumScalar{R_{T, \theta}}{\chi}}{R_{T,\theta}(1)} = \frac{\dblVirtualJacobiSumScalar{R_{T', \theta'}}{\chi}}{R_{T',\theta'}(1)}.$$
\end{lemma}

We recall\[
	\restrictionOfScalars{\finiteFieldExtension{m}}{\finiteField}{\algebraicGroup{\GL}_n} = \prod_{a \in \mathbb{Z}/m\mathbb{Z}} \left(\algebraicGroup{\GL}_n\right)_a
\]
with Frobenius root action $$\Frobenius(g_0,g_1, \hdots, g_{m-1}) = (\Frobenius(g_{m-1}), \Frobenius(g_0), \Frobenius(g_1), \hdots, \Frobenius(g_{m-2})).$$

\begin{remark}
Torus character pairs for $\restrictionOfScalars{\finiteFieldExtension{m}}{\finiteField}{\algebraicGroup{\GL}_n}$ are the same as torus character pairs for $\algebraicGroup{\GL}_n$ as a group over $\finiteFieldExtension{m}$.
\end{remark}

Given a rational maximal torus $\algebraicGroup{T} \subset \restrictionOfScalars{\finiteFieldExtension{m}}{\finiteField}{\algebraicGroup{\GL}_n}$, a character of $\algebraicGroup{T}(\mathbb{F})$, and a non-trivial additive character $\psi$ of $\finiteFieldExtension{m}$, we define a Guass sum
\[
	\tau_{\algebraicGroup{T}}(\theta,\psi) := \varepsilon_{\algebraicGroup{
    \algebraicGroup{T}}}\sum_{t\in \tilde{\algebraicGroup{T}}^{\Frobenius}} \theta(t)\psi(\trace(t_0)).
\]
\calvin{maybe should do inverses in the definition. Not sure if $\tau$ or $g$ make more sense.}

Given a torus character pair $(\algebraicGroup{T},\theta)$ with associated $\Frobenius$-twisted conjugacy class $(w) \in W(\algebraicGroup{T}_0)$ along with an embedding $\rho:\algebraicGroup{G}^* \to \restrictionOfScalars{\finiteFieldExtension{m}}{\finiteField}{\algebraicGroup{\GL}_n}$ and a Frobenius stable maximal tori $\algebraicGroup{T}^* \subset \restrictionOfScalars{\finiteFieldExtension{m}}{\finiteField}{\algebraicGroup{\GL}_n}$ such that $\rho(\algebraicGroup{T}_0^*) \subseteq \algebraicGroup{T}^*$, let $\tilde{\algebraicGroup{T}}$ be the torus of $\algebraicGroup{\GL}_n$ corresponding to the $\Frobenius$-twisted conjugacy class of $\rho(w) \in W(\algebraicGroup{T})$ and $\tilde{\theta}$ such that $(\rho(s_{\algebraicGroup{T},\theta})) = (s_{\tilde{\algebraicGroup{T}},\tilde{\theta}})$. Define
\[
    g_{\algebraicGroup{T},\rho}(\chi,\theta,\psi) := \tau_{\Tilde{\algebraicGroup{T}}}(\tilde{\theta} \times \chi(\det(t_0)),\psi).
\]
where $\chi$ is a multiplicative character of $\finiteFieldExtension{m}$ and $\psi$ and additive character of $\finiteFieldExtension{m}$.
The following lemma shows that this is well defined.

\begin{lemma}
\label{lemma:gl_invariance}
\begin{enumerate}[(i)]
	\item \label{item:gl-n-hasse-davenport-gauss-sum-implies-well-defined} In the case $\rho$ is the identity, $g_{\algebraicGroup{T},\rho}(\chi,\theta,\psi)$ depends only on the semisimple conjugacy class of $(s_{\algebraicGroup{T},\theta})$ and the character $\chi$.
	\item $g_{\algebraicGroup{T},\rho}(\chi,\theta,\psi)$ is well defined i.e. independent of choice of $\algebraicGroup{T}_0$, $\algebraicGroup{T}^*$ and conjugacy class representative $w$.
\end{enumerate}
\end{lemma}
\begin{proof}
\begin{enumerate}[(i)]
	\item In this case when $T = \prod_{i} \multiplicativegroup{\finiteFieldExtension{n_i}}$ and $\theta = \prod_{i} \alpha_i$, we recover the Gauss sum
	\[
	g_{T,\rho}(\chi,\theta,\psi)  = \prod_i \tau(\alpha_i \times \chi, \fieldCharacter_{n_i}),
	\]
	which by the Hasse-Davenport relation depends only on the geometric conjugacy class of $(T,\theta)$ hence the semisimple conjugacy class of $(s_{T,\theta})$ \calvin{maybe write more or less?}. 
	\item For a Frobenius stable semisimple conjugacy class $(s)$ in $\GL_n$, we define
	\[
	g_{(s)}(\psi,\chi) := g_{T,\rho}(\chi,\theta,\psi) \textnormal{ for some } (T,\theta) \textnormal{ such that } (s_{T,\theta}) = (s)
	\]
	By (\ref{item:gl-n-hasse-davenport-gauss-sum-implies-well-defined}) this is well defined. Then
	\[
	g_{T,\rho}(\chi,\theta,\psi)  = g_{(\rho(s_{T,\theta}))}(\psi,\chi)
	\]
	which is independent of of choice of $\algebraicGroup{T}^* \subset G^*$ and $\algebraicGroup{T} \subset \GL_n$.
\end{enumerate}
\end{proof}


\begin{corollary}
\label{cor:geoconj}
Suppose that $(\algebraicGroup{T},\theta)$ and $(\algebraicGroup{T}',\theta')$ are geometrically conjugate torus character pairs in $G$. Then $g_{\algebraicGroup{T},\rho}(\chi,\theta,\psi) = g_{\algebraicGroup{T}',\rho}(\chi,\theta',\psi)$.
\end{corollary}
\begin{proof}
Both are equal to $g_{(\rho(s_{T,\theta}))}(\psi,\chi) = g_{(\rho(s_{T',\theta'}))}(\psi,\chi)$.
\end{proof}

\begin{remark}
	The above work is not surpising as it is really a rephrasing that the eigenvalues of $\rho(s)$ are independent of conjugacy.
\end{remark}

The idea behind the proof of \Cref{lem:indep_geo_conj} is that the formula from \Cref{thm:computation-of-doubling-gauss-sum-scalar-for-deligne-lusztig-characters} relates $\dblVirtualJacobiSumScalar{R_{T, \theta}}{\chi}/R_{T,\theta}(1)$ to $g_{T,\rho}(\chi,\theta,\psi)$ for the appropriate $\rho$ in the table below. 
\begin{center}
\begin{tabular}{ c c}
 Group & Dual group representations $\rho$ \\ 
 $Sp(2n)$ &$SO(2n+1) \to \GL(2n+1)$\\  
 $GSp(2n)$  &$GSpin(2n+1) \to \GL(2n+1)$ \\
 $SO(2n)$ &$SO(2n) \to \GL(2n)$ \\
 $GO(2n)$ & ? \\
 $SO(2n+1)$ &$Sp(2n) \to \GL(2n)$ \\
 $GO(2n+1)$ & ? \\ 
  $U(n)$  & $U(n) \to \restrictionOfScalars{\quadraticExtension}{\finiteField}{\algebraicGroup{\GL}_n}$  
\end{tabular}
\end{center}
%Note that in the $SO$ and $GSO$ case, the map $\rho$ sends the split torus of $G^*$ into the split torus of $\GL(2n)$ and the induced map of cocharacters sends $x \mapsto (x,-x)$. In the $Sp,GSp$ cases the map is similar: the induced map of cocharacters sends $x \mapsto (x,1,-x)$. In the unitary case, the torus whose $\finiteField$ points are $(\NormOneGroup{2})^n \subset G^*(\finiteField)$ includes into the torus in $\GL(2n)$ whose $\finiteField$ points are $(\finiteFieldExtension{2}^\times)^n$ and the induced map of cocharacters is $(x_1,x_2, \hdots, x_n) \mapsto (x_1,-x_1,x_2,-x_2, \hdots ,x_n,-x_n)$. 


%In particular 
%To analyze $g_{\algebraicGroup{T},\rho}(\chi,\theta,\psi)$ for these $\rho$, we need to understand what torus character pair of $\GL$ corresponds to $\phi((s))$ under a map $\phi(x) = (x,-x)$:
%The following computation in the proof of theorem \ref{lem:indep_geo_conj}.

\begin{lemma}
\label{lem:cochar_computation}
    Let $\phi:X_*(\algebraicGroup{T}^*) \cong (\mathbb{Q}/\mathbb{Z})_{p'}^n \to X_*(\tilde{\algebraicGroup{T}}^*) \cong (\mathbb{Q}/\mathbb{Z})_{p'}^{2n}$ send $x$ to $(x,-w_n x)$. Then we have the following cases corresponding to conjugacy classes in the Weyl group
    \begin{enumerate}
        \item[B,C,D Case 1)] $w$ is conjugate to an $n$-cycle: \calvin{Add in note about the two type B cases. Maybe really should just talk about $\Frobenius$ action} In this case $\algebraicGroup{T} = \mathrm{Res}_{\mathbb{F}_n:\mathbb{F}}\algebraicGroup{G}_m$ with $\Frobenius$ acting by $w$ on the preimage and the cycle $(w,w^\ast)$ on the image, and $$x_{\finiteFieldExtension{n}^\times \times \finiteFieldExtension{n}^\times,\alpha \times \alpha^{-1}} = \phi(x_{\finiteFieldExtension{n},\alpha}).$$
        \item[B,C,D Case 2)] $w$ is conjugate to an $n$-cycle with one negative:
 In this case $\algebraicGroup{T} = \mathrm{Res}_{\mathbb{F}_n:\mathbb{F}}\algebraicGroup{U}(1)$ with $\Frobenius$ acting by $w$ on the preimage and by a $2n$ cycle on the image, and $$x_{\finiteField_{2n}^\times,\theta'}=\phi(x_{\NormOneGroup{2n},\theta})$$ where $\theta'(t) = \theta(t^{1-q^n}).$
    \item[$G=\algebraicGroup{U}$ Case 1)] $w$ is conjugate to $(1,2 \hdots n)$ an even cycle: In this case $\algebraicGroup{T} = \mathrm{Res}_{\mathbb{F}_n:\mathbb{F}}\algebraicGroup{G}_m$ with $\Frobenius$ acting by $-1\cdot w$ on the preimage and by the product of two $n$ cycles $\begin{pmatrix} 0 & w_nw \\
        w w_n & 0
        \end{pmatrix}$  on the image, and 
		$$x_{\finiteFieldExtension{n}^\times \times \finiteFieldExtension{n}^\times,\alpha \times \alpha^{-q}} = \phi(x_{\finiteFieldExtension{n},\alpha}).$$
    \item[$G=\algebraicGroup{U}$ Case 2)] $w$ is conjugate to $(1,2 \hdots n)$ an odd cycle: In this case $\algebraicGroup{T} = \mathrm{Res}_{\mathbb{F}_n:\mathbb{F}}\algebraicGroup{U}(1)$ with $\Frobenius$ acting by $-1\cdot w$ on the preimage and by the $2n$ cycle $\begin{pmatrix} 0 & w_nw \\
        w w_n& 0
        \end{pmatrix}$ on the image, and $$x_{\finiteField_{2n}^\times,\theta'}=\phi(x_{\NormOneGroup{2n},\theta})$$ where $\theta'(t) = \theta(t^{1-q^n})$
    \end{enumerate}
\end{lemma}
\begin{proof}
Recall that $x_{T,\theta}$ is defined by for sufficiently large $m$ with $\zeta_m = 1/(q^m-1)$ under the fixed identification $\multiplicativegroup{\algebraicClosure{\finiteField}} \cong (\mathbb{Q}/\mathbb{Z})_{p'}$
\[
    \exp(\langle x_{T,\theta},y\rangle) = \theta((N_{T(\finiteField_m):T(\finiteField)}(y(\zeta_m))).
\]
For different tori with Frob action we get
\begin{enumerate}
    \item ($n$-cycle) $\algebraicGroup{T}(\finiteField) \cong \finiteField_n^\times$ with $$\FieldNorm{\algebraicGroup{T}(\finiteField_{mn})}{\algebraicGroup{T}(\finiteField)}(y_1,y_2 \hdots y_n)(\zeta_{mn}) = (\zeta_n)^{y_1+qy_2+q^2y_3 \hdots +q^{n-1}y_n}$$
    \item ($n$-cycle with one negative) $\algebraicGroup{T}(\finiteField) \cong \{x \in \finiteField_{2n}^\times: \FieldNorm{2n}{n}(x) = 1\}$ with $$\FieldNorm{\algebraicGroup{T}(\finiteField_{2mn})}{\algebraicGroup{T}(\finiteField)}(y_1,y_2 \hdots y_n)(\zeta_{2mn}) = (\beta_n)^{y_1+qy_2+q^2y_3 \hdots +q^{n-1}y_n}$$
    where $\beta_n = \zeta_{2n}^{1-q^n}$.
    \item ($(-1) \cdot n$-cycle, $n$ even) $\algebraicGroup{T}(\finiteField) \cong \finiteField_n^\times$ with $$\FieldNorm{\algebraicGroup{T}(\finiteField_{mn})}{\algebraicGroup{T}(\finiteField)}(y_1,y_2 \hdots y_n)(\zeta_{mn}) = (\zeta_n)^{y_1-qy_2+q^2y_3-q^3y_4 \hdots -q^{n-1}y_n}$$
    \item ($(-1) \cdot n$-cycle, $n$ odd) $\algebraicGroup{T}(\finiteField) \cong \{x \in \finiteField_{2n}^\times: \FieldNorm{2n}{n}(x) = 1\}$ with $$\FieldNorm{\algebraicGroup{T}(\finiteField_{2mn})}{\algebraicGroup{T}(\finiteField)}(y_1,y_2 \hdots y_n)(\zeta_{2mn}) = (\beta_n)^{y_1-qy_2+q^2y_3-q^3y_4 \hdots +q^{n-1}y_n}$$
    where $\beta_n = \zeta_{2n}^{1-q^n}$.
\end{enumerate}
Now the results follow from the equalities
\begin{align*}
	\alpha(\zeta^{((y_1-y_{2n})+q(y_2-y_{2n-1}) + \hdots +q^{n-1}(y_{n}-y_{n+1}))})&=\alpha(\zeta^{(y_1+qy_2 + \hdots +q^{n-1}y_{n})}) \alpha^{-1}(\zeta^{(y_{2n}+qy_{2n-1} + \hdots +q^{n-1}y_{n+1})}) \\
	\theta(\beta^{((y_1-y_{2n})+q(y_2-y_{2n-1}) + \hdots +q^{n-1}(y_{n}-y_{n+1}))})&=\theta(\beta^{(y_1+qy_2 + \hdots + q^{n-1}y_{n} + q^ny_{2n}+q^{n+1}y_{2n-1} \hdots +q^{2n-1}y_{n+1})}) \\
	&=\theta'(\zeta^{(y_1+qy_2 + \hdots + q^{n-1}y_{n} + q^ny_{2n}+q^{n+1}y_{2n-1} \hdots +q^{2n-1}y_{n+1})})\\
	\alpha(\zeta^{((y_1-y_{2n})-q(y_2-y_{2n-1}) +q^2(y_3-y_{2n-2}) \hdots -q^{n-1}(y_{n}-y_{n+1}))})&= \alpha(\zeta^{(y_1+qy_{2n-1}+q^2y_3 +q^3y_{2n-3} \hdots +q^{n-1}y_{n+1})}) \\
	&\quad \cdot \alpha^{-1}(\zeta^{(qy_{2} + q^2y_{2n-2} + \hdots +q^{n-1}y_{n}+q^ny_{2n})}) \\
	\theta(\beta^{((y_1-y_{2n})-q(y_2-y_{2n-1}) +q^2(y_3-y_{2n-2}) \hdots +q^{n-1}(y_{n}-y_{n+1}))})&=\theta(\beta^{(y_1+qy_{2n-1} +q^2y_3 \hdots + q^{n-1}y_{n} + q^ny_{2n}+q^{n+1}y_{2} \hdots +q^{2n-1}y_{n+1})}) \\
	&=\theta'(\zeta^{(y_1+qy_{2n-1} +q^2y_3 \hdots + q^{n-1}y_{n} + q^ny_{2n}+q^{n+1}y_{2} \hdots +q^{2n-1}y_{n+1})})
\end{align*}
where $\zeta = \zeta_n = 1/(q^n-1)$ under the fixed identification $\multiplicativegroup{\algebraicClosure{\finiteField}} \cong (\mathbb{Q}/\mathbb{Z})_{p'}$ and $\beta = \beta_n = \zeta_{2n}^{1-q^n}$.
%Now the results follow from checking that
%\[
%    (\theta_1 \times \theta_1^{-1}) \circ \FieldNorm{\algebraicGroup{T}_1(\finiteField_r)}{\algebraicGroup{T}_1(\finiteField)} (\vec{y})(\zeta_r) = \theta_2\circ \FieldNorm{\algebraicGroup{T}_2(\finiteField_r)}{\algebraicGroup{T}_2(\finiteField)}(\vec{y})(\zeta_r)
%\]
%for $r$ divisible as above.
\begin{comment}
Thus the result follows from the computations:
\begin{enumerate}
	\item[B,C,D Case 1] \begin{align*}
		\exp(\langle\phi(x_{\finiteFieldExtension{n},\alpha}),(y_1,y_2)\rangle) &= \exp(\langle x_{\finiteFieldExtension{n},\alpha},\phi^*(y_1,y_2)\rangle) \\
		&= \alpha\left(\FieldNorm{mn}{n}(y_1(\zeta_{mn})\right) \alpha\left(\FieldNorm{mn}{n}(y_2(\zeta_{mn})\right)^{-1} \\
		&= \left(\alpha\times\alpha^{-1}\right)\left(\FieldNorm{mn}{n}((y_1,y_2)(\zeta_{mn}))\right)
	\end{align*}
	\item \begin{align*}
		\exp(\langle\phi(x_{\NormOneGroup{2n},\theta}),(y_1,y_2)\rangle) &= \exp(\langle x_{\NormOneGroup{2n},\theta},\phi^*(y_1,y_2)\rangle) \\
		&= \theta\left(\frac{\FieldNorm{2mn}{2n}(y_1(\zeta_{2mn})}{\FieldNorm{2mn}{2n}\left(y_1(\zeta_{2mn})^{q^n}\right)}\right) \theta\left(\frac{\FieldNorm{2mn}{2n}(y_2(\zeta_{2mn})}{\FieldNorm{2mn}{2n}\left(y_2(\zeta_{2mn})^{q^n}\right)}\right)^{-1} \\
		&= \theta'\left(\FieldNorm{2mn}{2n}((y_1,y_2)(\zeta_{mn}))\right)
	\end{align*}
\end{enumerate}
\end{comment}
\end{proof}

\begin{remark}\calvin{figure out how to type out better}
There is some ambiguity in choice of generator of each toris: for instance I could $\alpha \times \alpha^{-1}$ with $\alpha \times \alpha^{-q}$ in the first equality using the coordinate $y_{2n-1}$ instead of $y_{2n}$ for the second coordinate. In the type B,C,D case this ambiguity does not effect our sums since $\chi,\psi$ are defined over $\finiteField$ and hence stable under Frobenius i.e. $q$-th powers. In the type A unitary case, it is important to know which generators are in the first component which is any $y_i$ with $i\leq n$ since our characters $\chi$ take as input these entries.
\end{remark}

We let $\phi_{B,\lambda_i^+}, \phi_{B,\lambda_i^-}$ denote the $\phi$'s above in cases 1 and 2 respectivitly with $n = |\lambda_i|$. 
Let $\phi_{U,\lambda_i}$ denote the $\phi$'s above in cases 3 and 4 with $n = |\lambda_i|$.


\begin{proof}[Proof of \Cref{lem:indep_geo_conj}] 
By \Cref{thm:computation-of-doubling-gauss-sum-scalar-for-deligne-lusztig-characters}, it is enough to show that $g_{T}(\chi,\theta,\psi)$ and $\omega_{\pi}(-1) = \theta(-1)$ only depend of the geometric conjugacy of $(T,\theta)$. The latter only depends on geometric conjugacy by \Cref{prop:central_char}.

For the former we will show
\[
g_{T,\rho}(\chi,\theta,\fieldCharacter)= g_{T}(\chi^{-1},\theta^{-1},\fieldCharacter) \cdot \begin{cases}
        1  & G \neq Sp, GSp\\
        \tau\left(\chi^{-1}, \fieldCharacter\right) & G = Sp \textnormal{ or } GSp
    \end{cases}.
\]
\calvin{$\chi$ or $\chi^{-1}$? Check throughout proof!}
We will do this casewise, demonstrating in each case that the Frobeinus equivariant map $X_*(\algebraicGroup{T}) \to X_*(\algebraicGroup{G})$ decomposes into a direct sum of $\phi$'s appearing in \Cref{lem:cochar_computation}. Useful here is the description of the rational maximal tori $\algebraicGroup{T} \subseteq \algebraicGroup{G}$ and geometric Frobeinus action on $X_*(\algebraicGroup{T})$. 

\begin{enumerate}
    \item[($\algebraicGroup{U}$)] Given $(T,\theta)$, let $T$ correspond to the $w_n$-twisted conjugacy class $[w]$. Let $w \circ w_n$ have cycle decomposition $\lambda$ 
	and $\theta$ decompose as a product over $\lambda_i$ of $\alpha_i$ if $\lambda_i$ is even and $\theta_i$ if $\lambda_i$ is odd. Then
	$$\rho_* = \bigoplus_{\lambda_i} \phi_{U,\lambda_i}$$
	and applying \Cref{lem:cochar_computation}
    \begin{align*}
		g_{T,\rho}(\chi,\theta,\psi) &= \Pi_{\lambda_i^+} \GaussSumCharacter{\alpha_i}{\chi}{\fieldCharacter_{\quadraticFieldExtension{\lambda_i^+}}} \times \GaussSumCharacter{\alpha_i^{-q}}{\chi}{\fieldCharacter_{\quadraticFieldExtension{\lambda_i^+}}} \times \Pi_{\lambda_i^-} \GaussSumCharacter{\theta'_i}{\chi}{\fieldCharacter_{2\lambda_i^-}} 
	\end{align*}
	which is equal to $g_{T}(\chi^{-1},\theta^{-1},\psi)$.
	\begin{comment}old hacky version
	\calvin{right now can only deal with the case that $\chi$ is a square} Given $(T,\theta)$, let $T$ correspond to the $w_n$-twisted conjugacy class $[w]$. Let $w \circ w_n$ have cycle decomposition $\lambda$ 
	and $\theta$ decompose as a product over $\lambda_i$ of $\alpha_i$ if $\lambda_i$ is even and $\theta_i$ if $\lambda_i$ is odd. 
	Write $\chi = \chi'+\chi_{\mathbb{F}}\circ \FieldNorm{\quadraticExtension}{\mathbb{F}}$ and $\chi'^{-1} = (\chi')^c$. 
	Then
	$$\rho_* = \bigoplus_{\lambda_i} \phi_{U,\lambda_i}$$
	and applying \Cref{lem:cochar_computation}
	\begin{align*}
		g_{T,\rho}(\chi,\theta,\psi) &= \Pi_{\lambda_i} \begin{cases}
			\GaussSumCharacter{\left(\alpha_i \times \chi'\right)}{\left(\chi_{\mathbb{F}}\circ \FieldNorm{\quadraticExtension}{\mathbb{F}}\right)}{\fieldCharacter_{\quadraticFieldExtension{\lambda_i/2}}} \times \GaussSumCharacter{\left(\alpha_i^{-1}\times \chi'^c\right)}{\left(\chi_{\mathbb{F}}\circ \FieldNorm{\quadraticExtension}{\mathbb{F}}\right)}{\fieldCharacter_{\quadraticFieldExtension{\lambda_i/2}}} & \lambda_i \textnormal{ even} \\
			\GaussSumCharacter{\left(\theta'_i \times \chi'\right)}{\left(\chi_{\mathbb{F}}\circ \FieldNorm{\quadraticExtension}{\mathbb{F}}\right)}{\fieldCharacter_{2\lambda_i}} & \lambda_i \textnormal{ odd}
		\end{cases}
	\end{align*}
	which is equal to $g_{T}(\chi^{-1},\theta^{-1},\psi)$.\
	\end{comment}
	\item[($\algebraicGroup{SO}_{2n+1}$)] Given $(T,\theta)$, let $T$ correspond to the conjugacy class $[w]$ with cycle decomposition $(\lambda^+,\lambda^-)$ 
	and $\theta$ decompose as a product $\Pi_{\lambda_i^+ }\alpha_{i+} \times \Pi_{\lambda_i^-}\theta_{i-}$. Then
	$$\rho_* = \bigoplus_{\lambda_i^+} \phi_{B,\lambda_i^+} \oplus \bigoplus_{\lambda_i^-} \phi_{B,\lambda_i^-} $$
	and applying \Cref{lem:cochar_computation}
	\begin{align*}
		g_{T,\rho}(\chi,\theta,\psi) &= \Pi_{\lambda_i^+} \GaussSumCharacter{\alpha_i}{\chi}{\fieldCharacter_{\quadraticFieldExtension{\lambda_i^+}}} \times \GaussSumCharacter{\alpha_i^{-1}}{\chi}{\fieldCharacter_{\quadraticFieldExtension{\lambda_i^+}}} \times \Pi_{\lambda_i^-} \GaussSumCharacter{\theta'_i}{\chi}{\fieldCharacter_{2\lambda_i^-}} 
	\end{align*}
	which is equal to $g_{T}(\chi^{-1},\theta^{-1},\psi)$.
	\item[($\algebraicGroup{Sp}$)] Given $(T,\theta)$, let $T$ correspond to the conjugacy class $[w]$ with cycle decomposition $(\lambda^+,\lambda^-)$ 
	and $\theta$ decompose as a product $\Pi_{\lambda_i^+ }\alpha_{i+} \times \Pi_{\lambda_i^-}\theta_{i-}$. Then
	$$\rho_* = \bigoplus_{\lambda_i^+} \phi_{B,\lambda_i^+} \oplus \bigoplus_{\lambda_i^-} \phi_{B,\lambda_i^-} \oplus \left(0:X_*(\algebraicGroup{T}) \to \mathbb{Z}\right)$$
	applying \Cref{lem:cochar_computation}
	\begin{align*}
		g_{T,\rho}(\chi,\theta,\psi) &= \Pi_{\lambda_i^+} \GaussSumCharacter{\alpha_i}{\chi}{\fieldCharacter_{\quadraticFieldExtension{\lambda_i^+}}} \times \GaussSumCharacter{\alpha_i^{-1}}{\chi}{\fieldCharacter_{\quadraticFieldExtension{\lambda_i^+}}} \times \Pi_{\lambda_i^-} \GaussSumCharacter{\theta'_i}{\chi}{\fieldCharacter_{2\lambda_i^-}} \times \tau(\chi,\psi)
	\end{align*}
	which is equal to $g_{T}(\chi^{-1},\theta^{-1},\psi) \cdot \tau(\chi,\psi)$. \calvin{check inverses}
	\item[($\algebraicGroup{SO}_{2n}^+$)] Let $T$ correspond to the signed permutation conjugacy class $[w]$ with cycle decomposition $(\lambda^+,\lambda^-)$ 
	and $\theta$ decompose as a product $\Pi_{\lambda_i^+ }\alpha_{i+} \times \Pi_{\lambda_i^-}\theta_{i-}$. Then
	$$\rho_* = \bigoplus_{\lambda_i^+} \phi_{B,\lambda_i^+} \oplus \bigoplus_{\lambda_i^-} \phi_{B,\lambda_i^-}$$
	applying \Cref{lem:cochar_computation}
	\begin{align*}
		g_{T,\rho}(\chi,\theta,\psi) &= \Pi_{\lambda_i^+} \GaussSumCharacter{\alpha_i}{\chi}{\fieldCharacter_{\quadraticFieldExtension{\lambda_i^+}}} \times \GaussSumCharacter{\alpha_i^{-1}}{\chi}{\fieldCharacter_{\quadraticFieldExtension{\lambda_i^+}}} \times \Pi_{\lambda_i^-} \GaussSumCharacter{\theta'_i}{\chi}{\fieldCharacter_{2\lambda_i^-}} 
	\end{align*}
	which is equal to $g_{T}(\chi^{-1},\theta^{-1},\psi)$.
	\item[($\algebraicGroup{SO}_{2n}^-$)] Let $T$ correspond to the $(-n,n)$-twisted permutation conjugacy class $[w]$ with cycle decompositionof $[w \cdot (-n,n)]$ equal to $(\lambda^+,\lambda^-)$ 
	and $\theta$ decompose as a product $\Pi_{\lambda_i^+ }\alpha_{i+} \times \Pi_{\lambda_i^-}\theta_{i-}$. Then
	$$\rho_* = \bigoplus_{\lambda_i^+} \phi_{B,\lambda_i^+} \oplus \bigoplus_{\lambda_i^-} \phi_{B,\lambda_i^-}$$
	applying \Cref{lem:cochar_computation}
	\begin{align*}
		g_{T,\rho}(\chi,\theta,\psi) &= \Pi_{\lambda_i^+} \GaussSumCharacter{\alpha_i}{\chi}{\fieldCharacter_{\quadraticFieldExtension{\lambda_i^+}}} \times \GaussSumCharacter{\alpha_i^{-1}}{\chi}{\fieldCharacter_{\quadraticFieldExtension{\lambda_i^+}}} \times \Pi_{\lambda_i^-} \GaussSumCharacter{\theta'_i}{\chi}{\fieldCharacter_{2\lambda_i^-}} 
	\end{align*}
	which is equal to $g_{T}(\chi^{-1},\theta^{-1},\psi)$.
\end{enumerate}
   
\begin{comment}
We will have to treat the unitary group case seperatedly. First for the non-unitary group case:\\



Write $T \cong \prod_{j=1}^r \multiplicativegroup{\finiteFieldExtension{k_j}} \times \prod_{i=1}^s \finiteFieldExtension{2m_i}^1 \left[\times \{\pm 1\}\right]$ with $k_1 \geq k_2 \hdots \geq k_r$ and $m_1 \geq m_2 \hdots \geq m_r$ and under the above isomorphism $\theta = \alpha_1 \times \dots \times \alpha_r \times \theta_1 \times \dots \times \theta_s \times \epsilon$, where  $\left[\epsilon \colon \left\{\pm 1\right\} \to \multiplicativegroup{\cComplex},\right]$ $\alpha_j \colon \multiplicativegroup{\finiteFieldExtension{k_j}} \to \multiplicativegroup{\cComplex}$ and $\theta_i \colon \finiteFieldExtension{2m_i}^1 \to \multiplicativegroup{\cComplex}$. Then the Frobenius conjugacy class of the Weyl group corresponding to $T$ is represented by $((k_1,k_2, \hdots, k_r), (m_1,m_2 \hdots m_s))$. The map $\rho$ sends this conjugacy class (or union of two conjugacy classes) to the conjugacy class with cycle decomposition the union of cycles of lengths $k_1,k_1,k_2,k_2 \hdots, k_r,k_r, 2m_1,2m_2 \hdots 2m_r$ along with a cycle of length $1$ in the $Sp$ and $GSp$ case. Let $\tilde{T}$ be the torus in $\GL$ corresponding to the conjugacy class above and let $\tilde{\theta}$ be the character of $\tilde{T}$ corresponding to $\Phi(s)$. Then by theorem \ref{thm:thm:computation-of-doubling-gauss-sum-scalar-for-deligne-lusztig-characters}
\[
    \sum_{t \in \tilde{T}} \tilde{\theta}(t) \chi(det(t-1)) = \begin{cases}
        g_T(\theta,\psi)  & G \neq Sp \textnormal{ or } GSp\\
        g_T(\theta,\psi) \cdot g(1,\psi_1) & G = Sp \textnormal{ or } GSp
    \end{cases}.
\]
Since the sum above is a Gauss sum for $\GL$ it is equal to the Gauss sum attatched to the semisimple element $\Phi(s)$. \\
Now for the unitary case: \calvin{this needs to be checked... I used engineers induction to guess at how the Weyl group conjugacy classes match the torus}
Write $T \cong \prod_{j=1}^r \multiplicativegroup{\finiteFieldExtension{k_j}} \times \prod_{i=1}^s (\finiteFieldExtension{2m_i}^1)^2$ with $k_i$'s even and $m_i$'s odd \calvin{this is what I need to check... what tori you can get here} and under the above isomorphism $\theta = \alpha_1 \times \dots \times \alpha_r \times \theta_1 \times \dots \times \theta_s \times \epsilon$, where  $\left[\epsilon \colon \left\{\pm 1\right\} \to \multiplicativegroup{\cComplex},\right]$ $\alpha_j \colon \multiplicativegroup{\finiteFieldExtension{k_j}} \to \multiplicativegroup{\cComplex}$ and $\theta_i \colon \finiteFieldExtension{2m_i}^1 \to \multiplicativegroup{\cComplex}$. Then the Frobenius conjugacy class of the Weyl group corresponding to $T$ is represented by having even cycles of lengths $k_1,k_2 \hdots k_r$ and odd cycles of lengths $m_1,m_1,m_2,m_2, \hdots, m_s,m_s$. The map $\Phi$ sends this conjugacy class to the conjugacy class with even cycles $k_1,k_1,k_2,k_2 \hdots k_r,k_r$ and odd cycles $m_1,m_1,m_1,m_1,m_2,m_2,m_2,m_2, \hdots, m_s,m_s,m_s,m_s$. Let $\tilde{T}$ be the torus in $\GL$ corresponding to the conjugacy class with all even cycles of lengths $k_1,k_1,k_2,k_2 \hdots k_r,k_r$ and $2m_1,2m_1,2m_2,2m_2, \hdots, 2m_s,2m_s$. Then $\Phi(s)$ is in this torus \calvin{write out what that means} giving a character $\tilde{\theta}$ on it and 
\[
    \sum_{t \in \tilde{T}} \tilde{\theta}(t) \chi(det(t-1)) = g_T(\theta,\psi).
\]
Since the sum above is a Gauss sum for $\GL$ it is equal to the Gauss sum attatched to the semisimple element $\Phi(s)$.
\end{comment}
\end{proof}

\begin{comment}
\calvin{start alternate writeup of section 3.4/3.4.1:}
\subsection{Computation for Lusztig Series}
The goal of this section is to prove \Cref{thm:Phi_stable} in particular:
\begin{theorem}
\label{thm:doubling-method-gamma-factor-for-deligne-lusztig} If $c_{\pi}$ is nonzero in the virtual representation $R_T(\theta) = \sum c_{\pi} \pi$ then
$$\dblGammaFactorSpace{\hermitianSpace}{\pi}{\chi}{\fieldCharacter} = g_T\left(\theta, \fieldCharacter\right)$$
\end{theorem}
\calvin{add in $\chi$} the right hand side of which we showed in \Cref{lem:indep_geo_conj} only depends on the geometric conjugacy class of $(T,\theta)$ hence on the Lusztig series $\LusztigSeries{G}{s}$ which contains $\pi$.
\subsubsection{Proof of \Cref{thm:doubling-method-gamma-factor-for-deligne-lusztig}, \Cref{thm:Phi_stable}}
Our proof will be via induction on $\dim_{\quadraticExtension} \hermitianSpace$. First we will use \Cref{thm:multiplicativity-in-terms-of-gamma-factors}  and MULTIPLICAATIVITY-OF-GTTHETA to prove \Cref{thm:doubling-method-gamma-factor-for-deligne-lusztig} for any non cuspidal $\pi$ assuming \Cref{thm:doubling-method-gamma-factor-for-deligne-lusztig} for all Levi factors. \\

\begin{proof}
\calvin{should be short proof. Might not even need to put--just need to say both are multiplicative}
\end{proof}


By \Cref{lem:indep_geo_conj}, we know that for $R_T(\theta) = \sum c_{\pi} \pi$, 
$$\sum c_{\pi} \dblGammaFactorSpace{\hermitianSpace}{\pi}{\chi}{\fieldCharacter} = (\sum c_{\pi})g_T\left(\theta, \fieldCharacter\right)$$
is independent of geometric conjugacy class of $(T,\theta)$. Since we know the relevant equality for non cupsidal $\pi$, we get that
$$ \sum_{\pi \textnormal{ cuspidal}} c_{\pi} \dblGammaFactorSpace{\hermitianSpace}{\pi}{\chi}{\fieldCharacter} = \left(\sum_{\pi \textnormal{ cuspidal}} c_{\pi}\right)g_T\left(\theta, \fieldCharacter\right)$$.

To finish we will show that all cuspidal $\pi$ in a Lusztig packet $\LusztigSeries{G}{s}$ have the same $\dblGammaFactorSpace{\hermitianSpace}{\pi}{\chi}{\fieldCharacter}$ factor. This will follow from

\begin{proposition}[{\cite[Page 172]{Lusztig1977}}]
	Let $\GroupExtension{G}$ have connected center. For any semisimple element $s \in \DualFrobeniusFixedPoints{\GroupExtension{G}}$, the Lusztig series $\LusztigSeries{\FrobeniusFixedPoints{\GroupExtension{G}}}{s}$ contains at most one cuspidal element.
\end{proposition}

While our groups $G$ may not have connected center, the function $\Phi(g)$ extends by zero to a class function on $\tilde{G}$ (see section 3.2) which does have connected center. Hence if a representation $\pi$ of $\tilde{G}$ deposes intorepresentations $\pi_1 \oplus \pi_2 \hdots \oplus \pi_r$ of $G$, $\dblGammaFactorSpace{\hermitianSpace}{\pi_1}{\chi}{\fieldCharacter} = \dblGammaFactorSpace{\hermitianSpace}{\pi_2}{\chi}{\fieldCharacter} =\hdots =\dblGammaFactorSpace{\hermitianSpace}{\pi_r}{\chi}{\fieldCharacter}$.

\begin{proposition}[{\cite[Proposition 11.7]{Bonnafe2006}}]
	Let $\tilde{s}$ be a semisimple element of $\DualFrobeniusFixedPoints{\GroupExtension{G}}$ and let $s = i^{\ast}\left(\tilde{s}\right)$. Then
	\begin{enumerate}
		\item If $\tilde{\pi} \in \LusztigSeries{\FrobeniusFixedPoints{\GroupExtension{G}}}{\tilde{s}}$ and if $\pi$ is an irreducible subrepresentation of the restriction of $\tilde{\pi}$ to $\FrobeniusFixedPoints{G}$, then $\pi \in \LusztigSeries{\FrobeniusFixedPoints{G}}{s}$.
		\item Let $\pi \in \LusztigSeries{\FrobeniusFixedPoints{G}}{s}$. Then there exists $\tilde{\pi} \in \LusztigSeries{\FrobeniusFixedPoints{\GroupExtension{G}}}{\tilde{s}}$ such that $\pi$ is an irreducible subrepresentation of the restriction of $\tilde{\pi}$ to $\FrobeniusFixedPoints{G}$.
	\end{enumerate}
\end{proposition}
\calvin{maybe say that cuspidal breaks into cuspidal pieces}
So all the cuspidal $\pi$ in a Lusztig packet $\LusztigSeries{G}{s}$ are irreducible pieces in the restriction from $\GroupExtension{G}$ to $G$ of a single cuspidal representation $\tilde{\pi}$ of $\GroupExtension{G}$.

\calvin{end alternate writeup}
\end{comment}
\calvin{start alternate writeup 2 of section 3.4/3.4.1}
\subsection{Stability (one option)}\label{subsec:stability}
The goal of this section is to compute $\dblJacobiSumScalar{\pi}{\fieldCharacter}$ proving \Cref{thm:Phi_stable}:
\begin{theorem}
\label{thm:doubling-method-gamma-factor-for-deligne-lusztig} For either $G=G$ or its extension $\GroupExtension{G}$. If $c_{\pi}$ is nonzero in the virtual representation $R_T(\theta) = \sum c_{\pi} \pi$ then
$$\dblGammaFactorSpace{\hermitianSpace}{\pi}{\chi}{\fieldCharacter} = g_{T \cap G}\left(\chi,\theta|_{T \cap G}, \fieldCharacter\right)$$
\end{theorem}
\calvin{add in $\chi$} the right hand side of which we showed in \Cref{lem:indep_geo_conj} only depends on the geometric conjugacy class of $(T,\theta)$ hence on the Lusztig series $\LusztigSeries{G}{s}$ which contains $\pi$, implying \Cref{thm:Phi_stable}.

The proof strategy is to prove the following inductive statements
\begin{enumerate}[(A)]
	\item If for all representations with $\dim_{\mathbb{E}} V < n$, \Cref{thm:doubling-method-gamma-factor-for-deligne-lusztig} holds, then \Cref{thm:doubling-method-gamma-factor-for-deligne-lusztig} holds for all non cuspidal representations with $\dim_{\mathbb{E}} V = n$.
	\item If \Cref{thm:doubling-method-gamma-factor-for-deligne-lusztig} holds for all non cuspidal representations with $\dim_{\mathbb{E}} V = n$, then it holds for all cuspidal representations with $\dim_{\mathbb{E}} V = n$.
\end{enumerate}
The key input to (A) is the multiplicativity [CITE] of the $\dblGammaFactorSpace{\hermitianSpace}{\pi}{\chi}{\fieldCharacter}$ factor. The key input to (B) is \Cref{lem:indep_geo_conj} along with results of Lusztig controlling the cupsidal elements in a Lusztig series.

\subsubsection{Proof of \Cref{thm:doubling-method-gamma-factor-for-deligne-lusztig}}
We first reduce \Cref{thm:doubling-method-gamma-factor-for-deligne-lusztig} to the case $G$ has connected center. For any irreducible representation $\pi$ of $G$ which is the a component of an irreducible representation $\tilde{\pi}$ of $\GroupExtension{G}$ restricted to $\pi$,
\[
    \dblGammaFactorSpace{\hermitianSpace}{\pi}{\chi}{\fieldCharacter} = \dblGammaFactorSpace{\hermitianSpace}{\tilde{\pi}}{\chi}{\fieldCharacter}.
\]

The following proposition implies that if $\pi \in \LusztigSeries{\FrobeniusFixedPoints{G}}{s}$ then there exists $\tilde{\pi} \in \LusztigSeries{\FrobeniusFixedPoints{G}}{\tilde{s}}$ for $\tilde{s}$ only depending on $s$. Thus \Cref{thm:doubling-method-gamma-factor-for-deligne-lusztig} for $G$ follows from \Cref{thm:doubling-method-gamma-factor-for-deligne-lusztig} for $\GroupExtension{G}$ (see section 3.2). 

\begin{proposition}[{\cite[Proposition 11.7]{Bonnafe2006}}]
	Let $\tilde{s}$ be a semisimple element of $\DualFrobeniusFixedPoints{\GroupExtension{G}}$ and let $s = i^{\ast}\left(\tilde{s}\right)$. Then
	\begin{enumerate}
		\item If $\tilde{\pi} \in \LusztigSeries{\FrobeniusFixedPoints{\GroupExtension{G}}}{\tilde{s}}$ and if $\pi$ is an irreducible subrepresentation of the restriction of $\tilde{\pi}$ to $\FrobeniusFixedPoints{G}$, then $\pi \in \LusztigSeries{\FrobeniusFixedPoints{G}}{s}$.
		\item Let $\pi \in \LusztigSeries{\FrobeniusFixedPoints{G}}{s}$. Then there exists $\tilde{\pi} \in \LusztigSeries{\FrobeniusFixedPoints{\GroupExtension{G}}}{\tilde{s}}$ such that $\pi$ is an irreducible subrepresentation of the restriction of $\tilde{\pi}$ to $\FrobeniusFixedPoints{G}$.
	\end{enumerate}
\end{proposition}

We now prove \Cref{thm:doubling-method-gamma-factor-for-deligne-lusztig} for the extension $\GroupExtension{G}$.

Our proof will be via induction as descriped in \ref{subsec:stability}.
%First we will use \Cref{thm:multiplicativity-in-terms-of-gamma-factors}  and MULTIPLICAATIVITY-OF-GTTHETA to prove \Cref{thm:doubling-method-gamma-factor-for-deligne-lusztig} for any non cuspidal $\pi$ assuming \Cref{thm:doubling-method-gamma-factor-for-deligne-lusztig} for all Levi factors. \\
\begin{proof}[Proof of (A)]
\calvin{should be short proof. Might not even need to put--just need to say both are multiplicative}
\end{proof}
\begin{proof}[Proof of (B)]

%Combining this with our inductive hypothesis implies \Cref{thm:doubling-method-gamma-factor-for-deligne-lusztig} for non cupsidal $\pi$.
By \Cref{lem:indep_geo_conj}, we know that for $R_T(\theta) = \sum c_{\pi} \pi$, 
$$\sum c_{\pi} \dblGammaFactorSpace{\hermitianSpace}{\pi}{\chi}{\fieldCharacter} = (\sum c_{\pi})g_{T \cap G}\left(\theta|_{T \cap G}, \fieldCharacter\right).$$Subtracting our the equality for non-cupsidal $\pi$ we are left with:
$$ \sum_{\pi \textnormal{ cuspidal}} c_{\pi} \dblGammaFactorSpace{\hermitianSpace}{\pi}{\chi}{\fieldCharacter} = \left(\sum_{\pi \textnormal{ cuspidal}} c_{\pi}\right)g_{T \cap G}\left(\theta|_{T \cap G}, \fieldCharacter\right)$$.

But it is well known that there is only one cuspidal $\pi$ appearing in our sum:
\begin{proposition}[{\cite[Page 172]{Lusztig1977}}]
	Let $\GroupExtension{G}$ be a classical group with connected center. For any semisimple element $s \in \DualFrobeniusFixedPoints{\GroupExtension{G}}$, the Lusztig series $\LusztigSeries{\FrobeniusFixedPoints{\GroupExtension{G}}}{s}$ contains at most one cuspidal element.
\end{proposition}

So the proof of \Cref{thm:doubling-method-gamma-factor-for-deligne-lusztig} is complete.
\end{proof}
\calvin{end alternate writeup 2}

\subsection{Stability (another option)}
\calvin{retitle: Stability of Jacobi sums}

The goal of this section is to use \Cref{thm:computation-of-doubling-gauss-sum-scalar-for-deligne-lusztig-characters} and results of Lusztig to determine $\dblJacobiSumScalar{\pi}{\fieldCharacter}$ for any irreducible representation $\pi$ such that $\trace \pi$ appears with non-zero coefficient in $R_{T,\theta}$.

\begin{theorem}\label{thm:doubling-method-gamma-factor-for-deligne-lusztig}
	Let $T \subset H$ be a maximal torus and let $\theta \colon T \to \multiplicativegroup{\cComplex}$ be a character. Write $$R_{T,\theta} = \sum_{\Pi} c_{\Pi} \trace \Pi,$$
	where $\Pi$ goes over all the irreducible representations of $H$ and $c_{\Pi} \in \zIntegers$. Suppose that $\pi$ is an irreducible representation such that $c_{\pi} \ne 0$. Then $$\dblGammaFactorSpace{\hermitianSpace}{\pi}{\chi}{\fieldCharacter} = g_T\left(\theta, \fieldCharacter\right).$$
\end{theorem}
\subsubsection{Proof of the main result \calvin{\Cref{thm:Phi_stable}}}

\elad{Merge with Calvin's stuff}
The \emph{Lusztig series} $\LusztigSeries{G}{s}$ consists of (equivalence classes of) the irreducible representations $\pi$ of $R_{T, \theta}$ with $c_{\pi} \ne 0$ as in the theorem.

Let $L \subset G$ be a Levi subgroup of $G$, and suppose that $L \cong \prod_{j=1}^\ell \GL_{t_j}\left(\finiteField\right) \times G'$ where $G' = \IsometryGroup\left(\hermitianSpace'\right)$ where $\hermitianSpace'$ is a non-degenerate hermitian subspace and $\hermitianSpace = \xIsotropic \oplus \hermitianSpace' \oplus \yIsotropic$, where $\xIsotropic$ and $\yIsotropic$ are totally isotropic subspace in duality. Suppose that $s \in L$, and denote by $s_i$ the projection of $s$ to $\GL_{t_i}\left(\finiteField\right)$ for $1 \le i \le \ell$ and by $s_0$ the projection of $s$ to $G'$. We denote
$$\LusztigSeries{L}{s} = \left(\bigotimes_{j=1}^{\ell} \LusztigSeries{\GL_{t_j}\left(\finiteField\right)}{s_j}\right) \otimes \LusztigSeries{G'}{s_0}.$$

We collect some facts that will be used for proving \Cref{thm:doubling-method-gamma-factor-for-deligne-lusztig}.

The following proposition allows us to make use of the multiplicativity property (\Cref{thm:multiplicativity-in-terms-of-gamma-factors}) for non-cuspidal elements in a given Lusztig series.
\begin{proposition}[{\cite[Proof of Corollary 3.3.21]{GeckMalle2020}}]
	For any semisimple element $s \in \DualFrobeniusFixedPoints{\GroupExtension{G}}$, the Lusztig series $\LusztigSeries{\FrobeniusFixedPoints{\GroupExtension{G}}}{s}$ is a union of Harrish--Chandra series corresponding to Harrish-Chandra data of the form $\left(\FrobeniusFixedPoints{\algebraicGroup{P}},\sigma\right)$ where $\algebraicGroup{P}$ is a parabolic subgroup of $\algebraicGroup{\GroupExtension{G}}$ with Levi part $\algebraicGroup{L}$ with $s \in \DualFrobeniusFixedPoints{L}$ and $\sigma \in \LusztigSeries{\FrobeniusFixedPoints{L}}{s}$ is an irreducible cuspidal representation.
\end{proposition}

The following Proposition shows that is suffices to prove \Cref{thm:doubling-method-gamma-factor-for-deligne-lusztig} for the group $\GroupExtension{G}$.
\begin{proposition}[{\cite[Proposition 11.7]{Bonnafe2006}}]
	Let $\tilde{s}$ be a semisimple element of $\DualFrobeniusFixedPoints{\GroupExtension{G}}$ and let $s = i^{\ast}\left(\tilde{s}\right)$. Then
	\begin{enumerate}
		\item If $\tilde{\pi} \in \LusztigSeries{\FrobeniusFixedPoints{\GroupExtension{G}}}{\tilde{s}}$ and if $\pi$ is an irreducible subrepresentation of the restriction of $\tilde{\pi}$ to $\FrobeniusFixedPoints{G}$, then $\pi \in \LusztigSeries{\FrobeniusFixedPoints{G}}{s}$.
		\item Let $\pi \in \LusztigSeries{\FrobeniusFixedPoints{G}}{s}$. Then there exists $\tilde{\pi} \in \LusztigSeries{\FrobeniusFixedPoints{\GroupExtension{G}}}{\tilde{s}}$ such that $\pi$ is an irreducible subrepresentation of the restriction of $\tilde{\pi}$ to $\FrobeniusFixedPoints{G}$.
	\end{enumerate}
\end{proposition}

Finally, the following proposition shows that the multiplicativity property takes care of all but at most one element in a given Lusztig series.
\begin{proposition}[{\cite[Page 172]{Lusztig1977}}]
	For any semisimple element $s \in \DualFrobeniusFixedPoints{\GroupExtension{G}}$, the Lusztig series $\LusztigSeries{\FrobeniusFixedPoints{\GroupExtension{G}}}{s}$ contains at most one cuspidal element.
\end{proposition}

\Cref{thm:doubling-method-gamma-factor-for-deligne-lusztig} is now equivalent to the following statement.
\begin{proposition}\label{prop:gamma-factor-is-constant-on-lusztig-series}
	Keep the notations as above. Suppose that $\algebraicGroup{P}$ is a parabolic subgroup of $\algebraicGroup{\GroupExtension{G}}$ with Levi part $\algebraicGroup{L}$ with $s \in \DualFrobeniusFixedPoints{L}$. Then for any irreducible cuspidal $\sigma \in \LusztigSeries{\FrobeniusFixedPoints{L}}{s}$ and for any irreducible subrepresentation $\pi \subset \Ind{\FrobeniusFixedPoints{P}}{\FrobeniusFixedPoints{\GroupExtension{G}}}{\sigma}$, $$\dblGammaFactorSpace{\hermitianSpace}{\pi}{\chi}{\fieldCharacter} = \tau_{\GroupExtension{G}, \chi, \fieldCharacter}\left(s\right).$$
	\elad{Many things are not defined anymore. Need to define $\tau_{\GroupExtension{G}, \chi, \fieldCharacter}\left(s\right)$, etc.}
\end{proposition}
\begin{proof}
	The proof is by induction on $\dim_{\quadraticExtension} \hermitianSpace$.
	
	Suppose first that $L \ne G$. Then by using the multiplicativity property repeatedly (\Cref{thm:multiplicativity-in-terms-of-gamma-factors}), we have $$\dblGammaFactorSpace{\hermitianSpace}{\pi}{\chi}{\fieldCharacter} = \dblGammaFactorSpace{\hermitianSpace'}{\pi'}{\chi}{\fieldCharacter} \cdot \prod_{j=1}^r \tau_{\GL_{t_j}\left(\finiteField\right), \chi, \fieldCharacter}\left(s_j\right) \tau_{\GL_{t_j}\left(\finiteField\right), \chi, \fieldCharacter}\left(s_j^{-1}\right),$$
	where $\pi'$ is an irreducible cuspidal representation of $G'$ that lies in the Lusztig series $\LusztigSeries{\pi'}{s_0}$. By the induction hypothesis, we have $$\dblGammaFactorSpace{\hermitianSpace'}{\pi'}{\chi}{\fieldCharacter} = \tau_{G', \chi, \fieldCharacter}\left(s_0\right),$$
	which implies the result.
	
	Next, suppose that $L = G$ and that $\pi = \sigma$ is an irreducible cuspidal representation. Let us write $$R_{T, \theta} = \sum_{\Pi} c_\Pi \trace \Pi,$$
	where $\Pi$ goes over all the irreducible representations of $G$ and $c_{\Pi} \in \zIntegers$. For every $\Pi$ with $c_{\Pi} \ne 0$ and $\Pi \ne \pi$, we have that $\Pi$ lies in a Harish--Chandra series with $L \ne G$ as above, and therefore $$\dblVirtualJacobiSumScalar{\trace \Pi}{\chi} = \dim \Pi \cdot \centralCharacter{\Pi}\left(-1\right) \cdot c_{\hermitianSpace}\left(\chi, \fieldCharacter\right) \cdot \tau_{G, \chi, \fieldCharacter}\left(s\right).$$
	By \Cref{thm:computation-of-doubling-gauss-sum-scalar-for-deligne-lusztig-characters}, we have $$\dblVirtualJacobiSumScalar{R_{T,\theta}}{\fieldCharacter} = R_{T,\theta}\left(1\right) \cdot \theta\left(-1\right) \cdot c_{\hermitianSpace}\left(\chi, \fieldCharacter\right) \cdot \tau_{G, \chi, \fieldCharacter}\left(s\right).$$
	Since $\centralCharacter{\Pi}\left(-1\right) = \theta\left(-1\right)$ for every $\Pi$ with $c_{\Pi} \ne 0$,
	$$c_{\pi} \cdot \dblVirtualJacobiSumScalar{\trace \pi}{\fieldCharacter} = \left(R_{T,\theta}\left(1\right) - \sum_{\substack{\Pi \ne \pi\\
	c_{\Pi} \ne 0}} c_{\Pi} \dim \Pi\right) \cdot \theta\left(-1\right) \cdot c_{\hermitianSpace}\left(\chi, \fieldCharacter\right) \cdot \tau_{G, \chi, \fieldCharacter}\left(s\right).$$
	Since $$R_{T,\theta}\left(1\right) = \sum_{\Pi} c_{\Pi} \trace \Pi,$$ it follows that $$\dblVirtualJacobiSumScalar{\trace \pi}{\chi} = \dim \pi \cdot \centralCharacter{\pi}\left(-1\right) \cdot c_{\hermitianSpace}\left(\chi, \fieldCharacter\right) \cdot \tau_{G, \chi, \fieldCharacter}\left(s\right).$$
	Since $\dblVirtualJacobiSumScalar{\trace \pi}{\chi} = \dim \pi \cdot \dblJacobiSumScalar{\pi}{\chi}$, the result follows.
\end{proof}

\subsection{Kloosterman type sums}

Let $\chi_1, \dots, \chi_k \colon \multiplicativegroup{\finiteField} \to \multiplicativegroup{\cComplex}$ be characters such that $\chi_i^2 \ne 1$ for any $i$. Consider the assignment $K \colon G \to \cComplex$ given by $$K\left(x\right) = K_{\chi_1,\dots,\chi_k}\left(x\right) = \sum_{\substack{g_1, \dots, g_k \in G\\
g_1 \dots g_k = x}} \Phi_{\chi_1}\left(g_1\right) \dots \Phi_{\chi_k}\left(g_k\right).$$
It is clear that $K$ is a class function of $G$. Let $\pi$ be an irreducible representation of $G$. Then \begin{align*}
	 \innerproduct{K}{\conjugate{\trace \pi}} &= \frac{1}{\sizeof{G}}\trace \left(\sum_{g_1,\dots,g_k \in G} \Phi_{\chi_1}\left(g_1\right) \dots \Phi_{\chi_k}\left(g_k\right) \pi\left(g_1 \dots g_k\right) \right) \\
	 &= \frac{\sizeof{\lieAlgebra}^{\frac{k}{2}}}{\sizeof{G}} \trace\left( \dblJacobiSum{\pi}{\chi_1} \circ \dots \dblJacobiSum{\pi}{\chi_k} \right).
\end{align*}
Thus $$\innerproduct{K}{\conjugate{\trace \pi}} = \frac{\sizeof{\lieAlgebra}^{\frac{k}{2}} \dim \pi}{\sizeof{G}} \cdot \prod_{j=1}^k \dblJacobiSumScalar{\pi}{\chi_j}.$$
It follows that if
$$\DeligneLusztigInduction{T}{G}\theta = \sum_{\pi} c_{\pi} \cdot \trace \pi,$$
then $$\innerproduct{K}{\conjugate{\DeligneLusztigInduction{T}{G}\theta}} = \frac{\sizeof{\lieAlgebra}^{\frac{k}{2}} \prod_{j=1}^k \dblJacobiSumScalar{\pi}{\chi_j}}{\sizeof{G}} \sum_{\pi} c_{\pi} \dim \pi = \frac{\sizeof{\lieAlgebra}^{\frac{k}{2}} \prod_{j=1}^k \dblJacobiSumScalar{\pi}{\chi_j}}{\sizeof{G}} \left(\DeligneLusztigInduction{T}{G}\theta\right)\left(1\right),$$
where $\pi$ is any irreducible representation of $G$ such that $c_{\pi} \ne 0$.

Suppose that $T \cong \NormOneGroup{2m}$, and let $t \in T$ be an element such that $t$ is contained only in the torus $T$. Then
$$K\left(t\right) = \sum_{\theta} \theta\left(t\right) \innerproduct{K}{\DeligneLusztigInduction{T}{G}\left(\theta\right)},$$
where the sum is over all the characters of $T$. Using the formula $$\dblJacobiSumScalar{\pi}{\chi} = \tau\left( \theta' \times \chi^{-1}, \fieldCharacter_{2m} \right) \theta\left(-1\right) \tau\left(\chi^2, \fieldCharacter\right)^m,$$
we get
$$K\left(t\right) = \frac{\sizeof{\lieAlgebra}^{\frac{k}{2}} \cdot \DeligneLusztigInduction{T}{G}\left(\theta\right)\left(1\right)}{\sizeof{G}} \sum_{\theta} \theta\left( \left(-1\right)^k t\right) \prod_{j=1}^k \tau\left(\theta' \times \chi_j^{-1}, \fieldCharacter_{2m}\right) \tau\left(\chi_j^2, \fieldCharacter\right)^m.$$

Expanding the last sum, we have
\begin{align*}
	K\left(t\right) = & \frac{\left(-1\right)^k q^{-km}  \sizeof{\lieAlgebra}^{\frac{k}{2}} \DeligneLusztigInduction{T}{G}\left(\theta\right)\left(1\right)}{\sizeof{G}} \cdot \prod_{j=1}^k \tau\left(\chi_j^2, \fieldCharacter\right)^m \\
	 & \times \sum_{\theta} \sum_{\xi_1,\dots,\xi_k \in \multiplicativegroup{\finiteFieldExtension{2m}}} \theta\left( \left(-1\right)^k t \xi_1^{1-q^m} \dots \xi_k^{1-q^m} \right) \fieldCharacter_{2m}\left(\sum_{j=1}^k \xi_j\right) \prod_{j=1}^k \chi_j\left(\FieldNorm{2m}{1}\left(\xi_1^{-1} \right)\right).
\end{align*}
The sum on the second row becomes
$$\sum_{\substack{\xi_1,\dots,\xi_k \in \multiplicativegroup{\finiteFieldExtension{2m}}\\
\prod_{j=1}^k \xi_j^{1-q^m} = \left(-1\right)^k t^{-1}}} \left(\prod_{j=1}^k \chi_j^{-1}\left(\FieldNorm{2m}{1}\left(\xi_j \right)\right)\right) \fieldCharacter_{2m}\left(\sum_{j=1}^k \xi_j\right).$$

\appendix
\section{Evaluation of norm one exponential sums}

In this appendix, we evaluate several exponential sums.

\subsection{The case $\quadraticExtension = \finiteField$}
Let $\chi \colon \multiplicativegroup{\finiteField} \to \multiplicativegroup{\cComplex}$ be a non-trivial character. We compute $$\sum_{\substack{x \in \multiplicativegroup{\finiteFieldExtension{2}}\\
	\trace_{\finiteFieldExtension{2} \slash \finiteField}\left(x\right) \ne 0}} \chi\left(\FieldNorm{2}{1}\left(\frac{x}{\trace_{\finiteFieldExtension{2} \slash \finiteField} x}\right) \right).$$
As usual, we rewrite this sum as
$$\frac{1}{q} \sum_{z \in \finiteField} \sum_{y \in \multiplicativegroup{\finiteField}} \sum_{x \in \multiplicativegroup{\finiteFieldExtension{2}}} \chi\left(\FieldNorm{2}{1}\left(xy\right)\right) \fieldCharacter\left(z\left(y \trace_{\finiteFieldExtension{2} \slash \finiteField} x - 1\right)\right).$$
It is clear that if $z = 0$, then the sum over $x$ will vanish. Hence, we may reduce the sum to a sum over $z \in \multiplicativegroup{\finiteField}$. By replacing $y$ with $z^{-1} y$ and then replacing $z$ with $-z$, we arrive at the sum
$$\frac{1}{q} \sum_{y \in \multiplicativegroup{\finiteField}} \sum_{x \in \multiplicativegroup{\finiteFieldExtension{2}}} \chi\left(\FieldNorm{2}{1}\left(xy\right)\right) \fieldCharacter\left(y \trace_{\finiteFieldExtension{2} \slash \finiteField} x\right) \sum_{z \in \multiplicativegroup{\finiteField}} \chi^{-2}\left(z\right)\fieldCharacter\left(z\right).$$
Replacing $x$ with $y^{-1}x$, and using the Hasse--Davenport relation, we arrive at the sum
$$q^{\frac{1}{2}} \left(q-1\right) \tau\left(\chi, \fieldCharacter\right)^2 \tau\left(\chi^{-2}, \fieldCharacter\right).$$

\subsection{The case $\quadraticExtension \ne \finiteField$}
Let $\chi \colon \multiplicativegroup{\finiteFieldExtension{2}} \to \multiplicativegroup{\cComplex}$ be a non-trivial character and let $\theta \colon \NormOneGroup{2} \to \multiplicativegroup{\cComplex}$ be a character. Our goal is to compute $$\sum_{-1 \ne x \in \NormOneGroup{2}} \theta\left(x\right) \chi\left(1+x\right).$$
Using the Hilbert 90 map this is equivalent to
$$\frac{1}{q-1} \sum_{\substack{x \in \multiplicativegroup{\finiteFieldExtension{2}}\\
x \notin \delta \multiplicativegroup{\finiteField}}} \theta\left(\frac{x}{x^q}\right) \chi\left(1+\frac{x}{x^q}\right),$$
Where $\delta \in \multiplicativegroup{\finiteFieldExtension{2}}$ is a trace zero element. 
We arrive at the sum
$$\frac{1}{q-1} \sum_{\substack{x \in \multiplicativegroup{\finiteFieldExtension{2}}\\
		x \notin \delta \multiplicativegroup{\finiteField}}} \theta'\left(x\right) \chi^{-q}\left(x\right) \chi\left(\trace_{\finiteFieldExtension{2} \slash \finiteField} x\right),$$
	where $\theta' \colon \multiplicativegroup{\finiteFieldExtension{2}} \to \multiplicativegroup{\cComplex}$ is given by $\theta'\left(x\right) = \theta\left(\frac{x}{x^q}\right)$.
	
	We have that $x \in \multiplicativegroup{\finiteFieldExtension{2}}$ satisfies $x \notin \delta \multiplicativegroup{\finiteField}$ if and only if $\trace_{\finiteFieldExtension{2} \slash \finiteField}\left(x\right) \ne 0$. As usual, we may use this last condition to rewrite the sum as
	$$\frac{1}{q\left(q-1\right)} \sum_{x \in \multiplicativegroup{\finiteFieldExtension{2}}}\sum_{z \in \finiteField} \sum_{y \in \multiplicativegroup{\finiteField}}  \theta'\left(x\right) \chi^{-q}\left(x\right) \chi\left(y\right) \fieldCharacter\left(z\left(y - \trace_{\finiteFieldExtension{2} \slash \finiteField} \left(x\right)\right)\right).$$
	Since $\chi \ne 1$, the inner sum vanishes when $z = 0$, and therefore we can reduce the sum over $z$ to $z \in \multiplicativegroup{\finiteField}$. Changing variables $x \mapsto -z^{-1} x$ and $y \mapsto z^{-1} y$, and using the fact that $z^q = z$, we arrive at the sum
	$$\frac{1}{q} \sum_{x \in \multiplicativegroup{\finiteFieldExtension{2}}}\theta'\left(x\right)  \chi^{-q}\left(x\right) \fieldCharacter\left(\trace_{\finiteFieldExtension{2} \slash \finiteField} \left(x\right)\right) \sum_{y \in \multiplicativegroup{\finiteField}}   \chi\left(y\right) \fieldCharacter\left(y\right),$$
	which by changing variables $x \mapsto x^q$ equals
	$$q^{\frac{1}{2}} \GaussSumCharacter{\left(\theta'\right)^{-1}}{\chi^{-1}}{\fieldCharacter_{2}} \tau\left(\chi \restriction_{\multiplicativegroup{\finiteField}}, \fieldCharacter\right).$$

\bibliographystyle{abbrv}
\bibliography{references}
\end{document}
