\documentclass[12pt, reqno]{amsart}

\usepackage{amsmath, amsthm, amssymb}
\usepackage{enumerate}
\tolerance=500
\setlength{\emergencystretch}{3em}
\usepackage[margin=1.0in]{geometry}
\usepackage{xcolor}
\definecolor{cite}{rgb}{0.30,0.60,1.00}
\definecolor{url}{rgb}{0.00,0.00,0.80}
\definecolor{link}{rgb}{0.40,0.10,0.20}
\usepackage[pdfusetitle,colorlinks,linkcolor=link,urlcolor=url,citecolor=cite,pagebackref,breaklinks]{hyperref}
\usepackage{graphicx}
\usepackage{cleveref}
\usepackage{mathdots}
\usepackage{tikz-cd}
\usepackage{comment}
\usepackage{xypic}
\usepackage{mathtools}

% new environment
\newtheorem{theorem}{Theorem}[section]
\newtheorem{claim}[theorem]{Claim}
\newtheorem{proposition}[theorem]{Proposition}
\newtheorem{lemma}[theorem]{Lemma}
\newtheorem{conjecture}[theorem]{Conjecture}
\newtheorem{corollary}[theorem]{Corollary}

\theoremstyle{definition}
\newtheorem{definition}[theorem]{Definition}
\theoremstyle{definition}
\newtheorem{remark}[theorem]{Remark}
\theoremstyle{definition}
\newtheorem{example}[theorem]{Example}

%% frequently used symbols
% Common math notions
\newcommand{\zIntegers}{\mathbb{Z}}
\newcommand{\rReal}{\mathbb{R}}
\newcommand{\cComplex}{\mathbb{C}}
\newcommand{\multiplicativegroup}[1]{#1^{\times}}
\newcommand{\RealPart}{\mathrm{Re}}
\newcommand{\Hom}{\mathrm{Hom}}
\newcommand{\EndomorphismRing}{\operatorname{End}}
\newcommand{\Span}{\mathrm{span}}
\newcommand{\Supp}{\mathrm{supp}}
\newcommand{\Stab}{\mathrm{stab}}
\newcommand{\idmap}{\mathrm{id}}
\newcommand{\conjugate}[1]{\overline{#1}}
\newcommand{\indicatorFunction}[1]{\delta_{#1}}
\newcommand{\isomorphic}{\cong}
\newcommand{\lengthof}{\mathfrak{n}}
\newcommand{\abs}[1]{\left|#1\right|}
\newcommand{\sizeof}[1]{\left|#1\right|}
\newcommand{\lcm}{\operatorname{lcm}}

% Inner product
\newcommand{\innerproduct}[2]{\left\langle #1,#2\right\rangle}
\newcommand{\Norm}[1]{\left\Vert #1\right\Vert }
\newcommand{\standardForm}[2]{\left\langle #1,#2\right\rangle}

% Representation theory
\newcommand{\fieldCharacter}{\psi}
\newcommand{\centralCharacter}[1]{\omega_{#1}}
\newcommand{\Ind}[3]{\mathrm{Ind}_{#1}^{#2}\left(#3\right)}
\newcommand{\ind}[3]{\mathrm{ind}_{#1}^{#2}\left(#3\right)}
\newcommand{\Whittaker}{\mathcal{W}}
\newcommand{\Contragradient}[1]{#1^{\vee}}
\newcommand{\underlyingVectorSpace}[1]{V_{#1}}
\newcommand{\representationDeclaration}[1]{#1}
\newcommand{\besselFunction}{\mathcal{J}}
\newcommand{\besselFunctionOfFiniteFieldRepresentation}{\besselFunction_{\finiteFieldRepresentation, \fieldCharacter}}
\newcommand{\SpehRepresentation}[2]{\Delta\left(#1, #2\right)}

\newcommand{\gbesselSpehFunction}[2]{\mathcal{BS}_{#1, #2}}
\newcommand{\besselSpehFunction}[2]{\mathcal{BS}_{\SpehRepresentation{#1}{#2}, \fieldCharacter}}
\newcommand{\fourierTransform}[2]{\mathcal{F}_{#1}#2}
\newcommand{\GKGammaFactor}[3]{\gamma^{\mathrm{GK}}\left(#1 \times #2, #3\right)}
\newcommand{\LocalGKGammaFactor}[4]{\gamma^{\mathrm{GK}}\left(#1, #2 \times #3, #4\right)}

\newcommand{\GKPreGammaFactor}[3]{\Gamma^{\mathrm{GK}}\left(#1 \times #2, #3\right)}
\newcommand{\gGJPreGammaFactor}[3]{\Gamma\left(#1 \times #2, #3\right)}
\newcommand{\GJPreGammaFactor}[2]{\Gamma\left(#1, #2\right)}
\newcommand{\Irr}{\mathrm{Irr}}

% Group theory macros
\newcommand{\rquot}[2]{{#1}\slash{#2}}
\newcommand{\lquot}[2]{{#1}\backslash{#2}}
\newcommand{\grpIndex}[2]{\left[#1:#2\right]}

% Matrices macros
\newcommand{\transpose}[1]{\, {}^{t}#1}
\newcommand{\inverseTranspose}[1]{#1^{\iota}}
\newcommand{\IdentityMatrix}[1]{I_{#1}}
\newcommand{\diag}{\mathrm{diag}}
\newcommand{\antidiag}{\operatorname{\mathrm{antidiag}}}
\newcommand{\trace}{\operatorname{tr}}
\newcommand{\GL}{\mathrm{GL}}
\newcommand{\UnipotentSubgroup}{U}
\newcommand{\UnipotentRadicalForWss}[2]{N_{\left(#2^{#1}\right)}}
\newcommand{\UnipotentRadicalForWssRecursion}[2]{\mathcal{Y}_{c,k}}
\newcommand{\UnipotentRadical}{N}
\newcommand{\ParabolicSubgroup}{P}

% Finite field macros
\newcommand{\FieldNorm}[2]{\mathrm{N}_{#1:#2}}
\newcommand{\aFieldNorm}{\mathrm{N}}
\newcommand{\FieldTrace}{\mathrm{Tr}}
\newcommand{\finiteField}{\mathbb{F}}
\newcommand{\finiteFieldExtension}[1]{\finiteField_{#1}}
\newcommand{\FieldExtension}[2]{{#1} \slash {#2}}
\newcommand{\algebraicClosure}[1]{\overline{#1}}
\newcommand{\charactergroup}[1]{\widehat{\multiplicativegroup{\finiteFieldExtension{#1}}}}
\newcommand{\limitcharactergroup}{\Gamma}
\newcommand{\Galois}{\operatorname{Gal}}
\newcommand{\Frobenius}{\operatorname{Fr}}
\newcommand{\restrictionOfScalars}[3]{\operatorname{Res}_{#1 \slash #2}{#3}}
\newcommand{\multiplcativeScheme}{\mathbb{G}_m}
\newcommand{\affineLine}{\mathbb{A}^1}
\newcommand{\squareMatrix}{\operatorname{Mat}}
\newcommand{\Mat}[2]{\operatorname{Mat}_{#1 \times #2}}
\newcommand{\frobeniusDegree}{\operatorname{deg}}
\newcommand{\Steinberg}{\operatorname{St}}
\newcommand{\ProjectionOperator}{\operatorname{pr}}
\newcommand{\SymmetricGroup}{\mathfrak{S}}
\newcommand{\whittakerVector}[1]{v_{#1, \fieldCharacter}}
\newcommand{\gwhittakerVector}[2]{v_{#1, #2}}
\newcommand{\WhittakerProjection}{\ProjectionOperator_{\mathrm{Wh}}}
\newcommand{\ParabolicForSpeh}[2]{P_{\left({#1}^{#2}\right)}}
\newcommand{\UnipotentForSpeh}[2]{N_{\left({#1}^{#2}\right)}}
\newcommand{\PoincarePolynomial}[2]{P_{#2}}

%Partition macros
\newcommand{\localField}{F}
\newcommand{\ringOfIntegers}{\mathfrak{o}}
\newcommand{\residueField}{\mathfrak{f}}
\newcommand{\maximalIdeal}{\mathfrak{p}}
\newcommand{\depthZeroRepresentation}{\mathcal{T}}
\newcommand{\differential}{\mathrm{d}}
\newcommand{\mdifferential}{\differential^{\times}}
\newcommand{\quotientMap}{\nu}
\newcommand{\Lift}{\mathcal{L}}
\newcommand{\uniformizer}{\varpi}
\newcommand{\VolumeOf}{\operatorname{Vol}}

\newcommand{\Erdelyi}{Erd{\'e}lyi}
\newcommand{\Toth}{T{\'o}th}

\newcommand{\parabolicSection}{\Phi^{\left(z_1, \dots, z_c\right)}}
\newcommand{\intertwiningOperator}{M^{\left(z_1, \dots, z_c\right)}}
\newcommand{\holomorphicRepresentation}{\depthZeroRepresentation^{\left(z_1, \dots, z_c\right)}}
\newcommand{\WhittakerFunctional}[1]{\ell_{#1, \fieldCharacter}}
\newcommand{\gWhittakerFunctional}[2]{\ell_{#1, #2}}
\newcommand{\gSpehWhittakerFunctional}[3]{\ell_{\SpehRepresentation{#1}{#3}, \fieldCharacter_{\UnipotentRadicalForWss{#2}{#3}}}}
\newcommand{\gShortSpehWhittakerFunctional}[3]{\ell_{\SpehRepresentation{#1}{#3}}}
\newcommand{\GaussSum}[2]{\mathcal{G}\left(#1, #2\right)}
\newcommand{\dblGaussSum}[2]{\mathcal{G}^{\mathrm{dbl}}\left(#1, #2\right)}
\newcommand{\GaussSumScalar}[2]{\mathrm{G}\left(#1, #2\right)}
\newcommand{\dblGaussSumScalar}[2]{\mathrm{G}^{\mathrm{dbl}}\left(#1, #2\right)}
\newcommand{\GKGaussSum}[3]{\mathcal{G}\left(#1 \times #2, #3\right)}
\newcommand{\GKGaussSumScalar}[3]{\mathrm{G}\left(#1 \times #2, #3\right)}
\newcommand{\fieldCharacterkc}[2]{\fieldCharacter_{\left({#2}^{#1}\right)}}
\newcommand{\ExoticKloosterman}{\mathrm{Kl}}
\newcommand{\GaussSumCharacter}[4]{\tau_{#1}\left(#2 \times #3, #4\right)}
\newcommand{\IrrCusp}{\Irr_{\mathrm{cusp}}}
\newcommand{\convolutionWithCompactSupport}{\boldsymbol{\mathrm{R}}}
\newcommand{\ladicnumbers}{\algebraicClosure{\mathbb{Q}_{\ell}}}
\newcommand{\artinScrier}{\operatorname{AS}}
\newcommand{\KloostermanSumClassFunction}{\mathcal{K}}

\hypersetup{pdfauthor={Elad Zelingher},
	pdfsubject={Number theory, Representation theory},
	pdfkeywords={Kloosterman sums, Gauss sums}}

\title{Degenerate Gauss sums for $\GL_n\left(\finiteField_q\right)$}

\author{Elad Zelingher}
\address{Department of Mathematics, University of Michigan, 1844 East Hall, 530 Church Street, Ann Arbor, MI 48109-1043 USA}
\email{eladz@umich.edu}

\subjclass[2010]{20C33, 11L05, 11T24}

% 11L05    Gauss and Kloosterman sums; generalizations
% 11T24    Other character sums and Gauss sums
% 20C33    Representations of finite groups of Lie type

\begin{document}

\begin{abstract}
\end{abstract}
\maketitle

\section{Notation}
Let $\finiteField$ be a finite field with $q$ elements, and let $\fieldCharacter \colon \finiteField \to \multiplicativegroup{\cComplex}$ be a non-trivial character. For any $r \ge 0$, let $B_r \subset \GL_r\left(\finiteField\right)$ be the standard Borel subgroup
$$B_r = \left\{ \begin{pmatrix}
	t_1 & \ast & \ast & \ast\\
	& t_2 & \ast & \ast \\
	& & \ddots & \ast\\
	& & & t_r
\end{pmatrix} \mid t_1,\dots,t_r \in \multiplicativegroup{\finiteField} \right\}.$$

\section{Degenerate Gauss sums}
Let $\pi$ be an irreducible representation of $\GL_n\left(\finiteField\right)$. The purpose of this section is to study the following operator, which we refer to as a degenerate Gauss sum
$$\GaussSum{\pi}{\fieldCharacter}_r = \sum_{g \in \GL_n\left(\finiteField\right)} \pi\left(g\right) \fieldCharacter\left(\trace\left(g \begin{pmatrix}
	0_{r}\\
	& \IdentityMatrix{n-r}
\end{pmatrix}\right)\right).$$
Notice that $\GaussSum{\pi}{\fieldCharacter}_0$ lies in $\Hom_{\GL_n\left(\finiteField\right)}\left(\pi, \pi\right)$ and therefore, $$\GaussSum{\pi}{\fieldCharacter} \coloneqq \GaussSum{\pi}{\fieldCharacter}_0 = \GaussSumScalar{\pi}{\fieldCharacter} \cdot \idmap_\pi,$$
where $\GaussSumScalar{\pi}{\fieldCharacter} \in \cComplex$. The computation of this scalar is known due to Kondo.

We begin with the following simple observation. We have that for any $b \in B_r$ and any $X \in \Mat{r}{(n-r)}\left(\finiteField\right)$,
$$ \pi\begin{pmatrix}
b & X\\
& \IdentityMatrix{n-r}
\end{pmatrix} \circ \GaussSum{\pi}{\fieldCharacter}_r = \GaussSum{\pi}{\fieldCharacter}_r.$$
Therefore, we have that $\GaussSum{\pi}{\fieldCharacter}_r$ can be regarded as an operator $\GaussSum{\pi}{\fieldCharacter}_r \colon \pi \to J\left(\pi, r\right)$, where $$J\left(\pi, r\right) = \left\{ v \in \pi \mid \pi \begin{pmatrix}
	b & X\\
	& \IdentityMatrix{n-r}
\end{pmatrix} v = v, \forall b \in B_r, X \in \Mat{r}{(n-r)}\left(\finiteField\right) \right\},$$
which is a representation of the group $B_r \times \GL_{n-r}\left(\finiteField\right)$. This space is non-zero if and only if $\pi$ is subrepresentation of a parabolically induced representation of the form $1^{\circ r} \circ \tau = 1 \circ \dots \circ 1 \circ \tau$, where $\tau$ is an irreducible representation of $\GL_{n-r}\left(\finiteField\right)$. 

It follows from this observation that if $\pi$ is an irreducible representation of $\GL_n\left(\finiteField\right)$ such that $1$ is not in the cuspidal support of $\pi$, then for any matrix $A$ with rank $< n$, 
$$\GaussSum{\pi}{\fieldCharacter}_A \coloneqq \sum_{g \in \GL_n\left(\finiteField\right)} \pi\left(g\right) \fieldCharacter\left(\trace\left(A g\right)\right) = 0.$$

\section{Doubling method Gauss sums}

\subsection{The case of general linear groups}

Let $\chi \colon \multiplicativegroup{\finiteField} \to \multiplicativegroup{\cComplex}$ be a character.

Consider the following assignment $\GL_n\left(\finiteField\right) \to \multiplicativegroup{\cComplex}$
$$\Phi_{\chi}\left(g\right) = \begin{dcases}
	\chi\left(\det\left(g-\IdentityMatrix{n}\right)\right) & \text{if }\det\left(g-\IdentityMatrix{n}\right) \ne 0,\\
	0 & \text{otherwise.}
\end{dcases}$$
It is clear that $\Phi_{\chi}$ is a class function of $\GL_n\left(\finiteField\right)$. Given an irreducible representation $\pi$ of $\GL_n\left(\finiteField\right)$, we consider the following \emph{doubling method Gauss sum}:
$$\dblGaussSum{\pi}{\chi} = \sum_{g \in \GL_n\left(\finiteField\right)} \Phi_{\chi}\left(g\right) \pi\left(g\right).$$
Then $\dblGaussSum{\pi}{\chi}$ defines an element of $\Hom_{\GL_n\left(\finiteField\right)}\left(\pi, \pi\right)$. Therefore, by Schur's lemma there exists a complex number $\dblGaussSumScalar{\pi}{\chi} \in \cComplex$ such that $$\dblGaussSum{\pi}{\chi} = \dblGaussSumScalar{\pi}{\chi} \cdot \idmap_\pi.$$ 

The goal of this section is to express $\dblGaussSumScalar{\pi}{\chi}$ in terms of Kondo's Gauss sum.

Assume first that $1$ does not appear in the cuspidal support of $\pi$. Let us write
$$\dblGaussSum{\pi}{\chi} = \sum_{\substack{g \in \GL_n\left(\finiteField\right) \sum_{A \in \squareMatrix_n\left(\finiteField\right)}\\
\det\left(g - \IdentityMatrix{n}\right) \ne 0}} \chi\left(\det\left(g - \IdentityMatrix{n}\right)\right) \pi\left(g\right).$$
We rewrite this using a Fourier transform type expression
$$\dblGaussSum{\pi}{\chi} = q^{-n^2} \sum_{g, h \in \GL_n\left(\finiteField\right)} \sum_{A \in \squareMatrix_n\left(\finiteField\right)} \chi^{-1}\left(\det h\right) \pi\left(g\right) \fieldCharacter\left(\trace\left(A \left(\IdentityMatrix{n} - h \left(g - \IdentityMatrix{n}\right)\right)\right)\right).$$

Replacing $g$ with $h^{-1} g$, we arrive at the expression
\begin{align*}
	\dblGaussSum{\pi}{\chi} &= q^{-n^2} \sum_{A \in \squareMatrix_n\left(\finiteField\right)} \fieldCharacter\left(\trace A\right) \sum_{h \in \GL_n\left(\finiteField\right)} \chi^{-1}\left(\det h\right) \pi\left(h^{-1}\right) \fieldCharacter\left(\trace\left(Ah\right)\right)\\
	& \circ \sum_{g \in \GL_n\left(\finiteField\right)} \pi\left(g\right) \fieldCharacter\left(-\trace Ag\right).
\end{align*}
Since the last inner sum is $\GaussSum{\pi}{\fieldCharacter}_{-A}$, we have that the sum can be reduced to $A \in \GL_n\left(\finiteField\right)$. Replacing $h$ with $A^{-1} h$ and $g$ with $-A^{-1} g$, we get the sum
\begin{align*}
	\dblGaussSumScalar{\pi}{\chi} &= q^{-n^2} \centralCharacter{\pi}\left(-1\right) \sum_{A \in \GL_n\left(\finiteField\right)} \chi\left(\det A\right) \fieldCharacter\left(\trace A\right) \GaussSumScalar{\Contragradient{\pi} \times \chi^{-1}}{\fieldCharacter} \GaussSumScalar{\pi}{\fieldCharacter},
\end{align*}
which equals
\begin{align*}
	\dblGaussSumScalar{\pi}{\chi} &= \left(-1\right)^{n-1} q^{-n^2} \centralCharacter{\pi}\left(-1\right) \GaussSumScalar{\chi}{\fieldCharacter}^n \GaussSumScalar{\Contragradient{\pi} \times \chi^{-1}}{\fieldCharacter} \GaussSumScalar{\pi}{\fieldCharacter}.
\end{align*}

\bibliographystyle{abbrv}
\bibliography{references}
\end{document}