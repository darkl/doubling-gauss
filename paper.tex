\documentclass[12pt, reqno]{amsart}

\usepackage{amsmath, amsthm, amssymb}
\usepackage{enumerate}
\tolerance=500
\setlength{\emergencystretch}{3em}
\usepackage[margin=1.0in]{geometry}
\usepackage{xcolor}
\definecolor{cite}{rgb}{0.30,0.60,1.00}
\definecolor{url}{rgb}{0.00,0.00,0.80}
\definecolor{link}{rgb}{0.40,0.10,0.20}
\usepackage[pdfusetitle,colorlinks,linkcolor=link,urlcolor=url,citecolor=cite,pagebackref,breaklinks]{hyperref}
\usepackage{graphicx}
\usepackage{cleveref}
\usepackage{mathdots}
\usepackage{tikz-cd}
\usepackage{comment}
\usepackage{xypic}
\usepackage{mathtools}

% new environment
\newtheorem{theorem}{Theorem}[section]
\newtheorem{claim}[theorem]{Claim}
\newtheorem{proposition}[theorem]{Proposition}
\newtheorem{lemma}[theorem]{Lemma}
\newtheorem{conjecture}[theorem]{Conjecture}
\newtheorem{corollary}[theorem]{Corollary}

\theoremstyle{definition}
\newtheorem{definition}[theorem]{Definition}
\theoremstyle{definition}
\newtheorem{remark}[theorem]{Remark}
\theoremstyle{definition}
\newtheorem{example}[theorem]{Example}

%% frequently used symbols
% Common math notions
\newcommand{\zIntegers}{\mathbb{Z}}
\newcommand{\rReal}{\mathbb{R}}
\newcommand{\cComplex}{\mathbb{C}}
\newcommand{\multiplicativegroup}[1]{#1^{\times}}
\newcommand{\RealPart}{\mathrm{Re}}
\newcommand{\Hom}{\mathrm{Hom}}
\newcommand{\EndomorphismRing}{\operatorname{End}}
\newcommand{\Span}{\mathrm{span}}
\newcommand{\Supp}{\mathrm{supp}}
\newcommand{\Stab}{\mathrm{stab}}
\newcommand{\idmap}{\mathrm{id}}
\newcommand{\conjugate}[1]{\overline{#1}}
\newcommand{\indicatorFunction}[1]{\delta_{#1}}
\newcommand{\isomorphic}{\cong}
\newcommand{\lengthof}{\mathfrak{n}}
\newcommand{\abs}[1]{\left|#1\right|}
\newcommand{\sizeof}[1]{\left|#1\right|}
\newcommand{\lcm}{\operatorname{lcm}}
\newcommand{\hermitianSpace}{\mathrm{V}}
\newcommand{\xIsotropic}{\mathrm{X}}
\newcommand{\yIsotropic}{\mathrm{Y}}

% Inner product
\newcommand{\innerproduct}[2]{\left\langle #1,#2\right\rangle}
\newcommand{\Norm}[1]{\left\Vert #1\right\Vert }
\newcommand{\standardForm}[2]{\left\langle #1,#2\right\rangle}

% Representation theory
\newcommand{\fieldCharacter}{\psi}
\newcommand{\centralCharacter}[1]{\omega_{#1}}
\newcommand{\Ind}[3]{\mathrm{Ind}_{#1}^{#2}\left(#3\right)}
\newcommand{\ind}[3]{\mathrm{ind}_{#1}^{#2}\left(#3\right)}
\newcommand{\Whittaker}{\mathcal{W}}
\newcommand{\Contragradient}[1]{#1^{\vee}}
\newcommand{\underlyingVectorSpace}[1]{V_{#1}}
\newcommand{\representationDeclaration}[1]{#1}
\newcommand{\besselFunction}{\mathcal{J}}
\newcommand{\besselFunctionOfFiniteFieldRepresentation}{\besselFunction_{\finiteFieldRepresentation, \fieldCharacter}}
\newcommand{\SpehRepresentation}[2]{\Delta\left(#1, #2\right)}

\newcommand{\gbesselSpehFunction}[2]{\mathcal{BS}_{#1, #2}}
\newcommand{\besselSpehFunction}[2]{\mathcal{BS}_{\SpehRepresentation{#1}{#2}, \fieldCharacter}}
\newcommand{\fourierTransform}[2]{\mathcal{F}_{#1}#2}
\newcommand{\GKGammaFactor}[3]{\gamma^{\mathrm{GK}}\left(#1 \times #2, #3\right)}
\newcommand{\LocalGKGammaFactor}[4]{\gamma^{\mathrm{GK}}\left(#1, #2 \times #3, #4\right)}

\newcommand{\GKPreGammaFactor}[3]{\Gamma^{\mathrm{GK}}\left(#1 \times #2, #3\right)}
\newcommand{\gGJPreGammaFactor}[3]{\Gamma\left(#1 \times #2, #3\right)}
\newcommand{\GJPreGammaFactor}[2]{\Gamma\left(#1, #2\right)}
\newcommand{\Irr}{\mathrm{Irr}}

% Group theory macros
\newcommand{\rquot}[2]{{#1}\slash{#2}}
\newcommand{\lquot}[2]{{#1}\backslash{#2}}
\newcommand{\grpIndex}[2]{\left[#1:#2\right]}

% Matrices macros
\newcommand{\transpose}[1]{\, {}^{t}#1}
\newcommand{\inverseTranspose}[1]{#1^{\iota}}
\newcommand{\IdentityMatrix}[1]{I_{#1}}
\newcommand{\diag}{\mathrm{diag}}
\newcommand{\antidiag}{\operatorname{\mathrm{antidiag}}}
\newcommand{\trace}{\operatorname{tr}}
\newcommand{\GL}{\mathrm{GL}}
\newcommand{\UnipotentSubgroup}{U}
\newcommand{\UnipotentRadicalForWss}[2]{N_{\left(#2^{#1}\right)}}
\newcommand{\UnipotentRadicalForWssRecursion}[2]{\mathcal{Y}_{c,k}}
\newcommand{\UnipotentRadical}{N}
\newcommand{\ParabolicSubgroup}{P}

% Finite field macros
\newcommand{\FieldNorm}[2]{\mathrm{N}_{#1:#2}}
\newcommand{\aFieldNorm}{\mathrm{N}}
\newcommand{\FieldTrace}{\mathrm{Tr}}
\newcommand{\finiteField}{\mathbb{F}}
\newcommand{\finiteFieldExtension}[1]{\finiteField_{#1}}
\newcommand{\FieldExtension}[2]{{#1} \slash {#2}}
\newcommand{\algebraicClosure}[1]{\overline{#1}}
\newcommand{\charactergroup}[1]{\widehat{\multiplicativegroup{\finiteFieldExtension{#1}}}}
\newcommand{\limitcharactergroup}{\Gamma}
\newcommand{\Galois}{\operatorname{Gal}}
\newcommand{\Frobenius}{\operatorname{Fr}}
\newcommand{\restrictionOfScalars}[3]{\operatorname{Res}_{#1 \slash #2}{#3}}
\newcommand{\multiplcativeScheme}{\mathbb{G}_m}
\newcommand{\affineLine}{\mathbb{A}^1}
\newcommand{\squareMatrix}{\operatorname{Mat}}
\newcommand{\Mat}[2]{\operatorname{Mat}_{#1 \times #2}}
\newcommand{\frobeniusDegree}{\operatorname{deg}}
\newcommand{\Steinberg}{\operatorname{St}}
\newcommand{\ProjectionOperator}{\operatorname{pr}}
\newcommand{\SymmetricGroup}{\mathfrak{S}}
\newcommand{\whittakerVector}[1]{v_{#1, \fieldCharacter}}
\newcommand{\gwhittakerVector}[2]{v_{#1, #2}}
\newcommand{\WhittakerProjection}{\ProjectionOperator_{\mathrm{Wh}}}
\newcommand{\ParabolicForSpeh}[2]{P_{\left({#1}^{#2}\right)}}
\newcommand{\UnipotentForSpeh}[2]{N_{\left({#1}^{#2}\right)}}
\newcommand{\PoincarePolynomial}[2]{P_{#2}}

%Partition macros
\newcommand{\localField}{F}
\newcommand{\ringOfIntegers}{\mathfrak{o}}
\newcommand{\residueField}{\mathfrak{f}}
\newcommand{\maximalIdeal}{\mathfrak{p}}
\newcommand{\depthZeroRepresentation}{\mathcal{T}}
\newcommand{\differential}{\mathrm{d}}
\newcommand{\mdifferential}{\differential^{\times}}
\newcommand{\quotientMap}{\nu}
\newcommand{\Lift}{\mathcal{L}}
\newcommand{\uniformizer}{\varpi}
\newcommand{\VolumeOf}{\operatorname{Vol}}

\newcommand{\Erdelyi}{Erd{\'e}lyi}
\newcommand{\Toth}{T{\'o}th}

\newcommand{\parabolicSection}{\Phi^{\left(z_1, \dots, z_c\right)}}
\newcommand{\intertwiningOperator}{M^{\left(z_1, \dots, z_c\right)}}
\newcommand{\holomorphicRepresentation}{\depthZeroRepresentation^{\left(z_1, \dots, z_c\right)}}
\newcommand{\WhittakerFunctional}[1]{\ell_{#1, \fieldCharacter}}
\newcommand{\gWhittakerFunctional}[2]{\ell_{#1, #2}}
\newcommand{\gSpehWhittakerFunctional}[3]{\ell_{\SpehRepresentation{#1}{#3}, \fieldCharacter_{\UnipotentRadicalForWss{#2}{#3}}}}
\newcommand{\gShortSpehWhittakerFunctional}[3]{\ell_{\SpehRepresentation{#1}{#3}}}
\newcommand{\GaussSum}[2]{\mathcal{G}\left(#1, #2\right)}
\newcommand{\dblGaussSum}[2]{\mathcal{G}^{\mathrm{dbl}}\left(#1, #2\right)}
\newcommand{\GaussSumScalar}[2]{\mathrm{G}\left(#1, #2\right)}
\newcommand{\dblGaussSumScalar}[2]{\mathrm{G}^{\mathrm{dbl}}\left(#1, #2\right)}
\newcommand{\dblGammaFactor}[3]{\Gamma^{\mathrm{dbl}}\left(#1 \times #2, #3\right)}
\newcommand{\dblGammaFactorSpace}[4]{\Gamma^{\mathrm{dbl}}_{#1}\left(#2 \times #3, #4\right)}
\newcommand{\GKGaussSum}[3]{\mathcal{G}\left(#1 \times #2, #3\right)}
\newcommand{\GKGaussSumScalar}[3]{\mathrm{G}\left(#1 \times #2, #3\right)}
\newcommand{\fieldCharacterkc}[2]{\fieldCharacter_{\left({#2}^{#1}\right)}}
\newcommand{\ExoticKloosterman}{\mathrm{Kl}}
\newcommand{\GaussSumCharacter}[4]{\tau_{#1}\left(#2 \times #3, #4\right)}
\newcommand{\IrrCusp}{\Irr_{\mathrm{cusp}}}
\newcommand{\convolutionWithCompactSupport}{\boldsymbol{\mathrm{R}}}
\newcommand{\ladicnumbers}{\algebraicClosure{\mathbb{Q}_{\ell}}}
\newcommand{\artinScrier}{\operatorname{AS}}
\newcommand{\KloostermanSumClassFunction}{\mathcal{K}}
\newcommand{\IsometryGroup}{\mathrm{Isom}}
\newcommand{\lieAlgebra}{\mathfrak{g}}

\hypersetup{pdfauthor={Elad Zelingher},
	pdfsubject={Number theory, Representation theory},
	pdfkeywords={Kloosterman sums, Gauss sums}}

\title{On exponential sums arising from the doubling method}

\author{Elad Zelingher}
\address{Department of Mathematics, University of Michigan, 1844 East Hall, 530 Church Street, Ann Arbor, MI 48109-1043 USA}
\email{eladz@umich.edu}

\subjclass[2010]{20C33, 11L05, 11T24}

% 11L05    Gauss and Kloosterman sums; generalizations
% 11T24    Other character sums and Gauss sums
% 20C33    Representations of finite groups of Lie type

\begin{document}

\begin{abstract}
\end{abstract}
\maketitle

\section{Notation}
Let $\finiteField$ be a finite field with $q$ elements, and let $\fieldCharacter \colon \finiteField \to \multiplicativegroup{\cComplex}$ be a non-trivial character. For any $r \ge 0$, let $B_r \subset \GL_r\left(\finiteField\right)$ be the standard Borel subgroup
$$B_r = \left\{ \begin{pmatrix}
	t_1 & \ast & \ast & \ast\\
	& t_2 & \ast & \ast \\
	& & \ddots & \ast\\
	& & & t_r
\end{pmatrix} \mid t_1,\dots,t_r \in \multiplicativegroup{\finiteField} \right\}.$$

\section{Degenerate Gauss sums}
Let $\tau$ be an irreducible representation of $\GL_n\left(\finiteField\right)$. For a matrix $A \in \squareMatrix_n\left(\finiteField\right)$, denote
$$\GaussSum{\tau}{\fieldCharacter}_A \coloneqq q^{-\frac{n^2}{2}} \sum_{g \in \GL_n\left(\finiteField\right)} \tau\left(g\right) \fieldCharacter\left(\trace\left(A g\right)\right).$$
For any such $A$, there exist $h_1, h_2 \in \GL_n\left(\finiteField\right)$ and $0 \le r \le n$, such that $A = h_1 \begin{pmatrix}
	0_r\\
	& \IdentityMatrix{n-r}
\end{pmatrix} h_2$. Changing variables, we see that $$\GaussSum{\tau}{\fieldCharacter}_A = \tau \left(h_2^{-1}\right) \circ \GaussSum{\tau}{\fieldCharacter}_r \circ \tau\left(h_1^{-1}\right),$$
where
$$\GaussSum{\tau}{\fieldCharacter}_r = q^{-\frac{n^2}{2}} \sum_{g \in \GL_n\left(\finiteField\right)} \tau\left(g\right) \fieldCharacter\left(\trace\left(g \begin{pmatrix}
	0_{r}\\
	& \IdentityMatrix{n-r}
\end{pmatrix}\right)\right).$$
Notice that $\GaussSum{\tau}{\fieldCharacter}_0$ lies in $\Hom_{\GL_n\left(\finiteField\right)}\left(\tau, \tau\right)$ and therefore, $$\GaussSum{\tau}{\fieldCharacter} \coloneqq \GaussSum{\tau}{\fieldCharacter}_0 = \GaussSumScalar{\tau}{\fieldCharacter} \cdot \idmap_\tau,$$
where $\GaussSumScalar{\tau}{\fieldCharacter} \in \cComplex$. The computation of this scalar is known due to Kondo. Let us also denote for a character $\chi \colon \multiplicativegroup{\finiteField} \to \multiplicativegroup{\cComplex}$, the \emph{twisted Gauss sum}
$$\GaussSum{\tau \times \chi}{\fieldCharacter} = q^{-\frac{n^2}{2}} \sum_{g \in \GL_n\left(\finiteField\right)} \tau\left(g\right) \chi\left(\det g\right) \fieldCharacter\left(\trace g\right).$$
This element also lies in $\Hom_{\GL_n\left(\finiteField\right)}\left(\tau, \tau\right)$ and we denote
$$\GaussSum{\tau \times \chi}{\fieldCharacter} = \GaussSumScalar{\tau \times \chi}{\fieldCharacter} \cdot \idmap_\tau,$$
where $\GaussSumScalar{\tau \times \chi}{\fieldCharacter} \in \cComplex$.

Let us show that if $A$ is singular and $\tau$ does not contain $1$ in its cuspidal support then $\GaussSum{\tau}{\fieldCharacter}_A = 0$. We begin with the following simple observation. We have that for any $b \in B_r$ and any $X \in \Mat{r}{(n-r)}\left(\finiteField\right)$,
$$ \tau\begin{pmatrix}
	b & X\\
	& \IdentityMatrix{n-r}
\end{pmatrix} \circ \GaussSum{\tau}{\fieldCharacter}_r = \GaussSum{\tau}{\fieldCharacter}_r.$$
Therefore, we have that $\GaussSum{\tau}{\fieldCharacter}_r$ can be regarded as an operator $\GaussSum{\tau}{\fieldCharacter}_r \colon \tau \to J\left(\tau, r\right)$, where $$J\left(\tau, r\right) = \left\{ v \in \tau \mid \tau \begin{pmatrix}
	b & X\\
	& \IdentityMatrix{n-r}
\end{pmatrix} v = v, \forall b \in B_r, X \in \Mat{r}{(n-r)}\left(\finiteField\right) \right\},$$
which is a representation of the group $B_r \times \GL_{n-r}\left(\finiteField\right)$. This space is non-zero if and only if $\tau$ is subrepresentation of a parabolically induced representation of the form $1^{\circ r} \circ \tau = 1 \circ \dots \circ 1 \circ \tau$, where $\tau$ is an irreducible representation of $\GL_{n-r}\left(\finiteField\right)$. 

It follows from this observation that if $\tau$ is an irreducible representation of $\GL_n\left(\finiteField\right)$ such that $1$ is not in the cuspidal support of $\tau$, then for any matrix $A$ with rank $< n$, we have $\GaussSum{\tau}{\fieldCharacter}_A = 0$.

We actually use this later only for the case $\tau = \chi_{\GL_n}$, i.e., $\tau\left(g\right) = \chi\left(\det g\right)$ for some character $1 \ne \chi \colon \multiplicativegroup{\finiteField} \to \multiplicativegroup{\cComplex}$. Maybe rewrite for this case. The argument is also simpler.

\section{Doubling method Gauss sums}

\subsection{The case of general linear groups}

Let $\chi \colon \multiplicativegroup{\finiteField} \to \multiplicativegroup{\cComplex}$ be a character.

Consider the following assignment $\GL_k\left(\finiteField\right) \to \cComplex$
$$\Phi_{\chi}\left(g\right) = \begin{dcases}
	\chi\left(\det\left(g-\IdentityMatrix{k}\right)\right) & \text{if }\det\left(g-\IdentityMatrix{k}\right) \ne 0,\\
	0 & \text{otherwise.}
\end{dcases}$$
It is clear that $\Phi_{\chi}$ is a class function of $\GL_k\left(\finiteField\right)$. Given an irreducible representation $\tau$ of $\GL_k\left(\finiteField\right)$, we consider the following \emph{doubling method Gauss sum}:
$$\dblGaussSum{\tau}{\chi} = q^{-\frac{k^2}{2}} \sum_{g \in \GL_k\left(\finiteField\right)} \Phi_{\chi}\left(g\right) \tau\left(g\right).$$
Then $\dblGaussSum{\tau}{\chi}$ defines an element of $\Hom_{\GL_k\left(\finiteField\right)}\left(\tau, \tau\right)$. Therefore, by Schur's lemma there exists a complex number $\dblGaussSumScalar{\tau}{\chi} \in \cComplex$ such that $$\dblGaussSum{\tau}{\chi} = \dblGaussSumScalar{\tau}{\chi} \cdot \idmap_\tau.$$ 

The goal of this section is to express $\dblGaussSumScalar{\tau}{\chi}$ in terms of Kondo's Gauss sum.

Assume first that $1$ does not appear in the cuspidal support of $\tau$. Let us write
$$\dblGaussSum{\tau}{\chi} = q^{-\frac{k^2}{2}} \sum_{\substack{g \in \GL_k\left(\finiteField\right) \\
		\det\left(g - \IdentityMatrix{k}\right) \ne 0}} \chi\left(\det\left(g - \IdentityMatrix{k}\right)\right) \tau\left(g\right).$$
We rewrite this using a Fourier transform type expression
$$\dblGaussSum{\tau}{\chi} = q^{-\frac{3k^2}{2}} \sum_{g, h \in \GL_k\left(\finiteField\right)} \sum_{A \in \squareMatrix_k\left(\finiteField\right)} \chi^{-1}\left(\det h\right) \tau\left(g\right) \fieldCharacter\left(\trace\left(A \left(\IdentityMatrix{k} - h \left(g - \IdentityMatrix{k}\right)\right)\right)\right).$$

Replacing $g$ with $h^{-1} g$, we arrive at the expression
\begin{align*}
	\dblGaussSum{\tau}{\chi} =& q^{-\frac{3k^2}{2}} \sum_{A \in \squareMatrix_k\left(\finiteField\right)} \fieldCharacter\left(\trace A\right) \sum_{h \in \GL_k\left(\finiteField\right)} \chi^{-1}\left(\det h\right) \tau\left(h^{-1}\right) \fieldCharacter\left(\trace\left(Ah\right)\right)\\
	& \circ \sum_{g \in \GL_k\left(\finiteField\right)} \tau\left(g\right) \fieldCharacter\left(-\trace Ag\right).
\end{align*}
Since the last inner sum is $q^{\frac{k^2}{2}} \GaussSum{\tau}{\fieldCharacter}_{-A}$, we have that the sum can be reduced to $A \in \GL_k\left(\finiteField\right)$. Replacing $h$ with $A^{-1} h$ and $g$ with $-A^{-1} g$, we get the sum
\begin{align*}
	\dblGaussSumScalar{\tau}{\chi} &= q^{-\frac{k^2}{2}} \centralCharacter{\tau}\left(-1\right) \sum_{A \in \GL_k\left(\finiteField\right)} \chi\left(\det A\right) \fieldCharacter\left(\trace A\right) \GaussSumScalar{\Contragradient{\tau} \times \chi^{-1}}{\fieldCharacter} \GaussSumScalar{\tau}{\fieldCharacter},
\end{align*}
which equals
\begin{align*}
	\dblGaussSumScalar{\tau}{\chi} &=  \centralCharacter{\tau}\left(-1\right) \GaussSumScalar{\chi}{\fieldCharacter}^k \GaussSumScalar{\Contragradient{\tau} \times \chi^{-1}}{\fieldCharacter} \GaussSumScalar{\tau}{\fieldCharacter}.
\end{align*}
This is probably also true if $1$ is in the cuspidal support of $\tau$  and $\chi \ne 1$ (return to this later: should follow from a multiplicativity arguments). In the meantime, we will check this for $\tau = 1$, the trivial representation of $\GL_1\left(\finiteField\right)$, in which case we have
$$\dblGaussSum{1}{\chi} = q^{-\frac{1}{2}} \sum_{\substack{x \in \multiplicativegroup{\finiteField} \\
		x \ne 1}} \chi\left(x - 1\right) = q^{-\frac{1}{2}}\left(-\chi\left(-1\right) + \sum_{x \in \multiplicativegroup{\finiteField}} \chi\left(x\right) \right) = \begin{dcases}
		-q^{-\frac{1}{2}} \chi\left(-1\right) & \chi \ne 1\\
		q^{-\frac{1}{2}} \left(q - 2\right) & \chi = 1
	\end{dcases}.$$

\subsection{The case of classical groups}

Let $\hermitianSpace$ be a vector space of dimension $n$ over $\finiteField$, equipped with a non-degenerate bilinear form $\innerproduct{\cdot}{\cdot} \colon \hermitianSpace \times \hermitianSpace \to \finiteField$ which is either symmetric or anti-symmetric. Let $G = \IsometryGroup \left(\hermitianSpace\right)$ be the isometry group of $\hermitianSpace$, consisting of all the elements of $\GL\left(\hermitianSpace\right)$ that satisfy $\innerproduct{gx}{gy} = \innerproduct{x}{y}$ for every $x,y \in \hermitianSpace$. Let $\lieAlgebra$ be the Lie algebra of $G$, consisting of all elements $A \in \EndomorphismRing\left(\hermitianSpace\right)$ satisfying $\innerproduct{Ax}{y} + \innerproduct{x}{AY} = 0$ for every $x, y \in \hermitianSpace$.

Let $\chi \colon \multiplicativegroup{\finiteField} \to \multiplicativegroup{\cComplex}$ be a character. As before, the assignment $\Phi_{\chi} \colon \GL\left(\hermitianSpace\right) \to \cComplex$ given by $$\Phi_{\chi}\left(g\right) = \begin{dcases}
\chi\left(\det\left(g - \idmap_{\hermitianSpace}\right)\right) & \text{if }\det\left(g - \idmap_{\hermitianSpace}\right) \ne 0\\
0 & \text{otherwise,}
\end{dcases}$$
is a class function of $\GL\left(\hermitianSpace\right)$.

Let $\pi$ be an irreducible representation of $G$. Denote $$\dblGaussSum{\pi}{\chi} = \frac{1}{\sqrt{\sizeof{\lieAlgebra}}} \sum_{g \in G} \pi\left(g\right) \Phi_{\chi}\left(g\right),$$
where  Since $\Phi_{\chi}$ is a class function of $\GL\left(\hermitianSpace\right)$, it is also a class function of $G$, and therefore $\dblGaussSum{\pi}{\chi} \in \Hom_{G}\left(\pi, \pi\right)$. By Schur's lemma, there exists a constant $\dblGaussSumScalar{\pi}{\fieldCharacter} \in \cComplex$, such that $$\dblGaussSum{\pi}{\chi} = \dblGaussSumScalar{\pi}{\chi} \cdot \idmap_\pi.$$
We call $\dblGaussSumScalar{\pi}{\chi}$ the \emph{doubling method Gauss sum}.

\subsubsection{Multiplicativity property}
Our next goal is to understand how the doubling method Gauss sum behaves under parabolic induction.

Let $\xIsotropic$ and $\yIsotropic$ be isotropic spaces of $\hermitianSpace$ of dimension $k$, such that $\xIsotropic$ and $\yIsotropic$ are in duality with respect to form $\innerproduct{\cdot}{\cdot}$. Let us write $$\hermitianSpace = \xIsotropic \oplus \hermitianSpace' \oplus \yIsotropic,$$
where $\hermitianSpace' \subset \hermitianSpace$ is a non-degenerate subspace, orthogonal to $\xIsotropic$ and $\yIsotropic$. Let $P$ be the parabolic subgroup of $G$, consisting of all elements stabilizing the flag $$0 \subset \xIsotropic \subset \xIsotropic \oplus \hermitianSpace' \subset \xIsotropic \oplus \hermitianSpace' \oplus \yIsotropic = \hermitianSpace.$$
Write $P = L \ltimes N$, where $L$ is the Levi part of $P$ and $N$ is the unipotent radical of $P$. Then $L$ is isomorphic to $G' \times \GL_k\left(\finiteField\right)$, where $G' = \IsometryGroup\left(\hermitianSpace'\right)$.

\begin{theorem}\label{thm:multiplicativity-in-terms-of-gauss-sums}
	Let $\pi'$ be an irreducible representation of $G'$ and let $\tau$ be an irreducible representation of $\GL_k\left(\finiteField\right)$. Then for any irreducible representation $\pi$ of $G$ which appears as a subrepresentation the parabolic induction $\rho = \Ind{P}{G}{\tau \overline{\otimes} \pi'}$ and any $\chi \ne 1$, we have
	$$\dblGaussSumScalar{\pi}{\chi} = \centralCharacter{\tau}\left(-1\right) \GaussSumScalar{\chi^2}{\fieldCharacter}^k \GaussSumScalar{\tau \times \chi^{-1}}{\fieldCharacter} \GaussSumScalar{\Contragradient{\tau} \times \chi^{-1}}{\fieldCharacter} \dblGaussSumScalar{\pi'}{\chi}.$$
\end{theorem}
\begin{proof}
	Let $v_{\pi'} \in \pi'$ and $v_{\tau} \in \tau$. Consider $f \in \rho$ defined as follows. The function $f \in \rho$ is the unique element supported on $P$, such that $f\left(\idmap_{\hermitianSpace}\right) = v_{\tau} \otimes v_{\pi'}$. The projection $f$ to any non-zero invariant subspace of $\rho$ is non-zero. We compute $\dblGaussSum{\pi}{\chi} f$.
	
	Notice that for $x \in G$, $$\left(\dblGaussSum{\pi}{\chi} f\right)\left(x\right) = \frac{1}{\sqrt{\sizeof{\lieAlgebra}}} \sum_{g \in G} f\left(xg\right) \Phi_{\chi}\left(g\right) = \frac{1}{\sqrt{\sizeof{\lieAlgebra}}} \sum_{p \in P} f\left(p\right) \Phi_{\chi}\left(x^{-1} p\right),$$
	which can be rewritten as
	$$\left(\dblGaussSum{\pi}{\chi} f\right)\left(x\right) = \frac{1}{\sqrt{\sizeof{\lieAlgebra}}} \sum_{p \in P}  \Phi_{\chi}\left(x^{-1} p\right) \left(\tau \overline{\otimes} \pi'\right)\left(p\right) v_\tau \otimes v_{\pi'}.$$
	As before, we may write
	$$\Phi_{\chi}\left(g\right) = q^{-n^2} \sum_{h \in \GL\left(\hermitianSpace\right)} \sum_{A \in \EndomorphismRing\left(\hermitianSpace\right)} \chi^{-1}\left(\det h\right) \fieldCharacter\left(\trace\left(A \left(g - \idmap_{\hermitianSpace} - h^{-1}\right)\right)\right).$$
	Hence, \begin{align*}
		q^{n^2} \sum_{p \in P} \Phi_{\chi}\left(xp\right) \left(\tau \overline{\otimes} \pi'\right)\left(p\right) =& \sum_{A \in \EndomorphismRing\left(\hermitianSpace\right)} \sum_{h \in \GL\left(\hermitianSpace\right)} \chi\left(\det h\right)\fieldCharacter\left(\trace\left(Ah\right)\right) \\
		& \times \sum_{p \in P} \fieldCharacter\left(\trace\left(A \left(xp - \idmap_{\hermitianSpace}\right)\right)\right) \left(\tau \overline{\otimes} \pi'\right)\left(p\right).
	\end{align*}
	Assume that $\chi \ne 1$. Then since $\GaussSum{\chi_{\GL\left(\hermitianSpace\right)}}{\fieldCharacter}_A$ appears in the last equation, the sum can be reduced to $A \in \GL\left(\hermitianSpace\right)$. Let us replace $h$ with $A^{-1} h$. We get the sum
	\begin{align*}
		\sum_{A \in \GL\left(\hermitianSpace\right)} \chi^{-1}\left(\det A\right) \sum_{h \in \GL\left(\hermitianSpace\right)} \chi\left(\det h\right)\fieldCharacter\left(\trace h\right) \sum_{p \in P} \fieldCharacter\left(\trace\left(A \left(xp - \idmap_{\hermitianSpace}\right)\right)\right) \left(\tau \overline{\otimes} \pi'\right)\left(p\right).
	\end{align*}
	Consider the inner sum over $P$, written as $$\sum_{l \in L} \sum_{u \in N} \fieldCharacter\left(\trace\left(A x l u\right)\right) \left(\tau \overline{\otimes} \pi'\right)\left(l\right).$$
	The inner sum over $N$ will vanish unless $Ax \in P$. Thus we have
	\begin{align*}
		q^{n^2} \sum_{p \in P} \Phi_{\chi}\left(xp\right) \left(\tau \overline{\otimes} \pi'\right)\left(p\right) =& \sum_{h \in \GL\left(\hermitianSpace\right)}  \chi\left(\det\left(xh\right)\right) \fieldCharacter\left(\trace h\right) \\
		& \times \sum_{p' \in P} \sum_{p \in P} \chi^{-1}\left(\det p'\right) \fieldCharacter\left(\trace\left(p' \left(p - x^{-1}\right)\right)\right) \left(\tau \overline{\otimes} \pi'\right)\left(p\right).
	\end{align*}
	By decomposing the sum over $p' \in P$ into a sum over $N$ and $L$, we see that the inner sum over $N$ will vanish unless $p - x^{-1} \in P$, which implies that $x \in P$.
	
	Hence, we have that $\dblGaussSum{\pi}{\chi} f$ is supported on $P$. We move to compute $\left(\dblGaussSum{\pi}{\chi} f\right)\left(\idmap_{\hermitianSpace}\right)$. It is given by
	\begin{equation}\label{eq:recursive-doubling-gauss-sum}
		\frac{1}{\sqrt{\sizeof{\lieAlgebra}}} \sum_{\substack{p \in P\\
				\det\left(p - \idmap_{\hermitianSpace}\right) \ne 0}} \chi\left(\det\left(p - \idmap_{\hermitianSpace}\right)\right) \left(\tau \overline{\otimes} \pi'\right)\left(p\right) v_{\tau} \otimes v_{\pi'}.
	\end{equation}
	Decomposing the sum \eqref{eq:recursive-doubling-gauss-sum} as a sum over $L$ and $N$, and using the fact that if $p \in P$ has Levi part with image $\left(a, g'\right) \in \GL_k\left(\finiteField\right) \times G'$ then $$\det\left(p - \idmap_{\hermitianSpace}\right) = \left(-1\right)^k \det\left(a\right)^{-1}\det\left(a - I_k\right)^2 \det\left( g' - \idmap_{\hermitianSpace'}\right),$$
	and using the fact that $\sizeof{\lieAlgebra} = \sizeof{\lieAlgebra'} \sizeof{\squareMatrix_k\left(\finiteField\right)} \sizeof{N}^2$, where $\lieAlgebra'$ is the Lie algebra of $G'$,
	we get that \eqref{eq:recursive-doubling-gauss-sum} equals
	\begin{equation}
		\chi\left(-1\right)^k q^{-\frac{k^2}{2}} \sum_{a \in \GL_k\left(\finiteField\right)} \Phi_{\chi^2}\left(a\right) \chi^{-1}\left(\det a\right) \tau\left(a\right) v_{\tau} \otimes \frac{1}{\sqrt{\sizeof{\lieAlgebra'}}} \sum_{g' \in G'} \Phi_{\chi}\left(g'\right) \pi'\left(g'\right) v_{\pi'},
	\end{equation}
	which in turn is
	\begin{align*}
		&\chi\left(-1\right)^k \dblGaussSum{\tau \otimes \chi_{\GL_k}^{-1}}{\chi^2} v_{\tau} \otimes \dblGaussSum{\pi'}{\chi} v_{\pi'}\\
		=& \chi\left(-1\right)^k\dblGaussSumScalar{\tau \otimes \chi_{\GL_k}^{-1}}{\chi^2} \dblGaussSumScalar{\pi'}{\chi} v_{\tau} \otimes v_{\pi'}.
	\end{align*}
	The result now follows.
\end{proof}

\subsubsection{Gamma factors}
In order to make the computations more tolerable, it is beneficial to work with fully multiplicative functions. We introduce the doubling method gamma factors.

Let $\chi \colon \multiplicativegroup{\finiteField} \to \multiplicativegroup{\cComplex}$ be a character.

For an irreducible representation  $\tau$ of $\GL_k\left(\finiteField\right)$, define $$\dblGammaFactor{\tau}{\chi}{\fieldCharacter} = \centralCharacter{\tau}\left(-1\right) \GaussSumScalar{\tau \times \chi^{-1}}{\fieldCharacter} \GaussSumScalar{\Contragradient{\tau} \times \chi^{-1}}{\fieldCharacter}.$$
For an irreducible representation $\pi$ of $G$, define the \emph{doubling method gamma factor} $$\dblGammaFactorSpace{\hermitianSpace}{\pi}{\chi}{\fieldCharacter} = \frac{\dblGaussSumScalar{\pi}{\chi}}{\GaussSumScalar{\chi^2}{\fieldCharacter}^{\left[\frac{\dim \hermitianSpace}{2}\right]}}.$$
Using this notation and using the multiplicativity property of Kondo's Gauss sum and \Cref{thm:multiplicativity-in-terms-of-gauss-sums}, we have the following multiplicativity property.

\begin{theorem}
	\begin{enumerate}
		\item If $\tau_1$ and $\tau_2$ are irreducible representations of $\GL_{k_1}\left(\finiteField\right)$ and $\GL_{k_2}\left(\finiteField\right)$, respectively, then for any irreducible subrepresentation $\tau$ of the parabolic induction $\tau_1 \circ \tau_2$, we have
		$$\dblGammaFactor{\tau}{\chi}{\fieldCharacter} = \dblGammaFactor{\tau_1}{\chi}{\fieldCharacter} \dblGammaFactor{\tau_2}{\chi}{\fieldCharacter}.$$
		\item If $\tau$, $\pi'$ and $\pi$ are as in \Cref{thm:multiplicativity-in-terms-of-gauss-sums} and $\chi \ne 1$, then
		$$\dblGammaFactorSpace{\hermitianSpace}{\pi}{\chi}{\fieldCharacter} = \dblGammaFactorSpace{\hermitianSpace'}{\pi'}{\chi}{\fieldCharacter} \dblGammaFactor{\tau}{\chi}{\fieldCharacter}.$$
	\end{enumerate}
\end{theorem}

\subsubsection{Computation for irreducible cuspidal representations}

\begin{theorem}
	Suppose that $\pi$ is an irreducible cuspidal representation of $G$, parameterized by a character $\theta = \theta_1 \times \dots \times \theta_r$ of a torus $T$ such that $$T \cong \finiteFieldExtension{2n_1}^1 \times \finiteFieldExtension{2n_2}^1 \times \dots \times \finiteFieldExtension{2n_r}^1,$$
	where $2n_1 + 2n_2 + \dots + 2n_r = \dim \hermitianSpace$ if $\dim \hermitianSpace$ is even or $2n_1 + 2n_2 + \dots + 2n_r = \dim \hermitianSpace - 1$ if $\dim \hermitianSpace$ is odd, and where $$\finiteFieldExtension{2m}^1 = \left\{ x \in \multiplicativegroup{\finiteFieldExtension{2m}} \mid \FieldNorm{2m}{m}\left(x\right)= 1\right\}.$$
	Then up to maybe a sign, we have $$\dblGammaFactorSpace{\hermitianSpace}{\pi}{\chi}{\fieldCharacter} = \prod_{j=1}^r \tau\left( \theta'_j \times \chi, \fieldCharacter_{2n_j} \right),$$
	where $\theta'_j\left(x\right) = \theta_j\left(x^{1-q^{n_j}}\right)$ and $$\tau\left( \theta'_j \times \chi, \fieldCharacter_{2n_j}\right) = -q^{-n_j} \sum_{x \in \multiplicativegroup{\finiteFieldExtension{2n_j}}} \theta'_j\left(x\right) \chi\left(\FieldNorm{2n_j}{1}\left(x\right)\right)\fieldCharacter\left(\trace_{\finiteFieldExtension{2n_j} \slash \finiteField}\left(x\right)\right).$$
\end{theorem}

\bibliographystyle{abbrv}
\bibliography{references}
\end{document}