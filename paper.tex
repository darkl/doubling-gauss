\documentclass[12pt, reqno]{amsart}

\usepackage{amsmath, amsthm, amssymb}
\usepackage{enumerate}
\tolerance=500
\setlength{\emergencystretch}{3em}
\usepackage[margin=1.0in]{geometry}
\usepackage{xcolor}
\definecolor{cite}{rgb}{0.30,0.60,1.00}
\definecolor{url}{rgb}{0.00,0.00,0.80}
\definecolor{link}{rgb}{0.40,0.10,0.20}
\usepackage[pdfusetitle,colorlinks,linkcolor=link,urlcolor=url,citecolor=cite,pagebackref,breaklinks]{hyperref}
\usepackage{graphicx}
\usepackage{cleveref}
\usepackage{mathdots}
\usepackage{tikz-cd}
\usepackage{comment}
\usepackage{xypic}
\usepackage{mathtools}

% new environment
\newtheorem{theorem}{Theorem}[section]
\newtheorem{claim}[theorem]{Claim}
\newtheorem{proposition}[theorem]{Proposition}
\newtheorem{lemma}[theorem]{Lemma}
\newtheorem{conjecture}[theorem]{Conjecture}
\newtheorem{corollary}[theorem]{Corollary}

\theoremstyle{definition}
\newtheorem{definition}[theorem]{Definition}
\theoremstyle{definition}
\newtheorem{remark}[theorem]{Remark}
\theoremstyle{definition}
\newtheorem{example}[theorem]{Example}

%% frequently used symbols
% Common math notions
\newcommand{\zIntegers}{\mathbb{Z}}
\newcommand{\rReal}{\mathbb{R}}
\newcommand{\cComplex}{\mathbb{C}}
\newcommand{\multiplicativegroup}[1]{#1^{\times}}
\newcommand{\RealPart}{\mathrm{Re}}
\newcommand{\Hom}{\mathrm{Hom}}
\newcommand{\EndomorphismRing}{\operatorname{End}}
\newcommand{\Span}{\mathrm{span}}
\newcommand{\Supp}{\mathrm{supp}}
\newcommand{\Stab}{\mathrm{stab}}
\newcommand{\idmap}{\mathrm{id}}
\newcommand{\conjugate}[1]{\overline{#1}}
\newcommand{\indicatorFunction}[1]{\delta_{#1}}
\newcommand{\isomorphic}{\cong}
\newcommand{\lengthof}{\mathfrak{n}}
\newcommand{\abs}[1]{\left|#1\right|}
\newcommand{\sizeof}[1]{\left|#1\right|}
\newcommand{\lcm}{\operatorname{lcm}}
\newcommand{\hermitianSpace}{\mathrm{V}}
\newcommand{\xIsotropic}{\mathrm{X}}
\newcommand{\yIsotropic}{\mathrm{Y}}

% Inner product
\newcommand{\innerproduct}[2]{\left\langle #1,#2\right\rangle}
\newcommand{\Norm}[1]{\left\Vert #1\right\Vert }
\newcommand{\standardForm}[2]{\left\langle #1,#2\right\rangle}

% Representation theory
\newcommand{\fieldCharacter}{\psi}
\newcommand{\centralCharacter}[1]{\omega_{#1}}
\newcommand{\Ind}[3]{\mathrm{Ind}_{#1}^{#2}\left(#3\right)}
\newcommand{\ind}[3]{\mathrm{ind}_{#1}^{#2}\left(#3\right)}
\newcommand{\Whittaker}{\mathcal{W}}
\newcommand{\Contragradient}[1]{#1^{\vee}}
\newcommand{\underlyingVectorSpace}[1]{V_{#1}}
\newcommand{\representationDeclaration}[1]{#1}
\newcommand{\besselFunction}{\mathcal{J}}
\newcommand{\besselFunctionOfFiniteFieldRepresentation}{\besselFunction_{\finiteFieldRepresentation, \fieldCharacter}}
\newcommand{\SpehRepresentation}[2]{\Delta\left(#1, #2\right)}

\newcommand{\gbesselSpehFunction}[2]{\mathcal{BS}_{#1, #2}}
\newcommand{\besselSpehFunction}[2]{\mathcal{BS}_{\SpehRepresentation{#1}{#2}, \fieldCharacter}}
\newcommand{\fourierTransform}[2]{\mathcal{F}_{#1}#2}
\newcommand{\GKGammaFactor}[3]{\gamma^{\mathrm{GK}}\left(#1 \times #2, #3\right)}
\newcommand{\LocalGKGammaFactor}[4]{\gamma^{\mathrm{GK}}\left(#1, #2 \times #3, #4\right)}

\newcommand{\GKPreGammaFactor}[3]{\Gamma^{\mathrm{GK}}\left(#1 \times #2, #3\right)}
\newcommand{\gGJPreGammaFactor}[3]{\Gamma\left(#1 \times #2, #3\right)}
\newcommand{\GJPreGammaFactor}[2]{\Gamma\left(#1, #2\right)}
\newcommand{\Irr}{\mathrm{Irr}}

% Group theory macros
\newcommand{\rquot}[2]{{#1}\slash{#2}}
\newcommand{\lquot}[2]{{#1}\backslash{#2}}
\newcommand{\grpIndex}[2]{\left[#1:#2\right]}

% Matrices macros
\newcommand{\transpose}[1]{\, {}^{t}#1}
\newcommand{\inverseTranspose}[1]{#1^{\iota}}
\newcommand{\IdentityMatrix}[1]{I_{#1}}
\newcommand{\diag}{\mathrm{diag}}
\newcommand{\antidiag}{\operatorname{\mathrm{antidiag}}}
\newcommand{\trace}{\operatorname{tr}}
\newcommand{\GL}{\mathrm{GL}}
\newcommand{\UnipotentSubgroup}{U}
\newcommand{\UnipotentRadicalForWss}[2]{N_{\left(#2^{#1}\right)}}
\newcommand{\UnipotentRadicalForWssRecursion}[2]{\mathcal{Y}_{c,k}}
\newcommand{\UnipotentRadical}{N}
\newcommand{\ParabolicSubgroup}{P}

% Finite field macros
\newcommand{\FieldNorm}[2]{\mathrm{N}_{#1:#2}}
\newcommand{\aFieldNorm}{\mathrm{N}}
\newcommand{\FieldTrace}{\mathrm{Tr}}
\newcommand{\finiteField}{\mathbb{F}}
\newcommand{\finiteFieldExtension}[1]{\finiteField_{#1}}
\newcommand{\FieldExtension}[2]{{#1} \slash {#2}}
\newcommand{\algebraicClosure}[1]{\overline{#1}}
\newcommand{\charactergroup}[1]{\widehat{\multiplicativegroup{\finiteFieldExtension{#1}}}}
\newcommand{\limitcharactergroup}{\Gamma}
\newcommand{\Galois}{\operatorname{Gal}}
\newcommand{\Frobenius}{\operatorname{Fr}}
\newcommand{\restrictionOfScalars}[3]{\operatorname{Res}_{#1 \slash #2}{#3}}
\newcommand{\multiplcativeScheme}{\mathbb{G}_m}
\newcommand{\affineLine}{\mathbb{A}^1}
\newcommand{\squareMatrix}{\operatorname{Mat}}
\newcommand{\Mat}[2]{\operatorname{Mat}_{#1 \times #2}}
\newcommand{\frobeniusDegree}{\operatorname{deg}}
\newcommand{\Steinberg}{\operatorname{St}}
\newcommand{\ProjectionOperator}{\operatorname{pr}}
\newcommand{\SymmetricGroup}{\mathfrak{S}}
\newcommand{\whittakerVector}[1]{v_{#1, \fieldCharacter}}
\newcommand{\gwhittakerVector}[2]{v_{#1, #2}}
\newcommand{\WhittakerProjection}{\ProjectionOperator_{\mathrm{Wh}}}
\newcommand{\ParabolicForSpeh}[2]{P_{\left({#1}^{#2}\right)}}
\newcommand{\UnipotentForSpeh}[2]{N_{\left({#1}^{#2}\right)}}
\newcommand{\PoincarePolynomial}[2]{P_{#2}}

%Partition macros
\newcommand{\localField}{F}
\newcommand{\ringOfIntegers}{\mathfrak{o}}
\newcommand{\residueField}{\mathfrak{f}}
\newcommand{\maximalIdeal}{\mathfrak{p}}
\newcommand{\depthZeroRepresentation}{\mathcal{T}}
\newcommand{\differential}{\mathrm{d}}
\newcommand{\mdifferential}{\differential^{\times}}
\newcommand{\quotientMap}{\nu}
\newcommand{\Lift}{\mathcal{L}}
\newcommand{\uniformizer}{\varpi}
\newcommand{\VolumeOf}{\operatorname{Vol}}

\newcommand{\Erdelyi}{Erd{\'e}lyi}
\newcommand{\Toth}{T{\'o}th}

\newcommand{\parabolicSection}{\Phi^{\left(z_1, \dots, z_c\right)}}
\newcommand{\intertwiningOperator}{M^{\left(z_1, \dots, z_c\right)}}
\newcommand{\holomorphicRepresentation}{\depthZeroRepresentation^{\left(z_1, \dots, z_c\right)}}
\newcommand{\WhittakerFunctional}[1]{\ell_{#1, \fieldCharacter}}
\newcommand{\gWhittakerFunctional}[2]{\ell_{#1, #2}}
\newcommand{\gSpehWhittakerFunctional}[3]{\ell_{\SpehRepresentation{#1}{#3}, \fieldCharacter_{\UnipotentRadicalForWss{#2}{#3}}}}
\newcommand{\gShortSpehWhittakerFunctional}[3]{\ell_{\SpehRepresentation{#1}{#3}}}
\newcommand{\GaussSum}[2]{\mathcal{G}\left(#1, #2\right)}
\newcommand{\dblGaussSum}[2]{\mathcal{G}^{\mathrm{dbl}}\left(#1, #2\right)}
\newcommand{\GaussSumScalar}[2]{\mathrm{G}\left(#1, #2\right)}
\newcommand{\dblGaussSumScalar}[2]{\mathrm{G}^{\mathrm{dbl}}\left(#1, #2\right)}
\newcommand{\dblVirtualGaussSumScalar}[2]{\mathrm{g}^{\mathrm{dbl}}\left(#1, #2\right)}
\newcommand{\dblGammaFactor}[3]{\Gamma^{\mathrm{dbl}}\left(#1 \times #2, #3\right)}
\newcommand{\dblGammaFactorSpace}[4]{\Gamma^{\mathrm{dbl}}_{#1}\left(#2 \times #3, #4\right)}
\newcommand{\GKGaussSum}[3]{\mathcal{G}\left(#1 \times #2, #3\right)}
\newcommand{\GKGaussSumScalar}[3]{\mathrm{G}\left(#1 \times #2, #3\right)}
\newcommand{\fieldCharacterkc}[2]{\fieldCharacter_{\left({#2}^{#1}\right)}}
\newcommand{\ExoticKloosterman}{\mathrm{Kl}}
\newcommand{\GaussSumCharacter}[3]{\tau\left(#1 \times #2, #3\right)}
\newcommand{\IrrCusp}{\Irr_{\mathrm{cusp}}}
\newcommand{\convolutionWithCompactSupport}{\boldsymbol{\mathrm{R}}}
\newcommand{\ladicnumbers}{\algebraicClosure{\mathbb{Q}_{\ell}}}
\newcommand{\artinScrier}{\operatorname{AS}}
\newcommand{\KloostermanSumClassFunction}{\mathcal{K}}
\newcommand{\IsometryGroup}{\mathrm{Isom}}
\newcommand{\lieAlgebra}{\mathfrak{g}}
\newcommand{\DeligneLusztigInduction}[2]{\mathrm{R}_{#1}^{#2}}
\newcommand{\algebraicGroup}[1]{\boldsymbol{\mathrm{#1}}}

\hypersetup{pdfauthor={Calvin Yost-Wolff, Elad Zelingher},
	pdfsubject={Number theory, Representation theory},
	pdfkeywords={Kloosterman sums, Gauss sums}}

\title{On exponential sums arising from the classical doubling method}

\author{Calvin Yost-Wolff}
\address{Department of Mathematics, University of Michigan, 3084 East Hall, 530 Church Street, Ann Arbor, MI 48109-1043 USA}
\email{calvinyw@umich.edu}

\author{Elad Zelingher}
\address{Department of Mathematics, University of Michigan, 1844 East Hall, 530 Church Street, Ann Arbor, MI 48109-1043 USA}
\email{eladz@umich.edu}

\subjclass[2010]{20C33, 11L05, 11T24}

% 11L05    Gauss and Kloosterman sums; generalizations
% 11T24    Other character sums and Gauss sums
% 20C33    Representations of finite groups of Lie type

\begin{document}

\begin{abstract}
\end{abstract}
\maketitle

\section{Notation}
Let $\finiteField$ be a finite field with $q$ elements, and let $\fieldCharacter \colon \finiteField \to \multiplicativegroup{\cComplex}$ be a non-trivial character. 

\section{Kondo Gauss sums}
Let $\tau$ be an irreducible representation of $\GL_n\left(\finiteField\right)$. 
For a character $\chi \colon \multiplicativegroup{\finiteField} \to \multiplicativegroup{\cComplex}$, denote the \emph{twisted Gauss sum}
$$\GaussSum{\tau \times \chi}{\fieldCharacter} = q^{-\frac{n^2}{2}} \sum_{g \in \GL_n\left(\finiteField\right)} \tau\left(g\right) \chi\left(\det g\right) \fieldCharacter\left(\trace g\right).$$
This element lies in $\Hom_{\GL_n\left(\finiteField\right)}\left(\tau, \tau\right)$ and therefore by Schur's lemma there exists a complex number $\GaussSumScalar{\tau \times \chi}{\fieldCharacter} \in \cComplex$ such that
$$\GaussSum{\tau \times \chi}{\fieldCharacter} = \GaussSumScalar{\tau \times \chi}{\fieldCharacter} \cdot \idmap_\tau.$$ The computation of this scalar is known due to Kondo.

\section{Doubling method Gauss sums}

\subsection{The case of general linear groups}

Let $\chi \colon \multiplicativegroup{\finiteField} \to \multiplicativegroup{\cComplex}$ be a non-trivial character.

Consider the following assignment $\GL_k\left(\finiteField\right) \to \cComplex$
$$\Phi_{\chi}\left(g\right) = \begin{dcases}
	\chi\left(\det\left(g-\IdentityMatrix{k}\right)\right) & \text{if }\det\left(g-\IdentityMatrix{k}\right) \ne 0,\\
	0 & \text{otherwise.}
\end{dcases}$$
It is clear that $\Phi_{\chi}$ is a class function of $\GL_k\left(\finiteField\right)$.

Given an irreducible representation $\tau$ of $\GL_k\left(\finiteField\right)$, we consider the following \emph{doubling method Gauss sum}:
$$\dblGaussSum{\tau}{\chi} = q^{-\frac{k^2}{2}} \sum_{g \in \GL_k\left(\finiteField\right)} \Phi_{\chi}\left(g\right) \tau\left(g\right).$$
Then $\dblGaussSum{\tau}{\chi}$ defines an element of $\Hom_{\GL_k\left(\finiteField\right)}\left(\tau, \tau\right)$. Therefore, by Schur's lemma there exists a complex number $\dblGaussSumScalar{\tau}{\chi} \in \cComplex$ such that $$\dblGaussSum{\tau}{\chi} = \dblGaussSumScalar{\tau}{\chi} \cdot \idmap_\tau.$$ 

The goal of this section is to express $\dblGaussSumScalar{\tau}{\chi}$ in terms of Kondo's Gauss sum.

\begin{proposition}\label{prop:doubling-for-gln-in-terms-of-kondo}
	We have the identity
	$$\Phi_{\chi}\left(g\right) = q^{-\frac{k^2}{2}} \GaussSumScalar{\chi_{\GL_k}}{\fieldCharacter} \sum_{\substack{x, y \in \GL_k\left(\finiteField\right)\\
			y^{-1} x = -g}} \fieldCharacter\left(\trace x\right) \chi^{-1}\left(\det y\right) \fieldCharacter\left(\trace y\right).$$
\end{proposition}
\begin{proof}
	We write $$\Phi_{\chi}\left(g\right) = \frac{1}{\sizeof{\squareMatrix_k\left(\finiteField\right)}}\sum_{X \in \squareMatrix_k\left(\finiteField\right)} \sum_{h \in \GL_k\left(\finiteField\right)} \fieldCharacter\left(\trace \left(X\left(g-\IdentityMatrix{k}-h\right)\right)\right) \chi\left(\det h\right).$$
	We claim that for a singular matrix $X \in \squareMatrix_k\left(\finiteField\right)$ the inner sum vanishes. To show this, write $$X = h_1 \begin{pmatrix}
		\IdentityMatrix{k-r}\\
		& 0_r
	\end{pmatrix} h_2,$$
	where $1 \le r \le k$ and $h_1, h_2 \in \GL_k\left(\finiteField\right)$. Then
	$$\fieldCharacter\left(\trace Xh\right) = \fieldCharacter\left(\trace\left( \begin{pmatrix}
		\IdentityMatrix{k-r}\\
		& 0_r
	\end{pmatrix} h_2 h h_1\right)\right).$$
	Changing variables $h \mapsto h_2^{-1} h h_1^{-1},$
	we obtain the inner sum
	$$\sum_{h \in \GL_k\left(\finiteField\right)} \fieldCharacter\left(\trace \begin{pmatrix}
		\IdentityMatrix{k-r}\\
		& 0_r
	\end{pmatrix} h \right) \chi\left(\det h\right),$$
	which equals
	$$\frac{1}{\sizeof{\multiplicativegroup{\finiteField}}} \sum_{a \in \multiplicativegroup{\finiteField}} \sum_{h \in \GL_k\left(\finiteField\right)} \fieldCharacter\left(\trace \begin{pmatrix}
		\IdentityMatrix{k-r}\\
		& 0_r
	\end{pmatrix} \begin{pmatrix}
		\IdentityMatrix{k-1}\\
		& a
	\end{pmatrix} h \right) \chi\left(\det h\right).$$
	Changing variables $h \mapsto \left(\begin{smallmatrix}
		\IdentityMatrix{k-1}\\
		& a^{-1}
	\end{smallmatrix}\right) h$, we get the inner sum $$\sum_{a \in \multiplicativegroup{\finiteField}} \chi^{-1}\left(a\right),$$
	which vanishes because $\chi \ne 1$.
	Hence, we have
	$$\Phi_{\chi}\left(g\right) = \frac{1}{\sizeof{\squareMatrix_k\left(\finiteField\right)}} \sum_{X, h \in \GL_k\left(\finiteField\right)} \fieldCharacter\left(\trace \left(X\left(g-\IdentityMatrix{k}-h\right)\right)\right) \chi\left(\det h\right).$$
	Changing variables $h \mapsto X^{-1} h$, we get
	$$\Phi_{\chi}\left(g\right) = \frac{1}{\sizeof{\squareMatrix_k\left(\finiteField\right)}} \sum_{X, h \in \GL_k\left(\finiteField\right)} \fieldCharacter\left(\trace \left(Xg\right)\right) \fieldCharacter\left(-\trace h\right) \chi\left(\det h\right) \chi^{-1}\left(\det X\right) \fieldCharacter\left(-\trace X\right).$$
	Changing variables again $h \mapsto -h$ and setting $Xg = x$ and $-X = y$, we get  
	$$\Phi_{\chi}\left(g\right) = q^{-\frac{k^2}{2}} \GaussSumScalar{\chi_{\GL_k}}{\fieldCharacter} \sum_{\substack{x, y \in \GL_k\left(\finiteField\right)\\
			y^{-1} x = -g}} \fieldCharacter\left(\trace x\right) \chi^{-1}\left(\det y\right) \fieldCharacter\left(\trace y\right).$$
\end{proof}

We now use this to compute $\dblGaussSum{\tau}{\chi}$. We have that $$\dblGaussSum{\tau}{\chi} = q^{-k^2} \GaussSumScalar{\chi_{\GL_k}}{\fieldCharacter} \sum_{x, y \in \GL_k\left(\finiteField\right)} \fieldCharacter\left(\trace x\right) \chi^{-1}\left(\det y\right) \fieldCharacter\left(\trace y\right) \pi\left(-y^{-1} x\right).$$
The last equality implies the following result.
\begin{theorem}\label{thm:gln-doubling-gauss-sum-in-terms-of-kondo}For any irreducible representation $\tau$ of $\GL_k\left(\finiteField\right)$ and any non-trivial character $\chi \colon \multiplicativegroup{\finiteField} \to \multiplicativegroup{\cComplex}$, the following identity holds:
	$$\dblGaussSumScalar{\tau}{\chi} = \centralCharacter{\tau}\left(-1\right) \GaussSumScalar{\chi}{\fieldCharacter}^k \GaussSumScalar{\tau}{\fieldCharacter} \GaussSumScalar{\tau^{\vee} \times \chi^{-1}}{\fieldCharacter}.$$
\end{theorem}

\subsection{The case of classical groups}

Let $\hermitianSpace$ be a vector space of dimension $n$ over $\finiteField$, equipped with a non-degenerate bilinear form $\innerproduct{\cdot}{\cdot} \colon \hermitianSpace \times \hermitianSpace \to \finiteField$ which is either symmetric or anti-symmetric. Let $G = \IsometryGroup \left(\hermitianSpace\right)$ be the isometry group of $\hermitianSpace$, consisting of all the elements of $\GL\left(\hermitianSpace\right)$ that satisfy $\innerproduct{gx}{gy} = \innerproduct{x}{y}$ for every $x,y \in \hermitianSpace$. Let $\lieAlgebra$ be the Lie algebra of $G$, consisting of all elements $A \in \EndomorphismRing\left(\hermitianSpace\right)$ satisfying $\innerproduct{Ax}{y} + \innerproduct{x}{AY} = 0$ for every $x, y \in \hermitianSpace$.

Let $\chi \colon \multiplicativegroup{\finiteField} \to \multiplicativegroup{\cComplex}$ be a character. As before, the assignment $\Phi_{\chi} \colon \GL\left(\hermitianSpace\right) \to \cComplex$ given by $$\Phi_{\chi}\left(g\right) = \begin{dcases}
\chi\left(\det\left(g - \idmap_{\hermitianSpace}\right)\right) & \text{if }\det\left(g - \idmap_{\hermitianSpace}\right) \ne 0\\
0 & \text{otherwise,}
\end{dcases}$$
is a class function of $\GL\left(\hermitianSpace\right)$.

Let $\pi$ be an irreducible representation of $G$. Denote $$\dblGaussSum{\pi}{\chi} = \frac{1}{\sqrt{\sizeof{\lieAlgebra}}} \sum_{g \in G} \pi\left(g\right) \Phi_{\chi}\left(g\right).$$
Since $\Phi_{\chi}$ is a class function of $\GL\left(\hermitianSpace\right)$, it is also a class function of $G$, and therefore $\dblGaussSum{\pi}{\chi} \in \Hom_{G}\left(\pi, \pi\right)$. By Schur's lemma, there exists a constant $\dblGaussSumScalar{\pi}{\fieldCharacter} \in \cComplex$, such that $$\dblGaussSum{\pi}{\chi} = \dblGaussSumScalar{\pi}{\chi} \cdot \idmap_\pi.$$
We call $\dblGaussSumScalar{\pi}{\chi}$ the \emph{doubling method Gauss sum}.

\subsubsection{Multiplicativity property}
Our next goal is to understand how the doubling method Gauss sum behaves under parabolic induction.

Let $\xIsotropic$ and $\yIsotropic$ be isotropic spaces of $\hermitianSpace$ of dimension $k$, such that $\xIsotropic$ and $\yIsotropic$ are in duality with respect to form $\innerproduct{\cdot}{\cdot}$. Let us write $$\hermitianSpace = \xIsotropic \oplus \hermitianSpace' \oplus \yIsotropic,$$
where $\hermitianSpace' \subset \hermitianSpace$ is a non-degenerate subspace, orthogonal to $\xIsotropic$ and $\yIsotropic$. Let $P$ be the parabolic subgroup of $G$, consisting of all elements stabilizing the flag $$0 \subset \xIsotropic \subset \xIsotropic \oplus \hermitianSpace' \subset \xIsotropic \oplus \hermitianSpace' \oplus \yIsotropic = \hermitianSpace.$$
Write $P = L \ltimes N$, where $L$ is the Levi part of $P$ and $N$ is the unipotent radical of $P$. Then $L$ is isomorphic to $G' \times \GL_k\left(\finiteField\right)$, where $G' = \IsometryGroup\left(\hermitianSpace'\right)$.

\begin{theorem}\label{thm:multiplicativity-in-terms-of-gauss-sums}
	Let $\pi'$ be an irreducible representation of $G'$ and let $\tau$ be an irreducible representation of $\GL_k\left(\finiteField\right)$. Then for any irreducible representation $\pi$ of $G$ which appears as a subrepresentation the parabolic induction $\rho = \Ind{P}{G}{\tau \overline{\otimes} \pi'}$ and any $\chi$ such that $\chi^2 \ne 1$, we have
	$$\dblGaussSumScalar{\pi}{\chi} = \centralCharacter{\tau}\left(-1\right) \GaussSumScalar{\chi^2}{\fieldCharacter}^k \GaussSumScalar{\tau \times \chi^{-1}}{\fieldCharacter} \GaussSumScalar{\Contragradient{\tau} \times \chi^{-1}}{\fieldCharacter} \dblGaussSumScalar{\pi'}{\chi}.$$
\end{theorem}
\begin{proof}
	Let $v_{\pi'} \in \pi'$ and $v_{\tau} \in \tau$. Consider $f \in \rho$ defined as follows. The function $f \in \rho$ is the unique element supported on $P$, such that $f\left(\idmap_{\hermitianSpace}\right) = v_{\tau} \otimes v_{\pi'}$. The projection $f$ to any non-zero invariant subspace of $\rho$ is non-zero. We compute $\dblGaussSum{\rho}{\chi} f$.
	
	Notice that for $x \in G$, $$\left(\dblGaussSum{\rho}{\chi} f\right)\left(x\right) = \frac{1}{\sqrt{\sizeof{\lieAlgebra}}} \sum_{g \in G} f\left(xg\right) \Phi_{\chi}\left(g\right) = \frac{1}{\sqrt{\sizeof{\lieAlgebra}}} \sum_{p \in P} f\left(p\right) \Phi_{\chi}\left(x^{-1} p\right),$$
	which can be rewritten as
	$$\left(\dblGaussSum{\rho}{\chi} f\right)\left(x\right) = \frac{1}{\sqrt{\sizeof{\lieAlgebra}}} \sum_{p \in P}  \Phi_{\chi}\left(x^{-1} p\right) \left(\tau \overline{\otimes} \pi'\right)\left(p\right) v_\tau \otimes v_{\pi'}.$$
	As before, we may write
	$$\Phi_{\chi}\left(g\right) = q^{-n^2} \sum_{h \in \GL\left(\hermitianSpace\right)} \sum_{A \in \EndomorphismRing\left(\hermitianSpace\right)} \chi^{-1}\left(\det h\right) \fieldCharacter\left(\trace\left(A \left(g - \idmap_{\hermitianSpace} - h^{-1}\right)\right)\right).$$
	Hence, \begin{align*}
		q^{n^2} \sum_{p \in P} \Phi_{\chi}\left(xp\right) \left(\tau \overline{\otimes} \pi'\right)\left(p\right) =& \sum_{A \in \EndomorphismRing\left(\hermitianSpace\right)} \sum_{h \in \GL\left(\hermitianSpace\right)} \chi\left(\det h\right)\fieldCharacter\left(\trace\left(Ah\right)\right) \\
		& \times \sum_{p \in P} \fieldCharacter\left(\trace\left(A \left(xp - \idmap_{\hermitianSpace}\right)\right)\right) \left(\tau \overline{\otimes} \pi'\right)\left(p\right).
	\end{align*}
	Assume that $\chi \ne 1$. Then since $\GaussSum{\chi_{\GL\left(\hermitianSpace\right)}}{\fieldCharacter}_A$ appears in the last equation, the sum can be reduced to $A \in \GL\left(\hermitianSpace\right)$. Let us replace $h$ with $A^{-1} h$. We get the sum
	\begin{align*}
		\sum_{A \in \GL\left(\hermitianSpace\right)} \chi^{-1}\left(\det A\right) \sum_{h \in \GL\left(\hermitianSpace\right)} \chi\left(\det h\right)\fieldCharacter\left(\trace h\right) \sum_{p \in P} \fieldCharacter\left(\trace\left(A \left(xp - \idmap_{\hermitianSpace}\right)\right)\right) \left(\tau \overline{\otimes} \pi'\right)\left(p\right).
	\end{align*}
	Consider the inner sum over $P$, written as $$\sum_{l \in L} \sum_{u \in N} \fieldCharacter\left(\trace\left(A x l u\right)\right) \left(\tau \overline{\otimes} \pi'\right)\left(l\right).$$
	The inner sum over $N$ will vanish unless $Ax \in P$. Thus we have
	\begin{align*}
		q^{n^2} \sum_{p \in P} \Phi_{\chi}\left(xp\right) \left(\tau \overline{\otimes} \pi'\right)\left(p\right) =& \sum_{h \in \GL\left(\hermitianSpace\right)}  \chi\left(\det\left(xh\right)\right) \fieldCharacter\left(\trace h\right) \\
		& \times \sum_{p' \in P} \sum_{p \in P} \chi^{-1}\left(\det p'\right) \fieldCharacter\left(\trace\left(p' \left(p - x^{-1}\right)\right)\right) \left(\tau \overline{\otimes} \pi'\right)\left(p\right).
	\end{align*}
	By decomposing the sum over $p' \in P$ into a sum over $N$ and $L$, we see that the inner sum over $N$ will vanish unless $p - x^{-1} \in P$, which implies that $x \in P$.
	
	Hence, we have that $\dblGaussSum{\rho}{\chi} f$ is supported on $P$. We move to compute $\left(\dblGaussSum{\rho}{\chi} f\right)\left(\idmap_{\hermitianSpace}\right)$. It is given by
	\begin{equation}\label{eq:recursive-doubling-gauss-sum}
		\frac{1}{\sqrt{\sizeof{\lieAlgebra}}} \sum_{\substack{p \in P\\
				\det\left(p - \idmap_{\hermitianSpace}\right) \ne 0}} \chi\left(\det\left(p - \idmap_{\hermitianSpace}\right)\right) \left(\tau \overline{\otimes} \pi'\right)\left(p\right) v_{\tau} \otimes v_{\pi'}.
	\end{equation}
	Decomposing the sum \eqref{eq:recursive-doubling-gauss-sum} as a sum over $L$ and $N$, and using the fact that if $p \in P$ has Levi part with image $\left(a, g'\right) \in \GL_k\left(\finiteField\right) \times G'$ then $$\det\left(p - \idmap_{\hermitianSpace}\right) = \left(-1\right)^k \det\left(a\right)^{-1}\det\left(a - I_k\right)^2 \det\left( g' - \idmap_{\hermitianSpace'}\right),$$
	and using the fact that $\sizeof{\lieAlgebra} = \sizeof{\lieAlgebra'} \sizeof{\squareMatrix_k\left(\finiteField\right)} \sizeof{N}^2$, where $\lieAlgebra'$ is the Lie algebra of $G'$,
	we get that \eqref{eq:recursive-doubling-gauss-sum} equals
	\begin{equation}
		\chi\left(-1\right)^k q^{-\frac{k^2}{2}} \sum_{a \in \GL_k\left(\finiteField\right)} \Phi_{\chi^2}\left(a\right) \chi^{-1}\left(\det a\right) \tau\left(a\right) v_{\tau} \otimes \frac{1}{\sqrt{\sizeof{\lieAlgebra'}}} \sum_{g' \in G'} \Phi_{\chi}\left(g'\right) \pi'\left(g'\right) v_{\pi'},
	\end{equation}
	which in turn is
	\begin{align*}
		&\chi\left(-1\right)^k \dblGaussSum{\tau \otimes \chi_{\GL_k}^{-1}}{\chi^2} v_{\tau} \otimes \dblGaussSum{\pi'}{\chi} v_{\pi'}\\
		=& \chi\left(-1\right)^k\dblGaussSumScalar{\tau \otimes \chi_{\GL_k}^{-1}}{\chi^2} \dblGaussSumScalar{\pi'}{\chi} v_{\tau} \otimes v_{\pi'}.
	\end{align*}
	The result now follows.
\end{proof}

\subsubsection{Gamma factors}
In order to make the computations more tolerable, it is beneficial to work with fully multiplicative functions. We introduce the doubling method gamma factors.

Let $\chi \colon \multiplicativegroup{\finiteField} \to \multiplicativegroup{\cComplex}$ be a character. For an irreducible representation  $\tau$ of $\GL_k\left(\finiteField\right)$, define $$\dblGammaFactor{\tau}{\chi}{\fieldCharacter} = \GaussSumScalar{\tau \times \chi^{-1}}{\fieldCharacter} \GaussSumScalar{\Contragradient{\tau} \times \chi^{-1}}{\fieldCharacter}.$$
For an irreducible representation $\pi$ of $G$, define the \emph{doubling method gamma factor} $$\dblGammaFactorSpace{\hermitianSpace}{\pi}{\chi}{\fieldCharacter} = \centralCharacter{\pi}\left(-1\right) \frac{\dblGaussSumScalar{\pi}{\chi}}{\GaussSumScalar{\chi^2}{\fieldCharacter}^{\left[\frac{\dim \hermitianSpace}{2}\right]}}.$$
Using this notation and using the multiplicativity property of Kondo's Gauss sum and \Cref{thm:multiplicativity-in-terms-of-gauss-sums}, we have the following multiplicativity property.

\begin{theorem}
	\begin{enumerate}
		\item If $\tau_1$ and $\tau_2$ are irreducible representations of $\GL_{k_1}\left(\finiteField\right)$ and $\GL_{k_2}\left(\finiteField\right)$, respectively, then for any irreducible subrepresentation $\tau$ of the parabolic induction $\tau_1 \circ \tau_2$, we have
		$$\dblGammaFactor{\tau}{\chi}{\fieldCharacter} = \dblGammaFactor{\tau_1}{\chi}{\fieldCharacter} \dblGammaFactor{\tau_2}{\chi}{\fieldCharacter}.$$
		\item If $\tau$, $\pi'$ and $\pi$ are as in \Cref{thm:multiplicativity-in-terms-of-gauss-sums} and $\chi \ne 1$, then
		$$\dblGammaFactorSpace{\hermitianSpace}{\pi}{\chi}{\fieldCharacter} = \dblGammaFactorSpace{\hermitianSpace'}{\pi'}{\chi}{\fieldCharacter} \dblGammaFactor{\tau}{\chi}{\fieldCharacter}.$$
	\end{enumerate}
\end{theorem}


\subsection{Computation in terms of Deligne--Lusztig data}

The goal of this section is to compute $\dblGaussSumScalar{\pi}{\chi}$ for $\pi$ that appears in a Deligne--Lusztig virtual character $R_{T,\theta}$ for a suitable maximal torus $T$ and $\theta \colon T \to \multiplicativegroup{\cComplex}$.

Let us realize $G$ as the fixed points of the Frobenius map $\Frobenius$ acting on a connected reductive algebraic group $\algebraicGroup{G}$.

Given a class function $F \colon G \to \cComplex$ (i.e., a function that is invariant under conjugation by elements of $G$), let us define $$\dblVirtualGaussSumScalar{F}{\chi} = \frac{1}{\sqrt{\sizeof{\lieAlgebra}}} \sum_{g \in G} F\left(g\right) \Phi_{\chi}\left(g\right).$$

It is clear that if $\pi$ is an irreducible representation of $G$ then $$\dblVirtualGaussSumScalar{\trace \pi}{\chi} = \dblGaussSumScalar{\pi}{\chi} \cdot \dim \pi.$$
Notice that the assignment $g \mapsto \Phi_{\chi}\left(g\right)$ only depends on the semisimple part of $g \in G$. By \cite[Theorem in Section 1.2]{SaitoShinoda2000}, we have the following result.
\begin{proposition}
	For any $\Frobenius$-stable maximal torus $\algebraicGroup{T}$ of $\algebraicGroup{G}$, and any character $\theta \colon T \to \multiplicativegroup{\cComplex}$, where $T = \algebraicGroup{T}^{\Frobenius}$, we have
	$$ \dblVirtualGaussSumScalar{R_{T, \theta}}{\chi} = \frac{\grpIndex{G}{T}}{\sqrt{\sizeof{\lieAlgebra}}} \sum_{t \in T} \theta\left(t\right) \Phi_{\chi}\left(t\right).$$
\end{proposition}
In the special case that $\theta$ is in general position, the virtual character $\varepsilon_{\algebraicGroup{G}} \varepsilon_{\algebraicGroup{T}} R_{T, \theta}$ equals $\trace \pi$ for some irreducible representation $\pi$ of $G$ and we have that for this $\pi$,
$$\dblGaussSumScalar{\pi}{\chi} = \varepsilon_{\algebraicGroup{G}} \varepsilon_{\algebraicGroup{T}} \frac{\sizeof{G}_p}{\sqrt{\sizeof{\lieAlgebra}}} \sum_{t \in T} \theta\left(t\right) \Phi_{\chi}\left(t\right).$$
Here $\varepsilon_{\algebraicGroup{G}} = \left(-1\right)^{\mathrm{rel.rank} \algebraicGroup{G}}$ and $\varepsilon_{\algebraicGroup{T}} = \left(-1\right)^{\mathrm{rel.rank} \algebraicGroup{T}}$, and $\sizeof{G}_p$ is the size of the $p$-Sylow group of $G$, where $p$ is the characteristic of $\finiteField$. 

\subsubsection{Gauss sum notations}
If $\alpha \colon \multiplicativegroup{\finiteFieldExtension{n}} \to \multiplicativegroup{\cComplex}$ is a character, denote the \emph{Gauss sum} $$\tau\left(\alpha, \fieldCharacter_n\right) = -q^{-\frac{n}{2}}\sum_{x \in \multiplicativegroup{\finiteFieldExtension{n}}} \alpha\left(x\right) \fieldCharacter_n\left(x\right),$$
where $\fieldCharacter_n \colon \finiteFieldExtension{n} \to \multiplicativegroup{\cComplex}$ is given by $\fieldCharacter_n = \fieldCharacter \circ \trace_{\finiteFieldExtension{n} \slash \finiteField}$. If $\chi \colon \multiplicativegroup{\finiteField} \to \multiplicativegroup{\cComplex}$ is a character and if $\alpha$ is as above, denote the \emph{twisted Gauss sum}
$$\tau\left(\alpha \times \chi, \fieldCharacter_n\right) = -q^{-\frac{n}{2}}\sum_{x \in \multiplicativegroup{\finiteFieldExtension{n}}} \alpha\left(x\right) \chi\left( \FieldNorm{n}{1}\left(x\right)\right) \fieldCharacter_n\left(x\right).$$

If $m \ge 1$, we denote 
$$\finiteFieldExtension{2m}^1 = \left\{ x \in \multiplicativegroup{\finiteFieldExtension{2m}} \mid \FieldNorm{2m}{m}\left(x\right)= 1\right\}.$$

\subsubsection{Split torus computation I}
Let $\alpha \colon \multiplicativegroup{\finiteFieldExtension{k}} \to \multiplicativegroup{\cComplex}$ be a character and let $\chi \colon \multiplicativegroup{\finiteField} \to \multiplicativegroup{\cComplex}$ be a non-trivial character.

The goal of this section is to compute $$\sum_{x \in \multiplicativegroup{\finiteFieldExtension{k}}} \alpha\left(x\right) \chi\left(\FieldNorm{k}{1}\left(x-1\right)\right).$$
Let us rewrite this as
$$q^{-k} \sum_{z \in \finiteFieldExtension{k}} \sum_{x \in  \multiplicativegroup{\finiteFieldExtension{k}}} \sum_{y \in \multiplicativegroup{\finiteFieldExtension{k}}}\alpha\left(x\right) \chi\left(\FieldNorm{k}{1}\left(y\right)\right) \fieldCharacter_k\left(\left(y - \left(x - 1\right)\right)z\right).$$
Since $\chi$ is non-trivial, when $z = 0$, the inner sum over $y$ will vanish. Hence, we have that our sum can be rewritten as
$$q^{-k} \sum_{z \in \multiplicativegroup{\finiteFieldExtension{k}}} \sum_{x \in  \multiplicativegroup{\finiteFieldExtension{k}}} \sum_{y \in \multiplicativegroup{\finiteFieldExtension{k}}}\alpha\left(x\right) \chi\left(\FieldNorm{k}{1}\left(y\right)\right) \fieldCharacter_k\left(\left(y - \left(x - 1\right)\right)z\right).$$
Replacing $x$ with $-z^{-1} x$ and $y$ with $z^{-1} y$, this becomes
$$q^{-k} \alpha\left(-1\right) \sum_{z \in \multiplicativegroup{\finiteFieldExtension{k}}} \alpha^{-1}\left(z\right) \chi^{-1}\left(\FieldNorm{k}{1}\left(z\right)\right) \fieldCharacter_k\left(z\right) \sum_{x \in  \multiplicativegroup{\finiteFieldExtension{k}}} \alpha\left(x\right) \fieldCharacter_k\left(x\right) \sum_{y \in \multiplicativegroup{\finiteFieldExtension{k}}} \chi\left(\FieldNorm{k}{1}\left(y\right)\right) \fieldCharacter_k\left(y\right).$$
Hence, we have the identity
$$\sum_{x \in \multiplicativegroup{\finiteFieldExtension{k}}} \alpha\left(x\right) \chi\left(\FieldNorm{k}{1}\left(x-1\right)\right) = -q^{\frac{k}{2}} \alpha\left(-1\right) \GaussSumCharacter{\alpha^{-1}}{\chi^{-1}}{\fieldCharacter_k} \tau\left(\alpha, \fieldCharacter_k\right) \tau\left(\chi_{\GL_k}, \fieldCharacter_k\right).$$

\subsubsection{Split torus computation II}
We use the results from the previous section to compute the doubling method Gauss sum for a torus of the form $\multiplicativegroup{\finiteFieldExtension{k}}$. Let $\alpha \colon \multiplicativegroup{\finiteFieldExtension{k}} \to \multiplicativegroup{\cComplex}$ and $\chi \colon \multiplicativegroup{\finiteField} \to \multiplicativegroup{\cComplex}$ be characters such that $\chi^2 \ne 1$. Our goal is to compute $$\sum_{\substack{x \in \multiplicativegroup{\finiteFieldExtension{k}}\\
		x \ne 1}} \alpha \left(x\right) \chi\left(\FieldNorm{k}{1}\left(x - 1\right)\right) \chi\left(\FieldNorm{k}{1}\left(x^{-1} - 1\right)\right).$$
This can be rewritten as
$$\chi\left(-1\right)^k \sum_{\substack{x \in \multiplicativegroup{\finiteFieldExtension{k}}\\
		x \ne 1}} \alpha \left(x\right) \chi^{-1}\left(\FieldNorm{k}{1}\left(x\right)\right) \chi^2\left(\FieldNorm{k}{1}\left(x - 1\right)\right).$$
By the previous section, this equals
$$- q^{\frac{k}{2}} \alpha\left(-1\right)  \GaussSumCharacter{\alpha^{-1}}{\chi^{-1}}{\fieldCharacter_k} \GaussSumCharacter{\alpha}{\chi^{-1}}{\fieldCharacter_k} \tau\left(\chi^2, \fieldCharacter\right)^k.$$

\subsubsection{Elliptic torus computation}
In this section we compute the doubling method Gauss sum for a torus of the form $\finiteFieldExtension{2m}^1$.

Let $\theta \colon \finiteFieldExtension{2m}^1 \to \multiplicativegroup{\cComplex}$ and $\chi \colon \multiplicativegroup{\finiteField} \to \multiplicativegroup{\cComplex}$ be non-trivial characters, such that $\chi^2 \ne 1$. It follows that $\theta\left(x^{1-q^m}\right) \ne \chi \circ \FieldNorm{2m}{1}\left(x\right)$ for some $x \in \multiplicativegroup{\finiteFieldExtension{2m}}$. Our goal is to compute $$\sum_{\substack{x \in \finiteFieldExtension{2m}^1\\
		x \ne 1}} \theta \left(x\right) \chi\left(\FieldNorm{2m}{1}\left(x - 1\right)\right).$$

As usual, we rewrite this sum as follows
$$q^{-2m} \sum_{z \in \finiteFieldExtension{2m}} \sum_{y \in \multiplicativegroup{\finiteFieldExtension{2m}}} \sum_{x \in \finiteFieldExtension{2m}^1} \theta \left(x\right) \chi\left(\FieldNorm{2m}{1}\left(y\right)\right) \fieldCharacter_{2m}\left(z\left(y-x+1\right)\right).$$
If $z=0$ then the sum over $y$ will vanish. Hence we can reduce the sum to $z \in \multiplicativegroup{\finiteFieldExtension{2m}}$. Replacing $y$ with $z^{-1} y$ we get the sum
$$-q^{-m} \tau\left(\chi \circ \FieldNorm{2m}{1}, \fieldCharacter_{2m}\right) \sum_{z \in \multiplicativegroup{\finiteFieldExtension{2m}}} \chi^{-1}\left(\FieldNorm{2m}{1}\left(z\right)\right) \fieldCharacter_{2m}\left(z\right)  \sum_{x \in \finiteFieldExtension{2m}^1} \theta \left(-x\right) \fieldCharacter_{2m}\left(xz\right).$$
To proceed, we use the Hilbert 90 map $\multiplicativegroup{\finiteFieldExtension{2m}} \to \finiteFieldExtension{2m}^1$ given by $t \mapsto t^{1 - q^m}$. This map is surjective, and its kernel is $\multiplicativegroup{\finiteFieldExtension{m}}$. Define $\theta' \colon \multiplicativegroup{\finiteFieldExtension{2m}} \to \multiplicativegroup{\cComplex}$ by $\theta'\left(t\right) = \theta\left(t^{1-q^m}\right)$. Replacing the sum over $x \in \finiteFieldExtension{2m}^1$ with a sum over $t \in \multiplicativegroup{\finiteFieldExtension{2m}}$, we get \begin{align*}
	& \sum_{z \in \multiplicativegroup{\finiteFieldExtension{2m}}} \chi^{-1}\left(\FieldNorm{2m}{1}\left(z\right)\right) \fieldCharacter_{2m}\left(z\right) \sum_{x \in \finiteFieldExtension{2m}^1} \theta \left(x\right) \fieldCharacter_{2m}\left(xz\right) \\
	= & \frac{1}{q^m-1}\sum_{z \in \multiplicativegroup{\finiteFieldExtension{2m}}} \sum_{t \in \multiplicativegroup{\finiteFieldExtension{2m}}} \chi^{-1}\left(\FieldNorm{2m}{1}\left(z\right)\right) \fieldCharacter_{2m}\left(z^{q^m}\right) \theta' \left(t\right) \fieldCharacter_{2m}\left(t^{1-q^m} z\right).
\end{align*}
Replacing $z$ with $t^{q^m} z$, this becomes
\begin{align*}
	\frac{1}{q^m-1}\sum_{t,z \in \multiplicativegroup{\finiteFieldExtension{2m}}} \chi^{-1}\left(\FieldNorm{2m}{1}\left(z\right)\right) \chi^{-1}\left(\FieldNorm{2m}{1}\left(t\right)\right) \theta' \left(t\right) \fieldCharacter_{2m}\left(\trace_{\finiteFieldExtension{2m} \slash \finiteFieldExtension{m}}\left(z\right) t\right).
\end{align*}
Since $\chi \circ \FieldNorm{2m}{1} \ne \theta'$, if $\trace_{\finiteFieldExtension{2m} \slash \finiteFieldExtension{m}}\left(z\right) = 0$, we have that the inner sum over $t$ is zero. Hence, we may reduce the sum over $z$ to $z$ such that $\trace_{\finiteFieldExtension{2m} \slash \finiteFieldExtension{m}}\left(z\right) \ne 0$. Using the fact that $\theta'$ is trivial on $\multiplicativegroup{\finiteFieldExtension{m}}$, we have after replacing variables $t \mapsto \frac{t}{\trace_{\finiteFieldExtension{2m} \slash \finiteFieldExtension{m}}\left(z\right)}$,
\begin{align*}
	\frac{1}{q^m-1}\sum_{\substack{t,z \in \multiplicativegroup{\finiteFieldExtension{2m}}\\
	\trace_{\finiteFieldExtension{2m} \slash \finiteFieldExtension{m}}\left(z\right) \ne 0}} \chi^{-1}\left(\FieldNorm{2m}{1}\left(\frac{z}{\trace_{\finiteFieldExtension{2m} \slash \finiteFieldExtension{m}}\left(z\right) }\right)\right) \chi^{-1}\left(\FieldNorm{2m}{1}\left(t\right)\right) \theta' \left(t\right) \fieldCharacter_{2m}\left(t\right).
\end{align*}
By the appendix, this equals
\begin{align*}
	q^{\frac{m}{2}} \tau\left(\chi^{-1} \circ \FieldNorm{m}{1}, \fieldCharacter_m\right)^2 \tau\left(\chi^{2} \circ \FieldNorm{m}{1}, \fieldCharacter_m\right) \left(\sum_{t \in \multiplicativegroup{\finiteFieldExtension{2m}}} \chi^{-1}\left(\FieldNorm{2m}{1}\left(t\right)\right) \theta' \left(t\right) \fieldCharacter_{2m}\left(t\right)\right).
\end{align*}

To summarize, we have
$$\sum_{\substack{x \in \finiteFieldExtension{2m}^1\\
		x \ne 1}} \theta \left(-x\right) \chi\left(\FieldNorm{2m}{1}\left(x - 1\right)\right) = q^{\frac{m}{2}} \tau\left(\chi^{-1}, \fieldCharacter\right)^{2m} \tau\left(\chi, \fieldCharacter\right)^{2m} \tau\left(\chi^{2}, \fieldCharacter\right)^m \tau\left(\theta' \times \chi^{-1}, \fieldCharacter_{2m}\right),$$
	where $\theta'\left(t\right) = \theta\left(t^{1-q^m}\right)$.
Since $\tau\left(\chi^{-1}, \fieldCharacter\right) = \chi\left(-1\right) \conjugate{\tau\left(\chi, \fieldCharacter\right)}$ and since $\chi \ne 1$, this implies the identity
$$\sum_{\substack{x \in \finiteFieldExtension{2m}^1\\
		x \ne 1}} \theta \left(x\right) \chi\left(\FieldNorm{2m}{1}\left(x - 1\right)\right) = \theta\left(-1\right) q^{\frac{m}{2}} \tau\left(\chi^{2}, \fieldCharacter\right)^m \tau\left(\theta' \times \chi^{-1}, \fieldCharacter_{2m}\right).$$

\subsubsection{Computation for irreducible cuspidal representations}

\begin{theorem}
	Suppose that $\pi$ is an irreducible cuspidal representation of $G$, parameterized by a character $\theta = \theta_1 \times \dots \times \theta_r$ of a torus $T$ such that $$T \cong \finiteFieldExtension{2n_1}^1 \times \finiteFieldExtension{2n_2}^1 \times \dots \times \finiteFieldExtension{2n_r}^1,$$
	where $2n_1 + 2n_2 + \dots + 2n_r = \dim \hermitianSpace$ if $\dim \hermitianSpace$ is even or $2n_1 + 2n_2 + \dots + 2n_r = \dim \hermitianSpace - 1$ if $\dim \hermitianSpace$ is odd, and where $$\finiteFieldExtension{2m}^1 = \left\{ x \in \multiplicativegroup{\finiteFieldExtension{2m}} \mid \FieldNorm{2m}{m}\left(x\right)= 1\right\}.$$
	Then up to maybe a sign, we have $$\dblGammaFactorSpace{\hermitianSpace}{\pi}{\chi}{\fieldCharacter} = \prod_{j=1}^r \tau\left( \theta'_j \times \chi, \fieldCharacter_{2n_j} \right),$$
	where $\theta'_j\left(x\right) = \theta_j\left(x^{1-q^{n_j}}\right)$ and $$\tau\left( \theta'_j \times \chi, \fieldCharacter_{2n_j}\right) = -q^{-n_j} \sum_{x \in \multiplicativegroup{\finiteFieldExtension{2n_j}}} \theta'_j\left(x\right) \chi\left(\FieldNorm{2n_j}{1}\left(x\right)\right)\fieldCharacter\left(\trace_{\finiteFieldExtension{2n_j} \slash \finiteField}\left(x\right)\right).$$
\end{theorem}

\subsubsection{Computation for Lusztig series}

Let $T \cong \finiteFieldExtension{2n_1}^1 \times \finiteFieldExtension{2n_2}^1 \times \dots \times \finiteFieldExtension{2n_r}^1 \times \multiplicativegroup{\finiteFieldExtension{m_1}} \times \dots \multiplicativegroup{\finiteFieldExtension{m_s}}$ be a torus such that $\sum_{i=1}^r 2n_i + \sum_{j=1}^s 2m_j = \dim_{\finiteField} \hermitianSpace$. Let $\theta = \alpha_1 \times \dots \times \alpha_r \times \beta_1 \times \dots \times \beta_s$. Consider the Deligne--Lusztig induction $\DeligneLusztigInduction{T}{G}\theta$. Write $$\DeligneLusztigInduction{T}{G}\theta = \sum_{\pi} c_{\pi} \cdot \trace \pi,$$
where $\pi$ goes over all the irreducible representations of $G$ and $c_{\pi} \in \zIntegers$. Then for any irreducible $\pi$ such that $c_{\pi} \ne 0$, and any $\chi$ such that $\chi^2 \ne 1$, we have
$$\dblGammaFactorSpace{\hermitianSpace}{\pi}{\chi}{\fieldCharacter} = q^{-\frac{\dim \hermitianSpace}{2}} \prod_{j=1}^r \GaussSumCharacter{\alpha'_j}{\chi^{-1}}{\fieldCharacter_{2n_j}} \prod_{i=1}^s \GaussSumCharacter{\beta_i}{\chi^{-1}}{\fieldCharacter_{m_i}} \GaussSumCharacter{\beta_i^{-1}}{\chi^{-1}}{\fieldCharacter_{m_i}}.$$

\subsection{Kloosterman type sums}

Let $\chi_1, \dots, \chi_k \colon \multiplicativegroup{\finiteField} \to \multiplicativegroup{\cComplex}$ be characters such that $\chi_i^2 \ne 1$ for any $i$. Consider the assignment $K \colon G \to \cComplex$ given by $$K\left(x\right) = K_{\chi_1,\dots,\chi_k}\left(x\right) = \sum_{\substack{g_1, \dots, g_k \in G\\
g_1 \dots g_k = x}} \Phi_{\chi_1}\left(g_1\right) \dots \Phi_{\chi_k}\left(g_k\right).$$
It is clear that $K$ is a class function of $G$. Let $\pi$ be an irreducible representation of $G$. Then \begin{align*}
	 \innerproduct{K}{\conjugate{\trace \pi}} &= \frac{1}{\sizeof{G}}\trace \left(\sum_{g_1,\dots,g_k \in G} \Phi_{\chi_1}\left(g_1\right) \dots \Phi_{\chi_k}\left(g_k\right) \pi\left(g_1 \dots g_k\right) \right) \\
	 &= \frac{\sizeof{\lieAlgebra}^{\frac{k}{2}}}{\sizeof{G}} \trace\left( \dblGaussSum{\pi}{\chi_1} \circ \dots \dblGaussSum{\pi}{\chi_k} \right).
\end{align*}
Thus $$\innerproduct{K}{\conjugate{\trace \pi}} = \frac{\sizeof{\lieAlgebra}^{\frac{k}{2}} \dim \pi}{\sizeof{G}} \cdot \prod_{j=1}^k \dblGaussSumScalar{\pi}{\chi_j}.$$
It follows that if
$$\DeligneLusztigInduction{T}{G}\theta = \sum_{\pi} c_{\pi} \cdot \trace \pi,$$
then $$\innerproduct{K}{\conjugate{\DeligneLusztigInduction{T}{G}\theta}} = \frac{\sizeof{\lieAlgebra}^{\frac{k}{2}} \prod_{j=1}^k \dblGaussSumScalar{\pi}{\chi_j}}{\sizeof{G}} \sum_{\pi} c_{\pi} \dim \pi = \frac{\sizeof{\lieAlgebra}^{\frac{k}{2}} \prod_{j=1}^k \dblGaussSumScalar{\pi}{\chi_j}}{\sizeof{G}} \left(\DeligneLusztigInduction{T}{G}\theta\right)\left(1\right),$$
where $\pi$ is any irreducible representation of $G$ such that $c_{\pi} \ne 0$.

Suppose that $T \cong \finiteFieldExtension{2m}^1$, and let $t \in T$ be an element such that $t$ is contained only in the torus $T$. Then
$$K\left(t\right) = \sum_{\theta} \theta\left(t\right) \innerproduct{K}{\DeligneLusztigInduction{T}{G}\left(\theta\right)},$$
where the sum is over all the characters of $T$. Using the formula $$\dblGaussSumScalar{\pi}{\chi} = \tau\left( \theta' \times \chi^{-1}, \fieldCharacter_{2m} \right) \theta\left(-1\right) \tau\left(\chi^2, \fieldCharacter\right)^m,$$
we get
$$K\left(t\right) = \frac{\sizeof{\lieAlgebra}^{\frac{k}{2}} \cdot \DeligneLusztigInduction{T}{G}\left(\theta\right)\left(1\right)}{\sizeof{G}} \sum_{\theta} \theta\left( \left(-1\right)^k t\right) \prod_{j=1}^k \tau\left(\theta' \times \chi_j^{-1}, \fieldCharacter_{2m}\right) \tau\left(\chi_j^2, \fieldCharacter\right)^m.$$

Expanding the last sum, we have
\begin{align*}
	K\left(t\right) = & \frac{\left(-1\right)^k q^{-km}  \sizeof{\lieAlgebra}^{\frac{k}{2}} \DeligneLusztigInduction{T}{G}\left(\theta\right)\left(1\right)}{\sizeof{G}} \cdot \prod_{j=1}^k \tau\left(\chi_j^2, \fieldCharacter\right)^m \\
	 & \times \sum_{\theta} \sum_{\xi_1,\dots,\xi_k \in \multiplicativegroup{\finiteFieldExtension{2m}}} \theta\left( \left(-1\right)^k t \xi_1^{1-q^m} \dots \xi_k^{1-q^m} \right) \fieldCharacter_{2m}\left(\sum_{j=1}^k \xi_j\right) \prod_{j=1}^k \chi_j\left(\FieldNorm{2m}{1}\left(\xi_1^{-1} \right)\right).
\end{align*}
The sum on the second row becomes
$$\sum_{\substack{\xi_1,\dots,\xi_k \in \multiplicativegroup{\finiteFieldExtension{2m}}\\
\prod_{j=1}^k \xi_j^{1-q^m} = \left(-1\right)^k t^{-1}}} \left(\prod_{j=1}^k \chi_j^{-1}\left(\FieldNorm{2m}{1}\left(\xi_j \right)\right)\right) \fieldCharacter_{2m}\left(\sum_{j=1}^k \xi_j\right).$$

\appendix
\section{Evaluation of an exponential sum}

In this appendix, we evaluate the following exponential sum. Let $\chi \colon \multiplicativegroup{\finiteField} \to \multiplicativegroup{\cComplex}$ be a non-trivial character. We compute $$\sum_{\substack{x \in \multiplicativegroup{\finiteFieldExtension{2}}\\
	\trace_{\finiteFieldExtension{2} \slash \finiteField}\left(x\right) \ne 0}} \chi\left(\FieldNorm{2}{1}\left(\frac{x}{\trace_{\finiteFieldExtension{2} \slash \finiteField} x}\right) \right).$$
As usual, we rewrite this sum as
$$\frac{1}{q} \sum_{z \in \finiteField} \sum_{y \in \multiplicativegroup{\finiteField}} \sum_{x \in \multiplicativegroup{\finiteFieldExtension{2}}} \chi\left(\FieldNorm{2}{1}\left(xy\right)\right) \fieldCharacter\left(z\left(y \trace_{\finiteFieldExtension{2} \slash \finiteField} x - 1\right)\right).$$
It is clear that if $z = 0$, then the sum over $x$ will vanish. Hence, we may reduce the sum to a sum over $z \in \multiplicativegroup{\finiteField}$. By replacing $y$ with $z^{-1} y$ and then replacing $z$ with $-z$, we arrive at the sum
$$\frac{1}{q} \sum_{y \in \multiplicativegroup{\finiteField}} \sum_{x \in \multiplicativegroup{\finiteFieldExtension{2}}} \chi\left(\FieldNorm{2}{1}\left(xy\right)\right) \fieldCharacter\left(y \trace_{\finiteFieldExtension{2} \slash \finiteField} x\right) \sum_{z \in \multiplicativegroup{\finiteField}} \chi^{-2}\left(z\right)\fieldCharacter\left(z\right).$$
Replacing $x$ with $y^{-1}x$, and using the Hasse--Davenport relation, we arrive at the sum
$$q^{\frac{1}{2}} \left(q-1\right) \tau\left(\chi, \fieldCharacter\right)^2 \tau\left(\chi^{-2}, \fieldCharacter\right).$$
\bibliographystyle{abbrv}
\bibliography{references}
\end{document}