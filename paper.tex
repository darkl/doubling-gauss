\documentclass[12pt, reqno]{amsart}
\usepackage[normalem]{ulem}

\usepackage{amsmath, amsthm, amssymb}
\usepackage{enumerate}
\tolerance=500
\setlength{\emergencystretch}{3em}
\usepackage[margin=1.0in]{geometry}
\usepackage{xcolor}
\definecolor{cite}{rgb}{0.30,0.60,1.00}
\definecolor{url}{rgb}{0.00,0.00,0.80}
\definecolor{link}{rgb}{0.40,0.10,0.20}
\usepackage[pdfusetitle,colorlinks,linkcolor=link,urlcolor=url,citecolor=cite,pagebackref,breaklinks]{hyperref}
\usepackage{graphicx}
\usepackage{cleveref}
\usepackage{mathdots}
\usepackage{tikz-cd}
\usepackage{comment}
\usepackage{xypic}
\usepackage{mathtools}

% new environment
\newtheorem{theorem}{Theorem}[section]
\newtheorem{claim}[theorem]{Claim}
\newtheorem{proposition}[theorem]{Proposition}
\newtheorem{lemma}[theorem]{Lemma}
\newtheorem{conjecture}[theorem]{Conjecture}
\newtheorem{corollary}[theorem]{Corollary}

\theoremstyle{definition}
\newtheorem{definition}[theorem]{Definition}
\theoremstyle{definition}
\newtheorem{remark}[theorem]{Remark}
\theoremstyle{definition}
\newtheorem{example}[theorem]{Example}

%% frequently used symbols
% Common math notions
\newcommand{\zIntegers}{\mathbb{Z}}
\newcommand{\rReal}{\mathbb{R}}
\newcommand{\cComplex}{\mathbb{C}}
\newcommand{\multiplicativegroup}[1]{#1^{\times}}
\newcommand{\RealPart}{\mathrm{Re}}
\newcommand{\detQuadratic}{{\det}_{\quadraticExtension}}
\newcommand{\Hom}{\mathrm{Hom}}
\newcommand{\EndomorphismRing}{\operatorname{End}}
\newcommand{\Span}{\mathrm{span}}
\newcommand{\Supp}{\mathrm{supp}}
\newcommand{\Stab}{\mathrm{stab}}
\newcommand{\idmap}{\mathrm{id}}
\newcommand{\conjugate}[1]{\overline{#1}}
\newcommand{\indicatorFunction}[1]{\delta_{#1}}
\newcommand{\isomorphic}{\cong}
\newcommand{\lengthof}{\mathfrak{n}}
\newcommand{\abs}[1]{\left|#1\right|}
\newcommand{\sizeof}[1]{\left|#1\right|}
\newcommand{\lcm}{\operatorname{lcm}}
\newcommand{\hermitianSpace}{\mathrm{V}}
\newcommand{\xIsotropic}{\mathrm{X}}
\newcommand{\yIsotropic}{\mathrm{Y}}
\newcommand{\similitudeCharacter}{\operatorname{sim}}

% Inner product
\newcommand{\innerproduct}[2]{\left\langle #1,#2\right\rangle}
\newcommand{\Norm}[1]{\left\Vert #1\right\Vert }
\newcommand{\standardForm}[2]{\left\langle #1,#2\right\rangle}

% Representation theory
\newcommand{\fieldCharacter}{\psi}
\newcommand{\centralCharacter}[1]{\omega_{#1}}
\newcommand{\Ind}[3]{\mathrm{Ind}_{#1}^{#2}\left(#3\right)}
\newcommand{\ind}[3]{\mathrm{ind}_{#1}^{#2}\left(#3\right)}
\newcommand{\Whittaker}{\mathcal{W}}
\newcommand{\Contragradient}[1]{#1^{\vee}}
\newcommand{\underlyingVectorSpace}[1]{V_{#1}}
\newcommand{\representationDeclaration}[1]{#1}
\newcommand{\besselFunction}{\mathcal{J}}
\newcommand{\besselFunctionOfFiniteFieldRepresentation}{\besselFunction_{\finiteFieldRepresentation, \fieldCharacter}}
\newcommand{\SpehRepresentation}[2]{\Delta\left(#1, #2\right)}

\newcommand{\gbesselSpehFunction}[2]{\mathcal{BS}_{#1, #2}}
\newcommand{\besselSpehFunction}[2]{\mathcal{BS}_{\SpehRepresentation{#1}{#2}, \fieldCharacter}}
\newcommand{\fourierTransform}[2]{\mathcal{F}_{#1}#2}
\newcommand{\GKGammaFactor}[3]{\gamma^{\mathrm{GK}}\left(#1 \times #2, #3\right)}
\newcommand{\LocalGKGammaFactor}[4]{\gamma^{\mathrm{GK}}\left(#1, #2 \times #3, #4\right)}

\newcommand{\GKPreGammaFactor}[3]{\Gamma^{\mathrm{GK}}\left(#1 \times #2, #3\right)}
\newcommand{\gGJPreGammaFactor}[3]{\Gamma\left(#1 \times #2, #3\right)}
\newcommand{\GJPreGammaFactor}[2]{\Gamma\left(#1, #2\right)}
\newcommand{\Irr}{\mathrm{Irr}}

% Group theory macros
\newcommand{\rquot}[2]{{#1}\slash{#2}}
\newcommand{\lquot}[2]{{#1}\backslash{#2}}
\newcommand{\grpIndex}[2]{\left[#1:#2\right]}

% Matrices macros
\newcommand{\transpose}[1]{\, {}^{t}#1}
\newcommand{\inverseTranspose}[1]{#1^{\iota}}
\newcommand{\involution}[1]{#1^{c}}
\newcommand{\minusInvolution}[1]{#1^{-c}}
\newcommand{\involutionPlusOne}[1]{#1^{1+c}}
\newcommand{\IdentityMatrix}[1]{I_{#1}}
\newcommand{\diag}{\mathrm{diag}}
\newcommand{\antidiag}{\operatorname{\mathrm{antidiag}}}
\newcommand{\trace}{\operatorname{tr}}
\newcommand{\GL}{\mathrm{GL}}
\newcommand{\SO}{\mathrm{SO}}
\newcommand{\GSO}{\mathrm{GSO}}
\newcommand{\Sp}{\mathrm{Sp}}
\newcommand{\GSp}{\mathrm{GSp}}
\newcommand{\UnitaryGroup}{\mathrm{U}}
\newcommand{\UnipotentSubgroup}{U}
\newcommand{\UnipotentRadicalForWss}[2]{N_{\left(#2^{#1}\right)}}
\newcommand{\UnipotentRadicalForWssRecursion}[2]{\mathcal{Y}_{c,k}}
\newcommand{\UnipotentRadical}{N}
\newcommand{\ParabolicSubgroup}{P}
\newcommand{\GroupExtension}[1]{\widetilde{#1}}

% Finite field macros
\newcommand{\FieldNorm}[2]{\mathrm{N}_{#1:#2}}
\newcommand{\aFieldNorm}{\mathrm{N}}
\newcommand{\FieldTrace}{\mathrm{Tr}}
\newcommand{\finiteField}{\mathbb{F}}
\newcommand{\quadraticExtension}{\mathbb{E}}
\newcommand{\finiteFieldExtension}[1]{\finiteField_{#1}}
\newcommand{\quadraticFieldExtension}[1]{\quadraticExtension_{#1}}
\newcommand{\FieldExtension}[2]{{#1} \slash {#2}}
\newcommand{\NormOneGroup}[1]{\finiteFieldExtension{#1}^{\aFieldNorm = 1}}
\newcommand{\algebraicClosure}[1]{\overline{#1}}
\newcommand{\charactergroup}[1]{\widehat{\multiplicativegroup{\finiteFieldExtension{#1}}}}
\newcommand{\limitcharactergroup}{\Gamma}
\newcommand{\Galois}{\operatorname{Gal}}
\newcommand{\Frobenius}{\operatorname{Fr}}
\newcommand{\restrictionOfScalars}[3]{\operatorname{Res}_{#1 \slash #2}{#3}}
\newcommand{\multiplcativeScheme}{\mathbb{G}_m}
\newcommand{\affineLine}{\mathbb{A}^1}
\newcommand{\squareMatrix}{\operatorname{Mat}}
\newcommand{\Mat}[2]{\operatorname{Mat}_{#1 \times #2}}
\newcommand{\frobeniusDegree}{\operatorname{deg}}
\newcommand{\Steinberg}{\operatorname{St}}
\newcommand{\ProjectionOperator}{\operatorname{pr}}
\newcommand{\SymmetricGroup}{\mathfrak{S}}
\newcommand{\whittakerVector}[1]{v_{#1, \fieldCharacter}}
\newcommand{\gwhittakerVector}[2]{v_{#1, #2}}
\newcommand{\WhittakerProjection}{\ProjectionOperator_{\mathrm{Wh}}}
\newcommand{\ParabolicForSpeh}[2]{P_{\left({#1}^{#2}\right)}}
\newcommand{\UnipotentForSpeh}[2]{N_{\left({#1}^{#2}\right)}}
\newcommand{\PoincarePolynomial}[2]{P_{#2}}

%Partition macros
\newcommand{\localField}{F}
\newcommand{\ringOfIntegers}{\mathfrak{o}}
\newcommand{\residueField}{\mathfrak{f}}
\newcommand{\maximalIdeal}{\mathfrak{p}}
\newcommand{\depthZeroRepresentation}{\mathcal{T}}
\newcommand{\differential}{\mathrm{d}}
\newcommand{\mdifferential}{\differential^{\times}}
\newcommand{\quotientMap}{\nu}
\newcommand{\Lift}{\mathcal{L}}
\newcommand{\uniformizer}{\varpi}
\newcommand{\VolumeOf}{\operatorname{Vol}}

\newcommand{\Erdelyi}{Erd{\'e}lyi}
\newcommand{\Toth}{T{\'o}th}

\newcommand{\parabolicSection}{\Phi^{\left(z_1, \dots, z_c\right)}}
\newcommand{\intertwiningOperator}{M^{\left(z_1, \dots, z_c\right)}}
\newcommand{\holomorphicRepresentation}{\depthZeroRepresentation^{\left(z_1, \dots, z_c\right)}}
\newcommand{\WhittakerFunctional}[1]{\ell_{#1, \fieldCharacter}}
\newcommand{\gWhittakerFunctional}[2]{\ell_{#1, #2}}
\newcommand{\gSpehWhittakerFunctional}[3]{\ell_{\SpehRepresentation{#1}{#3}, \fieldCharacter_{\UnipotentRadicalForWss{#2}{#3}}}}
\newcommand{\gShortSpehWhittakerFunctional}[3]{\ell_{\SpehRepresentation{#1}{#3}}}
\newcommand{\GaussSum}[2]{\mathcal{G}\left(#1, #2\right)}
\newcommand{\dblJacobiSum}[2]{\mathcal{J}^{\mathrm{dbl}}\left(#1, #2\right)}
\newcommand{\GaussSumScalar}[2]{\mathrm{G}\left(#1, #2\right)}
\newcommand{\dblJacobiSumScalar}[2]{\mathrm{J}^{\mathrm{dbl}}\left(#1, #2\right)}
\newcommand{\dblVirtualJacobiSumScalar}[2]{\mathrm{j}^{\mathrm{dbl}}\left(#1, #2\right)}
\newcommand{\dblGammaFactor}[3]{\Gamma^{\mathrm{dbl}}\left(#1 \times #2, #3\right)}
\newcommand{\dblGammaFactorSpace}[4]{\Gamma^{\mathrm{dbl}}_{#1}\left(#2 \times #3, #4\right)}
\newcommand{\dblLanglandsGammaFactorSpace}[4]{\gamma^{\mathrm{dbl}}_{#1}\left(#2 \times #3, #4\right)}
\newcommand{\GKGaussSum}[3]{\mathcal{G}\left(#1 \times #2, #3\right)}
\newcommand{\GKGaussSumScalar}[3]{\mathrm{G}\left(#1 \times #2, #3\right)}
\newcommand{\fieldCharacterkc}[2]{\fieldCharacter_{\left({#2}^{#1}\right)}}
\newcommand{\ExoticKloosterman}{\mathrm{Kl}}
\newcommand{\GaussSumCharacter}[3]{\tau\left(#1 \times #2, #3\right)}
\newcommand{\IrrCusp}{\Irr_{\mathrm{cusp}}}
\newcommand{\convolutionWithCompactSupport}{\boldsymbol{\mathrm{R}}}
\newcommand{\ladicnumbers}{\algebraicClosure{\mathbb{Q}_{\ell}}}
\newcommand{\artinScrier}{\operatorname{AS}}
\newcommand{\KloostermanSumClassFunction}{\mathcal{K}}
\newcommand{\IsometryGroup}{\mathrm{Isom}}
\newcommand{\lieAlgebra}{\mathfrak{g}}
\newcommand{\DeligneLusztigInduction}[2]{\mathrm{R}_{#1}^{#2}}
\newcommand{\algebraicGroup}[1]{\boldsymbol{\mathrm{#1}}}
\newcommand{\LusztigSeries}[2]{\mathcal{E}\left(#1, (#2)\right)}
\newcommand{\ToriDualToriIsomorphism}{\kappa}
\newcommand{\characteristicPolynomial}{\operatorname{CharPoly}}

\newcommand{\calvin}[1]{\textcolor{orange}{\sffamily ((CALVIN: #1))}}
\newcommand{\elad}[1]{\textcolor{blue}{\sffamily ((ELAD: #1))}}

\hypersetup{pdfauthor={Calvin Yost-Wolff, Elad Zelingher},
	pdfsubject={Number theory, Representation theory},
	pdfkeywords={Kloosterman sums, Exponential sums}}

\title{On exponential sums arising from the classical doubling method}

\author{Calvin Yost-Wolff}
\address{Department of Mathematics, University of Michigan, 3084 East Hall, 530 Church Street, Ann Arbor, MI 48109-1043 USA}
\email{calvinyw@umich.edu}

\author{Elad Zelingher}
\address{Department of Mathematics, University of Michigan, 1844 East Hall, 530 Church Street, Ann Arbor, MI 48109-1043 USA}
\email{eladz@umich.edu}

\subjclass[2010]{20C33, 11L05, 11T24}

% 11L05    Gauss and Kloosterman sums; generalizations
% 11T24    Other character sums and Gauss sums
% 20C33    Representations of finite groups of Lie type

\begin{document}

\begin{abstract}
\end{abstract}
\maketitle

\subsection{TODO list}
\calvin{I mentioned this to Charlotte and will meet with her sometime after Arizona to discuss specifically the DL stuff to make sure that it ok. So we could try to have a draft }
\begin{enumerate}
	\item (Minor) \sout{Change terminology of Gauss sums $\dblJacobiSumScalar{\pi}{\chi}$ to Jacobi sums and the notation to $J^{\mathrm{dbl}}\left(\pi, \chi\right)$ as they are similar to the Jacobi sums}
	$$J\left(\alpha, \beta\right) = \sum_{\substack{x,y \in \multiplicativegroup{\finiteField}\\
	x+y = 1}} \alpha\left(x\right) \beta\left(y\right),$$
	compare with $$\dblJacobiSumScalar{\pi}{\chi} = \sum_{\substack{g \in G\\
			h \in \GL\left(\hermitianSpace\right)\\
	g + h = \idmap_{\hermitianSpace}}} \pi\left(g\right) \chi\left(\det\left(-h\right)\right).$$
	\sout{Consider replacing $\chi\left(\det\left(g - \idmap_{\hermitianSpace}\right)\right)$ with $\chi\left(\det\left(\idmap_{\hermitianSpace} - g\right)\right)$ for aesthetics (probably not worth it).}
	\item Maybe define the algebraic group versions $\algebraicGroup{G}$ and $\GroupExtension{\algebraicGroup{G}}$ in Section 3.2. For instance for $\algebraicGroup{\Sp}_{2n}$ we can define
	$$\algebraicGroup{\tilde{G}} = \left\{ \left(g,\lambda\right) \in \algebraicGroup{\GL}\left(\hermitianSpace\right) \times \multiplcativeScheme \mid \involution{\transpose{g}} J g = \lambda J,\ \right\}$$ for suitable $J$ and then define the torus $\algebraicGroup{S}$ that appears in \cite[Page 140]{DigneMichel2020} as $$\algebraicGroup{S} = \left\{\left(\begin{pmatrix}
		t \IdentityMatrix{n}\\
		& \IdentityMatrix{n}
	\end{pmatrix}, t\right) \mid t \in \multiplcativeScheme \right\} \subset \GroupExtension{\algebraicGroup{G}}.$$
	Actually it appears to me that we are not in the same situation as in \cite[Page 140]{DigneMichel2020}.
	\item In the computations of $\dblGammaFactor{\pi}{\chi}{\fieldCharacter}$ in terms of Deligne--Lusztig we should say that when we take a torus of $\GroupExtension{G}$ and intersect it with $G$ we get a torus of $G$ and hence it suffices to compute whatever we are computing.
	\item The proof of \Cref{prop:gamma-factor-is-constant-on-lusztig-series} should be modified. We should first use $\GroupExtension{G}$. For this group we have the ``at most one cuspidal'' property. Then we should use \cite[Page 140]{DigneMichel2020} $$R_{\GroupExtension{T}}^{\GroupExtension{G}}\left(\tilde{\theta}\right) \restriction_{G} = R_{T}^G \left(\tilde{\theta}\restriction_{T}\right),$$
	where we need to explain the notation.
	\item \Cref{subsec:tori-isomorphism} and \Cref{subsec:gauss-sums-attached-to-semi-simple} should be removed and replaced with Calvin's arguments.
\end{enumerate}

\section{Notation}
Let $\finiteField$ be a finite field with $q$ elements, and let $\fieldCharacter \colon \finiteField \to \multiplicativegroup{\cComplex}$ be a non-trivial character. Fix an algebraic closure $\algebraicClosure{\finiteField}$ of $\finiteField$. For any $k \ge 1$, let $\finiteFieldExtension{k} \slash \finiteField$ be the unique field extension of degree $k$ contained in $\algebraicClosure{\finiteField}$.

\section{Kondo Gauss sums}
Let $\tau$ be an irreducible representation of $\GL_n\left(\finiteField\right)$. 
For a character $\chi \colon \multiplicativegroup{\finiteField} \to \multiplicativegroup{\cComplex}$, denote the \emph{twisted Gauss sum}
$$\GaussSum{\tau \times \chi}{\fieldCharacter} = q^{-\frac{n^2}{2}} \sum_{g \in \GL_n\left(\finiteField\right)} \tau\left(g\right) \chi\left(\det g\right) \fieldCharacter\left(\trace g\right).$$
This element lies in $\Hom_{\GL_n\left(\finiteField\right)}\left(\tau, \tau\right)$ and therefore by Schur's lemma there exists a complex number $\GaussSumScalar{\tau \times \chi}{\fieldCharacter} \in \cComplex$ such that
$$\GaussSum{\tau \times \chi}{\fieldCharacter} = \GaussSumScalar{\tau \times \chi}{\fieldCharacter} \cdot \idmap_\tau.$$ The computation of this scalar is known due to Kondo.

\begin{lemma}\label{lem:sum-vanishes-for-singular-matrices}
	For any singular matrix $X \in \squareMatrix_n\left(\finiteField\right)$ and any $1 \ne \chi \colon \multiplicativegroup{\finiteField} \to \multiplicativegroup{\cComplex}$, the sum
	$$\sum_{h \in \GL_n\left(\finiteField\right)} \chi\left(\det h\right) \fieldCharacter\left(\trace\left(Xh\right)\right)$$
	is zero.
\end{lemma}
\begin{proof}
	Write $$X = h_1 \begin{pmatrix}
		\IdentityMatrix{n-r}\\
		& 0_r
	\end{pmatrix} h_2,$$
	where $1 \le r \le n$ and $h_1, h_2 \in \GL_n\left(\finiteField\right)$. Then
	$$\fieldCharacter\left(\trace Xh\right) = \fieldCharacter\left(\trace\left( \begin{pmatrix}
		\IdentityMatrix{n-r}\\
		& 0_r
	\end{pmatrix} h_2 h h_1\right)\right).$$
	Changing variables $h \mapsto h_2^{-1} h h_1^{-1},$
	we obtain the inner sum
	$$\sum_{h \in \GL_n\left(\finiteField\right)} \fieldCharacter\left(\trace \begin{pmatrix}
		\IdentityMatrix{n-r}\\
		& 0_r
	\end{pmatrix} h \right) \chi\left(\det h\right),$$
	which equals
	$$\frac{1}{\sizeof{\multiplicativegroup{\finiteField}}} \sum_{a \in \multiplicativegroup{\finiteField}} \sum_{h \in \GL_n\left(\finiteField\right)} \fieldCharacter\left(\trace \begin{pmatrix}
		\IdentityMatrix{n-r}\\
		& 0_r
	\end{pmatrix} \begin{pmatrix}
		\IdentityMatrix{n-1}\\
		& a
	\end{pmatrix} h \right) \chi\left(\det h\right).$$
	Changing variables $h \mapsto \left(\begin{smallmatrix}
		\IdentityMatrix{n-1}\\
		& a^{-1}
	\end{smallmatrix}\right) h$, we get the inner sum $$\sum_{a \in \multiplicativegroup{\finiteField}} \chi^{-1}\left(a\right),$$
	which vanishes because $\chi \ne 1$.
\end{proof}

\section{Doubling method Jacobi sums}

\subsection{The case of general linear groups}

Let $\chi \colon \multiplicativegroup{\finiteField} \to \multiplicativegroup{\cComplex}$ be a non-trivial character.

Consider the following assignment $\GL_k\left(\finiteField\right) \to \cComplex$
$$\Phi_{\chi}\left(g\right) = \begin{dcases}
	\chi\left(\det\left(\IdentityMatrix{k}-g\right)\right) & \text{if }\det\left(\IdentityMatrix{k}-g\right) \ne 0,\\
	0 & \text{otherwise.}
\end{dcases}$$
It is clear that $\Phi_{\chi}$ is a class function of $\GL_k\left(\finiteField\right)$.

Given an irreducible representation $\tau$ of $\GL_k\left(\finiteField\right)$, we consider the following \emph{doubling method Jacobi sum}:
$$\dblJacobiSum{\tau}{\chi} = q^{-\frac{k^2}{2}} \sum_{g \in \GL_k\left(\finiteField\right)} \Phi_{\chi}\left(g\right) \tau\left(g\right).$$
Then $\dblJacobiSum{\tau}{\chi}$ defines an element of $\Hom_{\GL_k\left(\finiteField\right)}\left(\tau, \tau\right)$. Therefore, by Schur's lemma there exists a complex number $\dblJacobiSumScalar{\tau}{\chi} \in \cComplex$ such that $$\dblJacobiSum{\tau}{\chi} = \dblJacobiSumScalar{\tau}{\chi} \cdot \idmap_\tau.$$ 

The goal of this section is to express $\dblJacobiSumScalar{\tau}{\chi}$ in terms of Kondo's Gauss sum.

\begin{proposition}\label{prop:doubling-for-gln-in-terms-of-kondo}
	We have the identity
	$$\Phi_{\chi}\left(g\right) = \chi\left(-1\right)^k q^{-\frac{k^2}{2}} \GaussSumScalar{\chi_{\GL_k}}{\fieldCharacter} \sum_{\substack{x, y \in \GL_k\left(\finiteField\right)\\
			y^{-1} x = -g}} \fieldCharacter\left(\trace x\right) \chi^{-1}\left(\det y\right) \fieldCharacter\left(\trace y\right).$$
\end{proposition}
\begin{proof}
	We write $$\Phi_{\chi}\left(g\right) = \frac{1}{\sizeof{\squareMatrix_k\left(\finiteField\right)}}\sum_{X \in \squareMatrix_k\left(\finiteField\right)} \sum_{h \in \GL_k\left(\finiteField\right)} \fieldCharacter\left(\trace \left(X\left(g-\IdentityMatrix{k}-h\right)\right)\right) \chi\left(\det\left(-h\right)\right).$$
	By \Cref{lem:sum-vanishes-for-singular-matrices}, we can reduce the sum to $X \in \GL_k\left(\finiteField\right)$. Thus	$$\Phi_{\chi}\left(g\right) = \frac{\chi\left(-1\right)^k}{\sizeof{\squareMatrix_k\left(\finiteField\right)}} \sum_{X, h \in \GL_k\left(\finiteField\right)} \fieldCharacter\left(\trace \left(X\left(g-\IdentityMatrix{k}-h\right)\right)\right) \chi\left(\det h\right).$$
	Changing variables $h \mapsto X^{-1} h$, we get
	$$\Phi_{\chi}\left(g\right) = \frac{\chi\left(-1\right)^k}{\sizeof{\squareMatrix_k\left(\finiteField\right)}} \sum_{X, h \in \GL_k\left(\finiteField\right)} \fieldCharacter\left(\trace \left(Xg\right)\right) \fieldCharacter\left(-\trace h\right) \chi\left(\det h\right) \chi^{-1}\left(\det X\right) \fieldCharacter\left(-\trace X\right).$$
	Changing variables again $h \mapsto -h$ and setting $Xg = x$ and $-X = y$, we get  
	$$\Phi_{\chi}\left(g\right) = \chi\left(-1\right)^k q^{-\frac{k^2}{2}} \GaussSumScalar{\chi_{\GL_k}}{\fieldCharacter} \sum_{\substack{x, y \in \GL_k\left(\finiteField\right)\\
			y^{-1} x = -g}} \fieldCharacter\left(\trace x\right) \chi^{-1}\left(\det y\right) \fieldCharacter\left(\trace y\right).$$
\end{proof}

We now use this to compute $\dblJacobiSum{\tau}{\chi}$. We have that $$\dblJacobiSum{\tau}{\chi} = \chi\left(-1\right)^k q^{-k^2} \GaussSumScalar{\chi_{\GL_k}}{\fieldCharacter} \sum_{x, y \in \GL_k\left(\finiteField\right)} \fieldCharacter\left(\trace x\right) \chi^{-1}\left(\det y\right) \fieldCharacter\left(\trace y\right) \tau\left(-y^{-1} x\right).$$
The last equality implies the following result.
\begin{theorem}\label{thm:gln-doubling-gauss-sum-in-terms-of-kondo}For any irreducible representation $\tau$ of $\GL_k\left(\finiteField\right)$ and any non-trivial character $\chi \colon \multiplicativegroup{\finiteField} \to \multiplicativegroup{\cComplex}$, the following identity holds:
	$$\dblJacobiSumScalar{\tau}{\chi} = \chi\left(-1\right)^k \centralCharacter{\tau}\left(-1\right) \GaussSumScalar{\chi}{\fieldCharacter}^k \GaussSumScalar{\tau}{\fieldCharacter} \GaussSumScalar{\tau^{\vee} \times \chi^{-1}}{\fieldCharacter}.$$
\end{theorem}

\subsection{The case of classical groups}

Let $\quadraticExtension \slash \finiteField$ be a field extension of degree $1$ or $2$ contained in $\algebraicClosure{\finiteField}$. For any $k \ge 1$, let $\quadraticFieldExtension{k} \slash \quadraticExtension$ be the unique field extension of degree $k$ contained in $\algebraicClosure{\finiteField}$. Let $x \mapsto \involution{x}$ be the generator of $\Galois\left(\quadraticExtension \slash \finiteField\right)$.

Let $\hermitianSpace$ be a vector space of dimension $n$ over $\quadraticExtension$, equipped with a non-degenerate sesquilinear form $\innerproduct{\cdot}{\cdot} \colon \hermitianSpace \times \hermitianSpace \to \quadraticExtension$ which is $\epsilon_{\hermitianSpace}$-symmetric for $\epsilon_{\hermitianSpace} \in \left\{\pm 1\right\}$. By this we mean that:
\begin{enumerate}
	\item For every $x_1,x_2,y \in \hermitianSpace$, $$\innerproduct{x_1 + x_2}{y} = \innerproduct{x_1}{y} + \innerproduct{x_2}{y}.$$
	\item For every $x,y \in \hermitianSpace$ and $t \in \quadraticExtension$, $$\innerproduct{tx}{y} = t\innerproduct{x}{y}.$$
	\item ($\epsilon_{\hermitianSpace}$-symmetric) For every $x,y \in \hermitianSpace$, $$\innerproduct{x}{y} = \epsilon_{\hermitianSpace} \involution{\innerproduct{y}{x}}.$$
	\item (non-degenerate) For every $0 \ne x \in \hermitianSpace$ there exists $y \in \hermitianSpace$ such that $$\innerproduct{x}{y} \ne 0.$$
\end{enumerate}
Let $\IsometryGroup \left(\hermitianSpace\right)$ be the isometry group of $\hermitianSpace$, consisting of all the elements of $\GL_{\quadraticExtension}\left(\hermitianSpace\right)$ that satisfy $\innerproduct{gx}{gy} = \innerproduct{x}{y}$ for every $x,y \in \hermitianSpace$. We denote by $G$ the identity component of $\IsometryGroup\left(\hermitianSpace\right)$. If $\quadraticExtension = \finiteField$, let $\GroupExtension{\IsometryGroup\left(\hermitianSpace\right)}$ be the similitude group version of $\IsometryGroup\left(\hermitianSpace\right)$, consisting of the elements $g \in \GL_{\quadraticExtension}\left(\hermitianSpace\right)$ with the property that there exists a constant $\similitudeCharacter\left(g\right) \in \multiplicativegroup{\quadraticExtension}$ such that for every $\innerproduct{gx}{gy} = \similitudeCharacter\left(g\right) \innerproduct{x}{y}$ for every $x,y \in \hermitianSpace$. Let $\GroupExtension{G}$ be the identity component of $\GroupExtension{\IsometryGroup\left(\hermitianSpace\right)}$. If $\quadraticExtension \ne \finiteField$, let $\GroupExtension{G} = G$.

We will assume that $\dim_{\finiteField} \hermitianSpace$ is even. This is automatic in all cases, expect for when $\quadraticExtension = \finiteField$ and $\epsilon_{\hermitianSpace} = 1$. The group $\GroupExtension{G}$ has connected center and contains $G$.

Let us summarize all possible cases for $G$ and $\GroupExtension{G}$. Let $$w_n = \begin{pmatrix}
	& & & 1\\
	& & 1\\
	& \iddots\\
	1
\end{pmatrix} \in \GL_n\left(\finiteField\right)$$ be the longest Weyl element.
\begin{enumerate}
	\item (Even special orthogonal groups): $\quadraticExtension = \finiteField$ and $\epsilon_{\hermitianSpace} = 1$. Denote $\dim_{\finiteField} \hermitianSpace = 2n$. In this case, we have that $G$ consists of all elements $g \in \IsometryGroup\left(\hermitianSpace\right)$ with $\det \hermitianSpace = 1$, and that $\GroupExtension{G}$ consists of all elements $g \in \GroupExtension{\IsometryGroup\left(\hermitianSpace\right)}$ such that $\det g = \similitudeCharacter\left(g\right)^n$. There are two cases here.
	\begin{enumerate}
		\item Split case: in this case, we can write $\hermitianSpace = \xIsotropic \oplus \yIsotropic$ where $\xIsotropic$ and $\yIsotropic$ are totally isotropic (that is, for every $x \in \xIsotropic$, $\innerproduct{x}{x} = 0$ and similarly for $\yIsotropic$) and these spaces are in duality with respect to $\innerproduct{\cdot}{\cdot}$, i.e., the map $\yIsotropic \to \Hom_{\finiteField}\left(\xIsotropic, \finiteField\right)$ given by $y \mapsto \left(x \mapsto \innerproduct{x}{y}\right)$ is an isomorphism. In this case, we have that $G$ is isomorphic to the subgroup $\SO_{2n}\left(\finiteField\right)$ of $\GL_{2n}\left(\finiteField\right)$ consisting of elements $g$ satisfying $$g \cdot w_{2n} \cdot \transpose{g} = w_{2n}$$ and $\det g = 1$.
		The group $\tilde{G}$ is isomorphic to the subgroup $\GSO_{2n}\left(\finiteField\right)$ of $\GL_{2n}\left(\finiteField\right)$, consisting of elements $g$ such that $$g \cdot w_{2n} \cdot \transpose{g} = \similitudeCharacter\left(g\right) w_{2n},$$
		and such that $\det g = \similitudeCharacter\left(g\right)^n$, where $\similitudeCharacter\left(g\right)$ is a non-zero scalar.
		\item Non-split case:
	\end{enumerate}
	\item (Symplectic groups): $\quadraticExtension = \finiteField$ and $\epsilon_{\hermitianSpace} = -1$. In this case, $\dim_{\finiteField} \hermitianSpace = 2n$ for some positive integer $n$ and $G$ is isomorphic to subgroup $\Sp_{2n}\left(\finiteField\right)$ of $\GL_{2n}\left(\finiteField\right)$ consisting of all elements $g$ satisfying $$g \cdot \begin{pmatrix}
		& w_n\\
		-w_n
	\end{pmatrix} \cdot \transpose{g} = \begin{pmatrix}
	& w_n\\
	-w_n
	\end{pmatrix}.$$ The group $\GroupExtension{G}$ is is isomorphic to the subgroup $\GSp_{2n}\left(\finiteField\right)$ of $\GL_{2n}\left(\finiteField\right)$ consisting of all elements $g$ such that $$g \cdot \begin{pmatrix}
	& w_n\\
	-w_n
	\end{pmatrix} \cdot \transpose{g} = \similitudeCharacter\left(g\right) \begin{pmatrix}
	& w_n\\
	-w_n
	\end{pmatrix},$$
	where $\similitudeCharacter\left(g\right)$ is a non-zero scalar.
	\item (Unitary groups): $\quadraticExtension \ne \finiteField$. Denote $\dim_{\quadraticExtension} \hermitianSpace = n$. In this case, the group $G = \GroupExtension{G}$ is isomorphic to the subgroup $\UnitaryGroup_n\left(\finiteField\right)$ of $\GL_{n}\left(\quadraticExtension\right)$ consisting of all elements $g$ such that $$g \cdot w_n \cdot \involution{\left(\transpose{g}\right)} = w_n.$$
\end{enumerate}

Let $\lieAlgebra$ be the Lie algebra of $G$, consisting of all elements $A \in \EndomorphismRing_{\quadraticExtension}\left(\hermitianSpace\right)$ satisfying $\innerproduct{Ax}{y} + \involution{\innerproduct{x}{AY}} = 0$ for every $x, y \in \hermitianSpace$.



Let $\chi \colon \multiplicativegroup{\quadraticExtension} \to \multiplicativegroup{\cComplex}$ be a character. As before, the assignment $\Phi_{\chi} \colon \GL_{\quadraticExtension}\left(\hermitianSpace\right) \to \cComplex$ given by $$\Phi_{\chi}\left(g\right) = \begin{dcases}
\chi\left(\detQuadratic\left( \idmap_{\hermitianSpace} - g\right)\right) & \text{if }\detQuadratic\left( \idmap_{\hermitianSpace} - g\right) \ne 0\\
0 & \text{otherwise,}
\end{dcases}$$
is a class function of $\GL_{\quadraticExtension}\left(\hermitianSpace\right)$.

Let $\pi$ be an irreducible representation of $H$ where $H \in \{G, \GroupExtension{G}\}$. Denote $$\dblJacobiSum{\pi}{\chi} = \frac{1}{\sqrt{\sizeof{\lieAlgebra}}} \sum_{g \in G} \pi\left(g\right) \Phi_{\chi}\left(g\right).$$
Since $\Phi_{\chi}$ is a class function of $\GL_{\quadraticExtension}\left(\hermitianSpace\right)$, it is also a class function of $G$. Since $G$ is a normal subgroup of $H$, we have that $\dblJacobiSum{\pi}{\chi} \in \Hom_{H}\left(\pi, \pi\right)$. By Schur's lemma, there exists a constant $\dblJacobiSumScalar{\pi}{\fieldCharacter} \in \cComplex$, such that $$\dblJacobiSum{\pi}{\chi} = \dblJacobiSumScalar{\pi}{\chi} \cdot \idmap_\pi.$$
We call $\dblJacobiSumScalar{\pi}{\chi}$ the \emph{doubling method Jacobi sum}.

\subsubsection{Multiplicativity property}
Our next goal is to understand how the doubling method Jacobi sum behaves under parabolic induction.

Let $\xIsotropic$ and $\yIsotropic$ be isotropic spaces of $\hermitianSpace$ of dimension $k$, such that $\xIsotropic$ and $\yIsotropic$ are in duality with respect to form $\innerproduct{\cdot}{\cdot}$. Let us write $$\hermitianSpace = \xIsotropic \oplus \hermitianSpace' \oplus \yIsotropic,$$
where $\hermitianSpace' \subset \hermitianSpace$ is a non-degenerate subspace, orthogonal to $\xIsotropic$ and $\yIsotropic$. Let $P$ be the parabolic subgroup of $H$, consisting of all elements stabilizing the flag $$0 \subset \xIsotropic \subset \xIsotropic \oplus \hermitianSpace' \subset \xIsotropic \oplus \hermitianSpace' \oplus \yIsotropic = \hermitianSpace.$$
Let $G' = \IsometryGroup\left(\hermitianSpace'\right)$. Write $P = L \ltimes N$, where $L$ is the Levi part of $P$ and $N$ is the unipotent radical of $P$. Then $L$ is isomorphic to $H' \times \GL_k\left(\quadraticExtension\right)$, where $H' \in \left\{G',\GroupExtension{G'}\right\}$ is of the same type as $H$. 

Let $\fieldCharacter_{\quadraticExtension} \colon \quadraticExtension \to \multiplicativegroup{\cComplex}$ be the character  $\fieldCharacter_{\quadraticExtension} = \fieldCharacter \circ \trace_{\quadraticExtension \slash \finiteField}$. Similarly, for any $k \ge 1$, let $\fieldCharacter_{\quadraticFieldExtension{k}} \colon \quadraticFieldExtension{k} \to \multiplicativegroup{\cComplex}$ be the character $\fieldCharacter_{\quadraticFieldExtension{k}} = \fieldCharacter \circ \trace_{\quadraticFieldExtension{k} \slash \finiteField}$.

Given a multiplicative character $\chi \colon \multiplicativegroup{\quadraticExtension} \to \multiplicativegroup{\cComplex}$, let $\involutionPlusOne{\chi} \colon \multiplicativegroup{\quadraticExtension} \to \multiplicativegroup{\cComplex}$ be the character given by
$$\involutionPlusOne{\chi}\left(x\right) = \chi\left(x \cdot \involution{x}\right).$$ Similarly, let $\minusInvolution{\chi} \colon \multiplicativegroup{\quadraticExtension} \to \multiplicativegroup{\cComplex}$ be the character given by
$$\minusInvolution{\chi}\left(x\right) = \chi^{-1}\left(\involution{x}\right).$$

\begin{theorem}\label{thm:multiplicativity-in-terms-of-gauss-sums}
	Let $\pi'$ be an irreducible representation of $H'$ and let $\tau$ be an irreducible representation of $\GL_k\left(\quadraticExtension\right)$. Then for any irreducible representation $\pi$ of $H$ which appears as a subrepresentation the parabolic induction $\rho = \Ind{P}{H}{\tau \overline{\otimes} \pi'}$ and any $\chi$ such that $\involutionPlusOne{\chi} \ne 1$, we have
	$$\dblJacobiSumScalar{\pi}{\chi} = \centralCharacter{\tau}\left(-1\right) \GaussSumScalar{\involutionPlusOne{\chi}}{\fieldCharacter_{\quadraticExtension}}^k \GaussSumScalar{\tau \times \minusInvolution{\chi}}{\fieldCharacter_{\quadraticExtension}} \GaussSumScalar{\Contragradient{\tau} \times \chi^{-1}}{\fieldCharacter_{\quadraticExtension}} \dblJacobiSumScalar{\pi'}{\chi}.$$
\end{theorem}
\begin{proof}
	Let $v_{\pi'} \in \pi'$ and $v_{\tau} \in \tau$. Consider $f \in \rho$ defined as follows. The function $f \in \rho$ is the unique element supported on $P$, such that $f\left(\idmap_{\hermitianSpace}\right) = v_{\tau} \otimes v_{\pi'}$. The projection $f$ to any non-zero invariant subspace of $\rho$ is non-zero. We compute $\dblJacobiSum{\rho}{\chi} f$.
	
	Notice that for $x \in G$, $$\left(\dblJacobiSum{\rho}{\chi} f\right)\left(x\right) = \frac{1}{\sqrt{\sizeof{\lieAlgebra}}} \sum_{g \in G} f\left(xg\right) \Phi_{\chi}\left(g\right) = \frac{1}{\sqrt{\sizeof{\lieAlgebra}}} \sum_{p \in P} f\left(p\right) \Phi_{\chi}\left(x^{-1} p\right),$$
	which can be rewritten as
	$$\left(\dblJacobiSum{\rho}{\chi} f\right)\left(x\right) = \frac{1}{\sqrt{\sizeof{\lieAlgebra}}} \sum_{p \in P}  \Phi_{\chi}\left(x^{-1} p\right) \left(\tau \overline{\otimes} \pi'\right)\left(p\right) v_\tau \otimes v_{\pi'}.$$
	As before, we may write
	$$\Phi_{\chi}\left(g\right) = q^{-k^2} \sum_{h \in \GL_{\quadraticExtension}\left(\hermitianSpace\right)} \sum_{A \in \EndomorphismRing_{\quadraticExtension}\left(\hermitianSpace\right)} \chi\left(\detQuadratic \left(-h\right)\right) \fieldCharacter_{\quadraticExtension}\left(\trace\left(A \left(g - \idmap_{\hermitianSpace} - h\right)\right)\right).$$
	Hence, \begin{align*}
		\chi\left(-1\right)^k q^{k^2} \sum_{p \in P} \Phi_{\chi}\left(xp\right) \left(\tau \overline{\otimes} \pi'\right)\left(p\right) =& \sum_{A \in \EndomorphismRing_{\quadraticExtension}\left(\hermitianSpace\right)} \sum_{h \in \GL_{\quadraticExtension}\left(\hermitianSpace\right)} \chi\left(\detQuadratic h\right)\fieldCharacter_{\quadraticExtension}\left(\trace\left(Ah\right)\right) \\
		& \times \sum_{p \in P} \fieldCharacter_{\quadraticExtension}\left(\trace\left(A \left(xp - \idmap_{\hermitianSpace}\right)\right)\right) \left(\tau \overline{\otimes} \pi'\right)\left(p\right).
	\end{align*}
	Assume that $\chi \ne 1$. By \Cref{lem:sum-vanishes-for-singular-matrices}, we can reduce the sum to $A \in \GL_{\quadraticExtension}\left(\hermitianSpace\right)$. Let us replace $h$ with $A^{-1} h$. We get the sum
	\begin{align*}
		\sum_{A \in \GL_{\quadraticExtension}\left(\hermitianSpace\right)} \chi^{-1}\left(\detQuadratic A\right) \sum_{h \in \GL_{\quadraticExtension}\left(\hermitianSpace\right)} \chi\left(\detQuadratic h\right)\fieldCharacter_{\quadraticExtension}\left(\trace h\right) \sum_{p \in P} \fieldCharacter_{\quadraticExtension}\left(\trace\left(A \left(xp - \idmap_{\hermitianSpace}\right)\right)\right) \left(\tau \overline{\otimes} \pi'\right)\left(p\right).
	\end{align*}
	Consider the inner sum over $P$, written as $$\sum_{l \in L} \sum_{u \in N} \fieldCharacter_{\quadraticExtension}\left(\trace\left(A x l u\right)\right) \left(\tau \overline{\otimes} \pi'\right)\left(l\right).$$
	The inner sum over $N$ will vanish unless $Ax \in P$. Thus we have
	\begin{align*}
		\chi\left(-1\right)^k q^{k^2} \sum_{p \in P} \Phi_{\chi}\left(xp\right) \left(\tau \overline{\otimes} \pi'\right)\left(p\right) =& \sum_{h \in \GL_{\quadraticExtension}\left(\hermitianSpace\right)}  \chi\left(\detQuadratic\left(xh\right)\right) \fieldCharacter_{\quadraticExtension}\left(\trace h\right) \\
		& \times \sum_{p' \in P} \sum_{p \in P} \chi^{-1}\left(\detQuadratic p'\right) \fieldCharacter_{\quadraticExtension}\left(\trace\left(p' \left(p - x^{-1}\right)\right)\right) \left(\tau \overline{\otimes} \pi'\right)\left(p\right).
	\end{align*}
	By decomposing the sum over $p' \in P$ into a sum over $N$ and $L$, we see that the inner sum over $N$ will vanish unless $p - x^{-1} \in P$, which implies that $x \in P \cap G$.
	
	Hence, we have that $\dblJacobiSum{\rho}{\chi} f$ is supported on $P$. We move to compute $\left(\dblJacobiSum{\rho}{\chi} f\right)\left(\idmap_{\hermitianSpace}\right)$. It is given by
	\begin{equation}\label{eq:recursive-doubling-gauss-sum}
		\frac{1}{\sqrt{\sizeof{\lieAlgebra}}} \sum_{\substack{p \in P \cap G\\
				\detQuadratic\left(p - \idmap_{\hermitianSpace}\right) \ne 0}} \chi\left(\detQuadratic\left(\idmap_{\hermitianSpace} - p\right)\right) \left(\tau \overline{\otimes} \pi'\right)\left(p\right) v_{\tau} \otimes v_{\pi'}.
	\end{equation}
	Decomposing the sum \eqref{eq:recursive-doubling-gauss-sum} as a sum over $L$ and $N$, and using the fact that if $p \in P \cap G$ has Levi part with image $\left(a, g'\right) \in \GL_k\left(\quadraticExtension\right) \times G'$ then $$\detQuadratic\left(\idmap_{\hermitianSpace} - p\right) = \left(-1\right)^k \detQuadratic\left(\involution{a}\right)^{-1}\detQuadratic\left(\IdentityMatrix{k} - a\right) \det\left(\involution{\left(\IdentityMatrix{k} - a\right)}\right) \detQuadratic\left(\idmap_{\hermitianSpace'} - g'\right),$$
	and using the fact that $\sizeof{\lieAlgebra} = \sizeof{\lieAlgebra'} \sizeof{\squareMatrix_k\left(\finiteField\right)} \sizeof{N}^2$, where $\lieAlgebra'$ is the Lie algebra of $G'$,
	we get that \eqref{eq:recursive-doubling-gauss-sum} equals
	\begin{equation}
		\chi\left(-1\right)^k q^{-\frac{k^2}{2}} \sum_{a \in \GL_k\left(\quadraticExtension\right)} \Phi_{\involutionPlusOne{\chi}}\left(a\right) \minusInvolution{\chi}\left(\detQuadratic a\right) \tau\left(a\right) v_{\tau} \otimes \frac{1}{\sqrt{\sizeof{\lieAlgebra'}}} \sum_{g' \in G'} \Phi_{\chi}\left(g'\right) \pi'\left(g'\right) v_{\pi'},
	\end{equation}
	which in turn is
	\begin{align*}
		&\chi\left(-1\right)^k \dblJacobiSum{\tau \otimes \minusInvolution{\chi_{\GL_k}}}{\involutionPlusOne{\chi}} v_{\tau} \otimes \dblJacobiSum{\pi'}{\chi} v_{\pi'}\\
		=& \chi\left(-1\right)^k\dblJacobiSumScalar{\tau \otimes \minusInvolution{\chi_{\GL_k}}}{\involutionPlusOne{\chi}} \dblJacobiSumScalar{\pi'}{\chi} v_{\tau} \otimes v_{\pi'}.
	\end{align*}
	The result now follows.
\end{proof}

\subsubsection{Gamma factors}
In order to make the computations more tolerable, it is beneficial to work with fully multiplicative functions. We introduce the doubling method gamma factors.

Let $\chi \colon \multiplicativegroup{\quadraticExtension} \to \multiplicativegroup{\cComplex}$ be a character.

We define a normalization factor as follows $$c_{\hermitianSpace}\left(\chi, \fieldCharacter\right) = \begin{dcases}
	\tau\left(\chi^2, \fieldCharacter\right)^{\frac{\dim_{\finiteField} \hermitianSpace}{2}} & \quadraticExtension = \finiteField\\
	\tau\left(\chi\restriction_{\multiplicativegroup{\finiteField}}, \fieldCharacter\right)^{\dim_{\quadraticExtension} \hermitianSpace} & \quadraticExtension \ne \finiteField
\end{dcases}.$$
Notice that if $\quadraticExtension \ne \finiteField$ then by the Hasse--Davenport relation $\tau\left(\chi\restriction_{\multiplicativegroup{\finiteField}}, \fieldCharacter\right)^{2} = \tau\left(\involutionPlusOne{\chi}, \fieldCharacter_{\quadraticExtension}\right)$.

For an irreducible representation  $\tau$ of $\GL_k\left(\quadraticExtension\right)$, define $$\dblGammaFactor{\tau}{\chi}{\fieldCharacter} = \GaussSumScalar{\tau \times \minusInvolution{\chi}}{\fieldCharacter} \GaussSumScalar{\Contragradient{\tau} \times \chi^{-1}}{\fieldCharacter}.$$
For an irreducible representation $\pi$ of $G$, define the \emph{doubling method gamma factor} $$\dblGammaFactorSpace{\hermitianSpace}{\pi}{\chi}{\fieldCharacter} = \centralCharacter{\pi}\left(-1\right) \frac{\dblJacobiSumScalar{\pi}{\chi}}{c_{\hermitianSpace}\left(\chi, \fieldCharacter\right)}.$$
Using this notation and using the multiplicativity property of Kondo's Gauss sum and \Cref{thm:multiplicativity-in-terms-of-gauss-sums}, we have the following multiplicativity property.

\begin{theorem}\label{thm:multiplicativity-in-terms-of-gamma-factors}
	\begin{enumerate}
		\item If $\tau_1$ and $\tau_2$ are irreducible representations of $\GL_{k_1}\left(\finiteField\right)$ and $\GL_{k_2}\left(\finiteField\right)$, respectively, then for any irreducible subrepresentation $\tau$ of the parabolic induction $\tau_1 \circ \tau_2$, we have
		$$\dblGammaFactor{\tau}{\chi}{\fieldCharacter} = \dblGammaFactor{\tau_1}{\chi}{\fieldCharacter} \dblGammaFactor{\tau_2}{\chi}{\fieldCharacter}.$$
		\item If $\tau$, $\pi'$ and $\pi$ are as in \Cref{thm:multiplicativity-in-terms-of-gauss-sums} and $\chi \ne 1$, then
		$$\dblGammaFactorSpace{\hermitianSpace}{\pi}{\chi}{\fieldCharacter} = \dblGammaFactorSpace{\hermitianSpace'}{\pi'}{\chi}{\fieldCharacter} \dblGammaFactor{\tau}{\chi}{\fieldCharacter}.$$
	\end{enumerate}
\end{theorem}

\begin{remark}
	The ``correct'' gamma factor (expected by Langlands functoriality) is given by $$\dblLanglandsGammaFactorSpace{\hermitianSpace}{\pi}{\chi}{\fieldCharacter} = \dblGammaFactorSpace{\hermitianSpace}{\pi}{\chi}{\fieldCharacter} \cdot \begin{dcases}
				\left(-1\right)^n & \quadraticExtension = \finiteField \text{ and } \epsilon_{\hermitianSpace} = 1\\
				\left(-1\right)^{n+1} \tau\left(\chi^{-1}, \fieldCharacter\right) & \quadraticExtension = \finiteField \text{ and } \epsilon_{\hermitianSpace} = -1 \\
				1 & \text{otherwise.}
	\end{dcases}.$$
	However, we will not need this version for our purposes.
\end{remark}

\subsection{Computation of character sums attached to Deligne--Lusztig data}

The goal of the next two sections is to compute $\dblJacobiSumScalar{\pi}{\chi}$ for $\pi$ that appears in a Deligne--Lusztig virtual character $R_{T,\theta}$ for a suitable maximal torus $T$ and $\theta \colon T \to \multiplicativegroup{\cComplex}$.

In this section, we focus on the computing the analogous character sum corresponding to $\dblJacobiSumScalar{\pi}{\fieldCharacter}$ for the virtual character $R_{T,\theta}$.

\calvin{I kinda prefer the normalization where we divide by $|G|$. I guess both for little and upper $G$. That way it is the honest product of characters in the representation ring.}

Let us realize $H$ as the fixed points of the Frobenius map $\Frobenius$ acting on a connected reductive algebraic group $\algebraicGroup{H}$.

Given a class function $F \colon H \to \cComplex$ (i.e., a function that is invariant under conjugation by elements of $G$), let us define $$\dblVirtualJacobiSumScalar{F}{\chi} = \frac{1}{\sqrt{\sizeof{\lieAlgebra}}} \sum_{g \in G} F\left(g\right) \Phi_{\chi}\left(g\right).$$

It is clear that if $\pi$ is an irreducible representation of $H$ then $$\dblVirtualJacobiSumScalar{\trace \pi}{\chi} = \dblJacobiSumScalar{\pi}{\chi} \cdot \dim \pi.$$
Notice that the assignment $g \mapsto \Phi_{\chi}\left(g\right)$ only depends on the semisimple part of $g \in H$. By \cite[Theorem in Section 1.2]{SaitoShinoda2000}, we have the following result.
\begin{proposition}\label{prop:reduction-of-gauss-sum-to-torus}
	For any $\Frobenius$-stable maximal torus $\algebraicGroup{T}$ of $\algebraicGroup{H}$, and any character $\theta \colon T \to \multiplicativegroup{\cComplex}$, where $T = \algebraicGroup{T}^{\Frobenius}$, we have
	$$ \dblVirtualJacobiSumScalar{R_{T, \theta}}{\chi} = \frac{\grpIndex{H}{T}}{\sqrt{\sizeof{\lieAlgebra}}} \sum_{t \in T} \theta\left(t\right) \Phi_{\chi}\left(t\right).$$
\end{proposition}
In the special case that $\theta$ is in general position, the virtual character $\varepsilon_{\algebraicGroup{H}} \varepsilon_{\algebraicGroup{T}} R_{T, \theta}$ equals $\trace \pi$ for some irreducible representation $\pi$ of $H$ and we have that for this $\pi$,
$$\dblJacobiSumScalar{\pi}{\chi} = \varepsilon_{\algebraicGroup{H}} \varepsilon_{\algebraicGroup{T}} \frac{\sizeof{H}_p}{\sqrt{\sizeof{\lieAlgebra}}} \sum_{t \in T} \theta\left(t\right) \Phi_{\chi}\left(t\right).$$
Here $\varepsilon_{\algebraicGroup{H}} = \left(-1\right)^{\mathrm{rel.rank} \algebraicGroup{H}}$ and $\varepsilon_{\algebraicGroup{T}} = \left(-1\right)^{\mathrm{rel.rank} \algebraicGroup{T}}$, and $\sizeof{H}_p$ is the size of the $p$-Sylow group of $H$, where $p$ is the characteristic of $\finiteField$. 

\subsubsection{Gauss sum notations}
If $\alpha \colon \multiplicativegroup{\finiteFieldExtension{n}} \to \multiplicativegroup{\cComplex}$ is a character, denote the \emph{Gauss sum} $$\tau\left(\alpha, \fieldCharacter_n\right) = -q^{-\frac{n}{2}}\sum_{x \in \multiplicativegroup{\finiteFieldExtension{n}}} \alpha\left(x\right) \fieldCharacter_n\left(x\right),$$
where $\fieldCharacter_n \colon \finiteFieldExtension{n} \to \multiplicativegroup{\cComplex}$ is given by $\fieldCharacter_n = \fieldCharacter \circ \trace_{\finiteFieldExtension{n} \slash \finiteField}$. If $\chi \colon \multiplicativegroup{\finiteField} \to \multiplicativegroup{\cComplex}$ is a character and if $\alpha$ is as above, denote the \emph{twisted Gauss sum}
$$\tau\left(\alpha \times \chi, \fieldCharacter_n\right) = -q^{-\frac{n}{2}}\sum_{x \in \multiplicativegroup{\finiteFieldExtension{n}}} \alpha\left(x\right) \chi\left( \FieldNorm{n}{1}\left(x\right)\right) \fieldCharacter_n\left(x\right).$$

If $m \ge 1$, we denote 
$$\NormOneGroup{2m} = \left\{ x \in \multiplicativegroup{\finiteFieldExtension{2m}} \mid \FieldNorm{2m}{m}\left(x\right)= 1\right\}.$$

\subsubsection{Split torus computation I}
Let $\alpha \colon \multiplicativegroup{\finiteFieldExtension{k}} \to \multiplicativegroup{\cComplex}$ be a character and let $\chi \colon \multiplicativegroup{\finiteField} \to \multiplicativegroup{\cComplex}$ be a non-trivial character.

The goal of this section is to compute $$\sum_{x \in \multiplicativegroup{\finiteFieldExtension{k}}} \alpha\left(x\right) \chi\left(\FieldNorm{k}{1}\left(1-x\right)\right).$$
Let us rewrite this as
$$q^{-k} \sum_{z \in \finiteFieldExtension{k}} \sum_{x \in  \multiplicativegroup{\finiteFieldExtension{k}}} \sum_{y \in \multiplicativegroup{\finiteFieldExtension{k}}}\alpha\left(x\right) \chi\left(\left(-1\right)^k\FieldNorm{k}{1}\left(y\right)\right) \fieldCharacter_k\left(\left(y - \left(x - 1\right)\right)z\right).$$
Since $\chi$ is non-trivial, when $z = 0$, the inner sum over $y$ will vanish. Hence, we have that our sum can be rewritten as
$$q^{-k} \sum_{z \in \multiplicativegroup{\finiteFieldExtension{k}}} \sum_{x \in  \multiplicativegroup{\finiteFieldExtension{k}}} \sum_{y \in \multiplicativegroup{\finiteFieldExtension{k}}}\alpha\left(x\right) \chi\left(\FieldNorm{k}{1}\left(y\right)\right) \fieldCharacter_k\left(\left(y - \left(x - 1\right)\right)z\right).$$
Replacing $x$ with $-z^{-1} x$ and $y$ with $z^{-1} y$, this becomes
$$q^{-k} \alpha\left(-1\right) \chi\left(-1\right)^k \sum_{z \in \multiplicativegroup{\finiteFieldExtension{k}}} \alpha^{-1}\left(z\right) \chi^{-1}\left(\FieldNorm{k}{1}\left(z\right)\right) \fieldCharacter_k\left(z\right) \sum_{x \in  \multiplicativegroup{\finiteFieldExtension{k}}} \alpha\left(x\right) \fieldCharacter_k\left(x\right) \sum_{y \in \multiplicativegroup{\finiteFieldExtension{k}}} \chi\left(\FieldNorm{k}{1}\left(y\right)\right) \fieldCharacter_k\left(y\right).$$
Hence, we have the identity
$$\sum_{x \in \multiplicativegroup{\finiteFieldExtension{k}}} \alpha\left(x\right) \chi\left(\FieldNorm{k}{1}\left(x-1\right)\right) = -q^{\frac{k}{2}} \chi\left(-1\right)^k \alpha\left(-1\right) \GaussSumCharacter{\alpha^{-1}}{\chi^{-1}}{\fieldCharacter_k} \tau\left(\alpha, \fieldCharacter_k\right) \tau\left(\chi \circ \FieldNorm{k}{1}, \fieldCharacter_k\right).$$

\subsubsection{Split torus computation II}
We use the results from the previous section to compute the doubling method Jacobi sum for a torus of the form $\multiplicativegroup{\finiteFieldExtension{k}}$. Let $\alpha \colon \multiplicativegroup{\finiteFieldExtension{k}} \to \multiplicativegroup{\cComplex}$ and $\chi \colon \multiplicativegroup{\finiteField} \to \multiplicativegroup{\cComplex}$ be characters such that $\involutionPlusOne{\chi} \ne 1$. Our goal is to compute $$\sum_{\substack{x \in \multiplicativegroup{\finiteFieldExtension{k}}\\
		x \ne 1}} \alpha \left(x\right) \chi\left(\FieldNorm{k}{1}\left(1 - x\right)\right) \chi\left(\FieldNorm{k}{1}\left(1 - \minusInvolution{x}\right)\right).$$
This can be rewritten as
$$\chi\left(-1\right)^k \sum_{\substack{x \in \multiplicativegroup{\finiteFieldExtension{k}}\\
		x \ne 1}} \alpha \left(x\right) \minusInvolution{\chi}\left(\FieldNorm{k}{1}\left(x\right)\right) \involutionPlusOne{\chi}\left(\FieldNorm{k}{1}\left(1 - x\right)\right).$$
By the previous section, this equals
\begin{equation*}
	-q^{\frac{k}{2}} \alpha\left(-1\right)  \GaussSumCharacter{\alpha^{-1}}{\chi^{-1}}{\fieldCharacter_{\quadraticFieldExtension{k}}} \GaussSumCharacter{\alpha}{\minusInvolution{\chi}}{\fieldCharacter_{\quadraticFieldExtension{k}}} \tau\left(\involutionPlusOne{\chi}, \fieldCharacter_{\quadraticExtension}\right)^k,
\end{equation*}
which equals
\begin{equation}\label{eq:identity-for-multiplicative-group-of-f-k-gauss-sum}
	-q^{\frac{k}{2}} \alpha\left(-1\right)  \GaussSumCharacter{\alpha^{-1}}{\chi^{-1}}{\fieldCharacter_{\quadraticFieldExtension{k}}} \GaussSumCharacter{\involution{\alpha}}{\chi^{-1}}{\fieldCharacter_{\quadraticFieldExtension{k}}} \tau\left(\involutionPlusOne{\chi}, \fieldCharacter_{\quadraticExtension}\right)^k,
\end{equation}

Notice that if $k = md$ and $\alpha = \beta \circ \FieldNorm{k}{m}$ where $\beta \colon \multiplicativegroup{\finiteFieldExtension{m}} \to \multiplicativegroup{\cComplex}$ is a character, then \eqref{eq:identity-for-multiplicative-group-of-f-k-gauss-sum} equals
$$-\left(q^{\frac{m}{2}} \GaussSumCharacter{\beta^{-1}}{\chi^{-1}}{\fieldCharacter_{\quadraticFieldExtension{m}}} \GaussSumCharacter{\involution{\beta}}{\chi^{-1}}{\fieldCharacter_{\quadraticFieldExtension{m}}} \tau\left(\involutionPlusOne{\chi}, \fieldCharacter_{\quadraticExtension}\right)^m\right)^{d}.$$

\subsubsection{Elliptic torus computation}
In the next two sections we compute the doubling method Jacobi sum for a torus of the form $\NormOneGroup{2m}$.

Let $\theta \colon \NormOneGroup{2m} \to \multiplicativegroup{\cComplex}$ and $\chi \colon \multiplicativegroup{\quadraticExtension} \to \multiplicativegroup{\cComplex}$ be non-trivial characters, such that $\involutionPlusOne{\chi} \ne 1$. Our goal is to compute $$\sum_{\substack{x \in \NormOneGroup{2m}\\
		x \ne 1}} \theta \left(x\right) \chi\left(\aFieldNorm_{\finiteFieldExtension{2m} \slash \quadraticExtension}\left(1 - x\right)\right).$$

We separate two cases: the case $\quadraticExtension = \finiteField$ and the case $\quadraticExtension \ne \finiteField$.

\subsubsection{Elliptic torus computation: the case $\quadraticExtension = \finiteField$}
In this section, we compute the above exponential sum for the case $\quadraticExtension = \finiteField$. Since $\chi^2 \ne 1$, it follows that $\theta\left(x^{1-q^m}\right) \ne \chi \circ \aFieldNorm_{\finiteFieldExtension{2m} \slash \quadraticExtension}\left(x\right)$ for some $x \in \multiplicativegroup{\finiteFieldExtension{2m}}$.

As usual, we rewrite the sum as follows
$$q^{-2m} \sum_{z \in \finiteFieldExtension{2m}} \sum_{y \in \multiplicativegroup{\finiteFieldExtension{2m}}} \sum_{x \in \NormOneGroup{2m}} \theta \left(x\right) \chi\left(\aFieldNorm_{\finiteFieldExtension{2m} \slash \finiteField}\left(-y\right)\right) \fieldCharacter_{2m}\left(z\left(y-x+1\right)\right).$$

If $z=0$ then the sum over $y$ will vanish. Hence we can reduce the sum to $z \in \multiplicativegroup{\finiteFieldExtension{2m}}$. Replacing $y$ with $z^{-1} y$ we get the sum
$$-q^{-m} \tau\left(\chi \circ \aFieldNorm_{\finiteFieldExtension{2m} \slash \finiteField}, \fieldCharacter_{2m}\right) \sum_{z \in \multiplicativegroup{\finiteFieldExtension{2m}}} \chi^{-1}\left(\aFieldNorm_{\finiteFieldExtension{2m} \slash \finiteField}\left(z\right)\right) \fieldCharacter_{2m}\left(z\right)  \sum_{x \in \NormOneGroup{2m}} \theta \left(-x\right) \fieldCharacter_{2m}\left(xz\right).$$
To proceed, we use the Hilbert 90 map $\multiplicativegroup{\finiteFieldExtension{2m}} \to \NormOneGroup{2m}$ given by $t \mapsto t^{1 - q^m}$. This map is surjective, and its kernel is $\multiplicativegroup{\finiteFieldExtension{m}}$. Define $\theta' \colon \multiplicativegroup{\finiteFieldExtension{2m}} \to \multiplicativegroup{\cComplex}$ by $\theta'\left(t\right) = \theta\left(t^{1-q^m}\right)$. Replacing the sum over $x \in \NormOneGroup{2m}$ with a sum over $t \in \multiplicativegroup{\finiteFieldExtension{2m}}$, we get \begin{align*}
	& \sum_{z \in \multiplicativegroup{\finiteFieldExtension{2m}}} \chi^{-1}\left(\aFieldNorm_{\finiteFieldExtension{2m} \slash \finiteField}\left(z\right)\right) \fieldCharacter_{2m}\left(z\right) \sum_{x \in \NormOneGroup{2m}} \theta \left(x\right) \fieldCharacter_{2m}\left(xz\right) \\
	= & \frac{1}{q^m-1}\sum_{z \in \multiplicativegroup{\finiteFieldExtension{2m}}} \sum_{t \in \multiplicativegroup{\finiteFieldExtension{2m}}} \chi^{-1}\left(\aFieldNorm_{\finiteFieldExtension{2m} \slash \finiteField}\left(z\right)\right) \fieldCharacter_{2m}\left(z^{q^m}\right) \theta' \left(t\right) \fieldCharacter_{2m}\left(t^{1-q^m} z\right).
\end{align*}
Replacing $z$ with $t^{q^m} z$, this becomes
\begin{align*}
	\frac{1}{q^m-1}\sum_{t,z \in \multiplicativegroup{\finiteFieldExtension{2m}}} \chi^{-1}\left(\aFieldNorm_{\finiteFieldExtension{2m} \slash \finiteField}\left(z\right)\right) \chi^{-1}\left(\aFieldNorm_{\finiteFieldExtension{2m} \slash \finiteField}\left(t\right)\right) \theta' \left(t\right) \fieldCharacter_{2m}\left(\trace_{\finiteFieldExtension{2m} \slash \finiteFieldExtension{m}}\left(z\right) t\right).
\end{align*}
Since $\chi \circ \aFieldNorm_{\finiteFieldExtension{2m} \slash \finiteField} \ne \theta'$, if $\trace_{\finiteFieldExtension{2m} \slash \finiteFieldExtension{m}}\left(z\right) = 0$, we have that the inner sum over $t$ is zero. Hence, we may reduce the sum over $z$ to $z$ such that $\trace_{\finiteFieldExtension{2m} \slash \finiteFieldExtension{m}}\left(z\right) \ne 0$. Using the fact that $\theta'$ is trivial on $\multiplicativegroup{\finiteFieldExtension{m}}$, we have after replacing variables $t \mapsto \frac{t}{\trace_{\finiteFieldExtension{2m} \slash \finiteFieldExtension{m}}\left(z\right)}$,
\begin{align*}
	\frac{1}{q^m-1}\sum_{\substack{t,z \in \multiplicativegroup{\finiteFieldExtension{2m}}\\
			\trace_{\finiteFieldExtension{2m} \slash \finiteFieldExtension{m}}\left(z\right) \ne 0}} \chi^{-1}\left(\aFieldNorm_{\finiteFieldExtension{2m} \slash \finiteField}\left(\frac{z}{\trace_{\finiteFieldExtension{2m} \slash \finiteFieldExtension{m}}\left(z\right) }\right)\right) \chi^{-1}\left(\aFieldNorm_{\finiteFieldExtension{2m} \slash \finiteField}\left(t\right)\right) \theta' \left(t\right) \fieldCharacter_{2m}\left(t\right).
\end{align*}
By the appendix, this equals
\begin{align*}
	q^{\frac{m}{2}} \tau\left(\chi^{-1} \circ \aFieldNorm_{\finiteFieldExtension{m} \slash \finiteField}, \fieldCharacter_m\right)^2 \tau\left(\chi^{2} \circ \aFieldNorm_{\finiteFieldExtension{m} \slash \finiteField}, \fieldCharacter_m\right) \left(\sum_{t \in \multiplicativegroup{\finiteFieldExtension{2m}}} \chi^{-1}\left(\aFieldNorm_{\finiteFieldExtension{2m} \slash \finiteField}\left(t\right)\right) \theta' \left(t\right) \fieldCharacter_{2m}\left(t\right)\right).
\end{align*}

To summarize, we have
$$\sum_{\substack{x \in \NormOneGroup{2m}\\
		x \ne 1}} \theta \left(-x\right) \chi\left(\aFieldNorm_{\finiteFieldExtension{2m} \slash \finiteField}\left(1 - x\right)\right) = q^{\frac{m}{2}} \tau\left(\chi^{-1}, \fieldCharacter\right)^{2m} \tau\left(\chi, \fieldCharacter\right)^{2m} \tau\left(\chi^{2}, \fieldCharacter\right)^m \tau\left(\theta' \times \chi^{-1}, \fieldCharacter_{2m}\right),$$
where $\theta'\left(t\right) = \theta\left(t^{1-q^m}\right)$.
Since $\tau\left(\chi^{-1}, \fieldCharacter\right) = \chi\left(-1\right) \conjugate{\tau\left(\chi, \fieldCharacter\right)}$ and since $\chi \ne 1$, this implies the identity
$$\sum_{\substack{x \in \NormOneGroup{2m}\\
		x \ne 1}} \theta \left(x\right) \chi\left(\aFieldNorm_{\finiteFieldExtension{2m} \slash \finiteField}\left(1 - x\right)\right) = \theta\left(-1\right) q^{\frac{m}{2}} \tau\left(\chi^{2}, \fieldCharacter\right)^m \tau\left(\theta' \times \chi^{-1}, \fieldCharacter_{2m}\right).$$

\subsubsection{Elliptic torus computation: the case $\quadraticExtension \ne \finiteField$}
In this section, we compute the above exponential sum for the case $\quadraticExtension = \finiteFieldExtension{2}$. Let $\chi \colon \multiplicativegroup{\finiteFieldExtension{2}} \to \multiplicativegroup{\cComplex}$. We wish to compute the following sum.
$$\sum_{\substack{x \in \NormOneGroup{2m}\\
		x \ne 1}} \theta \left(x\right) \chi\left(\aFieldNorm_{\finiteFieldExtension{2m} \slash \finiteFieldExtension{2}}\left(1 - x\right)\right).$$
By the appendix, this equals
$$\theta\left(-1\right) q^{\frac{m}{2}} \GaussSumCharacter{\left(\theta'\right)^{-1}}{\chi^{-1}}{\fieldCharacter_{2m}} \tau\left(\chi \restriction_{\multiplicativegroup{\finiteField}} \circ \FieldNorm{m}{1}, \fieldCharacter_m\right),$$
which equals
$$\theta\left(-1\right) q^{\frac{m}{2}} \tau\left(\chi \restriction_{\multiplicativegroup{\finiteField}}, \fieldCharacter\right)^m \GaussSumCharacter{\left(\theta'\right)^{-1}}{\chi^{-1}}{\fieldCharacter_{2m}}.$$

\subsubsection{Computation for Deligne--Lusztig characters}
Combining the results of the previous sections with \Cref{prop:reduction-of-gauss-sum-to-torus}, we arrive at the following result.

\begin{theorem}\label{thm:computation-of-doubling-gauss-sum-scalar-for-deligne-lusztig-characters}
	Suppose that $T \cong \prod_{j=1}^r \multiplicativegroup{\quadraticFieldExtension{k_j}} \times \prod_{i=1}^s \finiteFieldExtension{2m_i}^1$ is a maximal torus of $H$. Suppose that $\theta \colon T \to \multiplicativegroup{\cComplex}$ is a character and that under the above isomorphism $\theta = \alpha_1 \times \dots \times \alpha_r \times \theta_1 \times \dots \times \theta_s$, where $\alpha_j \colon \multiplicativegroup{\quadraticFieldExtension{k_j}} \to \multiplicativegroup{\cComplex}$ and $\theta_i \colon \finiteFieldExtension{2m_i}^1 \to \multiplicativegroup{\cComplex}$ are characters for every $i$ and $j$. Denote \begin{align*}
		g_T\left(\chi, \theta, \fieldCharacter\right)
		= & \prod_{i=1}^s \GaussSumCharacter{\left(\theta'_i\right)^{-1}}{\chi^{-1}}{\fieldCharacter_{2m_i}} \cdot \prod_{j=1}^r \GaussSumCharacter{\involution{\alpha_j}}{\chi^{-1}}{\fieldCharacter_{\quadraticFieldExtension{k_j}}} \GaussSumCharacter{\alpha_j^{-1}}{\chi^{-1}}{\fieldCharacter_{\quadraticFieldExtension{k_j}}}.
	\end{align*} Then
	\begin{align*}
		 \dblVirtualJacobiSumScalar{R_{T, \theta}}{\chi} = \varepsilon_{\algebraicGroup{H}} R_{T,\theta}\left(1\right) c_{\hermitianSpace}\left(\chi, \fieldCharacter\right) g_T\left(\chi, \theta, \fieldCharacter\right) \prod_{i=1}^s \theta_i\left(-1\right) \cdot \prod_{j=1}^r \alpha_j\left(-1\right).
	\end{align*}
\end{theorem}
\begin{proof}
	This follows from the results mentioned above and from the formula $$R_T\left(1\right) = \varepsilon_{\algebraicGroup{H}} \varepsilon_{\algebraicGroup{T}} \frac{\grpIndex{H}{T}}{\sizeof{H}_p},$$ the fact that $$ \sizeof{\lieAlgebra} = \sizeof{H}_p^2 \cdot q^{\frac{\dim_{\finiteField} \hermitianSpace}{2}},$$
	and the fact \begin{equation*}
		\varepsilon_{\algebraicGroup{T}} = \left(-1\right)^r.
	\end{equation*}
\end{proof}
\begin{remark}
	We have $$\varepsilon_{\algebraicGroup{G}} = \left(-1\right)^{\frac{\dim_{\finiteField} \hermitianSpace}{2}}.$$ If $\finiteField = \quadraticExtension$ then we also have $$\varepsilon_{\algebraicGroup{\GroupExtension{G}}} = \left(-1\right)^{\frac{\dim_{\finiteField} \hermitianSpace}{2} + 1}.$$
\end{remark}
\begin{remark}
	Suppose that $\pi$ is any irreducible representation of $H$ such that $\trace \pi$ appears with non-zero coefficient in $R_{T,\theta}$. Then the central character of $\pi$ satisfies
	\begin{align*}
		\centralCharacter{\pi}\left(-1\right) = \theta\left(-1\right) =& \prod_{j=1}^r \alpha_j\left(-1\right) \cdot \prod_{i=1}^s \theta_i\left(-1\right).
	\end{align*}
	Thus \begin{align*}
		\dblVirtualJacobiSumScalar{R_{T, \theta}}{\chi} =& R_{T,\theta}\left(1\right) c_{\hermitianSpace}\left(\chi, \fieldCharacter\right) \centralCharacter{\pi}\left(-1\right) g_T\left(\chi, \theta, \fieldCharacter\right).
	\end{align*}
\end{remark}

\subsubsection{Geometric Conjugacy and Lusztig Series}
(Following Chapter 13 of \cite{DigneMichel2020})
Let $\algebraicGroup{G}$ be a reductive group over $\finiteField$ and fix a torus $\algebraicGroup{T}$ defined over $\finiteField$. The \emph{dual group over $\finiteField$} of $\algebraicGroup{G}$ with respect to the torus $\algebraicGroup{T}$ corresponds to the group $\algebraicGroup{G}^*$ with a dual torus $\algebraicGroup{T}^*$ and root datum
\[
    (X^*(\algebraicGroup{T}^*),\Phi_{\algebraicGroup{T}^*},X_*(\algebraicGroup{T}^*),\Phi_{\algebraicGroup{T}^*}^\vee) \cong (X_*(\algebraicGroup{T}),\Phi_{\algebraicGroup{T}}^\vee,X^*(\algebraicGroup{T}),\Phi_{\algebraicGroup{T}})
\]
such that the isomorphism respects the Frobenius action on $X^*$ and $X_*$.

Fix an isomorphism $\multiplicativegroup{\algebraicClosure{\finiteField}} \cong (\mathbb{Q}/\mathbb{Z})_{p'}$ and an embedding $\exp:(\mathbb{Q}/\mathbb{Z})_{p'} \to \overline{\mathbb{Q}_\ell}^\times$. 
We get an isomorphism
\begin{equation}
\label{eq:DL}
\xymatrix{0 \ar[r] & X_*(\algebraicGroup{T})_{(\mathbb{Q}/\mathbb{Z})_{p'}} \ar[r]^{\Frobenius-\idmap} & X_*(\algebraicGroup{T})_{(\mathbb{Q}/\mathbb{Z})_{p'}} \ar[r] & \algebraicGroup{T}(\algebraicClosure{\finiteField}) \ar[r] & 0}
\end{equation}
with the last map being $y \mapsto \aFieldNorm_{\algebraicGroup{T}(\finiteFieldExtension{n}):\algebraicGroup{T}(\finiteField)}(y(\zeta_n))$
for $\zeta_n$ being a $q^n-1$ root of unity corresponding to $1/(q^n-1) \in (\mathbb{Q}/\mathbb{Z})_{p'}$ and sufficiently large $n$ (large enough so that $\algebraicGroup{T}\left(\finiteFieldExtension{n}\right)$ is split).

Hence for a torus character pair $(T,\theta)$ for $G$, we may conjugate $\algebraicGroup{T}$ to a fixed torus $\algebraicGroup{T}'$ and associate to $\theta$ a twisted Frobenius (twisted by an element of $W$ in the twisted Weyl group conjugacy class corresponding to $T$) stable element $x_{T,\theta}$ of $X^*(\algebraicGroup{T}')_{(\mathbb{Q}/\mathbb{Z})_{p'}}$ via the equality
\[
    \exp(\langle x_{T,\theta},y\rangle) = \theta(\aFieldNorm_{\algebraicGroup{T}(\finiteFieldExtension{n}):\algebraicGroup{T}(\finiteField)}(y(\zeta_n)))
\]
for large enough $n$, where $\aFieldNorm_{\algebraicGroup{T}(\finiteFieldExtension{n}):\algebraicGroup{T}(\finiteField)}(t) = t\cdot \Frobenius(t)\cdot \Frobenius^2(t)\cdot  \hdots \cdot \Frobenius^{n-1}(t)$ is the torus norm map. This $x_{T,\theta}$ is well defined up to the Weyl group, and in fact there is a correspondence between $(T,\theta)$ pairs and $\Frobenius$-stable Weyl group orbits in $X^*(\algebraicGroup{T}')_{(\mathbb{Q}/\mathbb{Z})_{p'}}$ \cite[Corollary 13.9]{DigneMichel2020}.  By composing with \eqref{eq:DL}, we associate to a torus character pair a Frobenius stable semisimple conjugacy class in $G^*$.

\begin{definition}
    Denote the Frobenius semisimple conjugacy class associated to a torus character pair $(T,\theta)$ by $(s_{T,\theta})$. Similarly denote the Frobenius stable Weyl group element of $X^*(\algebraicGroup{T})_{(\mathbb{Q}/\mathbb{Z})_{p'}}$ corresponding to $(T,\theta)$ by $(x_{T,\theta}).$

    The pairs $(T',\theta'),(T,\theta)$ are said to be \emph{geometrically conjugate} is the geometric conjugacy classes $(s_{T,\theta}), (s_{T,\theta'})$ are equal.
    
    The \emph{Lusztig series} $\LusztigSeries{G}{s}$ associated to a Frobenius stable geometric conjugacy class $(s) \subset \algebraicGroup{G}^*$ consists of (equivalence classes of) the irreducible representations $\pi$ which appear with non-zero multiplicity in $R_{T, \theta}$ for some $(T,\theta)$ such that $(s_{T,\theta}) = (s)$.
\end{definition}

\begin{center}
\begin{tabular}{ c c c c c c}
 Group & Fixed Torus & Frobenius on $X(T)$ & Frobenius on $W$ &Dual Group & Dual Torus \\ 
 $Sp(2n)$ & $(\multiplicativegroup{\finiteField})^n$ & 1 & 1 &$SO(2n+1)$ & $(\multiplicativegroup{\finiteField})^n$\\  
 $GSp(2n)$ & $(\multiplicativegroup{\finiteField})^{n+1}$ & 1 & 1 &$GSpin(2n+1)$ &  $(\multiplicativegroup{\finiteField})^{n+1}$ \\
 $SO(2n)$ & $(\multiplicativegroup{\finiteField})^{n}$ & 1 & 1 &$SO(2n)$ &  $(\multiplicativegroup{\finiteField})^{n}$ \\
 $GO(2n)$ & $(\multiplicativegroup{\finiteField})^{n+1}$ & 1 & 1 &? &  $(\multiplicativegroup{\finiteField})^{n+1}$ \\
 $SO(2n+1)$ & $(\multiplicativegroup{\finiteField})^{n}$ & 1 & 1 &$Sp(2n)$ &  $(\multiplicativegroup{\finiteField})^{n}$ \\
 $GO(2n+1)$ & $(\multiplicativegroup{\finiteField})^{n+1}$ &1 & 1 &? &  $(\multiplicativegroup{\finiteField})^{n+1}$ \\ 
  $U(n)$ & $(\NormOneGroup{2})^{n}$ & -1 & 1 & $U(n)$ &  $(\NormOneGroup{2})^{n}$ 
\end{tabular}
\end{center}

\subsubsection{Invariance under geometric conjugacy}
\calvin{todo:type in some motivation for class functions whose product doesn't depend on geometric conjugacy, gamma functions whatever}.

\begin{definition}
	A \emph{stable function} on $\algebraicGroup{G}(\finiteField)$ is a class function $f$ such that $\sum_{g \in G} f(g)\pi(g)$ is a scalar which only depends on the Lusztig packet for $\pi$. \calvin{FUNCTORIAL TRANSFER FOR REDUCTIVE GROUPS AND CENTRAL
COMPLEXES has this definition}
\end{definition}

A basic example of a class function which doesn't depend of geometric conjugacy 
\begin{proposition}[FIND SOMEWHERE TO CITE or put in appendix]
\label{prop:central_char}
    Any function $f$ supported on the center $\algebraicGroup{Z}(\mathbb{F})$ of $\algebraicGroup{G}(\finiteField)$ is a stable function. 
\end{proposition}
\begin{proof}
\calvin{can probably put this proof in an appendix}
Note it is enough to prove the statement for $f = 1_z$ an indicator function of an element $z \in \algebraicGroup{Z}(\mathbb{F})$. The he proof follows from the two claims:\\
Claim 1: $z$ acts by the scalar $\theta(z)$ in the virtual representation $R_{T,\theta}$.\\
Claim 2: If $(T,\theta),(T',\theta')$ are geometrically conjugate then $\theta|_{\algebraicGroup{Z}(\mathbb{F})} = \theta'_{\algebraicGroup{Z}(\mathbb{F})}$.\\
Proof of claim 1: The representation $R_T^G(\theta)$ is equal to $M \otimes_{\mathbb{C}[T]} \theta$ where $M$ is the virtual module corresponding to the Euler characteristic of cohomology on $X$ which as a variety is a subvariety of $\algebraicGroup{G}$ with the $G \times T$ action via $(g,\ell).x = gx\ell$. The center $\algebraicGroup{Z}$ then commutes across $X$ so the action of $(z,1)$ equals the action of $(1,z)$. The result is on the $\theta$ component of $M$, $z$ acts by $\theta(z)$.\\
Proof of claim 2: Geometric conjugacy is equivalent to for large enough $n$, there exists $g \in \algebraicGroup{G}(\mathbb{F}_n)$ \elad{Here you mean $\algebraicGroup{G}\left(\finiteFieldExtension{n}\right)$?}\calvin{oops yes} such that $g\algebraicGroup{T}g^{-1} = \algebraicGroup{T}'$ and $g(\theta \circ Nm) g^{-1} = (\theta' \circ Nm)$. So when restricted to the center, $\theta \circ Nm = \theta' \circ Nm$. Since $Nm$ is surjective onto the $\mathbb{F}$-points, $\theta|_{\algebraicGroup{Z}(\mathbb{F})} = \theta'|_{\algebraicGroup{Z}(\mathbb{F})}$.
\end{proof}


The main result of this subsection is 
\begin{theorem}
\label{thm:indep_geo_conj}
	$\Phi_{\chi}(g)$ is a stable function.
	%Equivalently	$\frac{\dblVirtualJacobiSumScalar{R_{T, \theta}}{\chi}}{R_{T,\theta}(1)}$ only depends on the geometric conjugacy class of $(T,\theta)$.
\end{theorem}

In the $\GL_n$ case this follows from \Cref{thm:gln-doubling-gauss-sum-in-terms-of-kondo} along with \calvin{Hasse-Davenport? Something else?}.


For a torus theta pair $(T,\theta)$ with associated Frobenius conjugacy classes $w \in W(\algebraicGroup{T})$ and $(s) \subset G^*$ and an embedding $\rho:G^* \to \GL_n$.  Fix a Frobenius stable maximal torus $\algebraicGroup{T}^* \subset G^*$ and a Frobenius stable maximal torus $\algebraicGroup{T} \subset \GL_n$ which contains $\rho(\algebraicGroup{T}^*)$. Let $\tilde{T}$ be the torus of $\GL_n$ corresponding to the Frobenius conjugacy class of $\rho(w) \in W(\algebraicGroup{T})$ and $\tilde{\theta}$ such that $(\rho(s)) = (s_{\tilde{T},\tilde{\theta}})$. Define
\[
    g_{T,\rho}(\theta,\psi,\chi) := \varepsilon_{\algebraicGroup{
    \tilde{T}}}\sum_{t\in \tilde{T}} \tilde{\theta}(t)\chi(\det(t))\psi(\trace(t)).
\]
The following lemma shows that this is well defined.

\begin{lemma}
\label{lemma:gl_invariance}
\begin{enumerate}[(i)]
	\item \label{item:gl-n-hasse-davenport-gauss-sum-implies-well-defined} In the case $G = \GL_n$ and $\rho$ is the standard representation, $g_{T,\rho}(\theta,\psi,\chi)$ depends only on the semisimple conjugacy class of $(s)$.
	\item $g_{T,\rho}(\theta,\psi,\chi)$ is well defined i.e. independent of choice of $\algebraicGroup{T}^* \subset G^*$ and $\algebraicGroup{T} \subset \GL_n$.
\end{enumerate}
\end{lemma}
\begin{proof}
\begin{enumerate}[(i)]
	\item In this case when $T = \prod_{i} \multiplicativegroup{\finiteFieldExtension{n_i}}$ and $\theta = \prod_{i} \alpha_i$, we recover the Gauss sum
	\[
	g_{T,\rho}(\theta,\psi,\chi)  = \prod_i \tau(\alpha_i \times \chi, \fieldCharacter_{n_i}),
	\]
	which by the Hasse-Davenport relation depends only on the geometric conjugacy class of $(T,\theta)$ hence the semisimple conjugacy class of $(s_{T,\theta})$ \calvin{maybe write more or less?}. 
	\item For a Frobenius stable semisimple conjugacy class $(s)$ in $\GL_n$, we define
	\[
	g_{(s)}(\psi,\chi) := g_{T,\rho}(\theta,\psi,\chi) \textnormal{ for some } (T,\theta) \textnormal{ such that } (s_{T,\theta}) = (s)
	\]
	By (\ref{item:gl-n-hasse-davenport-gauss-sum-implies-well-defined}) this is well defined. Then
	\[
	g_{T,\rho}(\theta,\psi,\chi)  = g_{(\rho(s_{T,\theta}))}(\psi,\chi)
	\]
	which is independent of of choice of $\algebraicGroup{T}^* \subset G^*$ and $\algebraicGroup{T} \subset \GL_n$.
\end{enumerate}
\end{proof}

\begin{corollary}
\label{cor:geoconj}
Suppose that $(T,\theta)$ and $(T',\theta')$ are geometrically conjugate torus character pairs in $G$. Then $g_{T,\rho}(\theta,\psi,\chi) = g_{T',\rho}(\theta',\psi,\chi)$.
\end{corollary}
\begin{proof}
Both are equal to $g_{(\rho(s_{T,\theta}))}(\psi,\chi) = g_{(\rho(s_{T',\theta'}))}(\psi,\chi)$.
\end{proof}

The idea behind the proof of \Cref{thm:indep_geo_conj} is that the formula from \Cref{thm:computation-of-doubling-gauss-sum-scalar-for-deligne-lusztig-characters} relates $\dblVirtualJacobiSumScalar{R_{T, \theta}}{\chi}/R_{T,\theta}(1)$ to $g_{T,\rho}(\theta,\psi,\chi)$ for the appropriate $\rho$ in the table below. 
\begin{center}
\begin{tabular}{ c c}
 Group & Dual group representations $\rho$ \\ 
 $Sp(2n)$ &$SO(2n+1) \to \GL(2n+1)$\\  
 $GSp(2n)$  &$GSpin(2n+1) \to \GL(2n+1)$ \\
 $SO(2n)$ &$SO(2n) \to \GL(2n)$ \\
 $GO(2n)$ & ? \\
 $SO(2n+1)$ &$Sp(2n) \to \GL(2n)$ \\
 $GO(2n+1)$ & ? \\ 
  $U(n)$  & $U(n) \to \GL(2n)$  
\end{tabular}
\end{center}
Note that in the $SO$ and $GSO$ case, the map $\rho$ sends the split torus of $G^*$ into the split torus of $\GL(2n)$ and the induced map of cocharacters sends $x \mapsto (x,-x)$. In the $Sp,GSp$ cases the map is similar: the induced map of cocharacters sends $x \mapsto (x,1,-x)$. In the unitary case, the torus whose $\finiteField$ points are $(\NormOneGroup{2})^n \subset G^*(\finiteField)$ includes into the torus in $\GL(2n)$ whose $\finiteField$ points are $(\finiteFieldExtension{2}^\times)^n$ and the induced map of cocharacters is $(x_1,x_2, \hdots, x_n) \mapsto (x_1,-x_1,x_2,-x_2, \hdots ,x_n,-x_n)$. 


In particular to analyze $g_{T,\rho}(\theta,\psi,\chi)$ for these $\rho$, we need to understand what torus theta pair of $\GL$ corresponds to $\phi((s))$ under a map $\phi(x) = (x,-x)$:
%The following computation in the proof of theorem \ref{thm:indep_geo_conj}.

\begin{lemma}
\label{lem:cochar_computation}
    Define $\phi:(\mathbb{Q}/\mathbb{Z})_{p'}^n \to (\mathbb{Q}/\mathbb{Z})_{p'}^{2n} = X_*(\GL_{2n})_{(\mathbb{Q}/\mathbb{Z})_{p'}}$ by $\phi(x) = (x,-x)$. We will consider two different Frobeinus actions on these characters:
    \begin{enumerate}
        \item When $\Frobenius$ acts by trivially on both, $x_{\finiteFieldExtension{n}^\times \times \finiteFieldExtension{n}^\times,\alpha \times \alpha^{-1}} = \phi(x_{\finiteFieldExtension{n},\alpha})$
        \item When $\Frobenius$ acts by $-1$ on the domain and by $\begin{pmatrix}
            0 & 1 \\
            1 & 0\end{pmatrix}$ on the codomain, $x_{\finiteField_{2n}^\times,\theta'}=\phi(x_{\NormOneGroup{2n},\theta})$ where $\theta'(t) = \theta(t^{1-q^n})$
    \end{enumerate}
\end{lemma}
\begin{proof}
Recall that $x_{T,\theta}$ is defined by for sufficiently large $m$ with $\zeta_m = 1/(q^m-1)$ under the fixed identification $\multiplicativegroup{\algebraicClosure{\finiteField}} \cong (\mathbb{Q}/\mathbb{Z})_{p'}$
\[
    \exp(\langle x_{T,\theta},y\rangle) = \theta(y(N_{T(\finiteField_m):T(\finiteField)}(y(\zeta_m))).
\]
Thus the result follows from the computations:
\begin{enumerate}
	\item \begin{align*}
		\exp(\langle\phi(x_{\finiteFieldExtension{n},\alpha}),(y_1,y_2)\rangle) &= \exp(\langle x_{\finiteFieldExtension{n},\alpha},\phi^*(y_1,y_2)\rangle) \\
		&= \alpha\left(\FieldNorm{mn}{n}(y_1(\zeta_{mn})\right) \alpha\left(\FieldNorm{mn}{n}(y_2(\zeta_{mn})\right)^{-1} \\
		&= \left(\alpha\times\alpha^{-1}\right)\left(\FieldNorm{mn}{n}((y_1,y_2)(\zeta_{mn}))\right)
	\end{align*}
	\item \begin{align*}
		\exp(\langle\phi(x_{\NormOneGroup{2n},\theta}),(y_1,y_2)\rangle) &= \exp(\langle x_{\NormOneGroup{2n},\theta},\phi^*(y_1,y_2)\rangle) \\
		&= \theta\left(\frac{\FieldNorm{2mn}{2n}(y_1(\zeta_{2mn})}{\FieldNorm{2mn}{2n}\left(y_1(\zeta_{2mn})^{q^n}\right)}\right) \theta\left(\frac{\FieldNorm{2mn}{2n}(y_2(\zeta_{2mn})}{\FieldNorm{2mn}{2n}\left(y_2(\zeta_{2mn})^{q^n}\right)}\right)^{-1} \\
		&= \theta'\left(\FieldNorm{2mn}{2n}((y_1,y_2)(\zeta_{mn}))\right)
	\end{align*}
\end{enumerate}
\end{proof}




\begin{proof}[Proof of \Cref{thm:indep_geo_conj}] 
By \Cref{thm:computation-of-doubling-gauss-sum-scalar-for-deligne-lusztig-characters}, it is enough to show that $g_{T}(\chi,\theta,\psi)$ and $\omega_{\pi}(-1) = \theta(-1)$ only depend of the geometric conjugacy of $(T,\theta)$. The latter only depends on geometric conjugacy by \Cref{prop:central_char}.

Suppose $T \cong \prod_{j=1}^r \multiplicativegroup{\finiteFieldExtension{k_j}} \times \prod_{i=1}^s \finiteFieldExtension{2m_i}^1$. We compute $g_{T,\rho}(\theta,\psi,\chi)$ using the the split torus $\algebraicGroup{T}^*$ in $G^*$ including into the split torus $\algebraicGroup{T}'$ in $\GL$ in the non unitary case and in the unitary case we use the tori $\algebraicGroup{T}^*$ whose $\finiteField$ points are $(\NormOneGroup{2})^n \subset G^*(\finiteField)$ includes into the torus $\algebraicGroup{T}'$ in $\GL(2n)$ whose $\finiteField$ points are $(\finiteFieldExtension{2}^\times)^n$. We then define $\tilde{T}$ as the $\finiteField$-rational torus corresponding to the Frobenius stable Weyl group orbit of $\rho(w)$ inside $W(\algebraicGroup{T}')$ where $(w)$ inside $W(\algebraicGroup{T}^*)$ is the Frobenius stable Weyl group orbit associated to $T$. \Cref{lem:cochar_computation} above implies that
\[
g_{T,\rho}(\theta,\psi,\chi)= g_{T}(\theta,\psi,\chi) \cdot \begin{cases}
        1  & G \neq Sp, GSp\\
        \tau(1,\psi_1) & G = Sp \textnormal{ or } GSp
    \end{cases}.
\]
\Cref{cor:geoconj} then implies that $g_T(\theta,\psi,\chi)$ only depends on the geometric conjugacy class of $(T,\theta)$.   
\begin{comment}
We will have to treat the unitary group case seperatedly. First for the non-unitary group case:\\



Write $T \cong \prod_{j=1}^r \multiplicativegroup{\finiteFieldExtension{k_j}} \times \prod_{i=1}^s \finiteFieldExtension{2m_i}^1 \left[\times \{\pm 1\}\right]$ with $k_1 \geq k_2 \hdots \geq k_r$ and $m_1 \geq m_2 \hdots \geq m_r$ and under the above isomorphism $\theta = \alpha_1 \times \dots \times \alpha_r \times \theta_1 \times \dots \times \theta_s \times \epsilon$, where  $\left[\epsilon \colon \left\{\pm 1\right\} \to \multiplicativegroup{\cComplex},\right]$ $\alpha_j \colon \multiplicativegroup{\finiteFieldExtension{k_j}} \to \multiplicativegroup{\cComplex}$ and $\theta_i \colon \finiteFieldExtension{2m_i}^1 \to \multiplicativegroup{\cComplex}$. Then the Frobenius conjugacy class of the Weyl group corresponding to $T$ is represented by $((k_1,k_2, \hdots, k_r), (m_1,m_2 \hdots m_s))$. The map $\rho$ sends this conjugacy class (or union of two conjugacy classes) to the conjugacy class with cycle decomposition the union of cycles of lengths $k_1,k_1,k_2,k_2 \hdots, k_r,k_r, 2m_1,2m_2 \hdots 2m_r$ along with a cycle of length $1$ in the $Sp$ and $GSp$ case. Let $\tilde{T}$ be the torus in $\GL$ corresponding to the conjugacy class above and let $\tilde{\theta}$ be the character of $\tilde{T}$ corresponding to $\Phi(s)$. Then by theorem \ref{thm:thm:computation-of-doubling-gauss-sum-scalar-for-deligne-lusztig-characters}
\[
    \sum_{t \in \tilde{T}} \tilde{\theta}(t) \chi(det(t-1)) = \begin{cases}
        g_T(\theta,\psi)  & G \neq Sp \textnormal{ or } GSp\\
        g_T(\theta,\psi) \cdot g(1,\psi_1) & G = Sp \textnormal{ or } GSp
    \end{cases}.
\]
Since the sum above is a Gauss sum for $\GL$ it is equal to the Gauss sum attatched to the semisimple element $\Phi(s)$. \\
Now for the unitary case: \calvin{this needs to be checked... I used engineers induction to guess at how the Weyl group conjugacy classes match the torus}
Write $T \cong \prod_{j=1}^r \multiplicativegroup{\finiteFieldExtension{k_j}} \times \prod_{i=1}^s (\finiteFieldExtension{2m_i}^1)^2$ with $k_i$'s even and $m_i$'s odd \calvin{this is what I need to check... what tori you can get here} and under the above isomorphism $\theta = \alpha_1 \times \dots \times \alpha_r \times \theta_1 \times \dots \times \theta_s \times \epsilon$, where  $\left[\epsilon \colon \left\{\pm 1\right\} \to \multiplicativegroup{\cComplex},\right]$ $\alpha_j \colon \multiplicativegroup{\finiteFieldExtension{k_j}} \to \multiplicativegroup{\cComplex}$ and $\theta_i \colon \finiteFieldExtension{2m_i}^1 \to \multiplicativegroup{\cComplex}$. Then the Frobenius conjugacy class of the Weyl group corresponding to $T$ is represented by having even cycles of lengths $k_1,k_2 \hdots k_r$ and odd cycles of lengths $m_1,m_1,m_2,m_2, \hdots, m_s,m_s$. The map $\Phi$ sends this conjugacy class to the conjugacy class with even cycles $k_1,k_1,k_2,k_2 \hdots k_r,k_r$ and odd cycles $m_1,m_1,m_1,m_1,m_2,m_2,m_2,m_2, \hdots, m_s,m_s,m_s,m_s$. Let $\tilde{T}$ be the torus in $\GL$ corresponding to the conjugacy class with all even cycles of lengths $k_1,k_1,k_2,k_2 \hdots k_r,k_r$ and $2m_1,2m_1,2m_2,2m_2, \hdots, 2m_s,2m_s$. Then $\Phi(s)$ is in this torus \calvin{write out what that means} giving a character $\tilde{\theta}$ on it and 
\[
    \sum_{t \in \tilde{T}} \tilde{\theta}(t) \chi(det(t-1)) = g_T(\theta,\psi).
\]
Since the sum above is a Gauss sum for $\GL$ it is equal to the Gauss sum attatched to the semisimple element $\Phi(s)$.
\end{comment}
\end{proof}

\subsection{Computation for Lusztig series}

The goal of this section is to use \Cref{thm:computation-of-doubling-gauss-sum-scalar-for-deligne-lusztig-characters} and results of Lusztig to determine $\dblJacobiSumScalar{\pi}{\fieldCharacter}$ for any irreducible representation $\pi$ such that $\trace \pi$ appears with non-zero coefficient in $R_{T,\theta}$.

\begin{theorem}
	Let $T \subset H$ be a maximal torus and let $\theta \colon T \to \multiplicativegroup{\cComplex}$ be a character. Write $$R_{T,\theta} = \sum_{\Pi} c_{\Pi} \trace \Pi,$$
	where $\Pi$ goes over all the irreducible representations of $H$ and $c_{\Pi} \in \zIntegers$. Suppose that $\pi$ is an irreducible representation such that $c_{\pi} \ne 0$. Then $$\dblGammaFactorSpace{\hermitianSpace}{\pi}{\chi}{\fieldCharacter} = g_T\left(\theta, \fieldCharacter\right).$$
\end{theorem}

\subsubsection{Proof of the main result}

\elad{Merge with Calvin's stuff}
The \emph{Lusztig series} $\LusztigSeries{G}{s}$ consists of (equivalence classes of) the irreducible representations $\pi$ of $R_{T, \theta}$ with $c_{\pi} \ne 0$ as in the theorem.

Let $L \subset G$ be a Levi subgroup of $G$, and suppose that $L \cong \prod_{j=1}^\ell \GL_{t_j}\left(\finiteField\right) \times G'$ where $G' = \IsometryGroup\left(\hermitianSpace'\right)$ where $\hermitianSpace'$ is a non-degenerate hermitian subspace and $\hermitianSpace = \xIsotropic \oplus \hermitianSpace' \oplus \yIsotropic$, where $\xIsotropic$ and $\yIsotropic$ are totally isotropic subspace in duality. Suppose that $s \in L$, and denote by $s_i$ the projection of $s$ to $\GL_{t_i}\left(\finiteField\right)$ for $1 \le i \le \ell$ and by $s_0$ the projection of $s$ to $G'$. We denote
$$\LusztigSeries{L}{s} = \left(\bigotimes_{j=1}^{\ell} \LusztigSeries{\GL_{t_j}\left(\finiteField\right)}{s_j}\right) \otimes \LusztigSeries{G'}{s_0}.$$

We will need the following two facts:
\begin{enumerate}
	\item For the groups $\GroupExtension{G}$ as the ones we consider in this paper, the Lusztig series $\LusztigSeries{\GroupExtension{G}}{s}$ contains at most one irreducible cuspidal representation.
	\item The Lusztig series $\LusztigSeries{\GroupExtension{G}}{s}$ is a union of Harrish--Chandra series corresponding to Harrish-Chandra data of the form $\left(P,\sigma\right)$ where $P$ is a parabolic subgroup of $\GroupExtension{G}$ with Levi part $L$ and $\sigma \in \LusztigSeries{L}{s}$ is an irreducible cuspidal representation.
	\item \elad{TODO} The correspondence between the Lusztig series $\LusztigSeries{\GroupExtension{G}}{\tilde{s}}$ and $\LusztigSeries{G}{s}$, where $\tilde{s} = i^{\ast}\left(s\right)$ (\cite[Proof of Theorem 1]{Li2023}, \cite[Proof of Theorem 3.2]{Li2019}, \cite[Proof of Lemma 4.8]{SchaefferFryTaylorVinroot2025}).	
\end{enumerate}
Our theorem is equivalent to the following statement.
\begin{proposition}\label{prop:gamma-factor-is-constant-on-lusztig-series}
	Keep the notations as above. Suppose that $P$ is a parabolic subgroup of $G$ with Levi part $L$. Then for any irreducible cuspidal $\sigma \in \LusztigSeries{L}{s}$ and for any irreducible subrepresentation $\pi \subset \Ind{P}{G}{\sigma}$, $$\dblGammaFactorSpace{\hermitianSpace}{\pi}{\chi}{\fieldCharacter} = \tau_{G, \chi, \fieldCharacter}\left(s\right).$$
\end{proposition}
\begin{proof}
	The proof is by induction on $\dim_{\quadraticExtension} \hermitianSpace$.
	
	Suppose first that $L \ne G$. Then by using the multiplicativity property repeatedly (\Cref{thm:multiplicativity-in-terms-of-gamma-factors}), we have $$\dblGammaFactorSpace{\hermitianSpace}{\pi}{\chi}{\fieldCharacter} = \dblGammaFactorSpace{\hermitianSpace'}{\pi'}{\chi}{\fieldCharacter} \cdot \prod_{j=1}^r \tau_{\GL_{t_j}\left(\finiteField\right), \chi, \fieldCharacter}\left(s_j\right) \tau_{\GL_{t_j}\left(\finiteField\right), \chi, \fieldCharacter}\left(s_j^{-1}\right),$$
	where $\pi'$ is an irreducible cuspidal representation of $G'$ that lies in the Lusztig series $\LusztigSeries{\pi'}{s_0}$. By the induction hypothesis, we have $$\dblGammaFactorSpace{\hermitianSpace'}{\pi'}{\chi}{\fieldCharacter} = \tau_{G', \chi, \fieldCharacter}\left(s_0\right),$$
	which implies the result.
	
	Next, suppose that $L = G$ and that $\pi = \sigma$ is an irreducible cuspidal representation. Let us write $$R_{T, \theta} = \sum_{\Pi} c_\Pi \trace \Pi,$$
	where $\Pi$ goes over all the irreducible representations of $G$ and $c_{\Pi} \in \zIntegers$. For every $\Pi$ with $c_{\Pi} \ne 0$ and $\Pi \ne \pi$, we have that $\Pi$ lies in a Harish--Chandra series with $L \ne G$ as above, and therefore $$\dblVirtualJacobiSumScalar{\trace \Pi}{\chi} = \dim \Pi \cdot \centralCharacter{\Pi}\left(-1\right) \cdot c_{\hermitianSpace}\left(\chi, \fieldCharacter\right) \cdot \tau_{G, \chi, \fieldCharacter}\left(s\right).$$
	By \Cref{thm:computation-of-doubling-gauss-sum-scalar-for-deligne-lusztig-characters}, we have $$\dblVirtualJacobiSumScalar{R_{T,\theta}}{\fieldCharacter} = R_{T,\theta}\left(1\right) \cdot \theta\left(-1\right) \cdot c_{\hermitianSpace}\left(\chi, \fieldCharacter\right) \cdot \tau_{G, \chi, \fieldCharacter}\left(s\right).$$
	Since $\centralCharacter{\Pi}\left(-1\right) = \theta\left(-1\right)$ for every $\Pi$ with $c_{\Pi} \ne 0$,
	$$c_{\pi} \cdot \dblVirtualJacobiSumScalar{\trace \pi}{\fieldCharacter} = \left(R_{T,\theta}\left(1\right) - \sum_{\substack{\Pi \ne \pi\\
	c_{\Pi} \ne 0}} c_{\Pi} \dim \Pi\right) \cdot \theta\left(-1\right) \cdot c_{\hermitianSpace}\left(\chi, \fieldCharacter\right) \cdot \tau_{G, \chi, \fieldCharacter}\left(s\right).$$
	Since $$R_{T,\theta}\left(1\right) = \sum_{\Pi} c_{\Pi} \trace \Pi,$$ it follows that $$\dblVirtualJacobiSumScalar{\trace \pi}{\chi} = \dim \pi \cdot \centralCharacter{\pi}\left(-1\right) \cdot c_{\hermitianSpace}\left(\chi, \fieldCharacter\right) \cdot \tau_{G, \chi, \fieldCharacter}\left(s\right).$$
	Since $\dblVirtualJacobiSumScalar{\trace \pi}{\chi} = \dim \pi \cdot \dblJacobiSumScalar{\pi}{\chi}$, the result follows.
\end{proof}

\subsection{Kloosterman type sums}

Let $\chi_1, \dots, \chi_k \colon \multiplicativegroup{\finiteField} \to \multiplicativegroup{\cComplex}$ be characters such that $\chi_i^2 \ne 1$ for any $i$. Consider the assignment $K \colon G \to \cComplex$ given by $$K\left(x\right) = K_{\chi_1,\dots,\chi_k}\left(x\right) = \sum_{\substack{g_1, \dots, g_k \in G\\
g_1 \dots g_k = x}} \Phi_{\chi_1}\left(g_1\right) \dots \Phi_{\chi_k}\left(g_k\right).$$
It is clear that $K$ is a class function of $G$. Let $\pi$ be an irreducible representation of $G$. Then \begin{align*}
	 \innerproduct{K}{\conjugate{\trace \pi}} &= \frac{1}{\sizeof{G}}\trace \left(\sum_{g_1,\dots,g_k \in G} \Phi_{\chi_1}\left(g_1\right) \dots \Phi_{\chi_k}\left(g_k\right) \pi\left(g_1 \dots g_k\right) \right) \\
	 &= \frac{\sizeof{\lieAlgebra}^{\frac{k}{2}}}{\sizeof{G}} \trace\left( \dblJacobiSum{\pi}{\chi_1} \circ \dots \dblJacobiSum{\pi}{\chi_k} \right).
\end{align*}
Thus $$\innerproduct{K}{\conjugate{\trace \pi}} = \frac{\sizeof{\lieAlgebra}^{\frac{k}{2}} \dim \pi}{\sizeof{G}} \cdot \prod_{j=1}^k \dblJacobiSumScalar{\pi}{\chi_j}.$$
It follows that if
$$\DeligneLusztigInduction{T}{G}\theta = \sum_{\pi} c_{\pi} \cdot \trace \pi,$$
then $$\innerproduct{K}{\conjugate{\DeligneLusztigInduction{T}{G}\theta}} = \frac{\sizeof{\lieAlgebra}^{\frac{k}{2}} \prod_{j=1}^k \dblJacobiSumScalar{\pi}{\chi_j}}{\sizeof{G}} \sum_{\pi} c_{\pi} \dim \pi = \frac{\sizeof{\lieAlgebra}^{\frac{k}{2}} \prod_{j=1}^k \dblJacobiSumScalar{\pi}{\chi_j}}{\sizeof{G}} \left(\DeligneLusztigInduction{T}{G}\theta\right)\left(1\right),$$
where $\pi$ is any irreducible representation of $G$ such that $c_{\pi} \ne 0$.

Suppose that $T \cong \NormOneGroup{2m}$, and let $t \in T$ be an element such that $t$ is contained only in the torus $T$. Then
$$K\left(t\right) = \sum_{\theta} \theta\left(t\right) \innerproduct{K}{\DeligneLusztigInduction{T}{G}\left(\theta\right)},$$
where the sum is over all the characters of $T$. Using the formula $$\dblJacobiSumScalar{\pi}{\chi} = \tau\left( \theta' \times \chi^{-1}, \fieldCharacter_{2m} \right) \theta\left(-1\right) \tau\left(\chi^2, \fieldCharacter\right)^m,$$
we get
$$K\left(t\right) = \frac{\sizeof{\lieAlgebra}^{\frac{k}{2}} \cdot \DeligneLusztigInduction{T}{G}\left(\theta\right)\left(1\right)}{\sizeof{G}} \sum_{\theta} \theta\left( \left(-1\right)^k t\right) \prod_{j=1}^k \tau\left(\theta' \times \chi_j^{-1}, \fieldCharacter_{2m}\right) \tau\left(\chi_j^2, \fieldCharacter\right)^m.$$

Expanding the last sum, we have
\begin{align*}
	K\left(t\right) = & \frac{\left(-1\right)^k q^{-km}  \sizeof{\lieAlgebra}^{\frac{k}{2}} \DeligneLusztigInduction{T}{G}\left(\theta\right)\left(1\right)}{\sizeof{G}} \cdot \prod_{j=1}^k \tau\left(\chi_j^2, \fieldCharacter\right)^m \\
	 & \times \sum_{\theta} \sum_{\xi_1,\dots,\xi_k \in \multiplicativegroup{\finiteFieldExtension{2m}}} \theta\left( \left(-1\right)^k t \xi_1^{1-q^m} \dots \xi_k^{1-q^m} \right) \fieldCharacter_{2m}\left(\sum_{j=1}^k \xi_j\right) \prod_{j=1}^k \chi_j\left(\FieldNorm{2m}{1}\left(\xi_1^{-1} \right)\right).
\end{align*}
The sum on the second row becomes
$$\sum_{\substack{\xi_1,\dots,\xi_k \in \multiplicativegroup{\finiteFieldExtension{2m}}\\
\prod_{j=1}^k \xi_j^{1-q^m} = \left(-1\right)^k t^{-1}}} \left(\prod_{j=1}^k \chi_j^{-1}\left(\FieldNorm{2m}{1}\left(\xi_j \right)\right)\right) \fieldCharacter_{2m}\left(\sum_{j=1}^k \xi_j\right).$$

\appendix
\section{Evaluation of norm one exponential sums}

In this appendix, we evaluate several exponential sums.

\subsection{The case $\quadraticExtension = \finiteField$}
Let $\chi \colon \multiplicativegroup{\finiteField} \to \multiplicativegroup{\cComplex}$ be a non-trivial character. We compute $$\sum_{\substack{x \in \multiplicativegroup{\finiteFieldExtension{2}}\\
	\trace_{\finiteFieldExtension{2} \slash \finiteField}\left(x\right) \ne 0}} \chi\left(\FieldNorm{2}{1}\left(\frac{x}{\trace_{\finiteFieldExtension{2} \slash \finiteField} x}\right) \right).$$
As usual, we rewrite this sum as
$$\frac{1}{q} \sum_{z \in \finiteField} \sum_{y \in \multiplicativegroup{\finiteField}} \sum_{x \in \multiplicativegroup{\finiteFieldExtension{2}}} \chi\left(\FieldNorm{2}{1}\left(xy\right)\right) \fieldCharacter\left(z\left(y \trace_{\finiteFieldExtension{2} \slash \finiteField} x - 1\right)\right).$$
It is clear that if $z = 0$, then the sum over $x$ will vanish. Hence, we may reduce the sum to a sum over $z \in \multiplicativegroup{\finiteField}$. By replacing $y$ with $z^{-1} y$ and then replacing $z$ with $-z$, we arrive at the sum
$$\frac{1}{q} \sum_{y \in \multiplicativegroup{\finiteField}} \sum_{x \in \multiplicativegroup{\finiteFieldExtension{2}}} \chi\left(\FieldNorm{2}{1}\left(xy\right)\right) \fieldCharacter\left(y \trace_{\finiteFieldExtension{2} \slash \finiteField} x\right) \sum_{z \in \multiplicativegroup{\finiteField}} \chi^{-2}\left(z\right)\fieldCharacter\left(z\right).$$
Replacing $x$ with $y^{-1}x$, and using the Hasse--Davenport relation, we arrive at the sum
$$q^{\frac{1}{2}} \left(q-1\right) \tau\left(\chi, \fieldCharacter\right)^2 \tau\left(\chi^{-2}, \fieldCharacter\right).$$

\subsection{The case $\quadraticExtension \ne \finiteField$}
Let $\chi \colon \multiplicativegroup{\finiteFieldExtension{2}} \to \multiplicativegroup{\cComplex}$ be a non-trivial character and let $\theta \colon \NormOneGroup{2} \to \multiplicativegroup{\cComplex}$ be a character. Our goal is to compute $$\sum_{1 \ne x \in \NormOneGroup{2}} \theta\left(x\right) \chi\left(1-x\right).$$
Using the Hilbert 90 map this is equivalent to
$$\frac{1}{q-1} \sum_{\substack{x \in \multiplicativegroup{\finiteFieldExtension{2}}\\
x \notin \multiplicativegroup{\finiteField}}} \theta\left(\frac{x}{x^q}\right) \chi\left(1-\frac{x}{x^q}\right).$$
Let $0 \ne \delta \in \multiplicativegroup{\finiteFieldExtension{2}}$ be a trace-zero element. Replacing $x$ with $\delta x$, we arrive at the sum
$$\frac{\theta\left(-1\right) \chi\left(-1\right)}{q-1} \sum_{\substack{x \in \multiplicativegroup{\finiteFieldExtension{2}}\\
		x \notin \delta^{-1} \multiplicativegroup{\finiteField}}} \theta'\left(x\right) \chi^{-q}\left(x\right) \chi\left(\trace_{\finiteFieldExtension{2} \slash \finiteField} x\right),$$
	where $\theta' \colon \multiplicativegroup{\finiteFieldExtension{2}} \to \multiplicativegroup{\cComplex}$ is given by $\theta'\left(x\right) = \theta\left(\frac{x}{x^q}\right)$.
	
	As usual, we may rewrite the last sum as
	$$\frac{\theta\left(-1\right) \chi\left(-1\right)}{q\left(q-1\right)} \sum_{x \in \multiplicativegroup{\finiteFieldExtension{2}}}\sum_{z \in \finiteField} \sum_{y \in \multiplicativegroup{\finiteField}}  \theta'\left(x\right) \chi^{-q}\left(x\right) \chi\left(y\right) \fieldCharacter\left(z\left(y - \trace_{\finiteFieldExtension{2} \slash \finiteField} x\right)\right).$$
	Since $\chi \ne 1$, the inner sum vanishes when $z = 0$, and therefore we can reduce the sum over $z$ to $z \in \multiplicativegroup{\finiteField}$. Changing variables $x \mapsto -z^{-1} x$ and $y \mapsto z^{-1} y$, and using the fact that $z^q = z$, we arrive at the sum
	$$\frac{\theta\left(-1\right)}{q} \sum_{x \in \multiplicativegroup{\finiteFieldExtension{2}}}\theta'\left(x\right)  \chi^{-q}\left(x\right) \fieldCharacter\left(\trace_{\finiteFieldExtension{2} \slash \finiteField} x\right) \sum_{y \in \multiplicativegroup{\finiteField}}   \chi\left(y\right) \fieldCharacter\left(y\right),$$
	which by changing variables $x \mapsto x^q$ equals
	$$q^{\frac{1}{2}} \theta\left(-1\right) \GaussSumCharacter{\left(\theta'\right)^{-1}}{\chi^{-1}}{\fieldCharacter_2} \tau\left(\chi \restriction_{\multiplicativegroup{\finiteField}}, \fieldCharacter\right).$$

\bibliographystyle{abbrv}
\bibliography{references}
\end{document}